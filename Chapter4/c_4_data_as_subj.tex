\startcomponent c_4_data_as_subj
\product prd_Chapter4
\project project_thesis

\subsection{Data as Subjective Observations}

The next interviews explore the nature of data as a subjective observation. These constructions differ strikingly from the data-as-communications realities as they accord data a place in the knowledge/information/data hierarchy. There is evidence of Ackoff's hierarchy in interview III that contrasts with a more cyclic hierarchy in interview IV. 

Present in both interviews is the acknowledgment that data is inherently subjective, that it is produced from observations made by people, and that it must be contextualized with information to provide useful guidance. Data's lack of inherent value due to its subjectivity requires that it be \quotation{filtered} so that only the most usable aspects remain. 

\subsubsection{Interview III}
\sidebar{
The participant explored business dynamics of a meeting. Data is the basis of information and knowledge, and formed from observations. 
}

This interview, exploring the interactions of a meeting, was one of my more theoretical interviews. While most of the other interviews focused on more technical topics, the communicative/business focus of this discussion provided an interesting counterpoint to the other interviews featured in the rest of this analysis. Observation is the fundamental component of data, allowing interpreted data to be useful information. Information becomes knowledge, with the hierarchy continuing beyond knowledge: 

\startextract

Knowledge is probably not the top of the hierarchy. Universal truths, e.g. things called \quotation{Laws} in science, could be called wisdom (W). 
\stopextract

In the interview, the participant began with a discussion of their opinion of the relationships between data, information, and knowledge. Later in the interview, however, they admitted to a change of their preconceptions due to the interview: \quotation{I consider [data and information] to be a hierarchy, but it's not very clear in my mind} is a comment made by participant at the start of the interview. Later in the interview, the participant observed that participation in the \SDFN produced a change of understanding of data:

\startextract
Coming into this interview, I thought that the hierarchy was Data, Information, Knowledge with maybe Wisdom or something above that. Or something crazy like that. Way out there. During the meeting, I think I... with your... non-directional coaching, you helped me to see where things were probably Data, where those Data were used by Os to form Information. And then, pieces of Information, there's something higher above information.
\stopextract

Structured and organized data, as delivered for a persuasive argument, becomes information. Data can be transformed into knowledge directly, such as a scientist predicting movement with Newton's first law of motion. In a prediction of the future as information, the predictor has to take data (observations of something happening), and input them into knowledge (a scientific model). Interestingly, data also encodes information and knowledge as well as forming the basis of those two concepts.

Knowledge is the theoretical basis of understanding-of-the-world. Two people discussing a simple mechanical operation are communicating information, but the theory driving that discussion is knowledge. Furthermore, the jargon used in that discussion also represents knowledge, whereas a generalized description is \quotation{somewhere between data and information.} Thus, an understanding of the state of the art and the status quo is knowledge of what is possible or effective to do. Operation of effective management practices is also knowledge, despite the management theories themselves not being transmitted. Because the knowledge of effective management inspired the communications to modify behavior, the communications themselves are knowledge, representing the practice of those theories. 

Information is data structured for a purpose. Structured, organized data as delivered for a persuasive argument becomes information just as an historical-relationship context is also information structured for purpose. Data becomes information by a person or machine collecting data and then \quotation{drawing something out of it.} Just as information is a structuring of data, the participant suggests that knowledge may be a structuring of information. 

Data is an observation. A listing of daily events is a data flow just as a progress update is data. Nonverbal cues between people across a meeting table, such as eye contact or arm movement, also constitute data. When one of the meeting participants aggregates the nonverbal cues to supplement persuasive mechanisms, the participant considers that aggregation to be data. 

Another example of data is an intuitive insight, a \quotation{moment of clarity,} into the fundamental nature of a problem. \quotation{Data gathering} is used in the context of an experiment differently than the term \quotation{data} alone, with few relationships in common between the two. \quotation{Experimental data}, unlike normal \quotation{data}, can have different levels of quality: \quotation{we can classify the data better if we can see it clearer.} There seems to be little support for data as encoded information or knowledge. 

The resolution of cognitive dissonance in this interview suggests that the \SDFN\ can be used as a way to reconcile a participant's theoretical understandings of categories with their everyday use. This consequence, while useful in the exploration of philosophy, can also be useful as a practical way to resolve \quotation{silos} as it is a way to change a participant's understanding of the items under discussion. 

\subsubsection[Interview IV]{Interview IV}

\sidebar{
The participant is an engineer. Data are subjective observations filtered by knowledge and localized into information.
}

This participant engaged in multiple interviews, producing quite a lot of evidence and strong elaboration of their ideas. There is a cyclic hierarchy of data as observations, localized by information, and filtered by knowledge, becoming information and knowledge. Knowledge then directs the gathering of the next cycle of information. There are no obvious demographic characteristics to explain the strong subjective nature of this participant's views, though they seemed to engage more strongly with databases than did those of some of the other participants. 

Data is used both in singular and plural senses, data can have qualities, and measurement generates a data point. Data relates to information and knowledge in a cyclic hierarchy\footnote{Reminiscent of Tuomi's ontology (page \at[Tuomi]).}, such that some assertions about the world (knowledge) form a context within which to view data. \quotation{Knowledge filters data, which produces information, which produces knowledge. The acceptance of new knowledge is predicated on extant filters.} 

One example of context modulating the nature of data is the understanding of a sensor's purpose changing the observer's interpretation of the data produced by that sensor. In this understanding, background education is a primary contributor to a piece of data's context. 

Information's localization of data from the \quotation{study of the issue at hand} helps the recipient to understand what the data means. Information emerges from a contextualization of data: \quotation{In order to have context [within] which to understand data, background education (knowledge) and the study of the problem at hand (information) must be combined.} Knowledge and information represent different levels of understanding, with knowledge formed by integrated conclusions versus the ordered and contextualized data of information.

Understanding in this case is an awareness of applicability. All data, information, and knowledge interact with mental models of the worlds. Locally true items are information, whereas knowledge informs models that are more global. Because the generation of knowledge is a function of incoming filtered data, we sometimes get conflicts in explanations: 

\startextract
Explanation is contextualizing [models] for specific circumstances. Explanations should expand knowledge. Sometimes they contradict understanding, potentially due to cultural factors or mistranslation.
\stopextract

Knowledge filters for meaningful data, and data is contextualized from information into information. Contextualizing knowledge frameworks allow for an understanding of incoming data by situating that data within knowledge. The framework does so by providing a relevancy filter for incoming data: 

\startextract
Information is more concrete -- in the hierarchy. Data is concrete. I measured it via a defined process at a particular time. We might argue what it means, which is knowledge, but it is what it is. But the information ... I've reviewed a whole pile of things. Those are information sources, it's clear how I obtained them. They may be wrong, their provenance may not be certain. Data is specific, discrete. Information can be a collection of data. 
\stopextract

It is the process of collecting and localizing data which turns data into information. This set of localized data is {\em contextualized}, which increases relevance at the same time that it increases potential error or subjectivity. Despite claims of concrete data, data in this construction must be subjective: it is filtered from human observations. However, the self-reflection of that subjectivity creates meta-data about reliability and thereby filters for useful data.

Data is a discrete observation. Discrete data means the ability to specify \quotation{what, where, when: would define a piece of that, what the measure of it was, and its location at the time.} Data has a specific provenance and reliability. Data is treated as if it is atomic, despite the ability to perform different, closer observations of the same thing. The presence of the term \quotation{filter,} one of the most important keywords that I found, suggests that data as observation is produced from any observation of the world and is thus a subjective construction, tied to the observer, rather than evidence of the world. 

Meaning is produced through the negative feedback of knowledge. Data can be formed into data sets, which may be evidence of a somewhat local scientific language translating between purely subjective data and the \quotation{reliable} data that some scientists demand. Relationships between sets are information. 

There is no link to technical data in this construction. Representations of data (bits) are not the thing. Categories of representations are meta-data and bit streams are merely representations of data. \quotation{When people can attach meaning to bit-streams, they can become information. Normally glyphs become information, but when they're just representations of data (like the 'A' signifying a part of the base pair in DNA) they're still data.}

\stopcomponent
