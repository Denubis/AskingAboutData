\startcomponent c_4_raw
\product prd_Chapter4
\project project_thesis

\section{Interview 2}
\placefigure[]
[fig:i2]
{The SDFN Diagram for Interview 2}
{\externalfigure[Chapter4/graphs/i2.pdf][factor=fit,frame=on]}

\startextract
Participant: So we can do a typical Information request for example. It starts off with somebody asking me a question.

Interviewer: Which entities to start with?

Participant: Entities as in an individual?

Interviewer: Role entities?

Participant: Let’s do with a {position 86} because they’re close.

Interviewer: To you as?

Participant: In my role as an {position 87}.

Interviewer: There is something... what is this flow?

Participant: That flow could be either a person 1:1 chat or maybe phone or email or something like that.

Interviewer: What is it a communication of?

Participant: Information I suppose you could say.

Interviewer: How can we label it? The {position 86} is communicating with you Information. What is the context of this Information? What are they communicating with you?

Participant: They’re asking a question.

Interviewer: So the {position 86} communicates with you a question. You would say this question is Information? This is the {noun 97} we’ll be doing.

Participant: Just keep prompting me.

Interviewer: Then what happens?

Participant: It depends upon the question. If we do a typical question. Say they’re looking for research work in a particular area. It would mean that I search a particular database.

Interviewer: What entities are involved in you figuring out what database to search?

Participant: That’s from experience I suppose.

Interviewer: Is there a flow from one entity to another or is this innate experience?

Participant: It’s experience.

Interviewer: Is this experience, Data, Information, Knowledge, or O?

Participant: Whether is it Data or Knowledge. I suppose it’s Knowledge really. Data to me is not -- I say it’s Knowledge.

Interviewer: You use your Knowledge as experience to find a database. Is it safe to generalize to “database” here or shall we be more specific?

Participant: No, database.

Interviewer: What flows are there from you to the database that your expertise has chosen?

Participant: The Knowledge is that I know which one to search.

Interviewer: You have internal Knowledge of which one to search. Is there some sort of confirmation action that you do to confirm that Knowledge or is that tacit?

Participant: It’s tacit.

Interviewer: you now know which database to search. What is the first thing you do when you have found that DB?

Participant: You answer that question.

Interviewer: So you send something to that database?

Participant: As I said, it’s looking for perhaps a research topic. So we’re going to that topic request.

Interviewer: From the question you would generate a topic request. This topic request -- this flow, can be categorized as Data, Information, Knowledge or O?

Participant: Data.

Interviewer: We start with you as {position 87} sending a topic request through data flow to the DB. Then what happens?

Participant: You either get a result or you don’t.

Interviewer: Let’s assume the best case.

Participant: You get a result. You get Data which will then flow back.

Interviewer: So we have a flow backwards which is Data and this is a result?

Participant: Which would then be communicated back to the {position 86}.

Interviewer: Do you perform a transformation on this result?

Participant: By transformation, if it’s actually a physical entity, you might actually provide that to the person. Not just Information, if it exists like a copy or something.

Interviewer: So one case would be that you and the {position 86} get this result. This is in one case. Are there other cases?

Participant: That’s assuming we’re successful. If it’s not successful then we may need to go elsewhere. That would be the first choice of checking that DB.

Interviewer: This iterates around and around until you...

Participant: It might not be a database, it might be asking individuals for example. So you go to a different source.

Interviewer: Let’s diagram that. Which entity are you talking to as a class.

Participant: It would be another {position 86}.

Interviewer: How can we label that other {position 86} different from this one?

Participant: {position 86} B. So I’ve gone down this loop, got nothing. Then I would contact that person.

Interviewer: Before we go into that, is the getting of nothing a data flow?

Participant: Yeah it’s a flow.

Interviewer: What it’s a flow of? Let’s start by indicating a flow. This is basically a null result.

Participant: It’d be a flow back to {position 86} A saying that didn’t work.

Interviewer: So we’ve got this null result. Is this null result flow from the database to you Data, Information, Knowledge or O?

Participant: ??? <From diagram it looks like Data>

Interviewer: Then you perform -- you send something back to the {position 86}?

Participant: It would be Information back to them.

Interviewer: And what is this Information?

Participant: Information saying “I now will have to actually look further.” That’s when we start to look at people like {position 86} B.

Interviewer: What covering name can we give to that “look further” communication? Is it like a status update?

Participant: Yes.

Interviewer: Just to step back for a moment... let me know when there’s enough. We give a status update back, they may or may not refine their question. Eventually you may need to talk to a {position 86}. What is it that you communicate with a {position 86}?

Participant: It is that original question, that original Information.

Interviewer: Do you perform any manipulation of this question?

Participant: Possibly.

Interviewer: What manipulation do you perform?

Participant: It may be just narrowing down. Which could be Knowledge I know of various... they may have the wrong year of the research project.

Interviewer: Perhaps a refined question?

Participant: Yes

Interviewer: Is this refined question Data, Information, Knowledge, O?

Participant: Information isn’t it?

Interviewer: You send, as {position 87}, to {position 86} B, a refined question? What do they do?

Participant: They may know it in their brain already or they could refer me to someone else, {position 86} C, or they may actually have their own systems, actually.

Interviewer: This, in a way, is atemporal. And, because of that, there’s no causation. Let’s model the set of normal responses that the other {position 86}s can give to you. What flows back are there from the {position 86}?

Participant: I suppose one is that they cannot help at all: “Dead end.”

Interviewer: Is this Dead End Data, Information, or Knowledge?

Participant: Information. The other one is that they may refer you. That would be Information.

Interviewer: referral as Information.

Participant: Another one is that they check their internal systems, they may have DBs, something like that, and provide the answer.

Interviewer: So they are communicating with an entity.

Participant: Let’s say they have their own DB.

Interviewer: They send a flow to a DB. What are they sending to the DB?

Participant: It would be that original question --the refined question.

Interviewer: The refined question as Data, Information, Knowledge?

Participant: It’s Information isn’t it? No, it’s Data, sorry. It’s Data.

Interviewer: Do they perform a transformation on this?

Participant: No, I don’t think so.

Interviewer: And the DB returns what?

Participant: Something positive, basically.

Interviewer: Could we say it’s the result / null result thing?

Participant: Yes. They get the same, a repeat.

Interviewer: They have a DB result, what then happens?

Participant: They refer it back to me.

Interviewer: As part of the referral or as a different Information flow?

Participant: It would be the same.

Interviewer: so they get a hit and they refer the hit back to you.

Participant: Yes that’s right

Interviewer: and you perform some sort of status update and give them a response. What other entities do they consult? In a generalized case?

Participant: The research system? Their own brain? They may talk to somebody else. And it just repeats.

Interviewer: If they have neither referral nor a dead end, is there something else they can send back to you?

Participant: No, that would be the end of it, wouldn’t it?

Interviewer: In their expertise they have some answer in their expertise is a referral?

Participant: Yes.

Interviewer: Say they dead end you, then what happens?

Participant: Go back to {position 86} A to find out just how important it is. That’s all to do with the priorities.

Interviewer: How can we model that flow? Is it part of the Status update or is it different?

Participant: I suppose it’s status, it would just be a different priority in status.

Interviewer: You give them a status update. Do they then give you anything?

Participant: They either say keep going or they might have a refined question. This could go round in a lot of circles.

Interviewer: I’m getting that sense. They send back to you ...

Participant: Information to say either stop or more Information to help assist.

Interviewer: Can we combine those in one covering statement.

Participant: It’s just a re-request or continuation -- priority.

Interviewer: Because they’re clarifying what is a priority.

Participant: Yeah, that’s right.

Interviewer: You send them a status update back one way or another. They prioritize...

Participant: Yeah.

Interviewer: Are there any other entities involved in this {noun 97}?

Participant: It again depends on the question. If it’s definitely internal. You stick with internal, you may get a dead end. Or if it’s something that perhaps somebody outside the organization might be able to assist with then you do the sort of steps with other people.

Interviewer: You then do {position 86} external.

Participant: Exactly, or DB, external. That sort of thing.

Interviewer: is there an entity which is neither a database nor a {position 86}? Or is there a different type of person that’s meaningfully different?

Participant: It could be another {position 87}.

Interviewer: Let’s model that flow. You as {position 87} A, talk to {position 87} B. And what do you communicate with them?

Participant: The refinement -- it would probably be the priority one by this stage.

Interviewer: Refined question with priority. Information?

Participant: Yes.

Interviewer: And then?

Participant: And then they would do something quite similar to what I just done. They would consult a system or an individual.

Interviewer: Same flows?

Participant: Same flows.

Interviewer: If they return, from their sources, dead end or null result.

Participant: Which is possible.

Interviewer: What happens?

Participant: Back to {position 86} A via myself.

Interviewer: What do you get and what does the {position 86} get?

Participant: Not a lot. But Information though. Or as in 0.

Interviewer: What label do we want to put on not a lot? We can absolutely call it not a lot.

Participant: No result, basically.

Interviewer: And you communicate that back to {position 56} A through status update?

Participant: Communication, status update.

Interviewer: If they don’t get no result ...

Participant: Sometimes they’re quite happy. Sometimes they’re just testing it out. That’s what {position 86}s... they don’t actually want you to find anything. Particularly if it’s a patent or something like that. They go through all that

Interviewer: due diligence.

Participant: But other times they’re very unhappy, so it depends.

Interviewer: Besides no result, what would you get from {position 87} B?

Participant: Probably a bill, an invoice. Just a status update basically saying they’ve searched this, this, and this, and that it hasn’t been successful.

Interviewer: Do we want to label this invoice or status update or both?

Participant: both.

Interviewer: Invoice is what?

Participant: It would be data, wouldn’t it? Data, Information, what was the other one? Knowledge? No. It would be Data.

Interviewer: And they search status update you.

Participant: Which is Information.

Interviewer: What happens if they do find a result

Participant: they do? Then they provide the Information as in Data. Or they may do a status saying we have to pay them for this part.

Interviewer: So you get search status and invoice one way or another.

Participant: Precisely.

Interviewer: You said you get the Data as Information?

Participant: The status would be Information, the actual article or whatever it is is Data. The result... If we had a result that would be Data.

Interviewer: The search result would be Data as opposed to no result which is Information. We have external {position 87}s iterate over that stuff according to the priority. Other entities? What do you think {position 86} A is doing with this request?

Participant: They’re probably -- they could be one of the people tasked in their team to work this out. There could be a team entity over there.

Interviewer: What interactions do you see {position 86} A having with their team?

Participant: There would be some sort of Information communication between themselves and the team.

Interviewer: Then, to team, Information.

Participant: It would be the status update.

Interviewer: Do they perform a manipulation of the status update

Participant: They would just convey that Information to the team.

Interviewer: Does the team send any flows back to {position 86} A?

Participant: They may to -- it depends on this loop, doesn’t it. It may have happened to get that priority for example. To say yes it was important to keep going. Then we went through this loop down here.

Interviewer: So is the team prompting A to send priority? Does A manipulate the team’s prompt?

Participant: They stick it in the e-mail for a month or so and wait.

Interviewer: That actually is a manipulation. It’s checking informally “is this important?”

Participant: Don’t know. They may or may not.

Interviewer: If they don’t, there’s nothing here. If they do?

Participant: I wouldn’t know unless it was in the trail from an e-mail, for example. They’ve sent it out and they sent the original team’s response back, then I can decide what they’ve done. But otherwise, I don’t know if they’ve manipulated or not.

Interviewer: What is your guess here?

Participant: Guess is that they don’t manipulate.

Interviewer: so there is a flow from team to you of priority. But there is also a flow from A to you for sending this. But the impetus for the question is always from A.

Participant: it looks like a bit of a mess. That’s what some days are like. You go around in circles.

Interviewer: Something slightly different. We have data flows to and from the DB system and the invoice. Are there any Data flows in that zone?

Participant: There would be if it turned out to be... we haven’t finished off the -- we’ve gotten the result.

Interviewer: Do you perform a manipulation of the search?

Participant: Maybe in a tiny way. Putting it in a format so that it’s more acceptable to us here.

Interviewer: Which is absolutely manipulation. What shall we label this manipulation as?

Participant: Just formatting isn’t it?

Interviewer: Formatted search result, which would happen with any of these entities.

Participant: Yes.

Interviewer: This formatted search result is Data, Information, or Knowledge?

Participant: It’s Data.

Interviewer: Is there a flow of Knowledge at any point in this {noun 97}. Can there be a flow of Knowledge?

Participant: It could be from {position 86} B? or it could be from the {position 87}. So we’re assuming here that they’re searching a system. They may, as I said, just be searching their tacit Knowledge.

Interviewer: Then they would communicate?

Participant: That’s right. Write it, or pop it in an e-mail, that sort of thing.

Interviewer: If {position 86} B is communicating back to you. What shall we label this Knowledge flow as?

Participant: It’s just their Knowledge isn’t it? Their experience.

Interviewer: So their experience?

Participant: Yeah.

Interviewer: What about Info Broker B?

Participant: Yes, I suppose it could be Knowledge there. It depends on -- I’m thinking of specific Information questions. They’re more like an intermediary that I’m thinking of {position 87} B in that they may have gotten Knowledge from somewhere else, but they themselves, that entity, no. It’s not the Knowledge, it’s just the Data.

Interviewer: Can we assert that Info Broker b would just have {position 86} Bx.

Participant: The sort of external systems similar to these internal. Yeah.

Interviewer: But they themselves wouldn’t. They would just pass --

Participant: Intermediary.

Interviewer: You are then the recipient of their experience qua Knowledge. What do you then return?

Participant: Right, OK. Then I either just relay it. Or if it’s too technical, I actually get them to talk to each other.

Interviewer: That sounds like two different flows. Flow the first is that you’re relaying it. What is the flow here between you and research A?

Participant: It’s just an exchange of Data.

Interviewer: It’s just formatted search results?

Participant: That would be it.

Interviewer: You’re formatting Knowledge as a search result.

Participant: That’s right.

Interviewer: And that’s Data?

Participant: Yes.

Interviewer: Or? You are referring B to A.

Participant: Yes.

Interviewer: Would this fall under formatted search results as well? With just different formatting? Or is it substantively different?

Participant: It’s just Information. Isn’t it? Not that much, no.

Interviewer: It’s just Information?

Participant: Yes.

Interviewer: What Information is going?

Participant: Just Information saying yes, I do know where that ... contact.

Interviewer: This is obviously a superposition of many generalized cases. Are there any other common avenues of exploration in this?

Participant: No, that’s good enough for that particular topic.

...

Interviewer: Let’s move to a more theoretical discussion. You label yourself, in one of your roles, as an {position 87}. To you, what is Information?

Participant: It’s something that’s not physical, basically. That means it could be a communication, it could be a conversation or a story. Something verbal.

Interviewer: So Information is any non-...

Participant: Data to me is physical, basically. It’s an entity that filters?? an entity of some sort.

Interviewer: You would say that letter is Data?

Participant: I would say that letter is Data. But what is on that Data is Information. Just to be a bit confusing.

Interviewer: Tell me more.

Participant: Coming from the field we’re in, in Libraries and Record {group 130}. You’ve got different areas. It would be, as I’ve said, Data or record. That’s just the physical entity.

Interviewer: Data is a container...

Participant: Yes, of the Information.

Interviewer: And Information is content of what type? Is there something common to all Information?

Participant: it’s not easy. I’m struggling.

Interviewer: Frankly, this is why I’m doing my PhD

Participant: It’s a struggle. Information, what is Information? It’s just... No, I’m drawing blank.

Interviewer: We have right now Data

Participant: And then of course you get Knowledge.

Interviewer: We have Data, Data is a container for Information.

Participant: I’m quite happy with that.

Interviewer: You say physical at some point?

Participant: Yes, physical in t??. It doesn’t actually have to be physical in a piece of paper, but it can be physical as in an e-mail message.

Interviewer: Physical as real?

Participant: Yeah, physical as in real.

Interviewer: Except that doesn’t really help.

Participant: What about in terms of semiotics.

Interviewer: Is there some sort of semiotic connotation or denotation of data? So this is Data containing the Information of the entities we described.

Participant: Yes.

Interviewer: Container is actually the correct word.

Participant: I think so.

Interviewer: Because it gets fuzzy of what the nature of the container is.

Participant: Yeah.

Interviewer: We have Information.

Participant: Information is ... Was talking about Knowledge. Knowledge to me is more than just Information. If we go backwards perhaps. Knowledge, you actually have to associate it with an experience.

Interviewer: So Knowledge is experiential?

Participant: Yes. Whereas Information is just there. It’s provided nobody actually read it or analyzed it or anything like that. Whereas Knowledge has been more analyzed and used. And I’m getting very vague.

Interviewer: Knowledge is Information which has been analyzed?

Participant: Yes, yes that’s right. It’s the next step up, I view it as.

Interviewer: Analyzed and used Information. Can there be a -- Knowledge is experiential. This experiential Knowledge allows us to do what?

Participant: That’s how a doctor knows how to operate, basically. It’s because they’ve had Information, and they’ve analyzed it. So that’s the tool for them to operate. It’s like somebody else coming in with just having some Information. But they don’t have the actual analysis or expertise. Which seems to be the Knowledge that the surgeon has.

Interviewer: So a surgeon has expertise about the world, and his expertise about the world is Knowledge.

Participant: Expertise about his Job, world, whatever.

Interviewer: Can this expertise about his or her job be verbed into Information?

Participant: I suppose it could be?

Interviewer: which verb?

Participant: I don’t know?

Interviewer: But there is a movement back.

Participant: Yes.

Interviewer: Is there a movement from Information to data?

Participant: No, not in my view. I view data as something completely different.

Interviewer: We’ve got up and down arrows [between Information and Knowledge] but we don’t have an arrow there [to Data]. We have Data as a bucket.

Participant: Yes, as a bucket, in my mind, whatever.

Interviewer: of Information or Knowledge?

Participant: and/or Knowledge

Interviewer: so there’s a difference in nature or kind between Data and these other two. But not between these two [Information and Knowledge]

Participant: No, these two get fuzzy.

Interviewer: Information can be turned into Knowledge?

Participant: Information can be turned into Knowledge.

Interviewer: Through analysis and use?

Participant: Yes.

Interviewer: Knowledge can be turned into Information through what?

Participant: I suppose it depends on who uses it basically. It’s the user of that. It’s somebody with ... not that. But what it was. Its like it’s been that level of ... It’s like dumbed down in a way. Somebody reads this and they’ve got the Knowledge, but somebody else reads it and it doesn’t make sense to them, it’s just Information. Because they don’t have...

Interviewer: Where is the sense making component here?

Participant: I don’t know.

Interviewer: The first question is: is there an external sense-making component?

Participant: I don’t know what you mean.

Interviewer: If I’m reading that, and I’m a lawyer,

Participant: that’s right, OK.

Interviewer: Versus that and I’m a painter with no Knowledge of the law.

Participant: That’s where I would say that the painter would just be reading that, and it would just be Information to them. But the lawyer would then become Knowledge. Because they would read into it their experience or whatever.

Interviewer: The lawyer would apply his or her Knowledge? Or incorporate that into his or her Knowledge?

Participant: Would incorporate that into.

Interviewer: So that’s part of the analysis

Participant: Yeah.

Interviewer: say you’re teaching someone as an apprentice. Is there a flow of any of this sort in that kind of teaching?

Participant: Teaching you would start off with the Information, and they would through the exercises and the training, it would be ... their Knowledge.

Interviewer: You would cause them to develop

Participant: Knowledge. Their Knowledge, yes.

Interviewer: You’re not communicating Knowledge to them.

Participant: No they have to do something to create the Knowledge.

Interviewer: would it be safe to say that Information is some sort of vehicle for Knowledge?

Participant: Yes.

Interviewer: What kind of vehicle is it?

Participant: It’s the primary vehicle, really.

Interviewer: So Information is the way that Knowledge is communicated?

Participant: Yeah.

Interviewer: Can Knowledge be communicated? Say you’re talking to {position 87} B, outside of this context, as equals. Can you ever give them a flow of Knowledge that’s Knowledge a opposed to a flow of Knowledge encoded in Information?

Participant: I’m not sure.

Participant: I suppose -- you do assume that they have a level of Knowledge, so yes, it would be an interchange. You’ll start to use, perhaps, jargon or something, assuming they know what you’re talking about.

Interviewer: Let’s talk about jargon. What is jargon?

Participant: Jargon, I suppose, would be a special language between specialists.

Interviewer: A verbal shorthand?

Participant: And each discipline would have its own.

Interviewer: Jargon is what?

Participant: Just a cryptic way of exchanging Information.

Interviewer: Jargon is a cryptic or short way of--

Participant: A shorthand

Interviewer: a shorthand for Information.

Participant: between specialists.

Interviewer: that’s not a Knowledge exchange.

Participant: No, it’s Information exchange.

Interviewer: Can there be a Knowledge exchange without using Information?

Participant: I’m still, I don’t know. There probably is but I can’t think of examples.

Interviewer: Say I’m writing my dissertation. Clearly, because I’m typing into a computer, Data is the container, yes?

Participant: Yes.

Interviewer: Am I typing my Knowledge into a computer, or am I typing my Information?

Participant: You’re typing your Knowledge. Because you’ve been through the {noun 97} of filing the Information.

Interviewer: So I’ve collected Information through whatever, and I’ve got all the Knowledge that I’ve got. The {noun 97} of putting Knowledge into a Data container, that does not touch Information? Or does it?

Participant: I don’t know. As you say, you could just be putting it in there, it’s not Knowledge, it’s just word for word from somebody else. You’re not relating it or transforming it or anything. So you’re not adding to it. As I said, if I come back to the feeling that Knowledge has been added to.

Interviewer: As you know I’m going to be transcribing this recording. The {noun 97} of me typing this recorded conversation falls where in this?

Participant: It’s Information. It just depends upon if you make any changes. If it’s a straight word for word, that’s just Information because you haven’t altered it in any way.

Interviewer: So our communications are Information.

Participant: Yes.

Interviewer: It’s recorded on the computer as Data.

Participant: Yes.

Interviewer: Me typing it would just be changing the Data container?

Participant: yeah. That’s right

Interviewer: Because it’s just Information, because there’s no analysis or use.

Participant: By the time it gets to your thesis, it turns into Knowledge.

Interviewer: Which I’m encoding into Data. But other people pick up a conference paper I generate from my thesis and theoretically read it. That conference paper, to them, is Knowledge or Information?

Participant: It could be either. It’s more Information. It’s initially Information until they choose to do something with it. Until it adds to their armory of whatever their skill set is.

Interviewer: Really, Information can produce Knowledge, but Knowledge can produce Information because someone is communicating their Knowledge to you as Information and it’s not until you analyze it that it becomes their Knowledge.

Participant: Yeah. That’s what it seems to be.

Interviewer: Could it be that the Knowledge to Information bit is abstraction?

Participant: Oh yeah.

Interviewer: If I’m communicating through a conference paper to you, my Knowledge, but you’re getting it as Information, I’m abstracting our my use and analysis of it, i.e. my internal tacit Knowledge. Can we think of a better word than abstraction?

Participant: No.

Interviewer: I want to make sure I’m not putting words into your mouth.

Participant: That’s OK. There’s no word there at the moment.

Interviewer: Final question. We have Data, Information, and Knowledge. First, is there anything to either side of Information or Knowledge?

Participant: Not that I can think of at the moment.

Interviewer: Knowledge is the worldview and Information is abstractions that can eventually be?

Participant: Yes.

Interviewer: If I’m making a prediction. Let’s do a really trivial one. I predict when I release this pen, it will drop. What is that?

Participant: That’s probably Knowledge, because ...

Interviewer: If you say to me, go research X, what is that?

Participant: That’s just Information, just telling you to a task of some sort.
\stopextract

\section{Interview 3}
\placefigure[]
[fig:i2]
{The SDFN Diagram for Interview 3}
{\externalfigure[Chapter4/graphs/i3.pdf][factor=fit,frame=on]}

\startextract
Participant: For about two and a half of the three years, I played the role of {position 100} of the project. And even that name was an interesting name, because it was felt that the invent, although a very clever person, was not a good {position 41} of people. So they needed somebody who could come in and {noun 41} the project. And, of course, keep the {position 101} happy at the same time. So there’s a little bit of interesting stuff there. It was fine, I did never fight with the {position 101}, we both fought Os together.

Interviewer: So we’ve got the {position 100}, we’ve got...

Participant: So there was a {group 104} that the {position 100} was the leader of. We were a fulltime group who mainly sat together.

Interviewer: Let’s look at this interaction. Did you send anything to the {group 104}?

Participant: I sat among the {group 104}. We’re talking about a time before there was personal computers, before there was e-mail. And so, there were of course, telephones, etc.. So we sat together. Typically, we would talk for a period each day about what we were trying to do and who was doing what.

Interviewer: Let’s model that talk. That thing is, even though you’re sitting together, there are flows of stuff. When you’re sitting together in his meeting each day and you’re talking to each other. What was one of the categories of things you talked about, that you told them?

Participant: In the early days, what we were trying to do, there was the {verb 102} part of it. So we were developing {verb 102}. Then there was another thing early on with equipment {verb 52}. It was quite important to get the right equipment because the aim that we were heading for, that we realized we were heading for, was some kind of automated {noun 75}. It was agreed without having been proved, that the {position 101}, he had the ability to control the {noun 103} better than the normal {position 90}s. He said it I do it by using the {noun 111}, and I use the {noun 111} this way. People were suspicious and thought that he was taking more cues than were actually in the {noun 111} and people wanted to say: “Can a machine understand this {noun 111}? and can a machine take the same kind of {noun 97} changes?”

Interviewer: Let’s look at the very trivial equipment {verb 52}. You as {position 100} are talking with these folks about equipment {verb 52}s. ...

Participant: What we within the {group 104} needed to do was to work out what was the most important pieces of equipment to have for purposes of conducting a {noun 75}. We were discussing it and working it out together. And of course we’ve got another bubble of the people who were actually going to fund it. That was the {group 105}.

Interviewer: So the {group 104} sent to the {group 105} a proposal?

Participant: yes.

Interviewer: Would you say that this proposal is Data, Information, Knowledge or something else?

Participant: Probably Information.

Interviewer: so the {group 104} sends to the {group 105} a proposal that is Information.

Participant: It is probably Information {noun 95}ed to persuade the group. It’s not Data, it’s brought together somehow. You mightn’t agree.

Interviewer: The thing is, I’m a passive receptor. I by definition agree. Does the {group 105} send anything back to the {group 104}?

Participant: The {group 105} will either agree with the proposal or not agree with the proposal or do nothing.

Interviewer: Are these separate flows, or is it a response flow?

Participant: It’s a response flow. Typically there would be two ways of doing this, there would be a formal way where there’s a meeting where the decision was taken or there would be a less formal way.

Interviewer: Are they different?

Participant: The result would be pretty much the same.

Interviewer: Is the result, Information, Data, Knowledge, O?

Participant: I dunno. We’ll go with Information at this stage.

Interviewer: Don’t feel compelled to choose Information just because of the other two.

Participant: It could be Knowledge, I suppose.

Interviewer: What’s the best fit?

Participant: I’m struggling with these definitions in my own mind.

Interviewer: If we’re struggling with the definitions we can do a theoretical conversation if you think that that would help you more or I can continue poking your intuition.

Participant: Let’s go intuition. You might discover what I actually think along the way.

Interviewer: {group 104} {position 100}, what are they sending you?

Participant: I think mostly Data.

Interviewer: what is one of the Data flows they send to you?

Participant: There’s lots of stuff about just the daily occurrences, progress.

Interviewer: Are daily occurrences data?

Participant: I think so. The way I’m thinking of them, it’s not anything that’s been considered or {noun 97}ed, it’s just Information. I shouldn’t use the word Information, it’s just what’s happening.

Interviewer: you can absolutely use the word Information if you want.

Participant: It’s just what’s happening.

Interviewer: If you want to use Data and Information as synonymous, you can.

Participant: I’ll try not to, I’ll try to use some sort of hierarchy of those words.

Interviewer: If you don’t consider it to be a hierarchy...

Participant: I do consider it to be a hierarchy, but it’s not very clear in my mind.

Interviewer: so they’re sending you daily occurrences in a data flow.

Participant: Lots of data, which I’m handling to try to understand what’s going on. As a {position 100}, in this particular thing, {position 100} was an euphemism for what I was being asked to do. And what I was really being asked to do was to be the leader of the group. Without offending the {position 101}.

Interviewer: Would you say that your role as group leader was different from your role as {position 100}, or was it just a polite label?

Participant: It was a polite label. But I tried to run a very democratic type of group. But I think when you work in a hierarchy, people don’t have to have it all that democratic. But you have to have some reality that people can say what they need to say anytime. And they can question any decision.

Interviewer: {position 100} back... So you’re assembling this. Are there flows of stuff there, perhaps to yourself or yourself in other roles?

Participant: I think in this particular case there were lots of flows of things, because in the {group 104}, in the main {group 104}, there was a {position 106}, a {position 107}. I’ll call him the {position 107}. There was the {position 101} and there was the {position 108}. Now, we all had to work together and my job was to lubricate all of these things to make sure that everything was happening correctly. There was a natural suspicion of the {position 101} with the {position 107}. The {position 101} and I were pretty reasonable friends. The {position 108} and I were pretty reasonable friends. And for a lot of the time the {position 107} and I were pretty reasonable friends. We never got to be enemies, but we got a little bit separated. There was a bit of tension across this way, and it was my job to make sure that everything happened as it should happen. The {position 101} couldn’t do the {noun 109}. And the {position 107}, wanted to be acknowledged for being a {position 107} and didn’t necessarily want to get ... he wanted to be a bit more of the -- not necessarily to delve into the depths of the detail. A bit more of an adviser. And the {position 108} just wanted to get on with it. Which is exactly the right sort of person to have. And it was my job to work on each of those issues. As I mentioned to you, this issue of {noun 110} in this {noun 111} was a tremendously important part of this endeavor, so I’ll talk about that a bit more. Things like knowing the hardware and so fort that the group wants, we dealt with it. We worked out what we wanted and the {group 105} gave it to us, etc.. But with this -- with respect to {noun 110}. ... The {noun 111} had various shapes that you could observe by eye .... And the {noun 110} thing became a classification thing. Now, in the early days of this, the best we could get to was about 65% correct. So the automatic method was only getting it right a bit more than half the time, but most of those errors weren’t bad errors. There’s good errors and bad errors. So just slightly wrong or very wrong. And one of the things that I did there, it happened really by just bringing everybody together, and we talked and talked and talked about this particular day. And eventually we got to the stage where we said “hey, we’re actually going to do this a slightly different way.” At least to the {position 107} I put it a slightly different way. I thought it was a fairly different way. When we talked around this, because if you do what you’ve always done, you’ll always get what you’ve always got. So we had to change. Because we might have got from 65% to 65.5%. But we weren’t going to go... So we made this change on how the {verb 112} would be done, and on the first attempt the 65% went to 95%. That was a tremendous breakthrough for the project, because at 65% it was not going to be good enough. At 95% it was clearly very close to good enough, and it was not making serious errors at all at 95%. That was one of those things of group dynamics. I always think of that afternoon where we thr{noun 137}ed through those things of being one of those magic times that happen 2-3 times in a lifetime when something decent happened and we got somewhere. I considered myself to be inside this group.

...

Interviewer: Your conjectures... So we’ve got {position 100}?

Participant: So therefore I became the {position 113} of that meeting, which everybody allowed me to be.

Interviewer: Is {position 113} a better term for modeling that meeting than {position 100}?

Participant: Probably. But I was a participant as well as a {position 113}.

Interviewer: So there’s you as {position 113}, and there’s you as participant. What role were you playing as participant?

Participant: I was the {position 114} ...

Interviewer: So we’ll go with {position 114}?

Participant: Sure.

Interviewer: Because there’s time when you could be sending stuff to the {position 113}.

Participant: Yes, I could be. The {position 113} always listened to me.

Interviewer: What other roles did we have at that meeting?

Participant: The {position 101}, and the {position 108}, and the {position 107}. That’s who they were. You’re asking me what their roles were? The {position 107} was mainly a {position 115} of the current {verb 112} technology. The {position 101} was very passive. So it was a passive role, but I don’t know if I could put another word to it. Write down passive until we get to another word. And the {position 108} was basically a supporter of the idea that we needed to change.

Interviewer: Did any of them play multiple roles?

Participant: So I played a multiple role. Those two roles. And I think actually our roles were changing during the day. .... And then I became the {position 114} of that method. At the end of that meeting, I was very much an {position 114} of that method. It’s more perhaps part of the {position 113} role. I had to keep this meeting to be constructive. A few, I think can remember, a few pretty straight truths were told to each other at various times of the meeting, but it was in a constructive way. The {position 101} was pretty passive. He was wanting, quietly, everything to move to a new state. ... And the {position 108} who was supporting it again was wanting to make sure that we got. I think there was probably in the three people, myself the {position 108} and the {position 101}, we were all probably coming into the meeting thinking that the {position 107} needed to move ground a little bit. There was this element of coercion, subtly there.

Interviewer: Let’s start by looking at some of the flows of the meeting. Maybe with this coercion {noun 97}. What flows were there?

Participant: I think we probably started in a facts and data way. I would, as {position 113}, would have asked the {position 107}/{position 115} to just give us an update on where he was getting to.

Interviewer: That request for update, should we call it request for update or something else?

Participant: Call for Information, I think, maybe Knowledge. If we had Knowledge.

Interviewer: Do we want to call it for something? What was it a call for?

Participant: I think in casual use of the word, it was a call for Information.

Interviewer: Was this call for Information, Data, Information, Knowledge or O?

Participant: It was Information and Knowledge. that’s what I was wanting.

Interviewer: What were you sending over this call? What is the nature of that call?

Participant: I don’t know how I would put it. Could you step out of your role for a moment and help me with what you think it might be. Do you have this often?

Interviewer: We don’t have to identify it as Data, Information, or Knowledge. On the other hand, what you’re doing here, you’re telling them that they need to give you Information and Knowledge.

Participant: Maybe, what they give back. I ask for Information and Knowledge, but I think what they will give back are three things. Data, Information, and Knowledge. Probably all mixed in together.

Interviewer: So they give back all three.

Participant: It goes backwards to everybody.

Interviewer: What is the label of their response? What are you asking them to give, Information and Knowledge about? Their model?

Participant: Yes, it was about the progress about their model.

Interviewer: Okay, so we can say Call for Progress Update, because what you want is for them to give you Information and Knowledge about...

Participant: Exactly. Progress...

Interviewer: Now, what they give back to you, was it the progress update, or was it something else?

Participant: It was the progress update, and it was a whole heap of other {noun 111}s there, including non-verbal {noun 111}s about how happy they were with the world and those other things.

Interviewer: Let’s break that out. They’ve got the progress update itself. Just that, would you classify that as Data, Information, Knowledge, or O? Just...

Participant: That’s Data in my opinion.

Interviewer: But, along with that, they send non-verbal cues. To the same daisy-chain?

Participant: You’ve modeled it as a daisy-chain.

Interviewer: In the model it’ll go there and then explode. But I need some way of... It’s not a daisy chain, it’s a simultaneous communication. They send the progress update qua Data. Then they send the same non-verbal cues in the same explody.

Participant: So the verbal thing can contain Data as well as Knowledge and Information.

Interviewer: What aspect of the progress update contained Information?

Participant: The Information about the overall performance about the {noun 110} program.

Interviewer: So we have the progress update as Data. We have overall performance discussion as Information. And the second one is overall performance?

Participant: Yes. There’s some details, some breakdown of that. It’s not just a single thing.

Interviewer: But it’s an informative flow.

Participant: Yes, it is.

Interviewer: Do they also send a Knowledge flow?

Participant: If you want to take it up to another level, the level of: are we achieving our goals and things like that, I think it’s getting to be a more inclusive thing. It might be heading from Knowledge towards...

Interviewer: So they are sending something about goal status maybe?

Participant: There’s something more than Information there.

Interviewer: So the goal status is Knowledge?

Participant: Yeah, it’s a higher thing, and I think the ... I’m not quite sure where you put those. It might be part of Information. It’s getting to be part of a bigger picture thing. It’s bringing together a number of pieces of Information.

Interviewer: This bringing together, would you say it’s Information or Knowledge?

Participant: It’s hard to say in the abstract.

Interviewer: Would you say that it’s both?

Participant: It can be, yeah. I think part of it’s Knowledge, part of it’s Information.

Interviewer: What about nonverbal?

Participant: I think that in this situation where, after a while in this meeting, I think although we were trying to be a very happy family, each of us would know if we were on the team of 3 or the team of 1. It would be some feeling of -- I don’t have much support here or I do have a bit of support here. Which is probably the reason why the {position 101} was a bit passive. He was trying not to be on one side or the other. As {position 113}, I should have been that way too. As {position 113}, I was in that role, but in my {position 114} role, I was pushing my own...

Interviewer: In that case, who do you think was picking up these nonverbal cues.

Participant: I think everybody picks up the nonverbal cues. We’re very used to working with each other. At this stage, we’ve worked with each other for a long time.

Interviewer: These non-verbal cues, Data, Information, Knowledge, O?

Participant: Mostly Data, I think.

Interviewer: So the {position 115}, in his progress update is spewing out all kinds of stuff, and you called for a progress update. You say “{position 115}, give me a progress update.” probably in nicer tones.

Participant: I think we all gave progress updates, he wasn’t the only one.

Interviewer: Did all of you give the same coordinated... set of flows?

Participant: I think so. So the person I called the {position 115}, the {position 107}. The {position 108} gave his update, and I think the {position 101} probably talked about the things he was particularly interested in. What he was doing was opening up a new idea. He was sort of thinking this other idea doesn’t work, let’s try something out.

Interviewer: So they all set progress updates to the {position 113}?

Participant: To the meeting. We’re all in the meeting.

Interviewer: So people are sending it to the meeting, not to anyone specific. So we have {position 113} to {position 115} going “Call for progress” That’s something, but we don’t know what that thing is yet. Do you have any intuitions?

Participant: I don’t know. It’s not -- I’m thinking it’s a lower level thing, really. It’s somewhere down -- it might just be Data.

Interviewer: The {position 115} to the meeting sent? A progress update, as Data, nonverbal cues as Data

Participant: But he also conveyed Knowledge. And Information as well as the Data.

Interviewer: He conveyed his overall performance as Information?

Participant: yes.

Interviewer: Or do you mean as part of these?

Participant: It was conveyed, the update contained all of those aspects of Data, Information, and Knowledge. But I don’t mind if you show them by separate flows.

Interviewer: So all of these flows are contained within the progress update.

Participant: Because it’s not as well organized as computers talking to each other.

Interviewer: We’ll kill the progress update as a flow because we can decompose it as these other flows. So we’ve got nonverbal cues as Data, We’ve got overall performance as Information. And all of these we understand as part of the progress. As Information, we have the goal state.

Participant: Now the goal state, now we’re drawing it this different way, I think the meeting is actually concluding with this Knowledge, from those feeds. It is producing the Knowledge.

Interviewer: Who does the meeting send this Knowledge to?

Participant: At this stage, the Knowledge stayed within the meeting. But we probably wrote it down at the time. We would have written it on the blackboard in those days.

Interviewer: So the meeting produced Knowledge sent to the blackboard. The Knowledge produced we can label as?

Participant: What went on the blackboard would have contained these conclusions, which are Knowledge. The blackboard also got some other things like to-do lists.

Interviewer: Did the meeting produce these to-do lists?

Participant: Yes.

Interviewer: Is this to-do list, Data, Information, Knowledge, O?

Participant: It’s more or less Data, I think.

Interviewer: So, the meeting, sent to the blackboard conclusions and a to-do list. The {position 115} sent to the meeting as a whole nonverbal cues and overall performance of his algorithm. Did the {position 115} send anything else to the meeting?

Participant: Later on in the meeting, the {position 115} was a strong participant in the discussion about how we were going to improve the {noun 110}.

Interviewer: Improvement discussion?

Participant: Yup. I think it starts with Data. And then you sort of start to integrate things.

Interviewer: The meeting integrates things. What does the meeting do with these integrated things?

Participant: Then it’s trying to make decisions. The decisions are finally part of the to-do list.

Interviewer: So improvement discussion as Data goes into the meeting, it goes grindgrindgrind and it outputs the to-do list onto the blackboard. Does the {position 115} get anything from the meeting?

Participant: He will get items on the to-do list. He will get a lot of nonverbal communication, etc.. about what goes on.

Interviewer: Is he getting that from the meeting, the {position 114}, or the supporter?

Participant: These are very hard to separate. I think the way that we were trying to run it, it was from the meeting. The collective.

Interviewer: So the meeting was sending back to the {position 115} nonverbal cues?

Participant: Yeah.

Interviewer: These nonverbal cues

Participant: Data

Interviewer: Would you say that the meeting changed the {position 115}’s mind?

Participant: Yes, because at the end of the meeting he agreed that we would change the approach.

Interviewer: Did the meeting send anything to him to change his mind or did someone else send something to him?

Participant: I think that at this particular occasion, the reason I said I was the {position 114}, once or twice in a lifetime you get moment of clarity. And what we were talking about then was what was called the learning set. So of these X {noun 116}s or Y {noun 116}s, however many there were, what had been done in the past was that the expert, our {position 101}, would say that this set of squiggles is 1a and this set is 3c. And the squiggles were something that, ... we’re not necessarily particularly perfect. But then, around that time, we had just got our automatic data logging stuff. And we could see what these squiggles looked like on a screen spread out. And so, as we started to talk through these things, I think that the position I put to everybody was: if the learning set isn’t perfect, how can we expect the recognizer to get it right? The basic thing that was said was: let’s use our new equipment ..., and we will get things that we believe are absolutely perfect examples for the full period of the window and they will be our new learning set. There was no loss of face in terms of the methodology, the learning set was improved because we now were able to see it for what it was really with our new equipment. I think that really was the difference between the {position 107} and those of us who were working a little bit more in the field. Was that the {position 107} was thinking of this as much more abstract way, and we were saying, hey, this is ... And so that was really the difference and we said that probably within a couple of days we had assembled a new learning set and we had gone from 65% to 95% using the same technology, just changing the learning set. And everybody won. Which was lucky.

Interviewer: You as {position 114} had a moment of clarity that you transmitted to the meeting.

Participant: Yes.

Interviewer: This moment of clarity, Data, Information, Knowledge?

Participant: It came out as Data.

Interviewer: So you communicated this moment of clarity as Data.

Participant: It’s one of those things that people who are a little more extroverted tend to speak before they think. So the stuff that comes out can be pretty rusty. And I think the meeting was able to build on that. I clarified my own position, etc..

Interviewer: Let’s look at that. You as {position 114} sent a moment of clarity into the meeting as Data. Who built on that?

Participant: I think I got a little bit of encouragement, in the first instance, I built on that.

Interviewer: Where does that flow come from?

Participant: I would imagine that what happened was, that when that idea was first put up, people said oh yeah, there might be something in that. Then we said, the {position 108}: how could we do this? The idea was accepted almost immediately as being reasonable and then we wanted to know whether it was feasible to do it. And when it was proved to be feasible, people agreed that we should have a go at that.

Interviewer: Then you would say that the {position 108}, who was the supporter, sent a flow into the meeting of “let’s discuss feasibility.” This flow, Data,Information,K?

Participant: It’s probably just Data, I think.

Interviewer: Then, people went yeah, let’s talk about feasibility. The passive {position 101} sitting there doing...

Participant: At this point, although he had an alternative thing up his sleeve, he would have then supported this idea because it was a new thing to try and after we talked about the feasibility we didn’t think it would be that hard to try it.

Interviewer: So the {position 101}’s support of idea, Data, Information, Knowledge?

Participant: Probably just data.

Interviewer: What other interactions did people have in the meeting?

Participant: I think that this is where, later in the meeting, where the to-do list starts to come out. After the moment of clarity and getting it clearer and clearer, writing up on the board what things look like

Interviewer: your conclusions?

Participant: yeah, I think.... That’s right. I think we concluded from the Data gathering stage that we were in trouble, we needed to do something. And then, that’s the stage... what can we do? I had my moment of clarity, I threw that into the meeting, we talked about it a lot. I think that the {position 107} wasn’t sure about it at all. And he was actually pretty concerned that it wouldn’t be right. Because in all the books that he read, you get the expert to classify, and we were saying the expert isn’t good enough to classify even though he invented this. Fortunately, the expert agreed himself that maybe that was true. We said “we can classify better if we can see it clearer.”

Interviewer: Did you just persuade him with nonverbal cues? Or?

Participant: I think that there was a very logical argument. So that, we’re talking about all the people in the room scientific techo types that you’re familiar with. People respond to a logical argument. I don’t want to estimate non-verbal, because most of us techos most often think nonverbal doesn’t even exist. I think, at the time, I certainly would have said that logical argument prevailed.

Interviewer: Who put the logical argument to whom?

Participant: I think that what happened, it’s interesting now that we’ve written it on here. Once I had this moment of clarity, one of only a few in my lifetime, I was advocating that, because I could see it really clearly, and it was only a few minutes afterwards that I saw it as an axiom, that it was so clear that we needed to try this. Then the {position 108}/supporter, he was jumping on board. And he was saying “Yeah, I think it’s a good idea and I know how to help to do this. Here’s this.” The {position 107} wasn’t sure because I don’t think he was really defending something as much as saying: “I’m not sure, I’m not sure.” But I think probably the logical argument was: we actually have the tools to do this now, we’re not going to be waiting months to try it, we’re can try it tomorrow, and who knows? And that would have been the persuading things.

Interviewer: and you as {position 114} persuaded the {position 115}?

Participant: the meeting persuaded. I think that as I mentioned early on, I actually had pretty good relationships with all of these people 1:1. But there was not such a good relationship here or here, and so maybe I was helping everything to come on board because of the personal relationship side of it.

Interviewer: So we’ve got logical argument: Data, Information, Knowledge? What is this logical argument thing?

Participant: I don’t know how we’d classify theoretical... it is one of those things that you might receive Data, but the Data are all naturally clicking to gear in the receiver’s mind.

Interviewer: Let’s look at that. You say that you receive Data, and that the receiver’s mind “clunk clunk clunk.” So... that set of organized, structured data, that really comprises a logical argument. “Given this, this, that, ...”

Participant: As soon as you say organized structured Data, which is exactly what it is, it starts to become Information.

Interviewer: We would say that this logical argument as a collection of ordered, structured, Data is Information?

Participant: Yes, I think so. I think that’s where that changed. Because, at my moment of clarity, it was probably better that I threw this into the meeting as Data. And then we all started clunking, clunking, clunking with it, which meant that this wasn’t seen as me -- my thing, because we all worked on it. But, as I say, it didn’t take me very long, where this in my own mind had gone from an idea to being an axiom. Which, fortunately, proved to be true.

Interviewer: so we’ve got this logical argument from the meeting as Information to the {position 115}. But you also mentioned that you had a personal relationship between {position 114} and {position 115}.

Participant: and that was just in terms of, mainly an out of work hobby. So I had quite a good relationship with him.

Interviewer: This personal relationship. Would you say that the drawing on this background, basically going: “I’m going to persuade you, because you know that I’m not completely full of it because of our personal relationship.” would you say that persuasive technique is drawing on or communicating Data, Information, or Knowledge?

Participant: It’s certainly got an organization to Data, at least. Whether it goes further than that, I don’t know how these things are classified. I’m a bit unsure about how to deal with that. It’s beyond data.

Interviewer: Is it Information?

Participant: I think he would have used it as Information.

Interviewer: He perceived the flow as Information. Because, he saw it as “here’s this guy, I trust this guy because of our shared hobby. He’s calling upon that shared trust to persuade me in this meeting. And he would see that as Information.

Participant: I don’t think that, for this meeting, that would have been a major factor. Okay? It might have been that straw that eventually... “Yes, we’ll do it.” I think that what he was -- he did tend to sometimes defend himself. I think, in this meeting, he was actually defending what he -- knowledge of the technique. And he didn’t want the technique to be bastardized in any way.

Interviewer: The meeting produced conclusions, these conclusions were Knowledge. Did anyone else produce Knowledge?

Participant: I imagine that various people quoted Knowledge.

Interviewer: They communicated their Knowledge that wasn’t theirs. Who would have done that?

Participant: I certainly think the {position 107}/{position 115} would have done that. Because, to some extent, his reticence to try new things was on the basis of his Knowledge.

Interviewer: So he was communicating his Knowledge to the meeting... And the Knowledge was, status quo?

Participant: I think it was about the state of the art of these techniques and that sort of thing. I think similarly the {position 101} would have been from time to time, injecting his Knowledge of {noun 117} into the meeting. And the {position 108} as well, Knowledge of programming. He would have been helping the meeting with the Knowledge of programming, because we had to use programming as a tool. These are all kind of constraints to how it is. I might have even been communicating Knowledge of {group 130}. {position 113}. So, in a way, although one is communicating Knowledge, you can possibly be communicating it as Data for everybody to churn around with.

....

Interviewer: Does Data become Information or Knowledge?

Participant: Coming into this interview, I thought that the hierarchy was Data, Information, Knowledge. Maybe wisdom or something above that. Or something crazy like that. Way out there. During the meeting, I think I... with your ... nondirectional coaching, you helped me to see where things were probably Data, where those Data were used by Os to form Information. And then, pieces of Information, there’s something higher above Information.

Interviewer: So Information becomes Knowledge?

Participant: I think it’s can become Knowledge.

Interviewer: Do any of these mediate any of the Os?

Participant: Can you run that by in a different method?

Interviewer: Before I do that. What happens with Knowledge? Does Knowledge jump back, somehow, to either of these? Can Knowledge become Information, can Knowledge become Data?

Participant: I think we were talking about it a minute ago. I actually think that you can go both ways. You can send Knowledge back as Data and you can probably send I back as Data, too. I don’t know if that complies with the theories. This is just off the top of my head.

Interviewer: So Data can become Information. How does Data become Information?

Participant: I think Data becomes Information by a person or machine collecting the Data and drawing something out of it.

Interviewer: Can we put a less general generalization to something?

Participant: Somehow, integrating it. Firstly, if you’ve got say: numbers as being the data. You can tell is it going up, is it going down, is it jumping around? Is it... Those are Information.

Interviewer: Can other things be Data?

Participant: I think we discussed all sorts of things as being Data there. nonverbal stuff. Verbal stuff, not just numbers.

Interviewer: So, when we have verbal stuff, what transformations happen to verbal stuff qua data to something else?

Participant: So, when I take in someone’s Data that they’re sending, they’re being assembled in various ways in my mind. And I’m starting to believe certain things from those assembled group of words.

Interviewer: From that assembled group of words that is Data, you generate Information or Knowledge?

Participant: Yes. I think you can do both.

Interviewer: So Data can precipitate Information or Knowledge.

Participant: I don’t know to get Knowledge it has to go through the Information stage or not. I’m not real sure. I think that the analogy I’ve got in my own mind here is something along the lines of: “I’ve got a Newton’s first law of motion or something in my mind, and I see the Data coming in, assembling it to some extent, and suddenly it fits that model of that, so the Data is almost become Knowledge without going through the Information.

Interviewer: You have a model in your head. Let’s use Newton’s laws of motion. You would say that those are... Information, Data, Knowledge?

Participant: I think they’re Knowledge.

Interviewer: They’re Knowledge because you can use them to...

Participant: I can use them in lots of different ways. I can certainly -- I can predict the future from Data.

Interviewer: So you can predict the future from Data... So you’ve got Knowledge, i.e. the equations. And you’ve got Data which are...

Participant: something happens, something happens, I’ll put it in the equation and I’ll say, so that’s the time using... and I can say “hey, that’s going to happen.”

Interviewer: Saying “Hey, that’s going to happen.” which one of those is that?

Participant: I think It might be Information. Because the Knowledge got you the Information in that case. So we’re back here, we can go back there too.

Interviewer: So you would say that Knowledge can become Information with the modulation of Data.

Participant: Yes.

Interviewer: Because Knowledge + Data = Information.

Participant: In that case. I think I’d be a flunk in philosophy.

Interviewer: Nonsense. How else can Knowledge become Information? Say you’re teaching me about how a {Noun 2} operates. What’s going on there? Besides me flunking...

Participant: If I was teaching you about how a {Noun 2} works, firstly I’d tell you something really simple about what you put in the top and what you put in the bottom and what comes out. And then I’d probably tell you one of the most important things to understand is ... And therefore, how the real-world is constrained so that you can only get certain outcomes. So there are models that I would have in my mind, and you would say: “OK, I want to make lots more {Noun 23}, so I’ll put I’ll put a lot more ... in the {Noun 2}.”

Interviewer: So you can make rust?

Participant: Remember there’s a lot of {noun 46} in there. ... That’s not very intuitive is it? So you were exactly right. And I would say, you put more ... in, you can’t, because the top of the {Noun 2} will go cold. And I only know that it goes cold because I understand the heat and mass balance of the {Noun 2}.

Interviewer: So telling me that simple stuff: Stuff goes in the top, stuff goes in the bottom, you would say that you’re communicating to me...

Participant: That’s more Information. But the model of how the {Noun 2} works is Knowledge.

Interviewer: So you start out with Information. But you build upon that foundation with models. But I don’t have the ability to understand the models without the Information.

Participant: No, I don’t think so. Because you don’t even have a language.

Interviewer: The creation of a jargon is the {noun 97} of what? You’re communicating your what to me?

Participant: In common parlance, you’d say Knowledge. I’m imparting my Knowledge. But a general description is hardly Knowledge.

Interviewer: What are you giving me in that jargon?

Participant: I think it’s more or less somewhere between Data and Information. It’s just a description.

Interviewer: Because you’re giving me Data and Information. And the Data you’re giving me is this word means that, or is that Information?

Participant: If I say all the words, that’s definitely Data. When I give you a meaning so that you can connect it with something you already know, that could be getting to this stage of Information. But these are all self definitions we’ve been changing in the last hour.

Interviewer: Let’s wrap on the oh so much fun concept of the normative assertion.

Participant: The who?

Interviewer: Saying “Should. You should do that. You should learn how a {Noun 2} works.” What’s that?

Participant: I think I would -- it depends on the circumstance. If I just met you and somehow we thought of ourselves as peers, rather than you being an old guru or something or other. So, if you said to me, you should do that, I’d take that as Data. Now, if you were the old guru, and you said to me: “You should do this.” I would take it as being more than Data, I’d have to... there’s it’s not just one of the things that I’m assembling with other things. It’s something that I’m almost being compelled to...

Interviewer: would you say the old guru is communicating to the newbie Knowledge?

Participant: By saying you should? In a way, I suppose. You could say that person in giving that piece of advice, has assembled a whole heap of things over a long period of time and that would be Knowledge to them.

Interviewer: So they’re drawing on their Knowledge, but you’re getting....

Participant: I’m getting Data, initially. And then I’m sort of saying, is that person an old guru that I should take notice of or should I just put this in with a whole heap of other pieces of Data and work out whether I’ll do it or not.

Interviewer: And if it’s someone you should take notice of it’s promoted out of Data into?

Participant: I’m not sure where it’s promoted. It’s promoted somewhere, because it doesn’t have to be digested the same way.

Interviewer: Could you say that is promoted into Information or Knowledge, depending? Can you make a case for either?

Participant: I’m not sure, when somebody says you should, I can’t quite get it to be Knowledge in my mind.

Interviewer: It can be promoted into Information, you recognize that they’ve done operations on a whole heap of Knowledge and Data, and you’re getting something that’s privileged over other you-shoulds. Whereas as a road sign, it’s just Data, right?

Participant: Well, I hardly have ever seen a road sign in my life. But there are some that are mandatory.

Interviewer: That railroad sign there.

Participant: That’s mandatory. So I would obey it.

Interviewer: But even in noticing and obeying it, you would consider that it’s communicating with its presence Data?

Participant: Yes. I think it’s Data. I just wonder when we’re talking about this how much research has been done on this. ...

\stopextract

\section{Interview 4}
\placefigure[]
[fig:i2]
{The SDFN Diagram for Interview 4}
{\externalfigure[Chapter4/graphs/i4.pdf][factor=fit,frame=on]}

\startextract

Participant: A trivial example is we hold a meeting with the {noun 54}. And we are giving them data on {noun 55}, they’re doing some interpretation and providing it back to us.

Interviewer: Let’s model this. Who’s us?

Participant: Us would be {position 56}s, {position 57}, and {position 58}ing.

Interviewer: What do I put in the bubble?

Participant: [company].

Interviewer: other bubble?

Participant: {position 59}

Interviewer: You, send them, ...

Participant: Operating Data. Measurements from the {noun 60}. usually together with contextual questions. We’ve got a problem, this is what’s happening, why?

Interviewer: Is it a separate flow?

Participant: It depends on what you want to model it as. The two go together...

Interviewer: Entirely up to you

Participant: Leave it together

Interviewer: This flow of operating data is Data, Information, or Knowledge?

Participant: Data.

Interviewer: You as [company] send operating data to the {position 59}. The {position 59} then?

Participant: The {position 59} -- we need a separate one which is contextual question...

Interviewer: These contextual questions are: Data, Information, Knowledge, O?

Participant: Well, it’s going to be both Information and Knowledge, because the way you ask the question, as far as: "I have this Data, I think this is happening (Knowledge), But, my explanation doesn’t explain this, this, and this. What’s happening? We actually put it like: "[company] thinks this, what do you think"

Interviewer: These contextual questions hold Information and Knowledge. Are I and Knowledge. Is it possible to distinguish a flow of just Information from a flow of just Knowledge?

Participant: They’re usually two clauses in the same sentence.

Interviewer: That’s fine. Information + Knowledge. The Information in the contextual questions is what?

Participant: The Information? Would be things like "we have found these {noun 61}s." The Data part was their analysis. But we found the {noun 61}s, that’s Information. The Knowledge is, having done the analysis, they are made up of this and the likely chemistry is this, but that doesn’t explain why they formed, tell me why they formed. So that one sentence would have the Information and the Knowledge.

Interviewer: What does the {position 59} then do?

Participant: Ignores us. ... If they genuinely don’t know, they probably tell you. This comes up in a few different forums. If they know something and it’s happened in this project -- and if they know something is potentially a problem for you but you haven’t encountered it yet, they won’t tell you. When you have the problem if you ask the question the wrong way, they won’t tell you. So it can be quite frustrating to do a whole pile of work and go "We found this, this, and this, and our explanation is this" "Yes, we saw this last year on this other {noun 60} that’s not yours." "Did you think to tell us?" "You didn’t ask." ... But the form is those questions: "[company] has a this. We see this, this, and this. What is your experience?" ... What should happen is that they do some analysis on that data, informed hopefully by our Knowledge, and they return to us their analysis and explanation.

Interviewer: Do we want to have separate flows for these?

Participant: Might as well.

Interviewer: Analysis, explanation. Analysis is?

Participant: Something like: model outputs or calculations. Explanation is what that actually means in context of your {noun 60}.

Interviewer: Class of analysis is Data, Information, Knowledge?

Participant: It’s probably more on the Information. Well, it’s Data and Information. And the explanation is Knowledge, it had better be.

Interviewer: When you say that, what do you mean?

Participant: You hope that when someone gives you their explanation, you know more than before they told you. Not always true. They can tell you stuff, and you can go "Well I understand even less than when I started." Because if it’s completely contradictory to your understanding, you are now really confused. ...

Interviewer: Let’s explore that for a second. When their explanation conflicts with your model...

Participant: Here, model might be our model of understanding about the system... In the first instance, you try to go "Well, hang on, well, this this and this. But hang on" What more usually happens and you try to get a lot done in the meeting, so we go away and we talk amongst ourselves and we follow up in an e-mail for clarification. ...

Interviewer: What other entities are here? Where are you getting the operating data from?

Participant: When we said [company], inside [company] the operating data came from the {noun 60} data acquisition system.

Interviewer: Are we zooming in?

Participant: We’re zooming into [company].

Interviewer: that’s a new page. Comes from?

Participant: It comes from the {noun 63}. ... Level 3 are computers that talk to {Noun 10} and do calculations. So when you request data out of a database which is Level 4, it ultimately came from a lower level.. but it might be derived data generated at levels 2 or 3. So we say the {noun 63}. ...

Interviewer: {noun 63}... We’re going to zoom into {noun 63} eventually. I can tell that this is going to be fractal.

Participant: The couple of major things that are hanging off of it. A database. Let’s just call it a database. The data repository database.

Interviewer: entity? or internal?

Participant: Strictly speaking, physically it’s external, network wise it’s logically... just a repository for data afterwards. {noun 64} database. I have no idea what {noun 64} stands for. {Noun 23}making something something something.

Interviewer: {noun 64} database to {noun 63}...

Participant: {noun 63} to database is one way. We are sending Data for long term storage. Both raw and derived.

Interviewer: raw what?

Participant: Raw Data, would be an actual {Noun 4} output.

Interviewer: and we’re sending... derived...

Participant: Which is the output of calculations on raw data. Just trying to keep my context here. That’s where that Data is coming from. Is actually from that database. We query it.

Interviewer: this is the [company] ...

Participant: In this context. Yeah. When we’re trying to work out something, one of the things we’ll do is data mining. Generally searching that database. That database is our primary place to start.

Interviewer: You say it’s a primary place to start, so we need a you here. So where is this going?

Participant: So this is going into me

Interviewer: can we label me?

Participant: There’s me as {position 65}. Me as {position 65} is requesting Information from {noun 64}. But I can’t write to {noun 64}.

Interviewer: What do you send it?

Participant: I send it an SQL query.

Interviewer: So you’re sending SQL. This SQL is Data, Information, Knowledge, O?

Participant: It’s other. I can’t really classify it as any one of those three.

Interviewer: What can you classify it as that’s some sort of category?

...

Participant: So the database contains a model of something. The purpose of a relational database is to model the real world. {noun 64} is an {position 58}ing database so it’s a hopeless model. So you’re trying to convince that ... in the query, you know what the form of the answer looks like, and you’re trying to describe what you want. I say trying because we’re not always successful.

Interviewer: the SQL is a representation of you trying to... what? Model....

Participant: Well, I’m looking for Data in a particular form. And I have what is a model of the real world in the database. And so by imposing some structure on the data that is selected...

Interviewer: so the form of your question is is that you’re sending this form of data wanted to the database. What would you classify that concept as? Is it itself, Data, Information, Knowledge?

Participant: It’s Information I guess. It’s not Data. {noun 97} of elimination. Is it Knowledge? It’s not Knowledge in in of itself.

Interviewer: But you feel comfortable classifying it as Information because?

Participant: Because that’s what’s left. That’s not the answer you wanted, I love doing this.

Interviewer: We don’t have to classify something as Data, Information, or Knowledge.

Participant: That’s why I said other originally.

Interviewer: But we do need to put a word in for other that I can add to the pretty colors. As opposed to other which I can’t.

Participant: it’s metadata. Data about Data.

Interviewer: That we can absolutely describe. So it’s metadata. Data about data. The database then sends back, what?

Participant: Data. In the form of nicely formatted tables, hopefully.

Interviewer: Are we going to call the flow nicely formatted tables, or are we going to categorize the content of the flow? We’ve got SQL going there, so we can balance this with nicely formatted tables.

Participant: We’re getting back our Operating Data.

Interviewer: So this flow back is operating data? Is this Operating Data flow the same as that operating data flow?

Participant: It’s a subset of that. Because I can get some Operating Data that isn’t in that database so I can get it from other places.

Interviewer: Can we differentiate this Operating Data flow Subset from Operating Data with a unique name?

Participant: {noun 67} Operating Data. ... I also can query part of the {noun 63} directly.

Interviewer: this query is in the form of?

Participant: This query is in the form of me selecting trends in a {noun 68}. A mouse driven {noun 68}. It’s actually the {noun 69}the {position 90}s have for operating the {noun 60}. We have access to that read only. We can’t actually flip a valve, but we can get the valve’s state at much higher frequency that’s in the database for a limited period of time. ...

...

Interviewer: You’re sending -- this is interesting. The {noun 97} of you sending this to the {noun 63} is clicking...

Participant: I’m selecting named variables.

Interviewer: Conceptually, what kind of container does that fall in? A set of selections of named variables is a? What kind of label data flow as? Because we can talk about computer events.

Participant: It’s Information isn’t it?

Interviewer: I want you to tell me this.

Participant: It provided the menu, I selected the thing from the menu.

Interviewer: Do we want to summarize that or do we want to go into... is it sending you menus and you’re sending selections back?

Participant: It sends me a menu, I ask it to send me a menu, it sends me a menu.

Interviewer: Menu. This flow is a flow of?

Participant: Information.

Interviewer: Sending back is?

Participant: Data

Interviewer: Of what?

Participant: of my selection

Interviewer: Are there any other flows?

Participant: Having made my selection, it will then send me the Data that’s in that tag. A live updating trend which I can choose to save. What I’m essentially doing is subscribing to a tag, and as new data comes into the system it keeps sending it at me. {noun 70}, 

Interviewer: This live updating {noun 70} is ?

Participant: Data. Conveniently.

Interviewer: You just said you send subscription as well.

Participant: When you’ve selected the menu, you’re actually asking it to subscribe. you say "give me a list of things I can subscribe to." It then ...

Interviewer: In that case, you’re sending it selection, but at higher level you’re sending it a subscription request. Is this subscription request as selection Data?

Participant: Isn’t it Meta-data again? I don’t think about this that often, I just do it.

Interviewer: So, subscription request.

Participant: The interaction diagram for the {noun 68}, oh god. That’s what we’re doing isn’t it?

Interviewer: Yup. Now what?

Participant: Don’t start drawing this bit, at any time within the Data buffer which is about 9 months for that system, I can slide back. Then it’s no longer updated data, it’s now just historical data, it’s just that tag at that time.

Interviewer: So live or historical?

Participant: Yeah, that’s good. Then there’s other things that I’d be drawing on. {noun 60} measurements, the {noun 71} people taking {noun 37}, having chemical analysis done.

Interviewer: {noun 60} measurements is another entity? Just {noun 60} measurements?

Participant: We can do {noun 71} for {noun 71}. That’s one group I rely on.

Interviewer: Now, what kind of relationships -- flows are we talking here?

Participant: There’s their regular work, which is {noun 72} Regular {verb 73} that they do -- so they have a program of work that they have to do every year. There’s a specification of "you will {noun 37} at these points for these {noun 74}."

Interviewer: That {noun 72} goes to you?

Participant: It goes to a bunch of people. There are many recipients. There’s a heap of people who use it for a bunch of different things. In this purpose, I’m one of the recipients. Let’s just call it programmed {verb 73}. It’s on their program

Interviewer: This programmed {verb 73} is a flow of?

Participant: Data. I can make a request to them to do something special.

Interviewer: Let’s label this request. What are you sending them?

Participant: Assuming I had approval I’d be sending them a program of work. {noun 75} work? Anything abnormal is a {noun 75}.

Interviewer: {noun 75} work is a flow of?

Participant: The request to them is Information. I’m asking them -- telling them what to do and they’re sending back Data again which is the results.

Interviewer: So they’re sending you back Data. This is {noun 75} results?

Participant: {noun 75} results. Then there are {noun 75}s I carry out as {noun 75} {position 100}.

Interviewer: So you as Data analysis talk to you as {noun 75} {position 100}? Is a different bubble. You as {position 65} send to you as {noun 75} {position 100} what?

Participant: I need Information on something.

Interviewer: You as {position 65}?

Participant: the {position 65} needs to know something so the {noun 75} {position 100} needs to {noun 95} a {noun 75} to actually find that out.

Interviewer: Okay, that flow is Information?

Participant: Yes. And that’s a request for Data.

Interviewer: And this request for Data is not meta-data because?

Participant: Um, Meta-data is... I’m saying SQL is meta-data. It’s because I’m interacting directly with a model of the system as opposed to here where it’s like "I need to know about those {noun 61}s that are over there, I want you to go and get some, get some analysis done on them."

Interviewer: So you send, to the {noun 75} {position 100} a request for Data which is Information. Then what happens?

Participant: Let’s condense everything that goes on around that into "the {noun 75} {position 100} runs a {noun 75}" which is -- actually collects the data.

Interviewer: The {noun 75} {position 100} {noun 97} returns what?

Participant: In terms of -- they’re going to return Data and Knowledge.

Interviewer: 2 flows?

Participant: 2 flows, Depending on what they’re ask. And then find something they wanted to ask, then they return that too. Depending on what we find, we have to go and return it to a person who didn’t ask, who might not be me.

Interviewer: Data?

Participant: If it’s a straight request to go and just get a measurement, that Data would be {noun 60} Measurements.

Interviewer: The Knowledge is

Participant: The Knowledge is the higher level stuff that happened. So I made these measurements, but while I was there I was watching what was going on and actually though we measured those heights, you can see that they’re all part of a flow. That kind of thing. The {noun 75} {position 100} imparts some useful abstraction about what’s going on.

Interviewer: the problem with fractal diagrams is the conservation of inputs and outputs. otherwise I get very confused.

Participant: You realize that if you did this again tomorrow, you’d get a completely different set of diagrams, but anyway... Well, the core of it would be the same. There would be differences.

Interviewer: .. I’m trying not to ask to do this again tomorrow. ...

Participant: Next time you’re down, you can do it. I’m actually relatively interested in this stuff.

...

Participant: what’s the {noun 75} {position 100} doing? We’re actually expanding the {noun 75} {position 100} out into their own...

...

Interviewer: That we need to model. Who receives requests for Data in this zoomed in model?

Participant: The request for Data, it would be going to... {noun 75} {position 100} is me but it’s the {noun 97} of ... having identified the need for Information separate for how we’re going to get it. We have to empanel a meeting of the important stakeholders...

Interviewer: I want conservation of inputs... let’s start with this flow. Which entity does that request for Data hit inside {noun 75} coordination?

Participant: It does depend a little bit on exactly what was asked. It could hit a number of people. That’s why I call it {noun 75} {position 100}, because it’s in the center of everything. When one or more of the people interested in this part of the {noun 60} decides that the request has come and it’s important, decided somehow that it’s important.

Interviewer: So this has already been a validated request.

Participant: so we’ve already decided it’s been important

Interviewer: which probably means we need someone to do validation.. who does validation? And is it at this level or this level?

Participant: There’s the validation that self does, there’s the validation that team that self is part of does, and then there’s the validation that the wider people paying for it do.

Interviewer: Do any of those three levels of validation need to be differentiated from the Os?

Participant: Well, if they need to be differentiated, then we need to fractal {noun 75} {position 100}. If they don’t need to be differentiated, then we don’t need to bother fractalling this guy.

Interviewer: is this of interest?

Participant: Probably not in this context. We’re now at the stage where we’re modeling how we run {noun 75}s.

Interviewer: I want more of this level than this happens then that happens.

Participant: There’s stuff that the {noun 60} meets by itself. There’s stuff that people have to go and observe. And we use the {noun 71} as an example. And there’s stuff that we have to go and make happen. Those are pretty much all the important observables about the {noun 60}. Important external/internal observables. which this bubble is responsible for collating and trying to put into some coherent explanation of the universe. And that’s why when [company] sits in this meeting, there’s one person who’s holding most of that, but there’s a whole bunch of people who are peripherally involved. Like the people who were involved in the original {noun 95} and {adjective 96} actually hold a heap of Knowledge. So they come to these meetings, but they’re there to facilitate the meeting and provide background context. And go :" And the reason we did that was..." ...

Interviewer: Conservation of inputs and outputs. We need some. Operating Data emerges from [company] emerges from what entity at this meeting?

Participant: Officially it will emerge from one of two people because they’ve got to be permitted to speak on behalf of the company. Officially it will emerge from either the {position 27} or the {position 58}ing {position 41}.

Interviewer: do they get bubbles or a bubble?

Participant: They get separate bubbles, I’m afraid

Interviewer: We’ve got bubbles, these bubbles send a dovetailed flow out of operating data?

Participant: They may coordinate or one may ask the other to do it or they may act independently. It depends. Primarily it will try and go through the {position 58}ing {position 41} because he’s a drinking buddy...

Interviewer: That actually matters.

Participant: It really matters, yeah.

Interviewer: How do we get that data flow out of this page?

Participant: How do we get the operating data flow out of this page? Information from the {position 65} is being collated by one or both of those entities.

Interviewer: *1 goes to one or both?

Participant: Yep.

Interviewer: This *1 is what?

Participant: This would be research findings, technical reports -- it’s going to in the context of one of these sorts of we’re going to ask a "question" in a few sentences. There’s usually lots and lots of reports and studies behind it that are the technical basis for that question.

Interviewer: What do we want to label this flow as.

Participant: Knowledge. I don’t know. It’s the sum of our Knowledge which we acknowledge to be incomplete otherwise we wouldn’t be asking the question. Incomplete Knowledge.

Interviewer: Now this is a flow of Knowledge which is labeled as incomplete Knowledge?

Participant: Yes.

Interviewer: Incomplete Knowledge as Knowledge is flowing from the {position 65} to the tech {position 41} and {position 58}ing {position 41} at the same time?

Participant: Can, it’s context dependent.

Interviewer: In this atemproal acausal diagram, it’s going to both? Unless there’s a reason to differentiate the flow

Participant: I should make your life easy and say that they’re the one person for the purpose of this then.

Interviewer: You don’t have to

Participant: When you explain it that way, it doesn’t make that much value separating them.

Interviewer: Tech {position 41} or {position 58}ing {position 41} -- or is there a header for both?

Participant: They sit in different business units. Let’s just say, yeah.

Interviewer: One flow from {position 65} to Tech or {position 58}ing {position 25} is a flow of Knowledge of incomplete Knowledge. Are there other flows?

Participant: If we’re physically sitting in the meeting together. I can directly tell the {position 59} something.

Interviewer: Do we want to model that?

Participant: Usually I’m doing that in response to some other query so probably not.

Interviewer: From tech or {position 58}ing {position 25}, how do we get to that flow?

Participant: He controls the {noun 97} by which we have this conversation. So either that flow has been provided to these guys, the incomplete Knowledge, which I’ve said, and this is Information and Knowledge. So there’s the Knowledge, the Information which is the question part of it is either formulated by him assisted by him...

Interviewer: but it’s a function of the incomplete Knowledge.

...

Interviewer: Operating Data is or is not a component of Incomplete Knowledge?

Participant: It is a component.

Interviewer: where does operating data emerge from this fractal zoom?

Participant: So the operating data -- the connection to here. Because if those guys could write a query, they’d do it themselves, but they can’t.

Interviewer: This is operating data qua data. Now, the Data + Information of the analysis from the {position 59} hits whom?

Participant: It hits the {position 58}ing {position 41}/Tech {position 41}

Interviewer: then what happens?

Participant: Sometimes they sit on it. No. It’s discussed, circulated, to interested parties. To which {position 65} is one.

Interviewer: For purposes of this discussion, are there other interested parties?

Participant: No.

Interviewer: *1 is what?

Participant: This stuff coming back is their analysis which is data and Information.

Interviewer: so the tech {position 41} is merely routing.. ?

Participant: He will also have his own interpretation

Interviewer: So therefore there’s a...

Participant: a transformation could occur

Interviewer: Therefore this new Data & Information flow ... Is this new flow Data, Information, Knowledge, or O?

Participant: We’d love it to be Knowledge, often it’s purely Data or Information. If we hit the jackpot, we get Knowledge. Let’s pretend we hit the jackpot.

Interviewer: How are we labeling this flow?

Participant: Let’s say it’s Knowledge. Let’s not model reality, it’s too hard. That provides stuff back to you. It may be the answer to the question you asked, it could be the answer to a different question. That was subtly related... that people misunderstanding, yeah it’s been doing it for years. Let’s hope it’s Knowledge. We want it to be Knowledge.

Interviewer: is it Knowledge, or just nonsense?

Participant: the purpose of us asking is to get Knowledge.

Interviewer: So this analysis turns into Knowledge? or is the explanation...

Participant: If they did a good enough job the explanation is Knowledge which we can incorporate into our model of the universe directly.

Interviewer: So you would directly get explanation as {position 65}?

Participant: We have done. It can happen.

Interviewer: Is there translation that the tech or {position 58} {position 41} performs?

Participant: He can do. It depends on what he was given.

Interviewer: How can we model this?

Participant: With difficulty.

Interviewer: Really what it means is that these flows dovetail. Because we’re modeling all of the causal fiddly bits. So,

Participant: If they’re able to tell me the answers to my questions, then you know, ??? is important. Stuff comes back and we deal with it

Interviewer: At some points of time you get Knowledge explanation back. You also directly get analysis back?

Participant: Yes.

Interviewer: other points in time, explanation goes through the tech {position 41} -- tech {position 41} gets explanation?

Participant: Explanation and analysis, it’s the same things.

Interviewer: Sometimes the {position 58}ing or Tech {position 41} takes the analysis and or explanation goes "wow this is shit" and does what with it?

Participant: Okay, in the context of our {noun 60} what that really means was...

Interviewer: Alright, so this is interpreted explanation? Who gets that interpreted explanation?

Participant: The {position 65}.

Interviewer: and that is Knowledge?

Participant: Yeah. It’s Knowledge... and Data and Information. It’s all three again. It depends on what it was.

Interviewer: So this flow of Data, Information, and Knowledge ... and or or?

Participant: It’s and.

Interviewer: can be labeled as?

Participant: The internal? Feedback. "What feedback did you get from those guys? Nothing!"

Interviewer: Are we missing anything?

Participant: For purpose of this, we’ve got conservation of Knowledge.

Interviewer: Let’s start with a discussion of the diagram you’ve looked over and that we were creating last time. We had in that diagram the {noun 71}, Database, and you as {noun 75} {position 100}. And we did not iterate into any of those because of time constraints. Are there any aspects of that diagram that bear iteration?

Participant: Unless we want to go and start modeling the actual {noun 75} planning {noun 97} and things, I don’t think so.

Interviewer: What I’m looking for is a difference in nature or kind. -- Last interview focused on a more technical point of view. You’re talking about interactions with databases and meta-data. Can we look at a scenario from your experience where there’s been confusion over understanding of data, Information, or Knowledge? In a procedural sense, not just in a one-off, but in some sort of really nasty conflict of philosophies.

Participant: If I’ve got the right idea, I’m thinking of an instance where we have some {noun 60} measurements of something, and they should be -- we’re trying to use them to understand the physical conditions that are in the {noun 60} and what actions we need to take, but there’s more than one way of analyzing those. And there were a number of opinions and this evolved over a couple of years as to gaining consensus. And of last week we still don’t have consensus.

Interviewer: That sounds like just the thing. Where do we start?

Participant: This is the {noun 76} again. It has some {noun 74}points.

Interviewer: How do we want to model those?

Participant: There’s 4 points and it’s the differences between those {noun 74}s that we’re trying to understand.

Interviewer: We’ll start with your interpretation, and then I’ll have you try to explain other peoples’ interpretations. With your interpretation, what entities should there be?

Participant: There would be 6 in total ... You don’t care what they are, you just care that they’re there. ... The actual {noun 74}s that we measure at these points is a function only of the {adjective 78} and the {adjective 79}. And the {adjective 79} responds to the {adjective 78}. If we measure the {noun 74}s, we should be able to infer what’s going on inside the box. We’d like this box to have all the same properties everywhere. However, since once of the purposes is a {noun 80}: if there’s new {noun 80} material being introduced ... and old {noun 80} material ..., it’s not going to be uniform .... We’d like to know about capacity and flows and stuff like that. So, if I look at this point and it’s higher {noun 74} than this point what does that imply about the system?

Interviewer: This point being an {noun 81} versus a {noun 82}?

Participant: So I’m interested in that case between that differential {noun 74}.

Interviewer: In that case in terms of the data, we have {noun 81} and {noun 82}...

Participant: My interpretation of this is that if we -- this is time and this is {noun 74} <drawing> -- if everything was uniform then those two would be exactly the same. If over time this happens and it’s the {noun 81}, we can have higher {noun 74} at the top and lower {noun 74} at the bottom. There are two ways to interpret this. One is, in my opinion, wrong. [explanation of model] And we justify this decision from numerical modeling. We can actually then independently create a numerical model which is consistent with that explanation. The alternate, incorrect view, is [alternate explanation] ... will give you the right answer most of the time. It’s a nice rule of thumb. {noun 83}s just happen to be different. And the difficult thing for us was, until we constructed a numerical model and said look: this numerical model starts off with the ... equations about the universe and it’s entirely self consistent and look it’s the same result. That was something which helped us convince people that well, yes, that is consistent with the {noun 60} measurements. And my explanation is not consistent with the laws of the universe. "OK, you’re probably right" say the Os. Where it goes wrong if, other things can also influence those {noun 74}s. Say if there’s {noun 61}. ... And that’s when I realized a lot of people were never really on board. Because you go to a meeting and they go, it’s just like you’re reading ??? trails. I don’t necessarily believe your explanation because other things could be effecting it. You know how in any defined problem there is: these are the assumptions I’ve made, this is the framework in which I use to analyze the problem. If my assumptions are violated, then my explanation is no longer necessarily the only explanation. You go back a little to change the assumption: well we’ve got a special cause. A special cause is driving this. Naah -- this is where we have some maths which are the {noun 74} differences, we have Knowledge which is our -- other people have studied the universe and we’ve implemented models based on their studies. The ... models which are based on the [name] equations. These are more well established. I’d class that as actual Knowledge. If I class that as Knowledge, I’m using that Knowledge to view the Data that I get from the {noun 60}. There are some assumptions embodied in that, and I have to be aware of those. If people reject that Knowledge or disagree with the assumptions then it becomes really hard to agree on what’s the interpretation of what’s actually happening.

Interviewer: This is a really useful discussion. DFDs are cute, but this is going to be more interesting. So we’ve got, Data, Information, and Knowledge. They’re in a line or however you want it. Walk me through your mental {noun 97} of coming up with some sort of explanation for this. The diversion where there’s an argument. We’ll be drawing lines in and out of these things, labeling lines.

Participant: Initially, we all started ... with the same Data. We are making these {noun 60} measurements.

Interviewer: Data here are what? Data is or are? I’m not actually being pedantic here. I’ve found instances of Data as singular, so I’m interested in how you use Data.

Participant: Completely inconsistently. In this case, we wouldn’t refer to it as Data. We may call them Data points.

Interviewer: So you start by getting Data points.

Participant: Originally the {noun 63} didn’t even measure it. It was originally a dial, and a guy had to go and read it once in the morning and once in the afternoon. Because it was a retrofit in the system. And later we connected it to the {noun 63}. Those are {noun 74}s I measured at known locations. At known locations and times. Initially we all agreed what that meant. Later on we began to fight over whether or not the {Noun 4} is a good {Noun 4}.

Interviewer: So you’re measuring {noun 74} at a location and time. Is there a consistent definition of {noun 74}, and where does it come from? An operationalization, say. It’s not a theoretical definition. When we use {noun 74}, we are referring to this.

...

Interviewer: So we’re measuring gauge {noun 74}. The concept of Gauge {noun 74} is what here, if anything?

Participant: That would be Knowledge.

Interviewer: So really, we have a Knowledge start of what? I’m trying to get at the really basic assumptions.

Participant: At the level of your standard {position 58} they have to take a great many things in their education on face value. And it’s an assertion.

Interviewer: You would assert that these assertions are Data, Information, Knowledge?

Participant: Knowledge.

Interviewer: So these assertions, as Knowledge, form what?

Participant: They form a context in which to view...

Interviewer: Context. Starting here from Knowledge, it produces a context for your Data?

Participant: Yeah.

Interviewer: Is there an incoming thing for Knowledge here? We have Knowledge which produces context for data. Is there anything which informs which context we produce?

Participant: We’ll take the view people have: it is a modern western educational system. Background education, everyone should have that.

Interviewer: Background education produces Knowledge, Knowledge is or produces context?

Participant: It produces context.

Interviewer: What is that production -- where are we getting context from in terms of production Knowledge? We’ve got background education and we somehow get context from that.

Participant: So what else produces the context?

Interviewer: Yes. What’s the transition phase there?

Participant: That context together with what I learn about the {noun 60} has to be applied to this particular problem.

Interviewer: So problem localization?

Participant: Yep.

Interviewer: What is problem localization?

Participant: That’s your initial study/investigation of the issue at hand.

Interviewer: This study of issue at hand is which category?

Participant: That’s collecting Information about it. The study of the problem at hand produces Information.

Interviewer: The Information, combined with background Knowledge, produces context?

Participant: yeah.

Interviewer: So we have, as the two inputs to this problem, we have background education and the study of the problem at hand. Are there any other inputs to this problem?

Participant: No.

Interviewer: now the study of the background at hand produces Information. Is there a transmutation of the study, or is this line the study of the problem at hand?

Participant: Not sure.

Interviewer: Is the study of the problem at hand Information or is compiled from stuff into Information?

Participant: It’s probably compiled from stuff.

Interviewer: So the study is also Information itself?

Participant: It generates Information.

Interviewer: so we have the study of the problem at hand which generates Information. This Information generated is what?

Participant: In our worldview, the difference between Information and Knowledge, is that Knowledge implies some level of understanding. I go off and I get the {noun 60} drawings and operational {noun 116}s, and I review all of that and I distill from that what is the important Knowledge which I hold in my head.

Interviewer: Let’s talk about that. You’re distilling Knowledge which you hold in your head.

Participant: I’m assembling a Knowledge framework in my head from Information.

Interviewer: What is the purpose of this framework?

Participant: To help me understand Data that I receive in the future.

Interviewer: So what you’re telling me is that from the study of the problem at hand and from background education you’re creating a contextual framework?

Participant: Yeah.

Interviewer: does the contextual framework do anything other than provide context?

Participant: Not really.

Interviewer: So this study of the problem at hand as Information plus background education as knowledge combine to create a contextual framework. This contextual framework is Knowledge and it is used as a basis for telling you what Data to collect? or for you to understand the collected data?

Participant: Both. there is lots of Data that I could be looking at, and my context is my filter, because a lot of it is not important.

Interviewer: Can you think of a useful nature analogy for this?

Participant: A molecular sieve?

Interviewer: A molecular sieve, fantastic.

Participant: It’s just an ordinary filter.

Interviewer: So the Knowledge acts as a filter? And the stuff going through the filter is Data?

Participant: Yeah.

Interviewer: When you say the stuff going through the filter is Data, give me an example of something that’s trapped by the filter and something that’s passed by the filter.

Participant: The vibration amplitude of the fans that are feeding the system. It is part of the {noun 60} and it is peripherally related to this problem, but it’s not important enough to actually pass through the filter. Whereas the actual flow through that fan which is also an input of the system, this far more important input, will pass through the filter, and I will try to analyze it in light of ???.

Interviewer: In order for you to analyze something that’s passed through the filter, is there some function of observation going on or has the observation happened upstream of the filter?

Participant: The observation happened upstream.

Interviewer: Context filter as Knowledge?

Participant: Yeah. I’d love for you to have an interview with [name]. You could have many interesting discussions about what is context and whether or not I’ve actually observed that or not.

Interviewer: So we have observations here. Observations of what?

Participant: In our world, they’d have to be observations of physical states.

Interviewer: Observations of physical things? Go into this {noun 97}? An observation is what?

Participant: An observation is a measurement of some kind.

Interviewer: So a measurement is generated and flows into the filter. The filter then winnows out irrelevant observations... and then we have relevant observations.

Participant: This is the important thing. Because people hold different contexts for these reasons. That includes what may or may not be superfluous around them. And chuck out the stuff that really matters.

Interviewer: So one area of conflict is the context filter. And that context filter conflict can come from either different priorities, a different background education, or a different study?

Participant: Yes.

Interviewer: Are there any other sources which perturb the context filter?

Participant: People’s political agenda.

Interviewer: Where should I put the box?

Participant: I guess it’s Knowledge.

Interviewer: What kind of Knowledge? Why is it Knowledge?

Participant: It exists by itself. It’s not Information which you can tie down. A person may be... in all cases of {position 58}ing problem, we can’t measure the things we actually want to measure. I want to know what the {adjective 78} is. I want to know what the {adjective 79} is there. Guess what, I can’t measure either of those two things. Alright, what can I measure? I want to measure the {noun 74}. That’s going to piss off that guy, because he’s got to put a hole in his {noun 60} and it’s going to corrode at that point. His political agenda is to stop me. He wants to solve the problem. He works for the same company as me and we want to make money. But he doesn’t want to put holes in his {noun 60}. So he’s got conflicting things: help make money, make his life easier.

Interviewer: So the political agenda is basically an estimation of personal needs, and this is Knowledge?

Participant: For that actor, yes.

Interviewer: Now you said it can’t be tied down like Information can?

Participant: Yes, because it’s fluid.

Interviewer: Now background education is also fluid?

Participant: Well, people will learn more. But they could be misapplying some of their education. They could review that...

Interviewer: It is also fluid?

Participant: to an extent, hopefully less.

Interviewer: Whereas Information’s not fluid?

Participant: Information should be more concrete.

Interviewer: When you say more concrete, what do you mean?

Participant: First off, Information is by more concrete -- in the hierarchy. Data is concrete. It might be -- I measured it via a defined {noun 97} at a particular time. We might argue what it means, which is Knowledge, but it is what it is. But the Information... I’ve reviewed a whole pile of things. Those are Information sources, they are -- it’s clear how I obtained them. They may be wrong, their provenance may not be... It’s less concrete.

Interviewer: Because the provenance may not be clear

Participant: It may not be clear.

Interviewer: is the nature different? Or is just a difference in reliability?

Participant: Between Data and Information? There’s always some overlap, so it’s not just provenance. ??? Data would be a nice discrete thing. That I could tie down. The Information ??? could be viewed as a collection of data.

Interviewer: So Data is something that could be tied down. Information is a collection of data?

Participant: It can be.

Interviewer: when you say tied down, what do you mean?

Participant: I can specify what, where, when. Probably what where when would define a piece of that. What the measure of it was, its location at the time.

Interviewer: When you say tied down, its tied down in terms of provenance? reliability?

Participant: Yep.

Interviewer: Information is a collection of Data. The {noun 97} of collecting it is a function of something we’ll get to. But the collecting it reduces its reliability?

Participant: It doesn’t reduce its reliability.

Interviewer: what does it change?

Participant: It places its -- it begins to place it in a context of other things. So you’ll often have conflicting data. They’re fine by themselves, but when viewed together in my collection of Information, now ??? Knowledge is going hang on,

Interviewer: Now let’s go up to here. We’ve got observation through Knowledge filter giving us relevant observations according to someone’s study, background education and political agenda. Relevant observations then what happens?

Participant: Then we are with that subset of relevant observations we try to produce a consistent context, a consistent view of the world. Yes I know I used context in the filter, but it...

Interviewer: are we using context in a different..?

Participant: This is a different context. Probably there’s like -- I have to decide what to look at.

Interviewer: In this case, context as what?

Participant: This is context as relevance.

Interviewer: when we’re using one word in different ways, we can just do "as foo". So we’ve got context as relevance filter, winnowing out relevant observations. We... combine?

Participant: Combine those observations

Interviewer: We combine these observations into what? More Data? Information?

Participant: Into a Data set.

Interviewer: We combine Data points into a Data set. This Data set is Data, Information, or Knowledge?

Participant: It’s still Data.

Interviewer: So we’ve got a data set of relevant observations? The fact that we have a set here. What produces the structure of the set? Why do we say that it’s a set?

Participant: We can say it’s a set, because we constructed our filter... It’s a set of things related to the problem at hand.

Interviewer: So their only relationship so far is their relationship to problem...?

Participant: If this is the first time we’re doing it, yeah. But later on it should be better than that.

Interviewer: But the Data points don’t have any relationship amongst themselves?

Participant: They will, but I can’t do this -- it’s hard to do abstractly. There will be subsets with multiple interrelationships.

Interviewer: So we’ve got sets as relevancy, but we also have sets as relationships. Set relationships are considered to be what? Data, Information, or Knowledge?

Participant: Information. Set relationship -- if we know a priori yes, {noun 74} is related to {Noun 12}, that’s Information. ??? But it may have to be something I have to discover. It depends where I -- where that relationship comes from. Sometimes I can just know where it comes from or I have to discover it through further manipulation.

Interviewer: That’s relevant. If we know a priori, we get that from background education? which is then turned into Information? What is that transformation right there? We’ve got lots of background education, from that we extract Information?

Participant: Uh-huh

Interviewer: The Information is that there is a relationship here? So, we have relevant observations and data sets. Some data sets have relational context internal to the set. Os have context external too the problem. Well, all have context external to the problem. Some have internal context based on prior Knowledge turned into Information. Given that we have some that don’t have an internal context that we have to discover, we could classify this as the {position 58}ing problem? Walk me through this.

Participant: In a concrete example?

Interviewer: Yeah.

Participant: This same problem? Ultimately, I’m interested in the {adjective 78} and distribution of that {noun 60}. There are a bunch of things which I immediately know are directly relevant. {adjective 79} rate, the {Noun 12}. ... I can do all of that -- any {position 58} should be able to do that. There is something about the nature of the {noun 84} that’s probably important. That’s probably going to have to do with: what {noun 97}es that produce the {noun 84}? What {noun 97}es that those raw materials were... It’s not -- I may suspect that they are related. I may not. There was an argument about that. I have to discover that relationship. In a nice case... the number of times when people don’t have any idea about an {position 58}ing problem. They go "right. Here’s the thing I’m trying to control. We currently have the {noun 85} table, there are 500 columns in the {noun 85} table, let’s join that with that table and let’s do bivariant plots to every one of them and get the r-squared. And we go through -- I’m not kidding. Ask [name] about {Noun 23} and {noun 28}. This is one of the only -- if the only tool you have is a hammer, everything looks like a thumb. It’s just like that. Can I do anything more sophisticated? Hopefully I can winnow out some of those 500 odd columns, because I know something. I have to be careful, because often -- the thing that’s driving this might be related to this. We don’t measure that. We measure some other things that are related to that. Hopefully I can get a weak relationship between them and the thing of interest. And maybe I’ll discover this thing one day. That’s sort of the {noun 97} where we’re trying to work out what drives this. And there are quite a few things we have that where we’ll find a subset of 5 things that can produce a decent linear model but allows us to describe, predict, what that’s going to do. And then time goes by and they don’t work so well anymore which tends to suggest that it was this thing down here that they were related to. And some other things that were also important are changing and we don’t really know what’s going on. That’s actually the key we’re trying to address: what research is trying to do and what the -- technology department solve today’s problems, solve next month’s problems. Research department, solve /the/ problem. Not the same thing. That’s the implication that we’re actually getting to the -- we’re aiming towards laws of nature. What is actually the real reason? As opposed to "How can I get a useful working model that allows us to make more money?" That’s more the technology {position 58}’s job. Which means we’re going to have overlap and conflict because we have different agendas.

Interviewer: You mentioned prediction.

Participant: Why do I want to predict?

Interviewer: a) why, but b) how do you predict?

Participant: If I had a causal model of the factors that control a variable that we’re going to observe, the reason I want to know the {noun 74}s is that it’s telling me about {adjective 78}. I want to control the {adjective 78}. if I know that these center 5 things have positive and negative influences on that {adjective 78}, and I want {adjective 78} to be low, I will try to select those 5 things such that I make it tend to be low. That’s why I want to be able to predict. So that when they say: "We’re thinking of buying this new thing.." and I’m saying "don’t do that because it’s high in this and this will make that bad and we’ll make less money." That’s why I want to predict.

Interviewer: A predictive model is what? Knowledge, I, Data?

Participant: It’s the embodiment of our Knowledge.

Interviewer: This embodiment of Knowledge is what?

Participant: I can’t be more concrete than that, it’s the embodiment of Knowledge. It should be an algorithm.

Interviewer: Does an algorithm even feature on this spectrum?

Participant: An algorithm should be Knowledge.

Interviewer: Why?

Participant: The {noun 97} of understanding is the {noun 129} of Knowledge. Because when I understand the problem, and I can specify: "you take these things and you do these operations on them and the outcome of that will tell you something." basic way of looking at an algorithm. Will tell you what’s going to happen in this other thing. The Knowledge is here, but I have to express it someway. An algorithm is a way for me to document the Knowledge I’ve developed.

Interviewer: An algorithm is an expression of Knowledge which is itself Knowledge.

Participant: I’m running out of words.

Interviewer: Feel free to make new ones. How does Information ... inform your ... It looks like we’ve got a couple types of Knowledge here. We’ve context and we’ve got predictive. Are they different?

Participant: In the view of the algorithm, when you’re providing context to the algorithm, you’re just providing the inputs into the algorithm. Which produces a prediction.

Interviewer: Do we have a type of Knowledge that’s not algorithmic?

Participant: Yeah, stuff that I just know.

Interviewer: what is the relationship between stuff that you just know and algorithmic Knowledge?

Participant: I tried to say that algorithmic Knowledge... having developed a Knowledge framework about something, how do I write down, tell someone else, tell a computer about it.

Interviewer: Good point. What’s computer code?

Participant: In a lisp sense?

Interviewer: Let’s work from the trivial sense, back.

Participant: In the trivial sense, it’s a series of instructions that you’re asking the computer to perform.

Interviewer: so each of these instructions, does any instruction fall under this chart?

Participant: An instruction tells the computer to operate on Data in the computer.

Interviewer: So it’s data operation. Is the instruction itself Data?

Participant: Well it certainly can be, depending on what your code is doing? Self-modifying code.

Interviewer: So code can modify itself, this modified code is Data?

Participant: Instructions.

Interviewer: This modified code is instructions, is Data?

Participant: Are we talking Von Neumann architecture, are we talking LISP? Or are we going to say a procedural language on a Von Neumann arch? I’m being a pain I know.

Interviewer: First question is: do we need to differentiate architectures?

Participant: Well, if we’re going to talk about something that’s a purely functional system. Where’s the state of the system? Well, it’s in the current state of the functions. They’re all and the data is immutable, and it’s a different way of thinking about the problem. If we leave it in a nice procedural world where Data resides in either memory or in registers and I carry out operations by moving Data from place to place or combining two pieces to produce another piece of Data. Then I have instructions and I have Data. In a functional architecture -- let’s say LISP. Let’s not talk FORTH. I’ve had limited experience with stack based languages.

Interviewer: The real question is: when does stuff in a computer become not-Data? Where does computer based stuff escape the classification of Data?

Participant: Well, that’s a meaning we ascribe to it. At the end of the day everything is a stream of bits. So this bit stream is Data. It depends on the level of abstraction you’ve chosen.

Interviewer: At what level of abstraction does that change? Is there a point where we get more and more abstract and suddenly those bit streams cease become data.

Participant: At a certain point, I’m able to attach meaning to it. When I’m looking at a character on the screen, from the screen’s point of view, some bits came along and it does stuff. And you have an A. Great. I look at those bits whether they’re on or off and say that’s an "a" At that point, I’ve got some Information.

Interviewer: So your perception of a character on a screen is not Data, but Information

Participant: It can be. If I was looking at DNA sequences, that "a" is just one of the base pairs. And that would be a representation of Data.

Interviewer: This representation of Data is itself Data, Information, Knowledge?

Participant: It’s just a representation.

Interviewer: is the representation significant?

Participant: How something’s represented is the significant act. Data is represented by something which is significant.

Interviewer: So you as {position 90} looking at computer screen see Information, but not Data, because you’re looking at representations of Data?

Participant: I will call it -- it will depend on how people decide to look at it. if they’re just looking at a number in a text box, And that’s changing with time or they’re looking at a trend line -- it’s the same underlying measurements of the {noun 60}. They’ll react to them differently.

Interviewer: Is how they react a function of something?

Participant: It will depend on all the things that are personal drivers on them. If it was something they were trying to control and they’re just looking at the current number, and they go yes, that’s less than my current set point limit. I’m happy. They’re living in the now. If they have a trend plot of it and they see that it’s under the set point limit but increasing, I’m worried. As opposed to previously: "Oh, fine. Is it going up or down, I don’t know, it depends on my context."

Interviewer: How does Data become Information?

Participant: When my theories?? get classified into data sets, collections...

Interviewer: You have the Data sets, and the Data sets with relationships are Information?

Participant: Yeah.

Interviewer: These Data sets that become Information, what happens to them?

Participant: In an ideal world, I’d write the perfect technical report which would form part of the institutions Knowledge and everybody would read it and we’d all have pushed back the frontiers of Knowledge a small amount.

Interviewer: Let’s unpack that. Candide’s universe, we’ve got el dorado. You have Information this information is your Data set.

Participant: Yeah

Interviewer: You then write a technical report. Is this technical report Information?

Participant: It probably is Information.

Interviewer: Does it represent Information?

Participant: It should represent Knowledge. It depends. In an ideal world, you’ve carried out a study, you draw some conclusions about the universe based on that study. You’re trying to impart that Knowledge to someone else along with the supporting Data and Information. We would say Data but it’s really Information, because measurements were taken and we use them in some way and that is Information from which we derive some Knowledge which hopefully is self-evident and we’re trying to impart that to people.

Interviewer: This Knowledge is a change in prediction modes, is a change in context, both?

Participant: Both.

Interviewer: So Information can cause changes in Knowledge, or is a change in Knowledge?

Participant: It can cause changes in Knowledge. It could always be rejected or it could be wrong.

Interviewer: We’ve got Knowledge filters Data, produces Information, produces Knowledge. Are there any other metarelationships here?

Participant: I think we’ve had enough

Interviewer: Nonsense, I’m only mostly confused.

Participant: ???

Interviewer: What kind of Knowledge does Information cause a change in?

Participant: I’m taking change to include simply: we’ve increased our Knowledge. So, all Information increases my contextual Knowledge. ??? If it’s more Information about something I already had Knowledge about it will either reinforce or cause me to do stuff: well, I’ve got 2 pieces of Information in conflict, it means the Knowledge worker is trying to resolve that conflict so that this bit might get weaker in this area.

Interviewer: When people you communicate with disagree with you, do they diverge from this?

Participant: Sometimes yes.

Interviewer: Lets look at both cases. When they diverge from this, where is their point of departure?

Participant: The annoying case is: you’re wrong, I don’t have any evidence but I know you’re wrong.

Interviewer: What does that annoying case stem from?

Participant: Some past politics. A lot of problems just aren’t really clear cut. Because you have a limited understanding, a limited data set about a thing, you can’t always... well I’ve developed a model about its behavior, and I’ve got maybe 20% of it. I can describe 20% of it. I think that’s interesting enough. And they go "No, you’ve missed 80% so it’s all bunk." At least I did something. There’s that case.

Interviewer: Is that a problem with contextual filtering?

Participant: Yeah. their contextual filters are different than mine.

Interviewer: Stemming from all these three sources . That’s the annoying case, what are the other cases?

Participant: Where lets say I didn’t have the right contextual filter and I say I’ve gone through and I’ve done this and here’s my case, this is great... "Yeah, you didn’t consider what happened in 2002" yeah, I didn’t put that in the Data set, I didn’t think it was important. "Yeah, it is." I look and go: "Actually it is important." So my contextual filter was incomplete. So they provided me with new Knowledge or Knowledge that was new to me, which I then have to adjust my contextual filter, and in light of that, the Knowledge coming from my study of my data points may not be correct.

Interviewer: Where do "shoulds" fall here? Either incoming or outgoing?

Participant: Should should be an outcome.

Interviewer: Specifically, if you’re producing normative assertions, what are they here?

Participant: For me, I would say that they’re Knowledge. For the recipient they’re probably Information.

Interviewer: When you are the recipient of normative recipient of normative assertions, you consider them to be?

Participant: I consider them to be Information. I don’t trust other people. When someone says you should do blank, you should go and measure this. Well, OK, but I have to decide myself if I’m going to actually believe you.

Interviewer: So these are untrusted assertions?

Participant: Until I do some vetting, everyone’s assertions are untrusted.

Interviewer: And an untrusted assertion is Information? Is it different from the Information from the study of problem at hand?

Participant: It can be Information that’s coming from the outside, yeah. While studying there will be other stakeholders who will tell you stuff. If they’re powerful, you have to at least pay lip service to them.

Interviewer: How do you vet these pieces of Information?

Participant: Who do you trust? If it’s somebody who I respect technically, then I will take their advice. If it’s someone who I think is a <bad person> then, unless what they’ve said is obviously true by inspection, then it gets far more weighting down. If the person is really powerful it’s no longer about trust. If someone really powerful in an open forum tells me to do something, I ignore that at my peril. It’s about weighing up the Information that’s come to me. Do I have to act on that? That will depend on the original relationships?

Interviewer: Where does meta-data fall here?

Participant: Sometimes it’s Information. You didn’t want that answer. The Data about Data.

Interviewer: In this flow, what is the meta-data? Is there any? Is it flowing past?

Participant: It’s part of the filter, yeah. To get the specific example of those {noun 74} tapings, they measure {noun 74}. What {noun 74} do they measure? The meta-data is how the measurement is actually taken.

Interviewer: and so that meta-data can be a ??? is Data itself, is Information, is Knowledge?

Participant: It’s Information in the main. I make the distinction in that case so when we’re discussing the meta-data of those {noun 74} tapings, there’s a pipe with X cells and has a hole in each cell. But the pipe... we have one {Noun 4} at the end. I’ve told with the other dimensions: you know how many holes there are, you know where they are, you know what the separation is. That’s all informing you about the measurement I’m making at the end of the pipe which is my original piece of Data. So I’ve given you a bunch of Data about the meta-data and you have some Information which is maybe "this isn’t reliable" or maybe its something measuring this thing or the other thing.

Interviewer: Is Data atomic?

Participant: In this example, we pretend that Data is. Because in every case leaving out quantum mechanics, these are continuous variables. But we {noun 37} them at discrete points in time and space. So we treat them in all analysis pretty much as discrete observations.

Interviewer: So there are no sub components of observations? You can’t have half an observation?

Participant: That’s true. We would like to think... it’s a thing.

Interviewer: Is there anything past Knowledge?

Participant: An overall Knowledge framework?

Interviewer: Is this overall Knowledge framework Knowledge?

Participant: I consider the whole thing is Knowledge. We’re just growing. I think Knowledge is just the boundary.

Interviewer: Is Information divisible?

Participant: Yes. A piece of Information, if I take the {noun 74} difference between two things is foo, that’s a piece of Information about the system, versus the difference between two things is immediately decomposable to two things.

Interviewer: Is there cases where it’s atomic, or is it always divisible into Data?

Participant: It could be atomic. It depends on what it’s Information about.

Interviewer: Give me a case where we have a piece of atomic Information.

Participant: The diameter of the pipe on the drawing.

Interviewer: The diameter of the pipe on the drawing is Information not Data, because...

Participant: If it was a measurement of the actual pipe it would be Data. But this is Information. It may not be correct. This is a drawing, it’s not the actual thing. It’s a thing about the thing. It’s not meta-data.

Interviewer: It’s a thing about the thing that’s not meta-data. Is Knowledge atomic?

Participant: If I could only point to something and say... we do do that and point to something and say "That’s Knowledge." except that that’s not -- it depends on the context we said that in.

Interviewer: Can we point at something and go that’s Knowledge?

Participant: Not realistically.

Interviewer: When we point at something and go that’s Knowledge, what are we pointing at?

Participant: In the abstract sense?

Interviewer: What category of thing is it that we are pointing at? I point at this and go this is Knowledge. When I point at this and go this is Knowledge. Is this Knowledge? Is this Information? is this Data?

Participant: The physical thing that you’re pointing at is the physical representation of the Knowledge that hopefully both you and I hold in our heads.

Interviewer: So it’s a Knowledge representation that’s not necessarily Knowledge, Information, nor Data? Representations of these categories do not necessarily require that the representation itself be in that category.

Participant: I agree.

Interviewer: Is there a category for representations?

Participant: Yeah, meta-data.

Interviewer: When you try to communicate these representations to people and an error occurs, where are those errors, what are those errors, and why?

Participant: The errors can be my communication of that. Faulty analogy, assumptions about recipient, that kind of thing. There’s also reception errors where the person doesn’t listen or is missing something that would help them that they require to receive that Knowledge.

Interviewer: Errors can apply on either side. The {noun 97} of communication is what? Errors occurs on the sending side of misreading of what?

Participant: Faulty assumptions, deficiencies in transmission.

Interviewer: So these transmissions are all representations of these three categories?

Participant: If we’re talking about transferring Knowledge to someone else, it’s all a representation.

Interviewer: In transferring Knowledge to someone else, the sender can have errors of assumptions or of representation. Can you transfer Information to someone else?

Participant: Yes, that’s what we defined in those examples.

Interviewer: Sending Errors there are what?

Participant: Straight out mistakes. Transcription errors.

Interviewer: Sending Errors in Data?

Participant: Transmission errors or corruption

Interviewer: Reception errors of Data are?

Participant: They just didn’t listen. That would apply to all categories.

Interviewer: Intentional deafness? What other errors are there receiving?

Participant: understanding errors. They got all that was transmitted but their own contextual filter rejected it or parts of it.

Interviewer: Contextual filter errors.
\stopextract

\section{Interview 5}

\placefigure[]
[fig:i2]
{The SDFN Diagram for Interview 5}
{\externalfigure[Chapter4/graphs/i5.pdf][factor=fit,frame=on]}

\startextract
Interviewer: What would you say -- can you differentiate into projects? Or has it just been one stream of continuous research.

Participant: The parts of the {noun 118} can be broken up to some degree. So a lot of time has been spent looking at how to put the materiel into top of the {noun 118}. How you feed a {noun 118}. ... And the bottom of the {noun 118}, the hearth, ... there’s a lot of work that happens on that. Because that is the blackest of black boxes for the {noun 118}. What happens there. Stuff goes in, stuff goes out, and you can measure {Noun 12}s around the side. And that’s it. And you cannot probe it. You just see it once every 20 years.

Interviewer: Because that’s when you flush it out?

Participant: That’s right. That’s when you do a reline. And even there, the mere act of cooling it off completely changes what you can look at. It’s vaguely representative of what was happening.

Interviewer: So the ... experiments ...

Participant: when they {verb 119}ed a {noun 118} and cut it in half basically.

Interviewer: Doesn’t really model...

Participant: Even that. You’ll get Information out, but it’s still not perfect Information. It’s pretty cool, but it’s not perfect.

Interviewer: When you say we get Information out, This is what I’m looking at. What do you mean by Information?

Participant: You can dig down, in a controlled way. You can excavate the {noun 118}, taking {noun 37} of materials and analyze their chemistry or their micro-structure or their metallurgy or something like that. And you’ll be looking at the physical shapes of what’s where. ... Plus also their micro-structure and things like that. So, as you go down through the {noun 118}, you’re looking for zones: “What’s the material structure in each zone of the {noun 118}.” Before things have melted, at the zone things are melting, below that point. And you’re trying to estimate what the physics or the chemistry was that’s happening when the {noun 118} was running at that point in space.

...



Interviewer: So, we start with a nice trivial flow. What is one of your roles that you play in this project?

Participant: ... {position 56}, doing research experiments.

Interviewer: Are there other ... {position 56}s?

Participant: No.

Interviewer: So we can just call it ... {position 56}. At some point, you either get a flow from somebody or send a flow to somebody. Who is your most common person or thing of interaction?

Participant: Specific names, or are you looking for the entities?

Interviewer: I’m looking for the roles. If names help you think, feel free to use them. I edit them out during the transcription.

Participant: During the experimental stage, it was the {position 127}. ...

Interviewer: Do you send stuff to him or does he send stuff to you?

Participant: Both ways.

Interviewer: Let’s model one of the things that you send to him. I’m using stuff here so I’m not going to prejudice you in terms of ... what we’re going to be doing is we draw a flow like this, we label the flow with the topic of the flow, and then we label the flow with category of the flow. Whether it’s Data, Information, or Knowledge. And if you say O, we go into depth with what you mean, and what the category implies. And I’ll keep asking you this question until we’re done. There’s no temporal model here, everything just happens at the same time. Because dealing with temporality is just... too much of a pain. We have a flow from you ... What do you send him?

Participant: During the experiments, I’ll send him observations from the results.

Interviewer: Observations? Or what shall we label this? Results?

Participant: I don’t send raw Data. I send interpreted Data. I won’t send raw measurements to him, but I’ll send an interpretation of those raw measurements.

Interviewer: Let’s go interpretation of measurements. Is this flow Data, Information, Knowledge, or O?

Participant: What would I call it? I’d be hazy on the difference between Information and Knowledge. Can we define it, or do you have a better definition in your head?

Interviewer: Here’s the thing. If there isn’t a difference, then there isn’t a difference. I’m putting the terms out there. You can use them as complete synonyms if you wish. Or you can differentiate them. I’m trying to model what’s in your head. Unfortunately I’ve got difficulty putting probes in.

Participant: It’s a black box. I’d probably call it Information then. And I’d call it that because it’s not just raw numbers. I’d plot an interpretation to it by some means. But it’s not Knowledge because it’s the experiment. And it’s a very contrived environment that it’s coming out of, so it’s interpreted Data about this contrived environment. Saying when we do this on the experimental rig, we see this effect happening. And I’ll often leave it fairly plain like that for them because I’m not familiar enough with their day to day observations of the {noun 118} to know if this is a relict observation of our experimental rig or if it’s really a phenomenon of the {noun 118}. It’s not Knowledge yet. We don’t want them to do something at the {noun 118} based on that yet.

Interviewer: what does the {position 127} do with this flow?

Participant: What they have done, in the past, considered it. Often it will be verbal or written. They’ll consider it, and then they’ll tell me whether it’s consistent with something they observe on the {noun 118}.

Interviewer: How would we label that flow back to you?

Participant: Again, I think it would be Information, because we both realize, me and the {noun 118} {position 58} realize we’re not talking about the same things. I’ve got a model as a chemical {noun 120}, and so we’re talking about observations that we’re making and an interpretation of what we’re seeing, and we’re looking to see if there’s a correlation between the two.

Interviewer: And so this Information flow back to you. What topic is it? He’s transforming your interpretation of measurements how?

Participant: I’ll give a comment about a specific phenomenon that I see. Let’s say it’s a variation with flow rate and time. And I’ll see, under a set of circumstances at a particular point. And that’s what I tell him and he’ll come back and may say: “that’s interesting, but we have situation x,y,z. We see a similar effect.”

Interviewer: So we would say that’s correlation with {noun 120}?

Participant: He’ll tell me if my Information correlates or doesn’t.

Interviewer: So really, it’s a confirmation of correlation. And you would say that that confirmation of correlation Information because of it interpreted but not situated?

Participant: It’s interpreted, but it’s not immediately relevant to the {noun 118}. We’re not going to change anything. We’re not going to change what we do at the {noun 118} based on that yet. It may change what I do with my experiment.

Interviewer: Let’s look at that. He sends back a confirmation of correlation, whether or not it’s positive or negative confirmation, he’s giving you something back. Then, what do you do?

Participant: It depends on where the experiments are going, and how critical it is with these experiments. Often I’ll say, “That’s interesting” but I’m going to keep going with my experimental plan because we’re only partway through or it was a side issue that’s not an immediate concern.

Interviewer: Is your experiment an entity here?

Participant: Yeah, I guess it would. If I dropped dead, I would hope that somebody would keep going with the experiment. The experimental plan doesn’t depend on me.

Interviewer: Do you send Data, Information, or Knowledge to the experiment?

Participant: I don’t. I build the experiment.

Interviewer: You send a flow of Data to the experiment and what would you classify this Data flow as?

Participant: It’s Data. I’m specifying the equipment geometry, the operational set points.

Interviewer: So you’re sending specifications to the experiment... Do you send anything else to the experiment?

Participant: The experiment itself is not a conscious entity. So I don’t think I could send anything else to it, I could only send it Data.

Interviewer: Does the experiment send anything to you?

Participant: Data.

Interviewer: It sends you Data.

Participant: I get numbers back from this.

Interviewer: Do we want to call that numbers or do we want to call that something else?

Participant: Data. I’ll get.

Interviewer: Sorry, the flow, the label for the flow. It’s Data, but what topic is it? Is it just numbers, is it results, or is it...

Participant: I’ll get numbers from the various instruments we have, be they cameras or weights versus time, that sort of stuff.

Interviewer: So you send a data flow to the experiment of specifications. The experiment sends you a data flow of numbers back. Then what happens? What other interactions do we have with these two entities?

Participant: We’ll also, for this particular project, we’ll also have {position 122}s. Who are {noun 95}ing new equipment.

Interviewer: What interactions are they having?

Participant: For there we’re, it’s a stronger influence than the {Position 22}’s. I’m actually wanting to give them Knowledge.

Interviewer: So you send them a flow of Knowledge or do you send them a flow of something else that they take...

Participant: They will get the same information as the {Position 22}.

Interviewer: So they get that flow?

Participant: They get that flow. They also get a flow from me about “I think you should do this.”

Interviewer: And you asserting: “I think you should do this?” is? Let’s first go: how do we label that?

Participant: I’ve done this experiment, I’ve asked the {Position 22} and it correlates with {noun 118} performance, therefore I think the new {noun 95} should have these features.

Interviewer: Can we summarize that?

Participant: I’d say that’s Knowledge.

Interviewer: That’s Knowledge. Can we have a 2-3 word description of this Knowledge that’s not that sentence?

Participant: Equipment {noun 95} recommendation.

Interviewer: So you send the {position 122}s your interpretation of measurements and you send them a Knowledge flow of equipment {noun 95} recommendations. Do you send them anything else?

Participant: As part of the Information flows to those, I may include a little bit of raw Data, but not very much.

Interviewer: Does this raw Data ...

Participant: usually photos or a graph.

Interviewer: So you would say that photos are raw data.

Participant: Yes.

Interviewer: Would you say that the raw data is contained in the Information? i.e. you send them interpretation of measurements. As part of that interpretation you have to send them some of the measurements that are really interesting. Would you say that the Data is inside the Information flow, and we can just label this as Information? Or would you say that it’s Information + Data?

Participant: I’d keep it inside the Information flow, because if it was just raw Data. They could very easily reach what I think is the wrong idea -- misinterpret.

Interviewer: Therefore you’re not going: “Here’s the Information, here’s the Data.” you’re going “Here’s the Information, here’s some Data inside the Information to back it up.”

Participant: Yeah, that’s right. But with just the raw numbers and no context, that’s Data.

Interviewer: You send them this Knowledge flow and this Information flow. Does the {position 22} send them anything? or are there any interactions that way?

Participant: As far as I know, yes there were. Part of that was direct transfer of people. They actually got some {Position 22}s into the {noun 95} team.

Interviewer: Would you say that’s a flow of Data, Information, Knowledge?

Participant: Knowledge.

Interviewer: So there’s a Knowledge flow, from the {position 22} to the {position 122}s of personal expertise?

Participant: Yes. Personal expertise and personnel, physically personnel. But personal expertise.

Interviewer: Personal expertise of Knowledge. Would you say that personnel transfer is a flow of Data, Information, Knowledge or O? Should we model a personnel transfer as some sort of conceptual flow?

Participant: No, it’s a convenient way of doing the Knowledge transfer.

Interviewer: So it’s basically sneakernet.

Participant: it’s better than just writing it down in a book and just posting the book and having them read the book. It’s much more convenient.

Interviewer: So personal expertise as Knowledge flow from the {position 127} to the {position 122}s... do the {position 122}s send anything back to the {Position 22}?

Participant: They’ll send Information about their current {noun 95}.

Interviewer: so would we say current {noun 95} or is there something more specific?

Participant: Current {noun 95} and intent of operation. No, it has to be all together.

Interviewer: So you’ve got a flow of personal expertise and they return current {noun 95} and intent.

Participant: and I’ll say that’s Information as opposed to... because of the intent side of it. It’s not just raw Data. Data I tend to think of as just numbers.

Interviewer: If it was just current {noun 95} it would be Data?

Participant: yes

Interviewer: But, because it’s current {noun 95} and intent, it’s Information. Would you say that the Data is encapsulated as this one is?

Participant: Yes.

Interviewer: Would you say that there’s no Data, there’s just Information?

Participant: No, because the intent is embodied in the {noun 95}. But you need to know it in case they’ve misunderstood an operation and the {Position 22} can correct, can feedback Knowledge, and say no no no, your {noun 95} won’t meet that intent.

Interviewer: And that feedback of Knowledge is the personal expertise?

Participant: Yes, that’s right.

Interviewer: So they send this Information flow of Here’s the Data + Information around it which is Data and Information, not just Information. Do you as {position 56} get anything from the {position 122}s?

Participant: Yes. I tend to get that same stream. the current {noun 95} and intent. But not so openly. I’ll get it filtered through because the {noun 95} is much bigger than what I do. There’s much more and there’s things that I don’t care about.

Interviewer: So therefore it’s a different flow because it’s filtered. Do we just want to put filtered in front of this?

Participant: Yes. For example, in my project, I very rarely get electrical Information. Because I was looking at material flows, not electrical flows.

Interviewer: This flow of filtered current {noun 95} and intent is... Information, Data, Knowledge, O?

Participant: Information. Information because it’s not a real thing yet. It’s not Knowledge about a real device. It’s still a hypothetical.

Interviewer: Do they interact with your experiment?

Participant: Yes, I may change, just as the {Position 22} feeds back, I might change the experiments because of that, likewise from the {position 122}’s side. I have changed experiments because of what they’ve said.

Interviewer: Is that just a function of changing specifications or is there another flow to your experiment based on their Information to you?

Participant: I’ve changed {noun 95}, the {noun 95} of my experiment.

Interviewer: Is there a {noun 95} flow to your experiment?

Participant: I’d say that’s part of the specifications. The geometry of it.

Interviewer: So geometry is part of specifications and all this is just Data.

Participant: It’s just numbers and stuff. Based on that, I’ll say I need to change dimension X by this or curvature Y by that.

Interviewer: Given this set of 4 entities, are there any other interactions?

Participant: We did have people building equipment. I don’t know if you’ll want to include that.

Interviewer: Are there stuff flows with these {position 123}?

Participant: Sometimes it was much more convenient for the {noun 95} team to talk to the {position 123} rather than me.

Interviewer: So the {position 123} are interacting with the {position 122}s in your experiment?

Participant: And with me. So: experimental {position 123} as opposed to people actually building the real {noun 118}. They were somebody else.

Interviewer: What flows are there from the {position 122}s to the experimental {position 123}?

Participant: I would give them plans.

Interviewer: Labeled as plans or drawings?

Participant: Drawings. Information. Wait, No I would say it’s Data, it’s numbers. I want a sheet of steel cut to these dimensions, surface treated in this way, joined in this way, join it in this way to this other piece of steel here.

Interviewer: And that’s just Data.

Participant: That’s just Data. They didn’t tell them the intent of what I was going to do. The {position 123} is not expected to interpret anything, they just manufacture.

Interviewer: If they were expecting to interpret, the {position 122}s would be sending?

Participant: Information, we would like the device to do this, please make sure that it can do this. But usually, if the {position 122}s were talking to them, they didn’t say that. They said: here’s the drawing, make that.

Interviewer: The experimental {position 123} then do what with that Data flow?

Participant: they’ll build equipment.

Interviewer: Does building of equipment represent some sort of flow of Data, Information, or Knowledge?

Participant: It’s the embodiment of the Data.

Interviewer: Is embodied Data a flow of Data?

Participant: Yeah, I’d say it would be. After that, the equipment {position 123} walks away and you have a device there which has the embodiment of Data in it. But you don’t know where it came from.

Interviewer: So there is a flow there.

Participant: Yes.

Interviewer: Where is that flow here? Do we need an equipment entity?

Participant: No, I’d put it into the experiment. Because everything there was included in the experiment.

Interviewer: So this embodiment of Data is a flow of Data?

Participant: Data, yes. So the piece of equipment is Data.

Interviewer: How do we want to label this flow. Piece of equipment or embodiment?

Participant: Piece of equipment.

Interviewer: The experimental {position 123} send a piece of equipment to the experiment. This piece of equipment is a Data flow because the piece of equipment embodies Data.

Participant: Yes.

Interviewer: Do the experimental {position 123} send anything else to the experiment?

Participant: Not to the experiment. They send a bill to me or someone else, the {position 124}.

Interviewer: Do we want to model the {position 124}?

Participant: No.

Interviewer: So they, to you, send what? A bill? They can send multiple things.

Participant: Usually the piece of equipment, we either pick it up on a vehicle or it gets delivered here. They don’t pass anything to me. What I get from them is the piece of equipment. It doesn’t have to come to me, so long as it ends up at my experiment, that’s all I care about.

Interviewer: So is there a flow from the {position 123} to you?

Participant: No.

Interviewer: Ok. Are there any other entities?

Participant: There is a flow from me to them. And that will be of the same form as what they get from the {position 122}.

Interviewer: So you send them drawings?

Participant: No. From me they’re more likely to get an intent. They’re going to get Information. Because I’m not a {position 122} or a mechanical {position 58}. And I know the function I want it to have, but I may not know that you need to make it out ... [of.]

Interviewer: So you, to the experimental {position 123}, send intent. Is this intent an expression of Information, Data, or Knowledge?

Participant: That will be Information.

Interviewer: What else do we have? Are there other entities that are interesting?

Participant: Involved with other {position 56}s. There are other {position 56}s who have helped from time to time.

Interviewer: And they’re interacting...

Participant: They interact with me.

Interviewer: {position 56}s do what?

Participant: They’ll, I’ll show them... sometimes they actually help doing the experiments. What they’ll do, they’ll provide a -- I’ll happily show them Data from the experiments.

Interviewer: So you send them numbers?

Participant: Yes.

Interviewer: Are these numbers the same numbers, or do you transform them somehow?

Participant: No, if I’m going to talk to them, I’ll send them real numbers.

Interviewer: Functionally you don’t perform any other task than switching.

Participant: Correct. It’s as if the experiment was sending them numbers. Not all like.. but for a time, a season... there’s a fellow {position 56} and I’ll need to double-check or get a second opinion or “can you help me with this kind of thing?” and they’ll get the raw numbers and we’ll have it.

Interviewer: So the numbers from the experiment go to the {position 56}s. This I assume is a Data flow?

Participant: Yes

Interviewer: Do you send anything to the {position 56}s?

Participant: Yes, I’ll be talking, in the system I’ve worked out here, Data is the raw numbers. Information is an interpretation of those numbers. And Knowledge is an embodiment of something, a real physical operating device of the company. So it’s something that we’re going to write down and keep for posterity and keep using. With the {position 56}s, ... I’ll stop modeling the Data for a while and just talk about it. It’ll be where... I see something interesting, but I want to make sure that it’s actually a reasonable conclusion. That it’s a sensible thing. That another person in my shoes will also reach. So I’ll do the interpretation of the Data myself, and I’ll also ask them to review it, to have a look at it. And see if it’s a sensible conclusion. So they often would have gotten some raw numbers, but they’ll also get the Information.

Interviewer: Is it the same Information that you would be sending to the {position 22}? Or is it different?

Participant: It’ll be different.

Interviewer: What shall we label it?

Participant: It varies, from time to time. It’ll be Information or it may even be Knowledge about an experiment.

Interviewer: Are these different flows?

Participant: Yes.

Interviewer: What is the Information flow about the experiment that you as {position 56} send to the {position 56}s?

Participant: I’m trying to thing of specific examples here so I’m not vague.

Interviewer: Vague is fine.

Participant: If it was an Information case, it would be, I have seen ... when we do x, y, z, I see behavior a, b, c in the rig. And I’d call that Information. And I’d usually ask them to review the Data, look at the Data, and say do you think that’s reasonable observation? However, if I’ve seen a class of behaviors or a systematic response to the rig to whatever it is I throw at it, then I would say -- then I’m talking to them about Knowledge. And it will be Knowledge about either the rig itself or about some phenomenon we may see on the {noun 118}. So it’s something to ...

Interviewer: So you send them an Information flow and a Knowledge flow. Let’s label the Knowledge flow first. It’s a Knowledge flow about phenomena or the rig?

Participant: It’ll be as though it’s a physical law of something or another.

Interviewer: So potential physical law?

Participant: Yes.. Potential physical law? It’s something on which I can predict the future.

Interviewer: whereas the Information is... correlation? request to confirm correlation?

Participant: It’s a request to confirm correlation. I think I’m seeing behavior x, y, z. Do you see it to?

Interviewer: But, because it’s only a correlation, it’s only Information?

Participant: Yes, that’s right, it’s not a predictive tool. I can’t... rather than saying, I think I understand the mechanism behind this thing that we’re seeing, and I think that this is the explanation for it, and if that explanation is correct than I can predict the future. If you throw x, y, z, at it it’s going to produce this effect. That’s probably the Information and Knowledge difference.

Interviewer: They have three flows going into them. Then what happens?

Participant: They’ll hate me because I’ve given them work. They’ll laugh at me...

Interviewer: Just a second, is that a flow? Is that expression of resentment?

Participant: Actually, no, it’s often more interest. Particularly when it’s the Knowledge kind of thing.

Interviewer: Is that a flow?

Participant: Their immediate response before anything else happens?

Interviewer: Yes.

Participant: I guess so, a positive response is always encouraging.

Interviewer: What it is a flow of?

Participant: Peer support.

Interviewer: There’s a flow back to you of peer support sometimes.

Participant: Peer/professional support. This is going to get more vague and nebulous as we go on. I’d say it’s just Information. It’s very ... it’s not numbers, it’s not ... although it could be... no, ... this morning, we talked about the case where [someone was] saying something ridiculously beyond the laws of physics. Even a first year undergraduate would recognize that it’s just a loopy kind of thing to say. And how do you correct that kind of thing? It’s more than just Knowledge, the professional support of a positive response to a hypothesis is kind of acknowledgement that things are... that you’re on the right track. So it’s Information there. Because the person has {noun 97}ed it at least slightly. Ever so little bit. And it hasn’t knocked up any red flags to say that this is stupid. So it’s Information. It’s an immediate interpretation of what’s in there. They haven’t looked at it in detail and found specific problems or errors with it or things like that.

Interviewer: Now are there any flows beyond that “Hey, that looks cool.”?

Participant: Eventually, I hope that they come back with a technical comment about it, that they’ve thought more deeply about it. That would be a critique.

Interviewer: So this critique is: Data, Information, Knowledge?

Participant: Information I would say. And there will also be times where they will also come back with Knowledge. Often based on the Data themselves.

Interviewer: So they can send to you Knowledge. This Knowledge flow is?

Participant: They look at the raw Data, and they think they can see systematic behavior which can be used in a predictive way to predict the future.

Interviewer: Would you say it’s the same Data flow that you send to them?

Participant: It’s of the same nature, yes.

Interviewer: Does it have a different label?

Participant: No I give it the same label.

Interviewer: Really, it’s a bidirectional data flow.

Participant: And that’s really good when that happens. In part, that’s the kind of the embodiment that I’m replaceable. And that we are actually doing science. The fact that somebody else can look at the raw Data and come up with that Knowledge is good. That’s how science operates. It should be independent of the observer.

Interviewer: One of my other hats is philosopher of science. I would say that you guys are engaged in a Lakatosian research programme

<explanation of Lakatos>

Interviewer: From what you described, the refinement of theories and the research of theories is a research programme. Because you’re not really questioning your inner assumptions of “this is {Noun 23}, this is what it does. We know it, we’re not going to go: ‘hey, I’ve discovered a new property of {Noun 23}’ we’re going ‘hey, here’s this refinement of the outer shell.’

Participant: that’s right. ... we don’t question the laws of physics. We try to find out how they’re expressed in a very specific set of circumstances.

...

Interviewer: Do we have any other interactions here? Perhaps a couple of Knowledge interactions? Interactions with ...

Participant: I guess we never got to the end of it, the other roles or things like that. Do we want to go onto other roles of me?

Interviewer: Absolutely.

Participant: Because then we have time... during the first {adjective 96} we were involved in, we would actually take measurements and giving immediate advice on those. It was convenient for me to be in that role, but it didn’t have to be a {position 56} as such.

Interviewer: But it was part of this {noun 97}?

Participant: It was part of the project, yes.

Interviewer: So what shall we call that role?

Participant: Whether the person in the {position 56} hat to be that person or not?

Interviewer: That’s the beauty of just labeling it as a role. You as the {position 56} can send flows to you in that role. We don’t have to instantiate roles really.

Participant: I’ll talk around it a bit first to help clean my mind. The person who fulfilled that role needed to be in charge of a small team. Who would take measurements using temporary pieces of equipment on the {noun 118}. This isn’t using normal {noun 118} {Noun 4}, we actually put extra cameras and extra this and extra that up there to take measurements of what was happening when the {noun 118} and bits of equipment were being run for the first couple of times. They weren’t being run with the {noun 118} being all operating, So it was slightly contrived. But it’s part of {adjective 96} to make sure that everything works properly. that you push button a and things happen as they should.

Interviewer: “If we push button A, does pipe A actually flow?”

Participant: Does it do the thing we expect it to do? During that there needed to be a bunch of measurements taken so that they could be analyzed to confirm that the {noun 95} intent was being met. So that it could be confirmed to be the {noun 95} intent as best as we could tell. And you needed somebody in that role who was familiar with the {noun 95} intent and with the equipment that was needed to take those extra measurements.

Interviewer: Let’s label this role.

Participant: That’s {adjective 96} ... are we separating the role from the actual measurements? I would say that’s that way. It’s different from experiments. {noun 126}s is what they were.

Interviewer: These {noun 126}s, what entity sends them flows?

Participant: The main purpose of the {noun 126}s is to find out that when you press button A that things happen as they should So it was the {position 122}s, they send to the {noun 126}s, I guess it was Information.

Interviewer: Information about?

Participant: Not the {noun 126}s themselves. They’re not a self-aware entity. So they just receive Data.

Interviewer: So here’s the thing. It depends if the {noun 75}s are

Participant: There wasn’t somebody who wasn’t on the {noun 95} team or in the {Position 22} camp who was running the {noun 75}s.

Interviewer: So if you want to put that person in the {noun 75}s you’re welcome to, or you can have them be a separate entity.

Participant: Yeah, {position 125}s.

Interviewer: do we want to make an entity for the {position 125}.

Participant: Yeah, separate the people from the actual {noun 97}.

Interviewer: So the {position 122}s to the {noun 75}s send what?

Participant: Data, numbers.

Interviewer: These numbers have what... what numbers are hey sending?

Participant: It could be embodied Data in physical equipment. So the actual {noun 118} top.

Interviewer: Equipment.

Participant: Equipment and that’s what they provided to the {noun 75}s.

Interviewer: What are they providing to the {position 58}?

Participant: They’re sending him Information. And that was the designs and the operating manuals.

Interviewer: Flows from the {position 58}s to the {noun 75}s.

Participant: From the {position 125} to the {noun 75}s?

Interviewer: All of the above.

Participant: The {position 125} also got Information from the {position 22}.

Interviewer: And what is this Information?

Participant: It would be a request. They tell them, when we usually run the {noun 118}, this is how we tend to operate the equipment. I want you to make sure we cover that kind of operation in your {noun 75}s.

Interviewer: So request for {noun 75}s maybe?

Participant: Request for performance is a better description. For example, there’s no need for the {position 125} to run the conveyor belt by startstopstartstop. You don’t need to test that because we never do that. It actually is physically capable of doing that, but we don’t care about that kind of testing. It’s not going to happen.

Interviewer: To test it the way we want the performance.

Participant: The way we usually do it and the reasonable boundaries that that...

Interviewer: and these requests for performance are Information. Do the {position 125}s receive anything else from anybody?

Participant: Yes, they’ve got Information also from me as {position 56}. And that would be Information of a similar nature, but to say this is the new behavior that we’re looking for, please make sure that the {noun 75}s test that area. We want to measure the behavior under this circumstance. Please make sure that circumstance happens during that {noun 75} and that we have the ability to measure it.

Interviewer: So request for measurement.

Participant: Request for measurement. And that was just from me though. I was the {position 56} so everything was coming through me, or that role. And then the {position 125} wouldn’t give feedback to any of us. They said what was possible. So they told us what their plan was.

Interviewer: Were they talking about plans or feasibility?

Participant: Plans.

Interviewer: Who were they sending plans back to?

Participant: Everybody who told them things. ... And then it was the same Information came back to all of us.

Interviewer: So they’re sending their plans as Information to {position 56}, {position 22}, and {position 122}s.

Participant: They found it convenient to send everything to everyone. Because then everyone could say: “No no no, you’ve missed X, please change your plans.”

Interviewer: Is that no-no-no you missed X a different Data flow?

Participant: no, it was reinforcement of the previous Information. ???

Interviewer: Are there any Knowledge or Data flows to the {position 125}?

Participant: Well, there would be some Data included with the Information. It’s usually interpreted Data. You don’t want to give raw numbers coming from anybody. We’re expecting them to generate or their {noun 126}s to generate raw numbers. Then they’d run the {noun 75}. So they send Data to the {noun 75}. They say the set points.

Interviewer: So {noun 75} specifications?

Participant: Yes.

Interviewer: These are Data?

Participant: Yes.

Interviewer: These {noun 75} specifications as Data are sent to the {noun 126}s. These {noun 75}s take the {noun 75} specifications and the equipment, the Data embodied in the equipment, and do what?

Participant: It just operates. When I think of {noun 126}s, I think of the piece of equipment that’s over there set up in the way it should be plus people who are there who might be actually operating or pressing buttons. But they’re not necessarily... these are my instructions. I must press this button at this time. Or talk to so and so. And we do the {noun 75}. Just do this.

Interviewer: These {noun 75}s send what to whom?

Participant: We get Data. The {position 125} gets Data from the {noun 75}. Everybody who’s going to get it. In actuality, all four, the {position 125}s, the {position 122}s, the {position 127}s and myself were all there.

Interviewer: Basically the {noun 126}s send a flow to *2 which is the distribution.

Participant: Yes, that’s right. As well as back to the {position 125}s.

Interviewer: That and this flow is?

Participant: {adjective 96} Data.

Interviewer: And this {adjective 96} Data is Data?

Participant: Yes.

Interviewer: Do the {noun 126}s send or get any other flows?

Participant: No.

Interviewer: Are we missing anything?

Participant: So the {noun 126}s are in a way similar to the experiments in the way that there’s Data going around. I think that’ll do, because then you get into {noun 118} operation. And that’s not part of the... most of us are gone, it’s just the {Position 22}s who are doing their thing at that point. Again, from the {noun 126}s the data that say I get as {position 56}, I’ll feedback Information to the {position 127}s, as Information. But I may also feed back Knowledge at that point.

Interviewer: So you’re sending interpretations of measurements, but they’re just different measurements. But it’s the same flow.

Participant: It’s the same flow. At that point, I may be sending Knowledge to the {Position 22}s. And that will be a different thing. That’ll be a further refinement and converting the Information to give it a predictive capability.

Interviewer: how do we want to label this?

Participant: Operational understanding.

Interviewer: Does anyone else get this operational understanding?

Participant: Various people in the technical community. That may also go to the other {position 56}s, but it’s not really of use to them. It’s more politeness to them.

Interviewer: Fundamentally speaking you’re sending the flow to the {Position 22}, though other people may receive it

Participant: It gets recorded in the library, that kind of thing.

Interviewer: Let’s talk about the library and books (which are different.) does the library have a role here? And do books have a role here?

Participant: Oh yeah.

Interviewer: Let’s start with books. What interactions do these entities have with books?

Participant: Internally generated documents? Or any kind of recorded Knowledge?

Interviewer: Do we want to differentiate those?

Participant: Probably not. Myself and the {position 122}s and the {Position 22}s. But mostly myself and the {position 122}s. We’ll get Knowledge from books, from sources.

Interviewer: Call it books?

Participant: Call it books, but it means books and printed matter or internet things. Not necessarily Wikipedia. Trusted sources of Information.

Interviewer: These books, as trusted sources of Information, are sending ....

Participant: Trusted sources of Knowledge.

Interviewer: These books, as trusted sources of Knowledge, are sending these three entities, although these two in particular, what?

Participant: It’ll be things like the laws of physics as applying to something close to our area, as best as we can apply to our area.

Interviewer: These laws of physics are Knowledge?

Participant: Yeah.

Interviewer: Do you, or any of these entities, send anything to books?

Participant: I’ll record. I try and write reports and documents of my experiments once they’re complete.

Interviewer: So you as {position 56} send documentation.

Participant: I’ll create documents, I’ll create books. Not very big ones.

Interviewer: But you’re sending to the entity books...

Participant: Information and Knowledge. Sometimes...

Interviewer: In the same flow or different flows?

Participant: In the same flow. Sometimes I’ll be able to deduce some Knowledge.

Interviewer: So you send a flow of Information and Knowledge... This flow of Information and Knowledge can be labeled...

Participant: What we did we call the data coming back to me?

Interviewer: Numbers.

Participant: Can we change that label to experimental Data?

Interviewer: Yes we can.

Participant: The experiment sends {position 56} and {position 56}s experimental Data. And my Information and Knowledge that goes into books or documents is, call it experimental interpretation. Analysis? Maybe analysis. Cross out interpretation. Experimental analysis. I use the term analysis in a lot of titles.

Interviewer: Do we want to generalize this documentation or books?

Participant: Documentation.

Interviewer: So you send Information and Knowledge in your experimental analysis to the docs. The docs send, to these three people, applied laws of physics, or pertinent laws of physics.

Participant: Yeah, pertinent. ... We differentiated here between the library and the books.

Interviewer: Is this here a worthwhile differentiation?

Participant: The library is merely an internal convenient store for it as opposed to, we can get into the local library down at [town name.].

Interviewer: So documentation is location independent, so therefore the library exist as an entity outside of its documentation?

Participant: Yeah.

Interviewer: How do you request documentation? Is that a flow, or is that a flow or do you just get appropriate flows from them?

Participant: They’ll construct catalogues and things of what they’ve got.

Interviewer: Is the catalogue pertinent to this project?

Participant: No, not really. For the sake of this project, we don’t generate that many books. Putting the library in it would be overkill.

...

Participant: I’m thinking that the overall project is for the {noun 95} of the equipment of {Noun 2}. And I’m thinking that we’ve pretty much covered everything in this project.

Interviewer: Let’s have a theoretical discussion. What happens in programming?

Participant: Computer programming?

Interviewer: Yes. What kind of Data, Information, Knowledge flows are there?

Participant: Do you want me to think of a specific computer program that I’ve written?

Interviewer: When you’re programming, what do you send into the program?

Participant: It depends if I’m writing a quick little thing to do some analysis of Data, there’ll be something I want to do to the numbers, so the program is going to be fed some numbers, and it’s going to spit out some numbers, or print a graph or something like that. But I want to transform those numbers in a particular way, If it’s quick like in an excel spreadsheet, if you’re doing a little formula, it could be just some Information. But I say {noun 97} these numbers in this way. Sum it over the ??? something like that.

Interviewer: and you would say that instruction to the computer is Information?

Participant: The intent of what I’m doing is Information to it. Because I don’t know how it’s actually going to do it in the CPU. But I’m telling the device... you fill out these numbers in this manner and tell me the numbers at the other end.

Interviewer: Whereas a more sophisticated program, what are you sending it?

Participant: Now, let’s say one of the models, I’ve written a {Noun 2} model or something like that. I usually think of them as an embodiment of Knowledge. There are laws of physics and the laws of chemistry and all of that that we can express as equations, and we can put them in in a generic kind of way then we have this model that has our Knowledge in there. And then we feed it numbers in and numbers will come out and they’ll be transformed according to those laws of physics and the laws of {noun 109} in there as well.

Interviewer: Physical laws.

Participant: Yeah.

Interviewer: So you would say that... your transmission to the program of physical laws is a flow of Knowledge? which the program then embodies?

Participant: No. I tell it Information.

Interviewer: You tell it Information... it then embodies...

Participant: I watch it transform, but then that Information is an embodiment of Knowledge. I know what the laws are, but the computer is just a machine. It doesn’t know that... it doesn’t know that the law of gravity is actually useful for saying that things fall down and break. We know that. It gets Information. I won’t even tell the hardware. You’re talking to the software that’s running on top of the hardware. “I want you to do whatever you do to manipulate these numbers in this way.”

Interviewer: So you would say that code is Information.

Participant: Yes. Based on Knowledge, but it’s just Information.

Interviewer: But what you’re sending is not Knowledge, it’s Information.

Participant: That’s right.

Interviewer: You would say that the ability to make predictive assertions about the universe is a function of?

Participant: That’s Knowledge. That’s an understanding of reality. So you need to have consciousness and self-awareness and that kind of thing. So my laptop, I don’t know about yours, but mine definitely doesn’t have self-awareness as far as I can tell.

Interviewer: Let’s look at this. Self-awareness is a vital prerequisite for.... sending flows of... Information and Knowledge?

Participant: Knowledge in particular.

Interviewer: Without self-awareness, Knowledge cannot be transmitted?

Participant: You can transmit it into a book, but then that book is frozen communication.

Interviewer: So it’s just an embodied... frozen communication as you said. An entity is talking to another entity via the communications medium of a book?

Participant: That’s right.

Interviewer: So the book itself does not have awareness, therefore it cannot operate upon its Knowledge. Even though it’s the vehicle for Knowledge transmission.

Participant: That’s right.

Interviewer: So how does ... it doesn’t actually sound like you’re describing a hierarchy here in terms of Data, Information, and Knowledge. Or does it? Tell me at any point if what I’m saying is incorrect. You would say Data are... would you say that Data is or Data are?

Participant: Data is plural. Datum is singular.

Interviewer: I’ve found different professions have evolved different plurality rules.

Participant: [name]’s very demanding about that.

Interviewer: So these Data are raw numbers?

Participant: Yes.

Interviewer: When you say raw numbers, what do those numbers represent?

Participant: They could represent anything, they’re just numbers.

Interviewer: So any numbers are Data? or are some numbers not Data?

Participant: There’s some Data that’s not numbers.

Interviewer: What Data is not numbers?

Participant: Qualities of things.

Interviewer: So qualitative and quantitative Data?

Participant: That’s right.

Interviewer: and qualitative Data are textual?

Participant: Bigger smaller higher. It’s something that you haven’t been able to put a number to, but you think it’s important or that it’s worth noting. It’s some... Information that you want to convey. That’s not the right word. It’s something that you want to convey, a quality about something. And so you’ll try and give it some numbers if you can, but otherwise it’ll be qualitative.

Interviewer: Would you say that qualitative Data are Data or Information? Despite the label?

Participant: It’s probably getting into Information then, because you have to know the context. And ways to understand the terms that are being used. So say it’s a hot day. What do you mean by that? Do you mean you’re an Englishman or an Australian when you say that? Or an Eskimo. An Eskimo and a Hawaiian are going to have very different ideas of a hot day. So it’s Information. Context is going to be something. But if I say it’s 42 degrees Celsius, then except for the Hawaiian everyone else probably thinks that’s hot. But it’s actually put into a ... it’s not a comparative ... You’ve given it in some kind of absolute term.

Interviewer: How does Data relate to Information?

Participant: Data itself is even useless without a context, you need to know what it’s about. but Information has a richer context and it’s less precise.

Interviewer: The maximal precision is in Data?

Participant: Yes.

Interviewer: And the minimal precision is in Information? Or is there something outside on the precision graph?

Participant: There’s more room for interpretation in Information rather than Data. Data is meant to give you very little room to wiggle in. It’s meant to be precise. But then it’s precision kind of restricts its usefulness to some extent. There’s nothing hanging off the sides of it.

Interviewer: So there’s an inversely proportional relationship between precision and scope? So we’ve got Data. Data is ... I don’t think we actually have a definition there.

Participant: Data is numbers.

Interviewer: And these numbers are? What do they represent?

Participant: They could represent anything. Any time you write numbers down it’s Data.

Interviewer: So Data is merely written numbers?

Participant: Yeah.

Interviewer: These written numbers can be turned into Information?

Participant: Yeah.

Interviewer: They can be. So there’s a relationship between Data and Information?

Participant: Yeah. There’ll be a context to interpretation.

Interviewer: So Data + context + interpretation?

Participant: Data + context is enough for Information.

Interviewer: equal Information.

Participant: Yeah.

Interviewer: What provides context?

Participant: Metadata on the side.

Interviewer: So metadata provides context. i.e. other Data, juxtaposed with Data. Does anything provide the addition? Or is it just Data + metadata = Information?

Participant: IST may come from the thing that’s generating the Data itself. So adding a title string to a file or something like that gives you a context. It helps you remember about it.

Interviewer: So there’s metadata which provides context, but there’s nothing which attaches the context to the Data save for the context.

Participant: Yeah, I guess so.

Interviewer: Can Information become Data?

Participant: No, not really.

Interviewer: So there’s a flow that way but not that way.

Participant: I guess it could. But you have to change what you’re looking at. Information can become Data but not when you’re talking about the initial thing the Data was about. You have to kind of change topics. Looking at a different scope of things...

Interviewer: Is there a backwards ...

Participant: No, not within a system.

Interviewer: It’s data + Context becomes Information. and Information is irreducible to Data within the confines of a given scope.

Participant: That’s how I think of it, yes.

Interviewer: This is what I’m getting at. Can Information become anything else? Or is Information the upper bound?

Participant: In the same way that Data flows into Information when it’s given a context, the Information can become Knowledge.

Interviewer: With the addition or interaction with?

Participant: I’m trying to think about what’s the thing you have to do to the Information to generate Knowledge. It’s stronger than interpretation. Plus understanding?

Interviewer: Information plus understanding.... Understanding of the Information? Is understanding a function or a separate thing?

Participant: I don’t think I kind of... a mental model is that there has to be some kind of consciousness looking at the Information to generate Knowledge.

Interviewer: So you go, Information and understanding are arguments to consciousness?

Participant: Yes.

Interviewer: Might as well use familiar terminology. Consciousness, taking in arguments of Information and understanding produces Knowledge.

Participant: It needs Knowledge. The understanding is actually Knowledge. So Information plus other Knowledge generates new Knowledge.

Interviewer: Can Knowledge become Information?

Participant: When you constrain it.

Interviewer: How do we want to represent that?

Participant: It’s applied Knowledge.

Interviewer: So Knowledge + .. - maybe? What metaphor is it?

Participant: Knowledge applied to a specific situation can give Information. So, for example, the law of gravity is Knowledge. And I apply it to the earth-moon system, and the Information I can get is the tides or something like that.

Interviewer: It’s not Knowledge, it’s Information.

Participant: Yeah. So that there’s going to be a high tide in 2 hours or something like that, that’s Information.

Interviewer: Are there direct interactions between Data and Knowledge?

Participant: Yeah, because the Knowledge is the law of gravity or something like that. And you’ll throw some numbers into that and it’ll spit out some numbers.

Interviewer: So Knowledge + Data = Data. And this new data is what, predictions?

Participant: Yes. So, Data + Knowledge = Data.

Interviewer: So Data + Knowledge is what kind of Data?

Participant: That’ll be Data. So Information + Knowledge will give Information. So like the earth-moon system, that’s probably more... So now we know that there is an earth-moon system. And we’re going to apply the law of gravity to apply the Knowledge to it, and it’ll become some Information about the earth moon, so I can make general comments about it.

Interviewer: Information, there is a solar system. Knowledge, here is law of gravity, Information comments applying law of gravity to earth-moon system. Deriving LaGrange points would be Information?

Participant: No, because you have to put in the masses of different things.

Interviewer: So that is Data + Knowledge is Data. So you’re going these numbers plus this way of manipulating these numbers produces other numbers.

Participant: It’s a number where you can put a rock and it’ll sit there in space.

Interviewer: Are there any other directions that Data can take here? Can Data become other things outside of Data, Information, or Knowledge?

Participant: I can make up words, but it would just come back to the same idea.

Interviewer: Is Data atomic, divisible?

Participant: Well, Data, except the plurality... I’d say it’s atomic. It’s as simple as it gets.

Interviewer: So you can’t get below a datum?

Participant: No.

Interviewer: Is Knowledge ... is there anything that is to knowledge as knowledge is to Information?

Participant: No.

Interviewer: So we have bounds. Data and Knowledge. Nothing outside of the bounds.

Participant: I don’t think so. Physical reality, but I’m not sure where you can philosophize then. Well, maybe that’s below Data. Physical reality is reality. Real is, that’s it.

Interviewer: So therefore, we have reality on this?

Participant: Yeah, put it in. Yeah, because numbers aren’t physical reality, numbers are numbers. They’re a quantification of physical reality. So it’s even more underlying than Data.

Interviewer: We have what applied to physical reality to get Data?

Participant: Usually a device of some kind.

Interviewer: Do we want to say device, or is device doing a verb?

Participant: OK, I’m going to read the {Noun 12} in the room at the moment. So we get a thermometer. But it needs to be calibrated and I need to know that thermometers tell me about {Noun 12}, so I need to have Knowledge about the operation of them.

Interviewer: So we have physical reality, but in order to get Data out of physical reality,....

Participant: We’ve got physical reality, but I actually need some Knowledge to quantify physical reality.

Interviewer: The act is quantifying physical reality via Knowledge to get Data.

Participant: Yeah.

Interviewer: But that’s a one way function. And then Knowledge floats around being Knowledge.

Participant: Yes. It’s frozen in books. And if conscious beings want to make their existence comfortable in this reality then you’re free to pick it up do with it as you wish.

Interviewer: Final question. Normative assertions. When you say you should do that, would you say that you’re passing them Data, Information, Knowledge or O?

Participant: When I say to my kids, you should do your homework, it’s Information.

Interviewer: When you say to an {position 58}, is that Information?

Participant: How many sentences am I allowed to add at the end of it? If I just say: “You should increase the blast {Noun 12} of the {noun 118}.” that’s Information. But they know how to do it. If they want to say why. They’ll probably come back with “Why” or “No.” But that’s the same thing.

Interviewer: If they come back with no, how do you respond, do you give them a flow of Information or Knowledge or Data?

Participant: It depends on the case. I may give any one of those three. It could be Data, because I might say to the {Noun 20} and say “you should increase the {Noun 12}” why? “because your blast is at 900 degrees.” And it doesn’t need to be said to them that that’s ridiculously low. That you need to up it to at least 11 preferably 12 hundred degrees. So I could give them a piece of Data and they have enough Knowledge to go, say “Oh yeah. That’s bad.” they know that it’s in error. Or a Knowledge situation is “I think you adjust the angle of this thing by .5 degree.” and at that point I may give Information or Knowledge according to how much Knowledge I think they have or need. Or don’t have.

Interviewer: But the initial statement is always Information?

Participant: Well, I may talk to them about a Knowledge thing, I assert a normative assertion, it’ll be, a you should... I may say Data, I may say Information, but if I’m saying you should, it’s not a Knowledge statement.

\stopextract
\section{Interview 6}
\placefigure[]
[fig:i2]
{The SDFN Diagram for Interview 6}
{\externalfigure[Chapter4/graphs/i6.pdf][factor=fit,frame=on]}

\startextract
Participant: Out of that, I became really fascinated in data and knowledge and information because it's -- that project -- people had different views on it and that's why I'm interested in talking to you because you can say "about {Noun 1}" but it means something different to everyone else there. Even the form that it takes.

Participant: It becomes an inherently personal thing in the {noun 129}. I was developing this thing in a very personal view, and it wasn't shared by everyone. "You don't do that, that's not what {Noun 1}'s about." but, to me it was. To some people it was about data. To me it's about knowledge. Other people it's about information. I guess we're here to ???. ...

...

Participant: Anyway, you might want to lead me through the {noun 97}.

Interviewer: Fair enough. ... You mentioned everyone has their own definition of {Noun 1}.

Participant: I'd say their own view. Probably not at the definition level. 

Interviewer: Understanding? View?

Participant: I'd say definition because they don't actually describe it. View is they articulate a vision. A definition would be they actually write something meaningful.

Interviewer: Tell me what your view of a {Noun 1} is, and what you think their views are.

Participant: My view of a {Noun 1}, it really should be about capturing different sources of Knowledge from different aspects, and that can be from different people, different approaches: You can have production thoughts, research thoughts. Mathematical models. Other motels, like data mining. ... And so you have disparate sources of how things work, and they all relate to making the {Noun 2} work better. They think you should do it this way, and the computer model says we should do it this way. They don't always agree, but they look at it in different perspectives. And they say: "oh the {Noun 1} is to grab those Knowledge and we refer to them as knowledge-bases and that's an extraordinarily old reference. Knowledge-bases together. And then to make a sensible input to provide a single piece of advice to a person. That was my view. A single piece of advice and I've implemented the {Noun 1} as such. ... we have another person who has a different view on a {Noun 1} and he see as much more as a Information system where you provide, as we might refer to: Information to people. And this is commonly, in our field, or philosophy, or where we come from, our research? as well as our [people] in other parts of the business see as {Noun 3}. So you have a valve that is open or closed, a 1 or 0, as a function of time. We have lots of {Noun 4} with calculated outputs and ... A very other common view is to say that they should be provided that information to make their own... A {Noun 1} doesn't tell anyone how to do anything, it just gives them stuff to look at. And maybe it's just a guided look, maybe you don't show them this valve, maybe you show them that valve. Nonetheless, it's about showing them things they want to look at. 

Participant: So that's really the opposing, for me, that's sort of the two main approaches we talk about taking. 

Interviewer: So advice versus providing information. Tell me about advice. What do you mean by advice?

Participant: With advice, I mean a specific action. You actually have to tell someone to do something. Advice is like: "Brian, catch the earlier train." Information to me would be: "Brian, here's the train schedule. You choose your best train. I'll just tell you the schedule and then you'll go on and sort it out for yourself." Where mine is if I want to help you make a decision about what train to catch I'll gather knowledge from the train master, from people who catch the train and present you: the 1:00 train is the best one for us. In comparison to other people saying "I'll collate the train timetable information, maybe I'll have indexes relating to crime levels. So you may then choose, 'I want that train, but it's got a high crime level, maybe I'll choose another train. It's a bit op{noun 71}ue, you don't know how people made the decisions, you're just giving them information to look at.

Interviewer: When you say knowledge capture, is that what you meant by {Noun 4}, or is there a different term you mean by...

Participant: In terms of knowledge. Well,it's funny. I use it in a different way, but it's probably pretty similar. When we talk about the bubble [DFD people] concepts as inputs, we refer to that -- like a {Noun 4} as data as such. And Knowledge would be someone which would be putting that data into context. You might say: "When the valve is above [{Noun 12}] Celsius, I would turn the valve off" that, to me, is Knowledge, where the value itself is Data. It's like a hierarchy. Wierdly enough I set my system up in a hierarchial f{noun 137}ion. I didn't want to do it, and I really love case-based reasoning. Well, we can talk separately if it comes out of -- the difference between case based reasoning and that form of knowledge compared to other expressions of knowledges, I refer to it as.

Interviewer: Data as {Noun 4}, Knowledge as the {Noun 4} stuff with context with memories, you'd say? What do you mean by context?

Participant: You might have such as limits, there might be arbitrary statements like: "When the data is trending up" It may not even be specific. Sometimes you might say: "When a valve is a {Noun 12} above [{Noun 12}] Celsius"

Interviewer: So it's generalized statements about a {Noun 4}?

Participant: Yes, that's right. That has been a reflection of a compromise how we've set up the work that they're doing over the last 5 years.

Interviewer: What about information? Is it different? is it the same?

Participant: In that hierarchial concept, Information sits between data and Knowledge. Again, A classic form of Information that I describe it as, it's multi-dimensional Information. That's the difference I draw here too. We're sort of getting into multi-dimensional information as we describe it. You should talk to [Person X] about this as well. You might have data versus time. It's two dimensional, it's a line. And we like making three-dimensional graphs of things so we have this versus that versus that. you can have a peak. We use a mathematical model to generate an optimal point and you talk to [Person X] at the university of [X] about this n-dimensional space which means nothing to me. ...

Participant: Information is really, its like... Data is something you can work with. Information is an expression of the Data in an almost trivial form. It can just be plotting it as a function of time on a graph or contour plot. Something like that. I hold Knowledge as the hierarchy of it all. Where it has that line in context of other things, or people's experience. It's where I come from. It's a personal sense, that it's a reflection of, that -- I believe in mathematical models. They're absolutely excellent. but they need to sit alongside experiential knowledge. Mathematical models are, to me, information. Because they take data in and they can transform it into something is maybe trivial or non-trivial, but you might get a predictive {Noun 12}. And that's good, but, them, themselves, in my experience, are not actually good on their own. They need knowledge. Someone says: "I only look at that when this happens or that happens. Or it's no good today because that thing's broken. We just can't use it at the moment. But if you looked at it solely in itself. Well, that's very important. it can or not be, that is people's experience the Knowledge can help transform the data into information into knowledge. 

Interviewer: Is there anything on either side of Data or Knowledge? Does the hierarchy continue even as not a hierarchy?

Participant: It's certainly a very [company] view. We often come to that ontology -- It's sort of a common ontology that's developed within 10 years around the technical areas about these sort of things. 

Participant: Do things exist on either side? Well, they do. and I think they would, and it's almost like other problems. To make my life easier, I ignore them almost. ... it's commonly accepted that I can top and tail that. ... Because I'm working within a known space. You could go out there, but I just don't. I can see its there ... it's good enough for me. ...

Interviewer: You said atomic. Do you believe that data is atomic, in the traditional sense, i.e. indivisible? Or is there there...

Participant: For our [company] ontology, yes it is. As my professional life, yes it is definitely. As a philosophical perspective, I can appreciate its not. But as far as philosophy in a [company], no its not. Even when we talk to the instrumentation guys, we still talk at that level, we don't go through it. You can sometimes touch on it. People sometimes allude to it that there's stuff there. "Well, we've got [number] minute data, and we want more data or different Data", but it's not commonly accepted"

Interviewer: The second part of this, as you know, is the drawing of the little circles. Let's identify two or three work related activities that we can diagram where you can illustrate what you meant in what you just told me. What do you think they should be? Here's a better question. What do you think your 2 or 3 most important or interesting work activities are, preferably ones involving small groups? 

Participant: Developing rules, or knowledebases.

Interviewer: {Noun 1} {noun 129} you'd say? No, you said developing rules for the knowledgebases.

Participant: Yeah, that's specific...

Interviewer: Okay, developing rules, what else?

Participant: Well, the O, the fascinating thing for me which I argue is the debate about the description about the new ... system. called {Noun 5}: ... The {Noun 5} system re {noun 129}. And then those two things are the most interesting related to data. The other things I really hold strong opinions on .... so they are probably the best place to start. 

...

Interviewer: So let's identify the entities. 

Participant: I never know quite which level to start at. There's always me, it it just me, or is it Os? I am the center of my own universe, I guess. 

Interviewer: So this is you as what?

Participant: As a {Noun 6}. 

Interviewer: So you as {Noun 6}. Are there any other roles that you play within the developing rules context

Participant: There are a few there. You can have me as the {Noun 1} developer. There is me as ... it's a little bit of my general day job? It's what I might commonly do? Me as a {Noun 6}? ... if I'm setting there today looking at my system, weirdly enough, they don't always happen to match up. Because that's, there's drivers driving me there. There's go and do this and I put on that hat. And I'm not quite in the same space. I can quite often self-justify anything you can do. Maybe I'll leave it like that, otherwise it gets a bit sort of hairy.

Interviewer: hairy isn't necessarily bad

Participant: But there are two main roles. I'm building systems and I'm typically using them as well, or using other systems. Those are the me roles. 

Interviewer: Who do you talk to?

Participant: There's other {position 56}s, sort of group members. If you talk to a few people, there's some sort of common ontology of who these people are will come out. ...

Interviewer: how do I label that?

Participant: Exactly as I said it. It won't be personally identifiable. Can I put [name] there? There's {Position 7} members. These are a very specific terminology. We have {Position 8}. These are different from {Position 9}. They're a different role in the controllers. These are various [company] specific positions. ...

Interviewer: We're talking about {Noun 1} {noun 129}? No.

Participant: {Noun 1} rule {noun 129}, Developing rules in the {Noun 1}. Interesting thing I do with data or how I look at these things.

Interviewer: Where do we begin? 

Participant: Me as the center the universe, of course.

Interviewer: You as what?

Participant: Me as {Noun 1} developer, because that will be my primary job.

Interviewer: Tell me about the flows of stuff.

Participant: Actually, another bubble there: it's a computer system I'm working on! It's definitely in there. We had that configured in [last interview] but -- we'll have the {Noun 1} as such, let's even say it exists. It goes weirdly through that metaphase it may or may not exist. As you're developing it it's sorta there there and sorta not.

Interviewer: It will have had existed?

Participant: Yeah, it might exist for a bit of a period during a {noun 75}, then it disappears. But it exists now, then it continues on. Then there's {Noun 5}, we talked about {Noun 5}. That's another computer system we deal with. I'll draw the line there, {Noun 5}. 

Interviewer: Well, you're the {Noun 1} developer, so let's do the nice simple start with {Noun 1}. 

Participant: That's right. I would stay -- I have my computer system. 

Participant: I send it, what I would refer to -- as a developer -- I'm encoding knowledge. I'm sending it knowledge as far as I'm concerned. It's not easy to try to express humans as an if-then-else statement. Maybe we can talk this as some sort of psychological session later. 

Interviewer: You're sending it knowledge...? Can we label this knowledge, or are you just...?

Participant: It's computer code. I'm sending it computer code. It's knowledge expressed as computer code, for sure. It actually sends me back Knowledge expressed as textual advice which is often been... ??? over the years.

Interviewer: It sends back knowledge as advice?

Participant: Yeah, as textual advice. I don't know why I make that distinction., but I've always tried to put the stuff it sends back, even to me, because I sit there and read it as textual things that sort of are vaguely sentences I best can do in 32 characters. Its like its text, it's words it might have a noun and it might have a verb or ... 

Interviewer: Is this medium? Or is this part of the message? The text?

Participant: It's the message? 

Interviewer: I don't see what you mean by medium there?

Participant: Is it sending the advice through text? Or is it sending text advice through something?

Participant: It's sending text advice through a range of different mediums. Because you can access it ... we didn't sort of go there, how people access it, but there is interfaces to how it goes out. And it changes a little bit depending on the interface, but there is sort of one interface which I consider which they really use in the {Noun 20}. I sort of focus, when I think about it, I think about that. That's my little box with little words on it. {Noun 11} which are bit more descriptive. I've actually got a {Noun 11} part which has a lot more stuff in it which you can do, no-one ever uses that so it's almost like it's ... I don't really support it.

Interviewer: What else do we have? So you send it code, it sends back to you advice. What else?

Participant: I'd put {Noun 5} over here, because it sends information to the {Noun 1}. Data! It sends it data, for sure. 

Participant: and that's expressed as values -- it can be both real values such as a {Noun 12} it can also be derived values such as an {Noun 13} like that. We sort of draw distinctions a little bit between that. 

Interviewer: So derived values as in calculation results?

Participant: Yeah. So you might calculate an {Noun 13}. So {Noun 5} calculates some -- an efficiency of something. I take that as well as I take raw data.

Interviewer: Do we want to say that as a different string?

Participant: Yeah we can. I certainly do derive differences between it. 

Interviewer: What are we labeling this string as, these calculated results?

Participant: Derived data, we refer to them as. So it would be "Data, derived." As much as these are data like real ...

Interviewer: what else?

Participant: Out of here... My next primary focus is {Noun 14}. 

Interviewer: What are they talking to?

Participant: They'll be talking to the {Noun 1} and to {Noun 5}, and to a few other of these guys as well. This is sort of getting complicated already.

Interviewer: you have no idea. And tell me about this?

Participant: We're sending, {Noun 1} is sending text advice as knowledge to those guys -- to their {Position 8}. 

Interviewer: and what else?

Participant: we've got to put the {Noun 2} here. It doesn't make any sense without having a {Noun 2} here, because they have to send stuff to the {Noun 2}. So that's an object. 

Participant: I'm going to put the {Noun 2} in the corner which may be very limiting in a way by cornering it like that.

Interviewer: we've got those little star escapes

Participant: oh, those little stars. 

Participant: So it's sending -- those guys send back data in the form of set points. So they'll want to change a valve position, say valve 7 to 45... it will want to return stuff to them. 

Interviewer: it will or will not?

Participant: It may or may not.

Interviewer: is it relevant?

Participant: no. We can say here that this is a really good line... these guys send... it sends data and we term as real values. it also sends data as calculated values. These guys send, really they send stuff I refer to data as derived values back into {Noun 5}. Which is a bit weird. It actually was super important because these guys have a value they operate the {Noun 2} by. 

...
Participant: ... They will sit there, they'll do calculations on a computer. they'll get some -- they get a reading from the {Noun 15} and they calculate a derived value which they'll operate. This is a new derived value we've got to operate

Interviewer: so do they send data to themselves?

Participant: Wierdly enough, a key part of the project early on. They have a whole heap of information -- I feel like I'm in front of a barrister, with the terminology. These guys send information back -- these guys have information which they didn't used to send to {Noun 5}, but I needed it over here because I was telling them to do something, but they said "well dude, it changed." but I said "I don't know about it." so I went through a {noun 97} with them entering derived data back into the system so I could feed it into my system to feed it back into them... there was definitely a loop there. There used to be a bit of a disconnect, because it was mostly data coming here, data going there. These guys would just spin around themselves.... 

Participant: Let's call that {Noun 16}? they only {noun 97} {Noun 17}, I can give you an example of one. It might be nice for your thesis 

Interviewer: I would love artifacts.
...

Participant: Yeah. They will send derived data back to the daily {noun 97} {Noun 17}. They'll also send data as real values as well. But that's a dead end. That's a sheet of paper. Previously, there wasn't this link, it didn't exist before...

Interviewer: this link being the {Noun 14} to {Noun 5} derived data? You but that link in so you could reverse {position 58} what they were doing... 

Participant: Yes

Interviewer: How does the {Noun 1} get that data from {Noun 5} ... as derived data? And so, {Noun 5} also sends it real values... where is it getting the real values from?

Participant: From {Noun 5} as well.

Interviewer: Where is {Noun 5} getting the real values?

Participant: That's why we needed the {Noun 2} in here, because the {Noun 2} obviously is doing stuff and it's sending real values. The {Noun 2} doesn't really calculate anything. It sorta does it sorta doesn't. As far as I'm concerned it doesn't. But it has real values, that's just {Noun 4}, composition and things like that. And it will send that real values -- it's weird, it's where you sit. Now we're talking about the philosophy of data, where you sit and things, because -- you're right, you can pin me down and say well: "Why when it calculates the top down composition that I see that as real value data when it's not really if you think about it, because it's a instrument that's measuring atomic percents and then doing some calculations and reports its as percentage to me. And obviously, percentage is not a real value in terms of what that is, but to me, and as you flow through the chain of data, it transforms. It becomes into different forms. Maybe it's sort of related to the ontology. Well its not really, if you pin me down, maybe it's not. But it sort of is to me. It's percentage. Well, it's not real, but it becomes real. It's a Pinocchio thing. They want to be a real boy. Eventually, after a while, you get these bizarre analogies from me. After a while, once you pass it around long enough, I guess this happens -- Probably well documented, Chinese whisper style, it can become real after a while. And this stuff goes from a non-existant calculated value to something that's real, measurable, and people love it. You look at its source, and often we have these problems. If you look at the source, the source may not actually maintain the love of over here, but, as you stretch out and try to transform people's opinion of it changes. It does sort of -- it can be loved, but if you have a look at it, and say no, "that machine broke ten years ago. Why are you still looking at the value?" and they say "Well, we love that. That's a real value." It's not doing the right thing, but through transformations and {noun 97}ing through systems peoples' experience becomes real when its sort of not.

Interviewer: And by real, you mean, a representation of a real thing?

Participant: That's right. It becomes a something that we can use. It comes back to data as an object or as a piece, a discrete sort of thing which is slightly different, as I said, to information, which isn't as discrete knowledge -- which is weirdly explicit but not discrete. 

Interviewer: wait, what is explicit but not discrete?

Participant: Knowledge. I'm being probably a bit loose in my terminology. But its like -- Knowledge is explicit because I have to, in the {Noun 1}, as the {Noun 1} developer. I have to code knowledge. So it has to be explicit. I've got to tell it to do something. 

Interviewer: is the code Knowledge? Or is the Knowledge embedded in the code?

Participant: Code is Knowledge. For us, that's the case. But the reverse is obviously true. You can't have one without ... The code is the expression of the Knowledge. The code is the expression of the Knowledge that I have gained through various means. 

Interviewer: Does this system output to you besides that textual analysis?

Participant: As a {Noun 1} developer? Well, yeah... I'd draw that one in, I guess. That's a tough line to describe. This is between {Noun 5} and the the {Noun 1} developer. That line between {Noun 5} and the {Noun 1} developer -- this is where it's complex, it is there because, as part of the {Noun 1} developer, I'm interpreting.

Interviewer: we can have more than one line. It might be helpful to just do all the trivial bits and see what's left.

Participant: At the moment, there's almost -- there's a data flow diagram for how shit happens. I get this and shit goes there and Knowledge goes there. But then there is, weirdly enough, the {noun 129} {noun 97} of making such a thing has its own different data flow diagram because this is where there are layers upon layers. Because I'm looking at {Noun 5}. {Noun 5}'s feeding me stuff, and I'm feeding that. And those guys are talking to me which is the {Noun 14}s to {Noun 1} developers. But maybe we're getting ahead of myself. Because we haven't quite -- this sort of DFD that we've got here with {Noun 1} developer, {Noun 5}, {Noun 14}, {Noun 16}, {Noun 2} is describing a bit of the -- maybe a different problem that we started off with. Maybe we can label this one as "{Noun 2} operation optimization" which is really the purpose of the project and it's above. And it's obviously different from developing the {Noun 1}. Obviously, we'll find similar things. Maybe it's hard to find where to draw the lines. 

Interviewer: do we need a line there?

Participant: I'd just take the line out between {Noun 5} and {Noun 1} for the time being. Depending on the context, I guess. In {Noun 2} operation optimization, I'd just..

Interviewer: in that domain, are there any other flows?

Participant: That's what I'm trying... maybe I can get a photocopy of this and go again? And we have that and we have {Noun 5} and the {Noun 2} has {Noun 5} data and we send it to the guys. And here you have more -- we left out a lot of circles here that we really should fill in. And we'll just stick to the {Noun 2} operation optimization which is different from the {Noun 1} {noun 129}. Maybe this is a more tractable problem. And interesting, and it's something interesting -- if we can get you in the {Noun 20}. No promises there. If we can get you over that hurdle? This is sort of like their experience and how I interact with those guys. It would be good to do {Noun 1} rule {noun 129} because that's how I interact with the university people which don't -- I can't quite -- I can squeeze on here. I'm holding my hands above each other. As the layers of information. That's a column of information which then something may sit above. It's not two-dimensional as ???. Obviously me as the {Noun 1} developer is dealing with multiple spaces. Maybe I'm playing it out maybe too two dimensionally. 

Participant: So now we're going to go to. These guys talk to the {Noun 9}. 

Interviewer: these guys being?

Participant: The {Noun 14}s 

Participant: The {Noun 14} and {Noun 9} often interchange roles. But they see people with vastly different experience. But often when the {Noun 9} is away, the {Noun 14} has to step into his role in terms of {Noun 2} operation. They send each other -- this is really a tough one. They talk to each other. So I'd describe it as they do literally talk to each other. It's so hard to capture. And I had a go at trying to capture verbal feedback and things like that. These guys, they talk to each O, and they'll say: "Oh man, the {Noun 18}'s broken." "Yeah, OK, change this." And my systems blithely -- because they get a phone call "Yeah, something's going on. Yeah, backed up {Noun 19} system" "Okay, yeah, we're going to make this change." Just lost, it's out there and thing. People have sorta become accepting of the limitation of this, over here, but it's really an issue. It's a really strong data flow I guess that I'm talking to... So they talk to each other. 

Interviewer: is this talk: Data, Information, Knowledge or just talk? 

Participant: I'd describe it as just ... there are a couple of different levels, but let's say on an operation optimization ... I'd describe it as Information. But they can actually. But no, no, it's true. We can put another line in which is really they transmit Knowledge. I can call that experience. And that's absolutely true. They transmit experience in this direction sort of going out into there and often these guys [{Noun 14}] are younger guys without a high degree. Maybe they haven't been to uni. And [{Noun 9}] will be telling them experience and how to do it while interacting with my system as well. But they don't quite -- often these are. I don't know how we can... maybe I can give it [the data flow] an AA. They're almost duplicate lines, but I'd like to acknowledge them as they're different. Because there are two people in the {Noun 20}. There's the {position 8} and the {position 9}. They're both looking at my box, but they aren't separate people. They deal with things separately. It would be too trivial to combine these guys together, because they're different not from my experience. But for your purposes, I'm copying things now, like the {Noun 1} provides the same knowledge by text advice to these guys as it does to them. 

Participant: And similarly, you know, a lot of these flows where they're putting set points and getting derived data and calculated data between {Noun 5} and the {Noun 14} are the same as between {Noun 5} and these guys. Similarly, they sink data into here, and so the {Noun 9} sinks data into the {Noun 16}, the same as the {Noun 14}. And that's "real value" and "derived value." Maybe we need to get to the bottom of why that is different. 

Interviewer: Why is that different?

Participant: It's literally, again, it's a hierarchy thing I think. ...

Interviewer: {Position 8} are the boss of...

Participant: I didn't draw any information back from the {Position 9} back to these guys because it doesn't flow that way. These guys aren't going to say, ... they might say: "Should I do something?" They won't change that. While those two guys are in the room together, this is not an absolute rule of course, but organizationally, that guy won't change -- won't do {Noun 2} operation optimization while that guy's in the room. He won't change those set points while he's there. But if he's out of the room, then it's his job to step up to make those changes. This is why I draw. Well, he tells him experience and knowledge, and he talks to him about ... "and so I'm leaving and I've got to go somewhere, have a look at that, if it goes up, put a bit of {Noun 21} on. sort of stuff." it doesn't come back the other way. The guy doesn't sit there and say "ohh, I'm going to make this change without your permission, it doesn't happen like that. So then, that's probably why they're a little bit different. Organizationally, this guy is higher than that guy. He looks after the machines, equipment, primarily. This guy's after the operation. 

Interviewer: do any of these systems get Information? We've got one Information flow here. Are there Os?

Participant: Well, we have all these replicant lines between the {Noun 5} ones.

Interviewer: we've got all these data lines. And obviously these replace that... Are there any Information flows? 

Participant: Yeah, there are. That's where we have. We almost have the data flowing away from the place we started.

Interviewer: this is a very useful discussion.

Participant: So we've got the {Noun 14}s just overlaid on the top. We have the {Noun 9}s... This is why I'm interested in your topic. This is real stuff for me. Wierdly important. And it's terrifyingly difficult to wrap your mind around. Especially when someone new starts in my sort of role. And they've just got to understand all of this. 

Participant: We talked about this one, The {Noun 2}{position 22}. The {Noun 2} {Position 22}. He is sending him information. It's weirdly enough, I'd raw that distinction between stuff. These might be stuff like daily targets. I want to make so many thousands of tons of {Noun 23} today. I want to make 7 thousand tons today. 

Interviewer: I really love living in a world where you can say that...

Participant: It's true. And I wanted it to be at this {Noun 24} rate. And I wanted it to be at this chemistry. And I do draw the distinction here because...

Interviewer: so page 3 is overlaid on page 2? 

Participant: It's sort of off to the side. I'm bringing the extra characters that are probably sitting up here, so maybe I could just ... do replicant boxes so we can just join them up together later as well. So they'll send them Information as daily targets. And we've got the {Noun 1} down there. How do these guys interact with here? It's really weird to explain. And maybe I think I'm crazy. Now why did I describe that line here ... text advice and knowledge between here... We'll label that line AA. These guys influence the ability of these guys --

Interviewer: wait, the {Noun 1} is getting the AA line? 

Participant: It's just the same line. There is a line in here as well. maybe that's the best way to describe it. There's a loop in here which is -- it's not a direct line ... that's not true either. That's definitely true. He sends this -- The {Position 22} sends that Knowledge into the {Noun 1}. He doesn't send a daily target, that's going this way.

Interviewer: He doesn't send that?

Participant: He does send it. He doesn't send it to me / he doesn't send it to the {Noun 1}. He just sends it straight to the ???

Interviewer: He's sending Knowledge to the {Noun 1}?

Participant: That's right. That's an SOP, for your purpose, SOP is standard operating procedure.

Interviewer: He's sending this as code?

Participant: No, as like a written document

Interviewer: To the {Noun 1}?

Participant: That is not true. He is sending it to me as the {Noun 1} developer, -- He sends me knowledge in the form of the SOP. And I send it back as Code. So that line is the same as that line. When you put them together, you start seeing that commonality

Interviewer: That would be BB, and this would be BB.

Participant: So that's true. But the point was -- it's almost like: Can you describe the strength of data as well? As part of the philosophy?

Interviewer: Sure! but what are we talking about?

Participant: The ability for this guy to accept my Knowledge is influenced by this person. 

Interviewer: Say again?

Participant: The ability of the {Noun 14} to accept textual knowledge from the {Noun 1} is influenced by the {Noun 2} {Position 22}.

Interviewer: Where is that influence? Here? Or is that influence not on this diagram?

Participant: That's what I was trying to...

Interviewer: That's what I couldn't quite, I was drawing the lines and I got a bit technical.

Participant: Maybe, it's best to describe now in terms of: "he's sending him" as I see it, "knowledge." which is really Knowledge about how to use the {Noun 1}. Because, you remember, I'm sending it all the time. It never stops. Like the mail. It keeps coming. He's his own person as well. But this person reports sort of vaguely to that person, and this person definitely does and [Name] tells these guys -- he influences as a form of Knowledge about what to do with my system. Whether you listen to it or you don't listen to it. Whether you pay attention to it...

Interviewer: Because there are certain times when you should and certain times when you shouldn't? according to him?

Participant: According to him, yes, that's right. -- He really is sending them Knowledge. Because it, again, to me is not just about information. that Data/Information/Knowledge hierarchy -- He's telling people how to use the {Noun 1} based on his own experience, based in some sort of context. And there is a {group 130} context as well. [Name] doesn't like that, don't do that. ... There's no rational sense in anything. That's why it's Knowledge, it's not Information. Because there's a context around it. 

Participant: So that's the {Noun 2} {Position 22}. So there's a little loop there. And that's true, because they won't send me -- so the {Noun 2} {Position 22} to the {Noun 1} developer. He won't send me much else apart from the procedure. Because this guy writes a procedure and says "These guys must follow that procedure." And he's also -- we must acknowledge that he is sending them Knowledge in the form of SOP. That's the CC line, same line. I just get a copy of it... and I reinterpret it. And even send it back to them. Which is apparently one of the benefits of -- perceived benefits of these systems. Maybe I just -- I haven't got a good context of a system {noun 75} within there. Similarly and functionally, these guys don't talk to these guys because these guys report to someone else who is the {Noun 2} {position 25}. He doesn't appear in any of these diagrams. Because he's more of a <grunt> what are you doing? <Grunt> sort of relationship. But similarly, but as the chains get a little bit weaker as they are going up we have ... And here we're actually progressing weirdly enough, we seem to be progressing through the hierarchy of these organizations. So now, maybe it's best to describe it as a hierarchy. as the {Noun 2}{Position 22}, we've got the {Noun 2} {position 26}. We've got feeding through... we haven't quite covered all the boxes, but 

Interviewer: we don't need to,

Participant: I might just... because it's always about me... {Noun 1} developer here. So we've acknowledged the line between the {Noun 2} {Position 22} and the {Noun 14} which is the CC line, the DD line, the EE line. 

Interviewer: Since we're labeling, we have those three lines as a set? FF... because {Noun 5} also sends FF to the {Noun 9}...

Participant: That's right. So these guys send Knowledge to these guys -- the {Noun 2} {position 26} will send Knowledge to the {Noun 2}{position 22}. I'd say actually this is is where I don't know ... it certainly does influence the {Noun 1}. Maybe I'm just trying to go back through the chain as I see it. This is just my personal opinion. It would be great to see what they think is different. As it's getting further away from me, it's getting more chunked up and not even true anymore. Well, they'll send the Knowledge back such as like, production conditions and like monthly targets, "We want to make so much {Noun 23} month" and these guys will say "we want to make so much {Noun 23} per day." And these guys will send Information back.

Interviewer: Information or Knowledge?

Participant: I used it, and I meant it. Info such as getting pretty general there. {Noun 2} {noun 97} condition.

Interviewer: what is that?

Participant: Maybe a health? How healthy is it? Like a person. It's weirdly enough, because its... because I did say Information, because there must be something special about that. because I perceive this as -- Those guys send those guys Information. They'll say, we're going OK. And may not have that much contextualized nature. It may be a graph and they'll sit there and look at the review at last week's worth of Information as a time-series graph. They'll say, "We're hitting our targets" weirdly enough I didn't use Knowledge, I used Information. {Noun 2} {noun 97} conditions like how are we traveling? Here is some graphs which describe. Traveling is a colloquial term of how the {Noun 2} is performing. Because if the {Noun 2} is not performing well, and the {Noun 2} is down here somewhere, it can effect those guys. The production conditions and the monthly target it. Those guys might acknowledge and will here rightly acknowledge problems in production, or in the {Noun 2} will affect his monthly target. So it will be the other loop which is getting really hairy for me up to the {Noun 23} Making {position 41}, he's like "Yearly targets." we're making X million tons a year. That's sort of, we're making X million tons this year. We'll make that target. ... But I didn't put the {position 41} of {Noun 23} making on here, because he does -- This guy sends me as the {Noun 1} developer, he really sends me Knowledge in terms of like a philosophy. We have regular meetings with this guy [Name.] We'll sit down -- we had a showdown meeting just before Christmas. And as the {Noun 1} developer and him we sit down and he expounds his philosophy on what we should -- I should be doing as the {Noun 1} developer. He's definitely sending me Knowledge about what he thinks should be happening. It is a philosophy saying, "you're going about this the right way. We want to look at this aspect" of it. To be honest, I don't send anything back. It's the hierarchical nature of this ... I just take it like a man. My project got shut down over Christmas. He decided, he expounded the philosophy that we don't want to do this anymore. He sent me an e-mail one month later, saying "how are you going with the project? Are you completing that work?" I wrote back: "Currently it is not a priority and I will do it when I have time." Dude, man, you told me to stop doing it and then he's asking me why I'm not doing it? As a hierarchy, he's there and he's above me in the organization. He tells me what he wants and I'm there to implement a vision. That actually draws an upper limit boundary for us! 

Participant: Wierdly enough, if you'd describe sort of from there to there, to here, we sort of captured that -- the business chain of this? It's really the business side of things. Now, we haven't delved into the other side. It really is the operations perspective. Which is a funny one for me to do. In terms of the business , it's where the money's at. It's the important one as far as the business . I wanted to describe it how I developed the {Noun 1}. Notice I've left out the {Noun 2} {position 27}. He hasn't appeared in here because he's not part of that business chain in terms of operation optimization. But he's very important in terms of rule development. Because he'll have a big influence over how I -- on another section of the diagram which we haven't filled out which is maybe on, as you look at the page, on the right hand side with {Noun 5} at the bottom, there's a little... He's over here and there's a whole other side which we haven't dealt with which is -- and maybe it's weird that I've drawn that distinction. That I've traveled off one side of this diagram. There's a whole other side here which we haven't dealt with which is {Noun 2} {position 27} and {Position 7} including other {position 56}s. ...

Interviewer: Before we stop, can you give me three examples (not on here) of your interactions, your data interactions, your Information interactions and your Knowledge interactions with your fellow {position 56}s? Just a little fluff...

Participant: Data interactions: well I sometimes get tidbits of Information (haha). That I send onto them. I'll find out about that there's a spreadsheet somewhere that there's a spreadsheet full of data that I might choose to share with people. So these guys have. People keep their own little personal records. I've been working on a task force about {Noun 23} and {noun 28}, and there's a whole record about the {Noun 23} levels in {noun 28}, which isn't recorded in any database, but which some guy with a spreadsheet who works for another company... And he sent it to me and now I've got it. And all of us have these little bits and we share it. "Have you got any data on that? Yeah I've got a little bit of data." that's the data side of things.

Participant: Information side, We certainly share reports and things like that. It probably comes to presentations and reports and things like that. I often describe them as Information that we'll share. Someone will say look at the data about the {noun 29}. "Who will look at the data? Who will write a report?" To me that will contain Information about stuff that I can learn about. Not necessarily Knowledge, because maybe I don't have the context.

Participant: And the Knowledge component, I see it with, as my other {position 56}s, it's that true sort of sharing of experience and things like that. It won't happen in reports. It'll happen in discussions and conversation. Maybe after a presentation, someone will present some information. This is this, this is that. And then we'll sort of flip into a Knowledge sharing mode -- there will be a bit of back and forth about clarification or context which will help then build your Knowledge. Which you can go away with because Information itself is OK, it doesn't... it's for now. But Knowledge, i guess, is something that's more general. It needs a bit of context. That's true now, but if I know the context I'll be able to generalize it for the future, which then will make it Knowledge. 

Interviewer: and so you would say a report's true now?

Participant: Yeah a report's Information because it's true now. If you looked at that same report a year later, I don't necessarily know its true anymore, because conditions might have changed which aren't documented within that thing. We try to document our assumptions or conditions, but it's not terribly important to us. Mostly we work like that and we look at it and we say that that was true then. And we go on and say: "is it true now?" Sometimes we don't even do that. We have a report and "are we going to confirm that's true? No, we'll just ignore the report. It's just Information. It's not true anymore. we can just dismiss it. And we're just going to do something now."

Participant: Data would be like values and bits and pieces to share with people. Information is like "we can share reports" again derived that often... sometimes... 

Interviewer: Time series stuff?

Participant: Yeah. And then the Knowledge part we share would be the conversations, the contextual stuff: "Yeah, you need to think about this, and yeah, that was happening at the time" and things like that. 

\stopextract

\section{Interview 7}
\placefigure[]
[fig:i2]
{The SDFN Diagram for Interview 7}
{\externalfigure[Chapter4/graphs/i7.pdf][factor=fit,frame=on]}
\placefigure[]
[fig:i2]
{The SDFN Diagram for Interview 7}
{\externalfigure[Chapter4/graphs/i7-2.pdf][factor=fit,frame=on]}

\startextract
Interviewer: ... I think a good way to segue into the entity diagram is to tell me what project you think you handle -- that are data-important or that are knowledge important.

Participant: Do you want me to distinguish the two?

Interviewer: Feel free to...

Participant: Clearly data is important -- not clearly. Data is important in a project such as the one that [Name] is working on. Data is also important in a project such as [Name] is working on. In both cases I’m not talking about any project I’m involved with. Data is also important in a project that I’m involved with which is related to changing the {Noun 30} {Noun 21} blend in the {noun 60}. Now you asked about knowledge, and information, as well?

Participant: Information is important in each of those. In my definition of knowledge. 

Interviewer: and knowledge as well?

Participant: Oh yeah, yeah.

Interviewer: So let’s do the one you’re involved in to start with, and if we have time, lets do your view and your role in the one that’s [name]’s involved in. But let’s do the one you’re involved in. And you mentioned a mathematical model.

Participant: That’s what I was involved with, yeah. It won’t impact on this one though. 

Interviewer: Hopefully if we have time after the modelling, I’d love to talk about how you use data to inform and to create and then to use inside the mathematical model. Just to have that simmering in the background. Let’s start with an entity dictionary.

Interviewer: How would you characterize yourself? Here, what we’re looking at is not just you as you, but each of the roles you’re in. So, for example, for me, at the university, one of my roles would be {noun 91}. student, and that would be separate from teaching assistant. So that’s the kind of roles that I think are best to start with when we’re working on our entity dictionary.
...

Interviewer: We begin by looking at a trivial flow between something and something else in the project you’re participating in.

Participant: The project involves {noun 35} with an external organization. So we have an entity called the {noun 36}. I can name it if it helps you.

Interviewer: Not really. So, what stuff is flowing to and from these guys?

Participant: {noun 37}. {Noun 21}.

...

Participant: So that’s a physical {noun 37}. They also send {noun 37} characteristics, so that’s data. {noun 37} characteristics or properties. Chemical properties if you want.

Interviewer: would it be better to say {noun 37} characteristics?

Participant: Characteristics is probably sufficient.

Interviewer: This is Data?

Participant: That’s Data. 

Interviewer: Would you characterize the physical {noun 37} as Data, Information, Knowledge, or just?

Participant: It’s the means, really. It’s not data. It’s not Knowledge. They’re just physical {noun 37}.

Interviewer: So we have our initial flows. What other flows are there?

Participant: So here’s the {position 32} over here, that’s me. We haven’t split -- this is the research organization, so there are people within this organization -- well, there’s probably one person if you want who’s the {position 100} here. So you could actually say “{position 40}” So, flow from here to here is the {noun 42} experiment. Over here we have {position 39} who help to define the {noun 42}.

Interviewer: so what do they provide?

Participant: They provide information to assist in {noun 95}ing the {noun 42}.

Interviewer: So the {noun 42} is information? or Knowledge?

Participant: It’s information. It’s information provided to the {position 41} here to allow that person to undertake test work of a specific nature. It’s guidance into what it is they’re going to test. We’ve sent them {noun 37} and a {noun 42}.

Interviewer: How can we label this flow from {position 39} to you of information? So you’re sending them a {noun 42}, and the {position 39} are sending you Information to assist in the {noun 42}. Do we label it as Information to assist in {noun 42}? 

Participant: I think information to help {noun 95} {noun 42}. 

Interviewer: What else?

Participant: Within here there’s also some work going on in terms of communication between this guy and their team. {position 43}. Information, {noun 37}, across there. Information, 

Interviewer: Information of the {noun 42}? Is this modulated by the {position 41}? Or just distributed?

Participant: It’s likely to be modulated. They’re likely to have their own standard operating procedures. 

Interviewer: How do we want to label this differently than {noun 42} to indicate that modulation?

Participant: Along the lines of {noun 42} conducted according to SOPs. 

Interviewer: And they’re sending Information, and they’re also sending {noun 37}? Which are just stuff?

Participant: a lot of test work goes on in here. A lot of analysis goes on. 

Interviewer: Do you know or care about the instrumentation they use for analysis or {verb 73}? Or is it just a blackbox thing?

Participant: We do care. In part, that’s covered in the {noun 42}. Is the question about: do we care about what sort of instrument they are using or the type of tests that they are doing?

Interviewer: Both. Because. We can treat this as a black box. And stuff will flow from this black box to these. Or we can treat this as part of your understanding of the project, at which point we’re going to label their external manipulations of your {noun 37} according to the SOP. 

Participant: I’ll explore one area. So these are physical {noun 37}. They actually have to manipulate those physical {noun 37}. They actually have to take and crush those {noun 37} and prepare the physical {noun 37}. These are physical {noun 37} from our {noun 60}... In order for the experimental team to do their experiments, they need to actually do some further ... and classification of the {noun 37}. The way they do that depends on the {noun 42}. 

Interviewer: So, is there an entity -- some technological device that is a useful catch-all or a useful... 

Participant: Let’s call it a properly sized {Noun 21} {noun 37}

Interviewer: So they’re sending properly sized {Noun 21} {noun 37}?

Participant: No. They receive the physical {noun 37}, then they have to generate this properly sized {Noun 21} {noun 37}. So they have to do some crushing and sizing.

Interviewer: So this would be a {noun 44}? So this goes physical {noun 37} in, and out comes properly sized and crushed... where does that go? 

Participant: It goes into an experimental {noun 45}. Crushed and sized {Noun 21}.

Interviewer: Are there any stuff flows, Information, Data, going along this route?

Participant: No, because it’s being done by this group.

Interviewer: So they’re not setting any settings on the {noun 44}.

Participant: Yes they are. That setting will be done to be consistent with the {noun 42} that’s been provided.

Interviewer: Are they setting the setting? Here’s the tricky bit. That setting of the setting, i.e. the transmission of the settings into the machine, is that Information or is that Data? Or is that something else?

Participant: Well, it’s data from us, but it’s Information into this machine, isn’t it? We tell them -- this was Information... Information can be Data as well, right? You don’t want so say?

Interviewer: The thing is, I can’t tell you what you think. 

Participant: The {noun 42} has Data on the size distribution. So that’s a set size distribution. So that’s Data that we’ve provided the people so that they’re informed so as to how to set. 

Interviewer: So, in the Information of the {noun 42}, which is compiled of multiple protocols

Participant: this is one of them

Interviewer: This is one of them, and you would say this individual protocol is Data? So in this packet of Information it goes through the {position 41}, the {position 41} adds SOP, and that packet plus SOP is still Information. Do they perform something interesting to this packet to extract out -- or can it even be extracted out -- the settings for this {noun 44}?

Participant: No, they just take Information and make the setting. 

Interviewer: So that would be transmitting information of size settings. -- Now, they’ve got an experimental {noun 45}. How are they interacting with the {noun 45}?

Participant: They’re running the {noun 45}.

Interviewer: Are they transmitting stuff to the {noun 45}? or is it 

Participant: Are they transmitting stuff?

Interviewer: Are they transmitting components of the tech plan to the {noun 45}?

Participant: No, because that’s what the SOP is about. The SOP guides them as to how they should run that {noun 45}. 

Interviewer: So there’s no data flow or Information flow into the {noun 45}.

Participant: No. 

Interviewer: What happens from the {noun 45}?

Participant: Produce new {noun 37} of {noun 46} which need to be analyzed. It goes to a Laboratory. 

Interviewer: And we’ve got new {noun 37} of {noun 46} to the lab: Who follow standard procedures as transmitted from him or them?

Participant: Probably by him.

Interviewer: So we’ve got SOP. And these SOP are Knowledge, Information...

Participant: Also, according to the {noun 42}

Interviewer: but it’s not that? Is it this flow?

Participant: In the {noun 42}, there is Information as to what lab tests are to be done. It’s like that. You could take it out of there if you wanted to?

Interviewer: What reflects reality best?

Participant: The reality is that it probably... as well as SOP, it would be {noun 42}. In fact, it’s a bit similar to what you’ve got there. {noun 42} conducted...

Interviewer: Is the {noun 42} conducted also Information when it’s sent over here? So {noun 42} conducted according to SOP goes to the lab along with {noun 46} {noun 37} which have been prepared through all that. What happens then?

Participant: Then Data from the lab comes back through to this person here. 

Interviewer: So this is Data. What is this Data?

Participant: The Data is the results of laboratory testing of {noun 46} {noun 37}. I’m being Generic, if that helps you. I could be more specific about what exactly the Data is, -- I’m happy to do that.

Interviewer: I would love an artifact... like a {verb 73} of the results? But unless you think that it explains what you think Data is more, I don’t think we need to go into any more detail.

Participant: I think the emphasis is that it is a result from the test.

Interviewer: Because that’s the second time you’ve used Data in this, which is why it is significant.

Participant: So this person prepares reports, and then the reports come back to the {position 32}.

Interviewer: so you get reports on the results

Participant: Correct. Via e-mail. 

Interviewer: is this significant?

Participant: No. 

Interviewer: It’s interesting. Some people would say that e-mail is significant and that it actually changes the way the other people go --it’s e-mail. So reports on lab results. Is this Data Information Knowledge, O?

Participant: At that point it’s Data. They are providing us Data. They haven’t made any -- Even in context of comparing other results, they would have compared Data from previous experiments that they’ve done with ours. 

Interviewer: And are these comparisons also Data, or are they privileged in some way?

Participant: I would have thought that would have been Information which they are not privy to disclosing to us anyway. As far as I’m concerned... we get the Data back.

Interviewer: Are there any other entities or steps we haven’t described?

Participant: Here there’s another one. Then we have the discussion internally, with {position 39}. Obviously, we each, individually analyze the data ourselves and then we do an excel spreadsheet analysis, if you want. 

Interviewer: Let’s stop there, let’s diagram that. How can we diagram that? We’ve got this line going to the {position 39}, what’s this first -- do you perform any operations on these reports before sending them to the {position 39}?

Participant: I do. I will probably summarize key points in an excel spreadsheet. Differences between successive tests. 

Interviewer: What do we want to label that, and is that Data as well?

Participant: It’s certainly data. “Results Analysis” Summary, I guess, of results analysis -- Results Summary. I guess I use the word Analysis there as well. ... and Analysis.

Interviewer: Then what happens?

Participant: Then we have our discussion.

Interviewer: How can we represent this discussion? Should we represent this discussion?

Participant: No. There’s all sorts of discussions going on elsewhere as well, but we haven’t presented that necessarily. You can leave it as that. And then, there’s a bit of recycle loop. This whole {noun 97} then out of discussion are conclusions based on the current set of results and then we may well review the {noun 42} and start this all again. 

Interviewer: So they send conclusions to you?

Participant: No no. 

Interviewer: you all generate conclusions

Participant: Yeah.

Interviewer: How do we want to represent these conclusions, or do we? Or is that stream the conclusions, from the earlier iteration?

Participant: Yeah, that’s fine. This was specific to the {noun 42}.

Interviewer: So we add another stream going here...

Participant: well, if you want, you can leave it as it is, then this discussion. The entity is that the {position 39} -- the {position 39} and myself get together -- therefore that stream ....

Interviewer: so we have {position 32} sends {position 39} and {position 32} a stream of Data which is results summary and analysis. 

Participant: Therefore, we can go from here with an updated {noun 42}. From then on we can follow the loop.

Interviewer: And the updated {noun 42} is?

Participant: It’s Data, Sorry it’s Information. I mean, It is both, right? I’m providing data, but it’s primarily information.

Interviewer: Unpack that for me.

Participant: So I’m providing numbers. So the numbers, in isolation, mean virtually nothing. So attached to the data is some Information. Particularly in this feedback loop, I would present, we said we were going to crush the {Noun 21}s to this specification, but based on the current results, we’ll change it, so here’s a new setting which we still refer to as Information anyway. 

Interviewer: is that correct?

Participant: I think so. In making that judgement, or in providing that Data, we are informing them of why we are changing the specifications, changing that Data. 

Interviewer: In a sense, this size setting is composed of numbers plus ... explanation?

Participant: Almost Intellectual Property, some explanation.

Interviewer: IP is a good word. We’ve got numbers plus IP. And you’d describe the numbers as?

Participant: As the data.

Interviewer: And you’d describe the IP as the Information. And that entire packet of data + Information, of numbers + IP, that entire packet is also Information.

Interviewer: Where does Knowledge happen, is there any Knowledge transmission in this whole {noun 97}?

Participant: Yeah. The Knowledge transmission probably takes place in this discussion.

Interviewer: in the discussion of the {position 39} and the {position 32}, what -- where’s the Knowledge interaction there?

Participant: Where is it, or how does it...?

Interviewer: All of the above.

Participant: So it’s a meaning??? Data’s on the table. Information is exchanged: what do I know, what do you know? What does the Data, what do the results mean? 

Interviewer: And that meaning of results is... Information?

Participant: The meaning of results is Information. But, that Information may be new. Or that’s an unexpected result. There is potential for Knowledge, for new Knowledge to be generated. 

Interviewer: And the new Knowledge which is generated comes from

Participant: From the analysis of the results and the discussion of the results and the implications that are derived from the analysis of those results. Am I confusing you?

Participant: Data, Information, and Knowledge. In my books, you analyze data. And generate new Information. New Information can generate a new... So, I have a model in my head, and this is what a {noun 46} should be. And that model is Knowledge. That model could be shifted slightly. A shift in some model (physical or whatever) is due to new Knowledge. 

Participant: Within that discussion

Interviewer: we have that iteration. I need to figure out how to represent that iteration. In here we’ve got Knowledge being generated, does it go anywhere?

Participant: For a start, it comes back -- the Knowledge is then transmitted back via information to these guys. We won’t provide that to them. There may be a little bit, but you’re not explicitly... It’s in the form of Information returned to the external organization via an external {noun 42}

Interviewer: By them analyzing the updated {noun 42} they see their own new Information, which means they can generate their own Knowledge? 

Participant: No, I don’t think they’d do that necessarily. The distinction here is these are research {position 41}... they’re an external service provider. If you have a contract for wanting you to do some work on your home for example, a plumber, you won’t necessarily tell them about what it is that you’re going to do with this device, you just tell them I want that fixed, and I want that service provided. and that’s whats happening here. We’re asking them to perform tests and asking them to provide us Data. We’ll assess the data, there’s no interaction with the {position 43}. They’re just providing that service.

Interviewer: Therefore do you sink your Knowledge somewhere else here so that other people can take it? 

Participant: Yes.

Interviewer: So let’s name that. 

Participant: We’ve named these {position 39} but we could probably name them more specifically as {position 47} because over here because the next offshoot here, the other entity is the {position 48}. 

Interviewer: You, to them, send...

Participant: Information. We provide them with Information, but they might ask for explanation as well, so therefore we give them Knowledge. 

Interviewer: First step is you’re sending them Information. You’re sending them what Information?

Participant: It will be something -- a summary of the summary that has just been produced. It could be a presentation. Executive summary if you want.

Participant: now, I put in this category, generally speaking what happens is when you get to this point you will have different levels within the organization. For example, you might have a general {position 41} of the operations, a {position 41} of the operations, and then a... So what we refer to as level 4, level 3 and level 2 {position 41}s. Let’s call them {position 41}s for a moment, I’ve clumped them all together under {position 49}. 

Interviewer: So you send to these level 4,3,2 {position 41}s... 

Participant: We would probably send them Information first.The alternative ... we may send them Information. More than likely we’ll sit down with them and present the Information.

Interviewer: So there’s some communication of Information, and then what?

Participant: Then that Information is discussed in a meeting, not that meeting, with these guys. And decisions are made as a result. The decision might be to continue the {noun 35}, continue the {noun 42}

b; So they send back to you ...

Participant: Remember I said we’d probably do it as a meeting

Interviewer: so this meeting sends back to you... 

Participant: In here {position 49} + {position 47}

Interviewer: and this dovetailed flow goes back to?

Participant: Ourselves.

Interviewer: And this dovetailed flow is the recommendations?

Participant: Recommendations, yeah.

Interviewer: And these recommendations are?

Participant: Recommendations of Information via definition...

Interviewer: So these recommendations are information. Do you transmit Knowledge from this meeting -- the {position 47} and yourself to the {position 49}? How can we render this Knowledge transfer? So you’re sending Knowledge...

Participant: It’s that IP. It’s like this. You say here is new Information, here is what we interpret as the new Knowledge, we’re providing that new Knowledge. 

Interviewer: So we could say that this is the IP... What do they do with the executive summary plus IP besides send back recommendations to you regarding the experiment?

Participant: So these guys in this project would then -- these are {position 49} people, they would communicate that Information, these people may well be in the same meeting, by the way, but let’s assume they’re not, to {noun 51} department.

Participant: Because what this is about is the purchasing of alternate {Noun 21} {noun 51}.

Interviewer: so {position 49} are sending Information

Participant: Information? So this would be a recommendation to {verb 52} another {Noun 21}. It is Information. And these guys would then talk to by phone or whatever, transmit in some way, that information to the {Noun 21} {noun 51}ers. {verb 52} and negotiate {noun 53} and that sort of thing.

Interviewer: and that’s just an Information flow? What do they do?

Participant: They will send stuff back.

Interviewer: Do you care about the stuff they send back?

Participant: Yes, we do. They will send specifications back to these guys here. 

Interviewer: And these specifications?

Participant: {Noun 21} specifications

Interviewer: and these {Noun 21} specifications are?

Participant: Data.

Interviewer: And the {noun 51} department then

Participant: Then they would provide that {noun 51} to the {position 49}

Interviewer: Are they modulating this or are they just passing it on?

Participant: They’re just passing it on. And that’s where it sits, because these are the guys that actually...

Interviewer: What else? Are there any other flows in this {noun 97}?

Participant: There are only other flows over here, but they’re outside... Probably there is, but that looks pretty good for me. 

...

Interviewer: This diagram is your how does one turn into the O, which means that you’re stating that one can turn into the other. Which is important because not everyone believes that. 

Participant: Mind you, having gone through that, I was thinking all along: this is no different to any other activity you might end up doing anyway. The fact is that there is an {position 43}, but in the end it could be anybody.

Interviewer: It’s just someone who applies the SOP. they’re a clearinghouse.

...

Interviewer: Let’s spend this time chatting about your philosophies of Data, Information, and Knowledge. Let’s begin by going: “Can you give me your own definition of these three terms?”

Participant: Of Data? I guess I was sort of doing it here, in a way. It’s what you do the Data... What is Data there? Data in itself, if you look at numbers on a sheet of paper, multidimensional, these are numbers in this case. An example using just numbers. Looking at that assemblage of Data points, it’s hard to come up with some mental model. Particularly if it’s multifactorial. 

Interviewer: So one of these is a data point?

Participant: All of them are

Interviewer: Each of them is a data point and collectively they are data?

Participant: Yeah. But it’s the interpretation of the data which is required. That can be done by your own mind, but it can be done in a more systematic way: steadfastly, there is tools. You have excel tools, excel software. You have other statistical analysis tools. 

Interviewer: So data + analysis through tools creates Information?

Participant: Creates Information, yeah. Ultimately, where this is going to end up is, I’ve got a model in my head: this causes that, for example. I’m looking at the Data and I’m analyzing the Data, and I want to understand whether the Data indicates that that is true or not. That there is a cause and an effect, that there is is a...

Interviewer: So the Data is informing Knowledge of causal relationships in your model?

Participant: Yeah. One of these may be a dependent variable, and the rest might be independent variables. I’m trying to understand if these dependencies are there?

Interviewer: And this understanding is Knowledge?

Participant: That’s part of the Knowledge, yes. 

Interviewer: And so we’ve got analysis... and so the analyzed Information in this... Where does Information fall in this causal... ?

Participant: The Information is almost an extraction ... Data + analysis provides Information. and Information then can be used to generate new Knowledge. 

Interviewer: Information + what? Is there a plus?

Participant: Well, I was talking about models here, so Information + model....

Interviewer: And a model is?

Participant: A model could be anything. It could be an equation, it could be a physical model.

Interviewer: and the class the model falls in is Information or is it Knowledge?

Participant: It’s probably old Knowledge in this definition. 

Interviewer: And so, in this definition, Information which is the combined Data + analysis of these data points, which are numbers.

Participant: They could be numbers they could be... You could also.... Data + analysis, but over here it could be Contextual Information. There may be things around... you could have a set of data, two sets of data, and the difference between those two sets of data is the constraints that were imposed by {group 130} or by some other entity, and that’s the explanation. Data is first, then constraints, analysis, and so on. 

Interviewer: And these constraints and this analysis are? Analysis is really a verb. These constraints and analysis are data?

Participant: No, constraints they’re... other data. I jumped to that because you were talking about the data points. So that’s just the data points. The constraints are, you’ve got this objective function, there are relationships that define X = Y, help to define that. But then bounding that are constraints. Maybe a way of thinking about it is: if you’re trying to optimize production rates or something like that, you could say OK, production rate = A+BC, but you know that C is bounded by something else, by Data. Therefore, it isn’t just the data, it’s Data + Constraints.

Interviewer: And these constraints are other data. You mentioned a very important word to me: relationships. Relationships are?

Participant: A relationship may such as that?

Interviewer: So this function is a relationship?

Participant: We call it ... you’re distinguishing mathematically now?

Interviewer: What I’m trying to get at: do you classify a relationship as Data, Information, or Knowledge?

Participant: I think a relationship, like that, is probably Knowledge. 

Interviewer: And that would be the model, basically.

Participant: Yes. 

Interviewer: you mentioned context?

Participant: So the context is that you could have a relationship like that unbounded, but the context is that’s knowing what the constraints are.

Interviewer: So context is the application of constraints. 

Participant: Constraints, yeah. That’s certainly one way to interpret context in this example.

Interviewer: This application of constraints as context is also Knowledge? Or is it...

Participant: The application of constraints? I think that’s probably Information. It’s either Data or Information. It’s not Knowledge.

Interviewer: Can we have arrows going the other direction here? We have this new Knowledge from this analysis {noun 97}. What happens then?

Participant: What happens back this way? New Knowledge is always, or should be, communicated and then -- on the practical side of it is: communicated, argued, and then agreed or discarded.

Interviewer: So you’ve got the Knowledge which is bound into or discarded from your knowledge-of-world, since we can’t have impressive German terms.... At some point in this chain, you issue normative orders: “You should do this.” Those normative orders, here, are function of your {noun 42}? 

Participant: They’re represented by the {noun 42}, and they’re represented by that Information. 

Interviewer: Your classification of these orders, these “You should do something” is it part of this model, or is it part of a different model?

Participant: I see what you’re getting back to this... which is a flat... I think it’s, in practice, I think it’s separate. particularly in practice --

Participant: Let’s say we’ve got scientific or technical Knowledge. And think of it back over here, it’s sort of like technical Knowledge is sort of in here. 

Interviewer: Technical Knowledge is where?

Participant: See, you have learnt new things. Sorry, these are results plus summary and analysis. You’re generating, performing

Interviewer: You’re performing the data + analysis into Information into Knowledge. 

Participant: So, now, it’s expressed here as IP. I guess in that case it is coming back up, it’s Information. So, in effect, we described it as Information.

Interviewer: We’ve described that Knowledge generation {noun 97}...

Participant: has provided new Information to be provided to the {noun 60} {position 90}s.

Interviewer: and they can then take that information and put it into their own cycle?

Participant: Yeah. They can. In fact I’m sure it happens like that, too. The reason I’m ??? is that there’s probably a filter here before it gets to {noun 60} {position 90}s. 

Interviewer: So we’ve got Technical Knowledge, and this Technical Knowledge is this IP? No. 

Participant: It’s that IP. Put through a filter, so that these people will understand better. 

Interviewer: This filter is?

Participant: What it is is the executive summary. What becomes is -- that’s the executive summary what it is over here a detailed summary, if you want, but it’s more than that.

Interviewer: Detailed summary and stuff. And then it’s winnowed into this executive summary.

Participant: It’s sort of like the questions you want -- you’ll have some questions in your mind that these people will want answered. so they’ve got objectives: do I, or don’t I change {Noun 21} {noun 51}? That’s the question. You’re anticipating that question, so therefore you generate a summary of or a distillation of this detailed stuff to present to them to address that. 

Interviewer: that distillation is?

Participant: Is -- over here we’ve got information. And we’ve said IP, in part. but over here the distillation is the application of -- no, it’s quite separate. So you’ve got new Knowledge, which is then put through a filter which is -- its nothing to do with the technical. it’s more about, let’s call it, a social. It’s non-technical. It’s about how to explain to you what I know? So I will try to put it into ... I’ll try to work out what you’re likely to understand, so I’m thinking: “Well, he’s not a social scientist, he’s a IST person, so therefore he’ll understand some {noun 109}. So I’ll put some maths or a relationship there. But if you were a salesperson with a marketing degree, probably forget about that.” The filter is a {verb 112} of -- it’s a device that permits more effective transmission of information between one entity and another. And in this case it’s probably a filter around “What will that person understand?”

Interviewer: I think one way of talking about that filter is a local language. That we evolve that we know, that when we’re speaking to each other that we have these specific terms that are locally true but not globally true.

Participant: It’s certainly jargon, if you want to put it that way. It’s more than that, it’s... the way that Information can be communicated is based on level of people. Meaning time, availability. But it’s also, what I was saying earlier, about Knowledge. 

Interviewer: The Knowledge of the listener?

Participant: Of the listener. And their level, and Knowledge of the listener which includes their understanding level. The other aspect is just they usually. It’s synthesis. It’s the synthesizing of all the Information. It’s the bottom line. If this, this, and this, are all true, what does it mean? What’s the bottom line? It’s being able to: that executive summary is being able to go that plus that plus that to get to the bottom line. This is the ultimate result. The ultimate piece of information that they require. 

Interviewer: And that executive summary, as Information, is functionally a normative recommendation to change {Noun 21} {noun 51}ers or not to change {Noun 21} {noun 51}ers.

Participant: Yeah, it’s a recommendation as opposed to a decision. 

Interviewer: That distinction is important.

Participant: This doesn’t necessarily provide -- that’s Knowledge, because it’s the IP, but the executive summary is Information via recommendation. But that’s really a decision.

Interviewer: And that decision is an order to change or nothing’s sent.

Participant: That’s right.

Interviewer: and you would say that this decision is Information in the same style that the executive summary is Information or the {noun 42} is Information. Or is the difference sort of Information?

Participant: Well, the decision is almost really Data? In a way. It’s Information, but it’s also ... no, it’s Information. I was thinking whether or not it was Data. you’re just saying, it’s yes or no. In that context “Do I or don’t I?” and you’re saying “Do”, to the {noun 51} person. It’s 1, go. 

Interviewer: And that Boolean decision is still Information?

Participant: I think that’s Data, that’s what I’m saying. 

Interviewer: Perhaps this decision ...

Participant: It’s informing the person to do something like the {noun 51}er to do it

Interviewer: It’s a container for Data? So, perhaps you’ve got the D: 1/0 change, don’t change. But it’s transmitted in some sort of information?

Participant: Yes, yeah. It’s yes or no, but behind the yes or no is this other Information. It’s what is supporting the yes or the no. What Information is supporting that decision. A yes or a no.

Interviewer: In this other context, In order to undertake something new , say you’re ordering someone to go do that research. An order is an example or instance of what?

Participant: An order is -- you’ve made a decision

Interviewer: and you’re transmitting that decision to me as the orderee. And that transmission of decision is? Is it on this line?

Participant: The transmission of the decision is there. It’s Data.

Interviewer: Because it boils down to a 1 or a 0?

Participant: Yeah. 

Interviewer: If you told me, Brian, go research that new IST system for CRM, that is also Data? 

Participant: Yeah.

Interviewer: The go research component is? Constraint?

Participant: Yeah. It’s probably Information. You’re saying context of... It has to be Information, surely. 

Interviewer: You have a pretty good sense of what I’m trying to tease out now. Do you have any parting words on ... conclusions I should come to in this analysis that might not be obvious from the diagrams

Participant: I’m not sure... I think the discussion about Data which is what you said there anyway, the Data, Information, Knowledge it’s quite confronting in a way for me, because I ... but I think it’s worthwhile for you to pursue it. You’ve probably held back your own views about what is and what isn’t data. And what is and what isn’t Information. I’d - I’m not sure. They are distinct. To a degree. At times, they will overlap. I think what I’ve been struggling with, through this interview if you want, is how to separate them. To make them more distinct. But maybe there isn’t a need for that. Maybe there are different flows here that are both Information and Data, and both Information and Knowledge. 

Interviewer: Thank you for this time.

Participant: It was a pleasure, and I say that sincerely, actually. What I do appreciate is being challenged. Not necessarily by Brian, but what he was trying to do.

\stopextract

\section{Interview 8}
\placefigure[]
[fig:i2]
{The SDFN Diagram for Interview 8}
{\externalfigure[Chapter4/graphs/i8.pdf][factor=fit,frame=on]}

\startextract
Interviewer: Let us do the training. Because we’re short on time, I’m going to skip making an entity dictionary. Functionally, an entity is any person or thing or {noun 97} that can manipulate Data, Information, and Knowledge in some interesting way. You, yourself, can be multiple entities depending on what your role is. When I’m at university for example, I can have the role of {noun 91} researcher, but I can also have the role of Teaching Assistant. And they’re different. My computer doesn’t necessarily have a role. If it’s acting like the pen in my hand, I don’t think about the pen I’m using, I just write. But if I suddenly, go “wait a minute, no, the pen is important because it’s doing some sort of transformation of what I’m doing, then suddenly it’s an entity.” We’ll start with you diagramming a role that you play in this {noun 97}. And then we’ll go from there to just a trivial flow of Data, Information, or Knowledge, and just build out. What would be a simple role that you play?

Participant: One is, of course, if you count the training, One would be putting together the material.

Interviewer: What could we label that role as?

Participant: I suppose its ... author? I’m lecturing.

Interviewer: We can call it author. Perhaps lecture author?

Participant: Everyone that has some training, they don’t go and get a book. We should sometimes, it would be a lot easier and take less time. It seems like everyone, every time there’s some training going on you start thinking about ‘what are we supposed to be telling them?’ what subject? What do I feel is important from a theoretical view? And also what’s important to the {noun 60}? You have to look at both sides. From that I produce a lecturing material.

Interviewer: Let’s model that. So we’ve got Lecture Author... now, you as Lecture Author, you tell me that you’re producing lecture material. Let’s start with ...

Participant: I usually base -- I use Data a lot. Because I start with a theory and then I actually use {noun 60} Data -- that theory is not just something I came up with...

Interviewer: Where do you get Data from?

Participant: From different production DBs so they’re collecting --

Interviewer: Production DBs? So we’ll call ... production...

Participant: You’re sort of really looking at numbers.

Interviewer: Are we looking at numbers?

Participant: No, we’re also looking at relationships between Data. Which in the theoretical case would be coming from books. Fundamental chemistry, I suppose. Characteristics.

Interviewer: So we’ve got the production DB and we’ve got books. What do you communicate to the production DB to get Data back? Or do you?

Participant: I just do the ordinary thing. I just use our tools to extract our Data. Manually we sort of -- that’s what we have to do all the time. We don’t have any kind of simple statistics. We normally don’t have any problems. We just have them SQL script, and just extract with whatever tools our IST department {noun 51}. And it could be different programs for different DBs, because in a {noun 60}, we have sort of a mix of different types of DBs from old to new and depending on system levels too. We normally work with 3 levels. One planning level, {noun 88}, and we certainly have the level 1 which is the {noun 60} sort of system, whatever that controls the {noun 60}. Level 1, and 2 is more ... calculation. And 3 is the planning. So you might get Data from all of them. And they all have different DBs.

Interviewer: Should we generalize your request to those DBs? Is it possible to put one request to any of those databases under one heading?

Participant: Yes, I suppose it is. We call that fact-finding or something. I wouldn’t sort of treat them differently. Because they’re sort of just numbers anyways.

Interviewer: This flow of fact-finding to the DBs, is it Data, Information, Knowledge or O? 

Participant: It’s just Data, numbers, from measurements. 

Interviewer: So you, to the DBs, are sending Data? 

Participant: No, I’m collecting Data. The equipment is...

Interviewer: We’ve got a flow from you to the DBs of Fact finding. What is that flow? Is it Data, Is it Information, is it Knowledge, or is it something else?

Participant: I’d just ask... 

Interviewer: So you asking to retrieve -- are you sending the computers, Data, Information, Knowledge, or something else? Or nothing?

Participant: Or nothing. no, no, I just ask for -- I want a list of numbers.

Interviewer: So that statement of list, you don’t categorize as Data, Information, or Knowledge?

Participant: No, I haven’t. But I suppose it is Knowledge, because I put together the question encoded to my Knowledge. I would be sending that Knowledge, even if the DB doesn’t acknowledge it.

Interviewer: You are sending it Knowledge because it’s a representation of your Knowledge?

Participant: Yeah. I choose, I suppose, And expand beyond that, to extract the Data, think on a little bit more would be someone else doing the same job might extract some other Data

Interviewer: because they don’t have the same Knowledge?

Participant: Yeah. 

Interviewer: Now, you send a Knowledge flow of fact-finding to the DBs. What do you get back?

Participant: I would get back numbers, I suppose? Again the facts. Hopefully the facts and not fiction. I wouldn’t know. I have to examine the Data. And that’s sort of -- when I retrieve the Data, you go through the Data, and you sort of quality control, I would think. Based on my Knowledge. That’s no fun. The scrutiny of Data.

Interviewer: Let’s set that aside for a second. You get back from the production DB Data, yes? It’s not Information or Knowledge, right? 

Participant: No, it’s just Data. 

Interviewer: The Data you get back, just like you’re sending fact-finding flow. What do you call a flow you get back from a DB?

Participant: I suppose it’s some Knowledge too, I suppose. Or results or ... It would be results. It’s the result of something. 

Interviewer: Of your fact-finding

Participant: Yeah.

Interviewer: This is what we’re looking at. You as your lecture author, send fact-finding Knowledge flow to the production DB. They respond with a Data flow of results. 

Participant: Yep. 

Interviewer: Now you mentioned books. How can we build something like that to books?

Participant: The books would have a big part in the fact finding. In what Knowledge I would sort of send... 

Interviewer: The first question is, if we’re treating a set of books as an entity, is there a flow from you to the books?

Participant: At this point, I couldn’t think of one. No, I wouldn’t be able to tell the books anything.

Interviewer: Would you consider the selection of a particular book from a set of books a sense of communication from you to the entity. Or is it just access which isn’t communication?

Participant: It’s just access.

Interviewer: From the books to you, however, is there a flow?

Participant: Of Knowledge. A Knowledge flow. 

Interviewer: so there’s a Knowledge flow. This Knowledge flow, what can we label?

Participant: Theories. And I suppose Theories and experiences. Depending on what...

Interviewer: so the books can provide a Knowledge flow of experiences?

Participant: Yeah.

Interviewer: Is it separate from the Knowledge flow of theories?

Participant: No, it would be probably one. I think books would be either/or. If it’s not something published where somebody did investigation. But if you look at printed books, they’re either theories or both. Even in the books, they would use sometimes examples based on production data. 

Interviewer: You as lecture author use books and the production database to generate your lecture. Do you use anything else to generate your lecture?

Participant: It would be more illustrations.

Interviewer: Where do you get the illustrations from?

Participant: It could be drawings... or if its equipment, that would be from our {noun 89}. 

Interviewer: What entity should we label it as, {noun 89}?

Participant: Yeah. 

Interviewer: now this {noun 89}, do you send it anything as part of this {noun 97}?

Participant: Not more than a request for a certain drawing. I just usually go to someone who has to print out the... I would ask... 

Interviewer: What do you do? You ask the archive...

Participant: for a particular text. 

Interviewer: how can we label this flow?

Participant: Drawing Request.

Interviewer: This drawing request, is it Data, Information, Knowledge, or O?

Participant: O, I suppose. Or it’s the same with the Data. We’ll ask for something specific because of what I want -- I suppose it’s a little bit of Knowledge in that. It’s based on what I want to show so it might just be -- I would relay when my word?? comes back, it’s more like Information. 

Interviewer: So you send a drawing request, which is an expression of expertise? i.e. Knowledge? Or is it something else?

Participant: Something else. I mean, I don’t know anything about it. 

Interviewer: you just say I want such and such. 

Participant: Because I want just show something practical or have need or practical use of it together with some other data or some event or whatever.

Interviewer: So that request, if we’re putting it in the other category, I need to label it something. What would you label it as, category wise?

Participant: So you have Knowledge,

Interviewer: Information, and Data.

Participant: I think I’m sending it Knowledge, I suppose, because I know something. I have Knowledge about something, that’s why I need this particular drawing.

Interviewer: Because you have Knowledge, you’re sending that Knowledge to them?

Participant: I never thought I was sending something more than the request for Information. Or get back, it’s just -- that’s a little bit harder.

Interviewer: Let’s have the flow back, what’s the flow back from the drawings?

Participant: That’s Information, yeah.

Interviewer: Now, what is this Information? What is the drawing archive sending you as Information?

Participant: what would you call a drawing?

Interviewer: You said illustration earlier. Would it be illustrations, would it be something else?

Participant: It would be a scaled drawing of some equipment.

Interviewer: And this drawing is Information?

Participant: Yeah.

Interviewer: But the drawing request is not Information?

Participant: No, I suppose it would be... 

Interviewer: I’m not trying to put words in your mouth. Don’t use the exclusion... don’t go: “Well, it’s not Data, and it’s not Knowledge, so it has to be Information”

Participant: Of course I have a reason why I want this particular...

Interviewer: But you’re not communicating this reason, are you?

Participant: No. You go to the computer “I want this drawing. and send it to this printer.”

Interviewer: Is that meta-data, is that communication? 

Participant: I suppose I’m sending it Data, because I’m just punching in a few numbers.

Interviewer: That’s different from the fact-finding Knowledge, yes?

Participant: Yeah. 

Interviewer: So you are sending the {noun 89} Data, because the drawing request is Data.

Participant: Yeah, that’s right.

Interviewer: But you get back Information as the drawing.

Participant: Yeah.

Interviewer: You then take results from the production database, theories and experiences from books, and drawings from the {noun 89} and what do you do with them?

Participant: I produce the material.

Interviewer: Now, what do you do with that material?

Participant: I would produce sort of lecture materials which I would -- would give us communicate with the {position 90}. 

Interviewer: Who would the lecture materials go to in terms of roles?

Participant: It goes to me as the lecturer. 

Interviewer: so the lecture author sends to the lecturer lecture material.

Participant: Yes.

Interviewer: This lecture material, is this Data, Information, or Knowledge?

Participant: I would say all three.

Interviewer: So this is Data and Information and Knowledge. 

Participant: I would say that it’s Information and Data. 

Interviewer: It’s Knowledge?

Participant: It’s Knowledge. I suppose it makes up your Knowledge if you say Knowledge is based on Information and Data.

Interviewer: Is it?

Participant: Knowledge... you learn from something. Experiences which can be either a specific experience Knowledge... but. Because if you’re just sending Data, and the receiver is the person who has to come up with the -- make up the Knowledge part.

Interviewer: So the author is sending Data, Information, and Knowledge to the lecturer.

Participant: That’s the material container.

Interviewer: Does the author send anything else to the lecturer?

Participant: Not that I can think of.

Interviewer: Now, what does the lecturer do with that lecture material? 

Participant: Communicate to the receivers. Now it’s the case of {position 90}s, or the class.

Interviewer: Lecturer sends to the {position 90}s, what?

Participant: I would say mainly the Knowledge part. That’s the big thing. That’s the purpose.

Interviewer: So the lecturer sends to the {position 90}s Knowledge of, what can we label this?

Participant: It’s Technical Knowledge. 

Interviewer: Since we have 4 minutes left, let’s do a very brief theory session. What is Data?

Participant: For me? It’s Information. It’s some kind of recorded readings, values of ??? discontinuous or continuous flow of events.

Interviewer: Now you mentioned the term Information. Can you define Data again for me?

Participant: Information I suppose... is a measure of the Data, as opposed to the measurable property.

Interviewer: So Data is a measure of a property?

Participant: Yeah, for me.

Interviewer: What is Information?

Participant: Information for me, would be more of the numbers.... but more sort of verbal text. For me, Information is text, and Data is this numbers. But even a text could be Data. 

Interviewer: How is text Data?

Participant: Because sometimes text could describe a measure better than a number, I think. Depending on what you want to use it for.

Interviewer: so a measure of a property can be either qualitative or quantitative. Whereas Information is what?

Participant: I would say it’s a quality, I think. Information, I suppose, can mix with all Data, Knowledge. Something that you relate -- it goes from one source to another. Any kind of Data or Knowledge is some kind of Information. 

Interviewer: So would you say Information is a container for Data and Knowledge? Some sort of communicative thing? 

Participant: Yeah, I would think. Or, ??? when you’re just looking at numbers... Information I suppose is a continuous flow of Data and numbers and Knowledge sort of ... 

Interviewer: Okay. What’s Knowledge?

Participant: That’s harder. Knowledge could be your perception of something. If that’s what you think you know, anyways. 

Interviewer: Knowledge is what you think you know?

Participant: For me, yeah. But I suppose someone has some other Knowledge which I could sort of receive too, and then it becomes my Knowledge. 

Interviewer: How can someone give you Knowledge? Do they just give you Knowledge, or do they give you Knowledge through something else?

Participant: You can receive Knowledge, I suppose. Text, verbally, visual. 

Interviewer: But the flow would still be a Knowledge flow, it wouldn’t be a Data flow or an Information flow? 

Participant: It could be all. You can find Knowledge in ... you could get direct Knowledge as sort of verbal, but you can also receive Data but then you sort of have to compute the Data for it to become Knowledge. I think the Data itself does not become Knowledge until you do something with the Data. 

Interviewer: What about Information?

Participant: It’s the same, I think. You still have to do something. You have to put it in perspective. 

Interviewer: Three very quick questions. First, is Data atomic, can you divide data? Is there something that’s more... fundamental than Data?

Participant: Not that I can think of.

Interviewer: Is there something above Knowledge? Is there something that combines Knowledge into something bigger than itself?

Participant: That’s a religious question.

Interviewer: Yes, it can be.

Participant: I wouldn’t be able to describe it. I sometimes think that there’s something even above Knowledge. 

Interviewer: But we don’t have a ...

Participant: No.

Interviewer: When someone tells you to do something: “You should do this” are they communicating Data, Information, Knowledge, or something else to you?

Participant: I think they’re giving me Information, I would think. Instruction is ... that comes with a consequence. If I have to do something. I think they’re sending me Information, I would think. Which I would probably transform into some kind of Knowledge. Of what the consequences would be if I don’t go along with the Informational instruction.

Interviewer: If you’re making predictive statements about the future, or from the past to the present, are those predictive statements, Data, Information, or Knowledge? And what are you using to create them?

Participant: It would be Knowledge. And I would be using Information and Data to produce these Knowledge. 

Interviewer: How does experience turn into predictions?

Participant: Intuition.

Interviewer: You’d say experience is Knowledge?

Participant: It’s not perhaps, it’s part of the Knowledge. 

Interviewer: and you’d say that causal models are part of Knowledge? Like if I’ve got these inputs, this output would happen?

Participant: Yeah, sort of, predictive models ... they would be Knowledge, yeah.
\stopextract
\section{Interview 9}
\placefigure[]
[fig:i2]
{The SDFN Diagram for Interview 9}
{\externalfigure[Chapter4/graphs/i9.pdf][factor=fit,frame=on]}

\startextract
Interviewer: Let’s model both topics.

Participant: Do we have time?

Interviewer: No, but I want to model both. Let’s do the {noun 91} first.

Participant: it’s probably more relevant to this particular group. In that there was very little interaction from the reline in terms of work that I did.

Interviewer: I’m going to skip the entity dictionary. Normally what I do with people is: let’s just go through and figure out what entities there are. What I’m going to do instead is describe to you what an entity could be, and we’ll just jump in because I’d love to get both. An entity is a role that someone or something plays. Tools can be entities if they’re not ready to hand. A pen isn’t an entity because when I’m writing, I don’t go “pen.” But, SAS is probably an entity because I’ll be doing lexical analysis through it. If it’s providing a transformation, it’s an entity. You, yourself, can have multiple entities. As a {noun 91} student, I’ve got {position 56} hat, I’ve got teaching assistant hat, I’ve got grader hat, I’ve got database {position 31} hat. Just because I’m one person doesn’t meant I’m one entity. So, starting with the {noun 91}.

...

Interviewer: This is looking at both your {noun 91} methodology and the {noun 91} {noun 97}. The surrounding... We’ll start by modeling a trivial {noun 97} in your {noun 91}. because it’s best to start slow. Which entity would you like to characterize yourself as first?

Participant: Me?

Interviewer: You as?

Participant: I guess as a {position 56}? The one doing the work.

Interviewer: As a {position 56}. You’ve got flows of stuff. I use the word stuff because I’m not going to day Data, Information, or Knowledge. Or other. To someone else. What is a trivial flow of stuff to someone else and who is that other person?

Participant: I guess the trivial stuff would be reporting: talking about progress, results. So my main {position 92}s here are [name] and [name].

...

Interviewer: So, {position 56} to {position 92}s. What are you sending them?

Participant: I’m sending them Information. I meet with them about once a week to talk about what I’ve done. It might be results that I’ve gotten, that I can show them. It might just be talking about how things are going.

Interviewer: Do we want to differentiate that? This diagram is atemporal. Everything is superimposed because dealing with reality is such a pain. Let’s break that down into multiple flows. You’re telling them what you’ve done.

Participant: yeah.

Interviewer: Do we have a better label than “what you’ve done?”

Participant: Progress?

Interviewer: Now, you send them a flow of stuff about progress. Now this flow of stuff could just be a conversation, but it’s still a transfer of... stuff. The stuff you identified is Information?

Participant: As in results?

Interviewer: Your progress update. Is your progress update Data, Information, Knowledge, or O?

Participant: Information?

Interviewer: You may also send them Results. These results are?

Participant: Data.

Interviewer: Do you send them anything else in these usual meetings?

Participant: What about concerns? Can you send concerns? Emotions? Frustration?

Interviewer: Do you want to summarize them as one flow?

Participant: Concerns, I guess, is the right one.

Interviewer: These concerns that we all have are what?

Participant: Well they’re not Data. They’re not really Knowledge. So I’m assuming they’re Information.

Interviewer: Don’t use exclusions.

Participant: Can they be emotion?

Interviewer: Sure. Is the communication of emotions a unique concept? Or is covered under the other categories? Is emotion distinct in itself or is it part of some larger superset?

Participant: I would have thought that it would be distinct in itself. ... I’m going to think that emotions are separate. [someone else] probably wouldn’t look at that.

Interviewer: This is a really useful interview.

Participant: Because it’s true! I’m a lot more emotional than a lot of the [group] are. And so, for me, communicating emotions and getting feedback and getting my emotions right is important for me...

Interviewer: And this is distinct from Data, Information, and Knowledge?

Participant: yeah.

Interviewer: Do you send them anything else?

Participant: No, I think that covers my part.

Interviewer: What do they send you?

Participant: I’d probably break it down into feedback and guidance.

Interviewer: Feedback is?

Participant: It’s Information. Yeah, I’d put that one as Information. That’s responding to what I’ve done and telling me how they think things are going. Whereas guidance is probably more Knowledge, because that’s more pushing me in certain directions or giving me the understanding that I need in order to progress in different directions.

Interviewer: What you’re identifying here are normative statements: “You should do this.” I’ll want to unpack those later. So, your {position 92} sends back feedback and guidance. Do they send anything else?

Participant: Occasionally they’ll send -- there’s a problem where I’ve done something and it hasn’t given the results we expected. And so [name’s] done the work himself and then provided results back for me to compare to.

Interviewer: So these are comparative results?

Participant: yeah. They’d be Data.

Interviewer: Who/what else do you work with? Do you get or send flows to as a {position 56}?

Participant: My other {position 92} who is my [location] {position 92}. Because he’s kind of separate. Because I don’t talk to him that often. 

Interviewer: {adjective 93} {position 92}? or ignored {position 92}?

Participant: Hey, it goes both ways.

Interviewer: You to your {adjective 93} {position 92} send what?

Participant: The main thing would be progress, than anything.

Interviewer: as Information?

Participant: Yes.

Interviewer: Is it the same progress?

Participant: More condensed. So it’s not the same progress. I’ll talk to him every few months or so and I’ll talk to them every week.

Interviewer: Condensed you would say?

Participant: Maybe summary.

Interviewer: Summary of progress is?

Participant: Information.

Interviewer: Does he send anything back?

Participant: ...

Interviewer: Sorry, in the best of all possible worlds, does he send anything back?

Participant: Occasionally?

Interviewer: What does he send back?

Participant: Occasionally he sends back guidance, Knowledge.

Interviewer: Is it the same guidance?

Participant: No, different Knowledge.

Interviewer: How can we differentiate these guidances?

Participant: I don’t know. Direction?

Interviewer: Tell me this guidance versus this guidance. They send you what?

Participant: They [local {position 92}s] send me kind of week to week guidance. At the moment I’ve been developing a ... model to try and describe [undesired chemical interaction]. And so I’ve developed something to counteract that. So, over the {noun 97} of developing that, [name] has kind of given me direction in terms of: “Oh, should we consider this model instead?” and I’ll try that and it won’t work. And I’ll say OK, that one didn’t work, what about this. And point me in different directions. and then I’ll do the work and come back and say none of the work serves what I’ve done, which is what happens. Whereas with my {adjective 93} {position 92} it’s more of, I’ll present him a summary of what I’ve worked on for the last 2-3 months and say “This is where I’m planning to go.” More of the higher up {position 92} than a day to day thing. And I’ll kind of talk more about this is my plan for where I’m heading based on what I’ve done. And he’ll provide kind of direction and, if necessary, Information on different things I’m planning on doing.

Interviewer: So this directed guidance, as Knowledge, so he provides ...

Participant: when I say Information, I mean papers, or contacts, or things like that. And that kind of stuff I would classify as Knowledge. Because it’s stuff he has that I don’t.

Interviewer: so his Knowledge contains Information?

Participant: Yes.

Interviewer: He sends back directed guidance as Knowledge which contains Information. But the flow is labeled Knowledge, because it’s an expression of his Knowledge?

Participant: That makes sense.

Interviewer: It actually makes complete sense to me. What else?

Participant: I guess the only other thing he does, as an {adjective 93} {position 92}, is that he can put me in touch with other people. I don’t know if you have any points for networking.

Interviewer: of course I do. Let’s start with a flow from him to you. He creates networking opportunities by sending you what?

Participant: Contacts?

Interviewer: Contacts. These contacts are?

Participant: people? names?

Interviewer: These people and names are what?

Participant: So they’re Information. They’re not really Knowledge. I guess they could be Data. It depends on your definition.

Interviewer: Of course. Are they Data? Are they a different sort of Data than your other Data?

Participant: If they are Data, then yes they are.

Interviewer: Are they Data?

Participant: I’m an {position 58} by background, I can’t tell you anything.

Interviewer: It’s so much fun pinning {position 58}s down, because they squirm.

Participant: Yeah, I can feel myself squirming. I would say: “Look, you can put them in a list. There is a defined quantity. They can be grouped, they can be ordered or whatever. They probably are Data. You can put them in a spreadsheet.

Interviewer: But they’re different Data. How can we characterize this Data as different Data?

Participant: They’re people.

Interviewer: They’re not numbers.

Participant: People don’t like to be defined as numbers. They get all kind of emotional at me.

Interviewer: So, contacts are Data. Do we want to differentiate this flow of Data from your other flows of Data? Is it a difference in nature and kind or just nature? Are you comfortable with us labeling this Data in the sense that it will be perceived as the same as this other Data?

Participant: Meh.

Interviewer: Excellent. What other entities do we have?

Participant: With my {position 56} hat on?

Interviewer: Yes.

Participant: There are other people who I send Information to but don’t get anything from.

Interviewer: which entity is this?

Participant: [name] as my research {position 41}. As in the [local company] research {position 41}.

Interviewer: Research {position 41}. You’ve got a flow to him of?

Participant: summary of progress.

Interviewer: The same flow. And the type of flow?

Participant: Information?

Interviewer: Are there any flows back to you? Are there any flows back to other people regarding this interaction?

Participant: Not particularly. It’s just one sided. It’s just as a courtesy, keeping him in the loop sort of thing. So he knows what’s going on but doesn’t necessarily have a direct hand.

Interviewer: What else do we have?

Participant: I guess you can put other {position 56}s at [academic institution], but I don’t do much with them.

Interviewer: If you don’t do much with them, then they’re not relevant. Two other areas to explore. Communications with yourself wearing other hats and communication with tools.

Participant: Does my computer count as a tool? That’s really my only tool.

Interviewer: Do you use different programs on your computer in meaningful ways?

Participant: Not heaps. I do all my coding in C. Well, yeah, I’ve done a bit of computer programming before but nothing too extensive before this. But so far I’ve learned C, C++, FORTRAN. They’re not too bad, when you don’t know anything better, they’re not too bad. So I’ve done most of my coding in that. I often use one visualization package that I use to look at results or I just put them in excel.

...

Interviewer: Okay, let’s go with visualization software, and we can lump excel and your other vis stuff in here unless you think we should

Participant: No that’s fine.

Interviewer: do you send different stuff to your visualization software than excel?

Participant: Yes and no. I get the same kind of things back from them. I don’t do an awful lot of transformation stuff. I don’t do a lot of formulas in excel. I just put the numbers in and get graphs. The visualization stuff I’m putting numbers in and I’m getting pictures.

Interviewer: And you’ve got, and what entity would you say your programming is? How do you conceptualize it in your head?

Participant: It’s code.

Interviewer: You as {position 56}, send the entity known as code, what?

Participant: I send it Data. I created it though, so how does...

Interviewer: That’s the question. You send it Data. What Data do you send it?

Participant: Well I send it the lines that it uses to create the program. So I send it... I don’t know how you would describe that. I write the basis of it. I tell it what to do. But I also send it numbers to work with.

Interviewer: Let’s differentiate those into two flows. The numbers it needs to work with are?

Participant: Data.

Interviewer: And we can label them?

Participant: inputs.

Interviewer: You also send it what?

Participant: I guess I’d classify it as Knowledge. No, it’s Information. It’s kind of all of the above.

Interviewer: we can certainly combine elements.

Participant: Not really Information.

Interviewer: so you send it what?

Participant: I would call it Knowledge. Because I’m imparting my Knowledge into the code and telling it what to do.

Interviewer: and what will we label this flow?

Participant: Maybe model theory or something?

Interviewer: What other flows are there?

Participant: Well it just sends back results.

Interviewer: Same results or are these results different from these results?

Participant: They are different. But not in nature. Just in ... obviously I’m not going to take every result I take from the code and send it on. Because that would be ridiculous.

Interviewer: So can we say selected results over here?

Participant: Yeah.

Interviewer: What else?

Participant: That’s really about it.

Interviewer: when you do debugging, does it send anything?

Participant: Only when I tell it to.

Interviewer: when you tell it to send you things, does it send you a flow outside of results?

Participant: not really. I would say...

Interviewer: just different kinds of results? How do you interact with the visualization software and does your code directly interact?

Participant: no.

Interviewer: How do you interact with the visualization?

Participant: I send it the selected results.

Interviewer: the same selected results?

Participant: Different selected results. You can make it the same as the one that comes out of the code because, for all intents and purposes that’s what I use. The visualization package doesn’t always use all of the results that I send it.

Interviewer: but it’s the same results, there’s no transformation, there’s just filtering.

Participant: the filtering is usually done inside the package.

Interviewer: you send it results, what does it do?

Participant: it gives me pictures or pretty graphs. like circles and balls. I’ve got very unexciting pictures.

Interviewer: It sends you back pictures and graphs...

Participant: The pictures come from one software and the graphs come from the other.

Interviewer: These pictures are?

Participant: Data?

Interviewer: Why are they Data?

Participant: Because they’re based on particular things.

Interviewer: So it’s just Data as representation of Data.

Participant: Yeah.

Interviewer: It also sends you back graphs. These graphs are?

Participant: the same thing.

Interviewer: Data. What else do we have?

Participant: That’s mostly it.

Interviewer: Do any of these entities exchange stuff... about your work?

Participant: to a minor extent. I would say that there’s a minor extent link between my {position 92}s and my {adjective 93} {position 92}.

Interviewer: OK, what links are there?

Participant: All I can think about is discussion.

Interviewer: Who is discussing with whom?

Participant: Well, it’s a double sided arrow.

Interviewer: This discussion is?

Participant: Information.

....

Interviewer: Let’s see what we can do on {verb 94}. Which entity do we start with for {verb 94}?

Participant: if we’re looking at it from my point of view, so I was an {position 58}.

Interviewer: And what flows are there, to what?

Participant: So my {position 41} on the project, we’ll call him the lead {position 58}. He was just responsible for our group.

Interviewer: flows, you to lead {position 58}.

Participant: I’d send him any work that I’ve done. It’s kind of hard to quantify what I did.

Interviewer: Are there categories of what you did?

Participant: Yeah, there are. Maybe I could break it down to three main categories. {position 58}ing {noun 95} work, so {noun 95} calculations and stuff like that. Maybe {noun 95} calcs?

Interviewer: And these {noun 95} calcs are?

Participant: They’re Data. {noun 95} calcs, and {adjective 96} stuff, documentation I guess I’d call it. {adjective 96} stuff is kind of stuff that’s done, so we do {adjective 96} and we go use a {Noun 18} or test a valve. Documentation is all of the documentation that we wrote up ...

Interviewer: Your ... documentation is?

Participant: I don’t know what you’d classify it as. I guess it’s a type of Data.

Interviewer: Can we separate it out into sub-categories?

Participant: Not really.

Interviewer: So it’s not Information, it’s not Knowledge.

Participant: It’s a kind of a bit of Knowledge and a bit of Information and a bit of Data all rolled into one document. So what is the document?

Interviewer: Let’s start by: “What is the document?”

Participant: It depends on which document.

Interviewer: What categories of documents are there?

Participant: The main ones are descriptions of the {noun 97}, describing what the {noun 97}es do. I need to explain that I worked mostly on the {noun 98} stuff.

Interviewer: So we’ve got {noun 97} description documentation. This {noun 97} description documentation is?

Participant: Well it’s all the above. It’s all of them.

Interviewer: So Knowledge + Information + Data?

Participant: Yeah.

Interviewer: And documentation not including {noun 97} description documentation has what other bits in it?

Participant: {adjective 96} reports.

Interviewer: So we have {adjective 96} stuff, but that’s different from {adjective 96} reports.

Participant: Yes.

Interviewer: These reports are?

Participant: They’re Information and Data.

Interviewer: But not Knowledge?

Participant: Nope.

Interviewer: Why not?

Participant: Because the Knowledge contained in the {noun 97} descriptions is a kind of background behind what was done and why. Whereas the {adjective 96} report is just a description of what was done during {adjective 96} and the results that were obtained. So the dumb stuff is the Information. And the results are the Data.

Interviewer: are there other categories that we can extract out from documentation?

Participant: That’ll do for now or we’ll be here all day.

Interviewer: In documentation, the remainder of the documentation is what?

Participant: Mostly just Information.

Interviewer: The {adjective 96} stuff is what?

Participant: I would classify it as {adjective 96} results and call it Data.

Interviewer: All right, what else do we have? What other interactions?

...

Interviewer: are there any flows between {position 58} and lead {position 58}?

Participant: The lead {position 58} tells me what to do?

Interviewer: What can we label this telling you what to do as?

Participant: Directions sounds a bit too nice but it will do.

Interviewer: Direction is?

Participant: Information, I guess.

Interviewer: Does the lead {position 58} send anything else back to you?

Participant: I guess they send back Knowledge. I’m trying to think of what form that would come under. But say there’s something in the {adjective 96} results that’s not quite as you’d expect it, then they can give advice as to what could be causing it, or what to do to test something else. So you could call it {noun 97} Knowledge. The point of the lead {position 58} is for them to have more comprehensive understanding of how things work. You can put some other entities in if you want to. “other {Position 22}s.” My main flow of communication with them... My background with the project was that I knew how all of the systems worked because I {noun 95}ed the system. ... I knew how things worked and so I would explain to them how things worked. ... I guess my main flow to them would be {noun 97} understanding. Call it Knowledge.

Interviewer: Do they send anything to you?

Participant: They would me {adjective 96} results.

Interviewer: The same ones?

Participant: The same Data.

Interviewer: How are they different?

Participant: Well, it could be... they do {adjective 96} on?? other systems. So different results. Whereas the stuff that I send would be whatever I was involved in. If you want another arrow there, they send those {adjective 96} results to the lead {position 58}.

Interviewer: so these results are?

Participant: They’re Data.

Interviewer: What else?

Participant: The main one I had was the {noun 63} {position 58}. I’d teach them about {Position 22}ing. I’d probably give them the same {noun 97} understanding as the Os. They would send back. Well, they would send back understanding of how the {noun 63} works. So that would be -- I guess that would be Knowledge. It’s very hard to break it down into data.

Interviewer: where do you get these {noun 95} calculations from?

Participant: I do them. If we want an entity for them, the best thing is that I do {noun 99} modeling, and I use the {noun 99} package. That’s a nice little name for it.

Interviewer: What do you send to the package, and what do you get back?

Participant: I’d send input, Data. And it gives me, results, Data.

Interviewer: Anything else?

Participant: Well, I don’t know where all the equipment is.

Interviewer: Is the equipment important?

Participant: Probably not. I mean it is important for the job, but probably not.

Interviewer: Does the equipment provide Data, Information, Knowledge, or O?

Participant: Of course it does. It provides all of the above. Well, maybe not Information or Knowledge... It’s very weird thinking of it as a Knowledge transferring. Maybe you could call it the {noun 63} that’s giving us Information back. The [object] through it doesn’t really tell you anything.

Interviewer: OK, the {noun 63}. What do you send to it? What does it send to you?

Participant: We don’t really send it anything. Well, I don’t send it anything.

Interviewer: What does it send to you?

Participant: It sends {noun 97} results. Data, because I don’t have anything better.

Interviewer: now you mentioned Information at some point, with regards to this.

Participant: Well, it’s the Information that comes from the results.

Interviewer: where does that Information get...

Participant: transformed? Through the {position 58}. Well, an {position 58} looks at it and goes “Yes, that number looks right. Or no that number doesn’t look right. It’s broken.”

Interviewer: And that’s a transformation into what?

Participant: It’s a transformation into whether things are working.

...

Interviewer: Define for me, Knowledge. Are there types of Knowledge?

Participant: I would define Knowledge as something that an individual or entity ... something that an entity possesses. That typically has been learned from somewhere else or gained from experience or whatever that enables them a better understanding then they would have otherwise. My understanding of how things would work is that Knowledge is the {noun 97} by which Data is turned into Information. So I would say Data is things. It’s numbers, it’s raw Information. It is something that, on its own, doesn’t mean very much. It’s just stuff. Information is an interpretation of that. So it’s kind of something that is understandable to someone without Knowledge, without even concept of what the raw Data is.

Interviewer: So it’s encapsulated something? What is it that’s encapsulated?

Participant: I mean, it can be an explanation of results. It can have nothing to do with it whatsoever.

Interviewer: So we’ve got Data is transformed by Knowledge into Information. What does Information do and where does it go? It doesn’t just sit there.

Participant: Information is used as a means of communication.

Interviewer: What is it communicating and does it cause any changes in anything?

Participant: I don’t know.

Interviewer: Where does emotion fall into this if anywhere?

Participant: Well I’d say it’s fairly separate. However, it can influence how things are perceived. So it probably has an influence on the Information.

Interviewer: So emotion changes the transformation of Data into Information?

Participant: I’d say it just changes the perception of Information.

Interviewer: So Data goes into Knowledge, what’s this flow?

Participant: Data doesn’t really go into Knowledge.

Interviewer: Sorry...

Participant: Data doesn’t turn into Knowledge. Data turns into Information.

Interviewer: So Data turns into Information. Knowledge...

Participant: creates that link. So if you don’t have Knowledge, the Data won’t transform very well.

Interviewer: does Knowledge go into Information?

Participant: Yes, Information can be a communication of Knowledge.

Interviewer: Does anything go into Knowledge? How do you get more?

Participant: Through Information.

Interviewer: What is Knowledge?

Participant: understanding inherent to an entity, is how I’d describe it.

Interviewer: Three questions: Is there anything to either side of Data, Information, or Knowledge? Is there anything that’s lower than Data? Or is Data atomic?

Participant: I would say Data can be atomic. It stretches across that width of the spectrum. I would say that it goes from as far as atomic to as big as macroscopic. It can be anything within that. I’m saying that in any sense.

Interviewer: Data can be divided or it cannot. If Data is divided, what is it divided into?

Participant: Subdata?

Interviewer: Is subdata a thing? Or is subdata Data?

Participant: I would say it’s not any different. You could have different scales of it, but... you could have a group of numbers. You could select some of that group of numbers. That’s not any different. It’s just still Data.

Interviewer: So you have a subset of Data which is Data.

Participant: Yeah.

Interviewer: and at some point you just can’t subset any more.

Participant: You could get it right down to the last point... It’s still Data

Interviewer: You could have many points, and that collection of points is still Data.

Participant: Yeah.

Interviewer: But you have a collection of points, and you apply Knowledge and you get ... Information? And then you’ve got this loop of Information to Knowledge and Knowledge to Information. Is there anything beyond Knowledge?

Participant: I would say no. I would say the only thing beyond Knowledge is god. I mean God is all Knowledge. That just gets confusing. <inaudible>

Interviewer: Because it is part of some people’s ontologies, but it’s outside the scope of this investigation ... When someone gives you a normative assertion, it’s what of these, if anything? When someone orders you to do something, is that Data, Information, or Knowledge?

Participant: None of them.

Interviewer: What is it? Do we have a handy label for it?

Participant: I don’t know. If anything, I’d say it falls under Information, but I don’t think it does.

Interviewer: So what is Information?

Participant: It’s a lot of stuff.

Interviewer: What’s a component of Information?

Participant: It could be an interpretation of something. It could just be words. Understanding, words. I would say if someone asks someone to do something, provides them a direction or a directive, it’s not necessarily a should, but, provides them guidance to what they should do, I would say that’s Information, they’re giving them Information on what to do.

Interviewer: The guidance is what to do based on what?

Participant: Based on that entity’s understanding. Based on their Knowledge or their Information. Or the interpretation they’ve got of things.

Interviewer: Any final thoughts?

Participant: Nope.

....

Interviewer: ... your use of Emotion as a perception of manipulation engine is fascinating. So, thank you for that.

Participant: That’s OK. Maybe that’s where should comes into it. Maybe it that when someone tells you that you should do something, they’re kind of applying their emotiveness behind it. In that, ??? said, it’s a manipulative thing. It’s not necessarily a “I’m guiding you to do this.” It’s a I’m making or telling you that you must do this. So they’re exercising their manipulativeness over them. Which can be a good thing. Like a {position 41} with an employee might help them to do something. And that’s manipulation.

Interviewer: and so it’s an emotive communication of...

Participant: status in the hierarchy. Maybe that’s emotive, maybe that’s not really emotive.

Interviewer: You tell me

Participant: I don’t know. I’ll just confuse myself.
\stopextract

\section{Interview 10}
\placefigure[]
[fig:i2]
{The SDFN Diagram for Interview 10}
{\externalfigure[Chapter4/graphs/i10.pdf][factor=fit,frame=on]}

\startextract
Interviewer: Let’s model this. A brief discussion on entities. An entity is something, someone, that takes in these flows, performs some sort of transformation on them, and outputs flows. Or produces flows, or takes in flows. I would say that, when I was doing my masters, I had three roles: I was a student, I was a teaching assistant, and I was the database {position 31}. For purposes of this diagram, I would be three different entities because not only do they deal with Data, Information, Knowledge, in different ways, but they can talk to each other. Even though it’s me talking to me, it’s me as DBA going: I just wrote this lab me, here are the questions we need to deal with. It’s the DBA talking to the graduate assistant, there are different ways of dealing with stuff. We will ignore things that are ready-at-hand. This pen is not an entity because I don’t think about it when I’m writing, I just write. Say I use that computer and I load up SAS, and I load up the 100,000 words of transcript that I’ve written so far. And it munges them and it produces something. SAS is an entity because it’s taking stuff in and giving me something. It’s {noun 97}ing it, and I’m aware that it’s {noun 97}ing it. It’s an entity to me because it’s not just an invisible flow it’s go there, come back. Scientific instruments can be entities, they can take stuff... What I’m trying to do is be very vague here so I don’t prejudice your philosophy. This is why I’m saying if you think doing an entity dictionary is useful we’ll do it. If you’ve got a fairly decent idea, we’ll start trivial and expand out.

Participant: we’ll start trivial?

Interviewer: Let’s start by talking about one of the really beautiful {noun 37} cases we just talked about with that semipermiable membrane. With ..

Participant: technology / business .

Interviewer: we start out by going what is one of the central entities in this {noun 97}? probably a role that you play, but it doesn’t have to be. 

Participant: As the collector, just call it the {noun 129} entity. {noun 129}. 

Interviewer: multiple people can be in one entity.

Participant: {noun 129}. What happens in the {noun 129} well: the inputs are observations....

Interviewer: What is effectively sending you observations?

Participant: {position 90}s. We’ll probably do this around [name]. So the {position 90}s. These are corridor conversations. You’ll be sitting there and some {position 90} will say: “Do you know such and such?” ... and the {position 90}s will come around for a little chat. Because they know I’m there. And you pick up all sorts of things that are going on. That’s the informal observations.

Interviewer: So we have {position 90}s sending a flow to {noun 129}. This flow is observations?

Participant: Just observations. They push the buttons in the {noun 97}. They hear things, they see things. They know how the {noun 97}. They know the differences between the shifts. And if you want to put a credibility factor on it, around 20%. So 20% of those observations would be supported by serious observations by a trained professional. And 80% of it is just... rattle... it’s perceptions, it’s “I can’t back this up.” Just the random Knowledge. But 20% of that material will be well worth taking on board. That’s the kind of stuff you have to stack into the observational database, for want of a better word. It’ll be scribbled on a piece of paper. I’ll just keep that in mind for later.

Interviewer: {position 90}s send you a flow of observations. Here’s the fun bit. Would you categorize this flow as Data, Information, Knowledge, or something different?

Participant: Can I pick two?

Interviewer: absolutely.

Participant: Information and Knowledge. Because the Information is the things that I can reduce to something technical. The Knowledge is their understanding of the {noun 97}. ... what’s their perception of the basis of [their choices]. Information to me would be the technical side of it. The Knowledge is the operational expertise. So you’re getting some of those two things coming through. 

Interviewer: So they send you observations. The observations are Knowledge, or are Information. 

Participant: Pretty nebulous. Handwavy stuff. Some as good as gold.

Interviewer: Is there a flow from {noun 129} back to the {position 90}s?

Participant: Yes. There probably are many. But again, it’s limited to the ones that you talk to. And it’s, for me, trying to “this is the next level down of telling the story” of why, why do these things happen? hang on fellas, that’s not the way it works, this is it. Some of them? [company] ... ??? Interesting story, giving a presentation. This was a few years ago. It was {noun 131}s. These things are like big bowls. So we’re doing a formal presentation, there’s about half a dozen, a {position 106} from here... I start giving this background of {noun 131}s. And this guy says: “[name], what you’re really saying, independent of type of {noun 61}, they all form for the same reason” and I say “spot on.” Did {group 130} get that? no. But one of the {position 90}s picked it up in 5 minutes of me talking. And so, that was a formal feedback to the {position 90}s, but when I go back, there’s a lot of informal value in the corridor. {noun 131}s is all about this. {Noun 21} {noun 132} is all about this. {noun 137} is about this. So, for me personally, there is an informal feedback. 

Interviewer: So there’s feedback back, whether it’s formal or informal, it takes about the same form?

Participant: Yes. Just talking. Just start drawing lines on a whiteboard. Putting together the pieces of the jigsaw puzzle into some sort of digestible ... my job is to tell a story. The complexity of that story depends on the audience.

Interviewer: Would you say the feedback is Data, Information, Knowledge, or O?

Participant: It’s my Knowledge, but I’m giving it back as Information. 

Interviewer: Let’s unpack that. Would you say the flow is absolutely generated from your Knowledge, and you assert that they perceive the flow as Information?

Participant: Yes.

Interviewer: Would you say the flow itself is a flow of Knowledge, or a flow of Information?

Participant: Put the two together I’d say.

Interviewer: Why?

Participant: Because the Knowledge base that I’m drawing on, is much more substantial than these.

Interviewer: So you’re passing them Knowledge, but that they accept as Information?

Participant: Yes.

Interviewer: What do they do with that Information?

Participant: They probably stew about it. Would they incorporate that into their actions? No.

Interviewer: Which is why you would categorize it as Information?

Participant: Yes. Because I have no power to direct which way they want to {noun 97}. That would hopefully be operating through the form of {noun 88}. My role in that loop is to just give them the background why? They might agitate for that, “But [name] says this.” But [name] doesn’t have a clue. Every now and then, if it’s the right sort of segment, they will push that in.

Interviewer: So you would say that it’s not so much a flow of Information and Knowledge, as that it comes from you as a Knowledge flow, and then becomes an Information flow they accept. ... Are there other flows between these two entities?

Participant: in the example I’m thinking about, no. Because I’m here as the blowing consultant. You’re hoping that the formal loop of {noun 88} would be reinforcing that message. That would be from this entity up to the {noun 88} entity. 

Interviewer: What do we want to label that as?

Participant: {noun 88}? This is the group that sets the operational procedures and standards. This is how you shoot this wheel into that position. You will run this {noun 97} at this speed. You will do this if the {noun 132} in the {Noun 21} exceeds this value. That’s the formal bit. So, what I provide to them is the written report which is Knowledge, I would suspect. 

Interviewer: So there’s a flow from you to them, of Knowledge. 

Participant: Formal Knowledge. This is documented.

Interviewer: So you would categorize it as a different kind of Knowledge?

Participant: Yes. This is the next level up in technical understanding. 

Interviewer: This formal Knowledge flow... what’s the content of the flow? Just as the content of this is observations. 

Participant: This will be datasets, analysis, and some observational material that supports it. This is really about the formal Data set.

Interviewer: So you’re passing to {noun 88}, the distilled version... 

Participant: you probably do release your Data sets. There will be tables and numbers. If it’s an analysis, there will be a table of numbers. 

Interviewer: But this is your interpreted {noun 97}, refined...

Participant: This is my certified... Data set. I believe the Data set that best and most reliably represents the analyses that we’ve taken as part of this. Simple things like the tables add up to a sensitive number. If you look at {noun 133} analyses, the table is not necessarily 100%, but you know, if I express it this way, the table should be about 95%. If I express it this way, the table is just over 100. 

Interviewer: But they’re certified, because...

Participant: They fit more or less internal guidelines that you have for quality of Data. If it’s outside, there’s something wrong with it. I’ve gotten analyses out of there that only total 70%. There’s something wrong here [name]. I want this fixed. It’s not good, there’s something wrong with it. Usually it’s traced to a problem that they’re not running the right standard. Certified Data set, this is the most reliable Data set that we can have from these analyses given these {noun 37}. The same for if you’ve got a series of {noun 97} measurements, you’ll usually attach some sort of quality, a standard deviation... it looks like it fits the general trend, yes it’s consistent. So effectively the summary of that is part of the certified Data set. If someone was to go back and take the original set of Data off the {noun 97} computer and do exactly the same set they should end up with exactly the same numbers. Even if they don’t do the exclusions, they should end up with the same value. That in essence is what you would call a certified Data set. 

Interviewer: Where do you get the Data for this certified Data set?

Participant: From {noun 37} and analyses.

Interviewer: In the chain of entities that may provide you with this Data set, what is the penultimate one?

Participant: For me personally, it is a physical {noun 37} for which I have an analysis. 

Interviewer: So we have {noun 37}?

Participant: {noun 37}. That’s only because of how I take problems, and the sort of problems I take. The {position 134} says I want the {noun 37}. The {position 135} says I want the analysis. For the people here, it will be a block of {noun 97} Data. [Name]: “I’ve got two years {noun 97} Data.” I’ve got a relationship.” His {noun 37} is actually the set of {noun 97} Data. And for me, occasionally, that’s probably my second set. I’ll look down to the {noun 97} Data. 

Interviewer: So we’ve an entity “{noun 97} Data”

Participant: Yes. 

Interviewer: Do we want to say that the entity is {noun 97} Data, or say that it’s a {noun 97} DB? 

Participant: I’m extracting specific variables from a DB. [name], could you get me tail {Noun 23} and {noun 28} chemistry? Can you get me {Noun 21} {noun 132}s from {noun 136} such and such? Can you get me the {Noun 21} {noun 137}? can you get me the delivery {noun 137} stock? That’s a DB. Someone else will extract that for me.

Interviewer: Let’s start with {noun 37}. Do you send any flows to the {noun 37}?

Participant: I hope not. No. 

Interviewer: What flows do you get from the {noun 37}?

Participant: I get {noun 37}, observation. I see the physical nature of the {noun 37}.

Interviewer: Let’s go with physical nature?

Participant: Yes. This looks like normal {Noun 21} for [company.] This stuff’s undergoing [term]. It’s different. It shouldn’t have done this in three days. There’s something about the physical nature.

Interviewer: So you’re taking physical nature. The flow of observations of physical nature is Data,Information, Knowledge, O?

Participant: Information. But I’m using my Knowledge to assign the Information, parameters, based on my previous experience.

Interviewer: Unpack that for me. You would say that it’s Information...

Participant: It’s Information. But that Information is actually based on an experience and technology???. 

Interviewer: Here’s a better question. You say that this Information is a function of your prior Knowledge. Where is the input of your prior Knowledge such that this Information is produced.

Participant: It’s my long term experience with ... Or anything for that matter. If we went to: “What do I do with these {noun 37}?” First is observation. The second is some sort of chemical analysis. And the third thing, which i’ll come back to is what it looks like at the end of a microscope. And one of the more or less concepts that I push to people is that the story should be independent of scale. In the sense that what I see down the microscope and what I see in the specimen. Or what I see from the scanning electron microscope, they should be a consistent story. When I write the summary those observations should be independent of scale. They should be what I see. A lot of it is what I see down the microscope has to match in some way, shape, or form, what the SEM is telling me. They tell you different bits of the story, but they must be consistent. They must be the same story. Or compatible with the same story. And, I keep talking about what you see in textures and materials should in some way or form, be explainable between the two. The SEM and the optical microscope. Some people don’t see it that way. “Oh! but this is different.” No, there the same materials. How you see them may be slightly different, but the story should be consistent. The next level down. It may not look the same, you’re just seeing different visualizations of the same thing. That’s probably something about the {noun 37}.

Interviewer: Okay, what it looks like is that physical nature and information isn’t a direct flow. That story tells me that there’s a role that acts as an intermediary between {noun 37} and {noun 129}. What is that role? 

Participant: That role, that’s the Knowledge role? Knowledge or expertise?

Interviewer: It’s the person making the observations. What name can we put to that role?

Participant: I’m loathe to say. But it’s almost like an expert, it’s an expertise I have. 

Interviewer: what I see is that {noun 129} sends to this role Knowledge.

Participant: yes.

Interviewer: This role gets something from the {noun 37}, combines it with Knowledge, and sends physical nature as Information back into the {noun 129}.

Participant: into the development to write the story. 

Interviewer: So we’ve got role X here ... you as {noun 129} sends Knowledge to this role. This Knowledge is expertise?

Participant: Yes. Accumulated expertise which goes right back to {noun 134} training in a lot of ways. Everything has a precedent in the past. An organization that forgets its past doesn’t have a future. It’s the same thing. 

Interviewer: And you get the physical nature from this entity. You as {noun 129}. I’m pretty sure they’re both you. Now, this entity, we can call it expert. Can we call it something else? What would you call it if you were telling someone else to take {noun 37} for you. 

Participant: If I had pre-knowledge of what we were {verb 73}, I’d say look out for.... I was just thinking that yes I had been in a situation. Usually the critical thing is: just get me a {noun 37}. “Oh, oh, we’ve got the {noun 37}.” But that, in the industrial sense is extremely hard. Usually if you’ve got a bit more control, you want to refine your location. The {noun 138} that you’d like to take a {noun 37} from. ...

Interviewer: Not just I want a {noun 37}, but I want a {noun 37} from that thing.

Participant: The {noun 139} there. I want a {noun 37} out of that{noun 140} and I want it pretty soon... And I will go to the pile and pick it out. “Well, that’s different, that’s different.” Because you’re almost doing the analytical stage in your head. “Oh, we’ve got a {noun 37}.” and the next question is: well where was it? I don’t remember. That’s part of it’s: what is its relationship to its other entities? What was its orientation? Is this the top or the bottom? Simple things like that. Usually in giving guidelines for {verb 73}, it’s “get me a {noun 37}.”

Interviewer: {position 141}.

Participant: Yes, {position 141}. And usually you keep it s simple as possible. The best option for me is get me a whole {noun 37} and I’ll go over it, and I’ll pick out the {noun 37} I really need. And that’s the way I usually work when I can. But if its ??? just get us a {noun 37}. And with pictures. That’s probably the best thing. Digital cameras are absolutely terrific. Take a few pictures of where it’s coming from. And then go ‘I’ve got use for the things.” 

Interviewer: So that means that {noun 37} is sending to the {position 141} something. The {position 141} is also sending to {noun 129} pictures. 

Participant: Observations. Recorded observations that you can make something out of. Recorded site observations. Here’s the picture. It’s usually got a time on the bottom of it. We usually know where it’s come from. And out of the picture you can probably scale things. “Well, that’s usually there too...” It’s .25 mil, it’s about half a mil across. So this digital image that’s an observation, but it’s recorded. Once you’ve got it recorded, you’re in business.

Interviewer: This flow of recorded site observations is: Data, Information, Knowledge, O?

Participant: I’d say it would have to be Data. This is stuff you can do things... this is something you can reduce to a number. It also has observations, but it’s reducable to a number. 

Interviewer: The {noun 37} flow to {position 141} is what? The {position 141} is the one that uses your expertise to produce the physical nature. So it’s taker, but it’s also the person who processes the {noun 37}. So the {noun 37} to the sample taker is what?

Participant: It’s embodied Information, isn’t it? Shoved another qualifier on the front of it. He doesn’t inherently know what it is. But the {noun 37} embodies Information. Information, it might, and if you do an analysis on it, it’s Data. 

Interviewer: So where’s the Data here?

Participant: Probably just {noun 37} needs to be split into its an observation, it’s a physical {noun 37}, and it’s also what you do with that {noun 37}. 

Interviewer: Let’s look at flows here. We’ve got physical

Participant: A physical body that you can hold in your hand.

Interviewer: We would say that the physical body is an embodiment of Information? 

Participant: Yes. And ultimately it can be reduced to hard numbers. The Data.

Interviewer: That reduction of hard numbers is performed by the {position 141} and given to {noun 129}?

Participant: Yes.

Interviewer: So we’ve got another flow here. These hard numbers, what’s the label of it?

Participant: A chemical analysis. This chemical analysis is... this is all hard numbers. Here is an analysis. This is real. This is hard Data. This comes down to numbers which hopefully will ??? properly. By standards, it’s real Data. 

Interviewer: We also have other things going from the {noun 37} to the {position 141}, besides the physical body. Right?

Participant: Yes.

Interviewer: What are those?

Participant: Information. Simple as “this is the date I collected it, it was raining.”

Interviewer: environmental?

Participant: environmental. It was picked off this {noun 136}. This {noun 136}, it had changed color. It’s thats sort of observations.

Interviewer: These environmental observations are?

Participant: They’re not Data. Information? 

Interviewer: As a note, don’t use exclusion. find some word that it fits even if we have to make up new words.

Participant: I think it’s Information. it’s creating a picture of the problem. 

Interviewer: Do we want to say provenance instead of environmental observations?

Participant: Provenance is a slightly different sort of thing.

Interviewer: Is there a flow of provenance or is provenance a function of?

Participant: If you’re looking where other people have taken {noun 37}, provenance becomes extremely important. If you know, the {position 141}, but there’s another {position 141} out here, it’s like an expansion of this -- {noun 129} request and we do have examples of this. ... {position 141}, then there’s another loop out here,... research doing things, and they’re doing various things. And the provenance of any of their {noun 37} becomes critical. Where it’s like this is the {noun 37} and I’m the {position 141}, provenance is not an issue because it’s embodied in the observation. But if it was on the other side there, you’d like to know something of the provenance of the {noun 37}. 

Interviewer: Therefore, where’s the provenance flow? And what is it?

Participant: Provenance flow, it’s probably a subset of this interaction here, but it is another point removed. 

Interviewer: Point removed from what?

Participant: The immediacy of the {noun 37}? There’s a flow there, and it’s really Information. 

Interviewer: Are there any other flows of Data, Information, Knowledge, or other coming from the {noun 37} to anyone?

Participant: No, I don’t think so. It’s all being more or less channeled through this loop here. It’s channeling through one entity. Sometimes this will be circumvented because they’ll do it themselves, but really, in the way I’m thinking about things, it’s this. This is the {noun 129} role. I take in all this Information, distill it down to a story, and pass it on. 

Interviewer: Is there a flow of story here?

Participant: Usually a story back this way so they know what they’re doing. 

Interviewer: So {noun 129} sends to {position 141} story?

Participant: Story. Why do I want to do this? What is the background thing? What is the rationality for doing this? This is not a stupid exercise. 

Interviewer: So {noun 129} sends to {position 141}, rationale.

Participant: Rationale. Why are we doing this?

Interviewer: This rationale is?

Participant: It’s Knowledge. Because I know the people involved, I know that they like to have some idea of why are we doing this? Oh, [name] says this about it. otherwise it’s shit. But if they understand why, they’re likely to do it. If they don’t understand ???

Interviewer: Even if the role of {position 141} is in your own head, you still have to justify to yourself the rationale.

Participant: Why would I be spending money on this thing if it won’t go? There’s a rationale for myself and there’s a rationale for other people.

Interviewer: Does the rationale go to anyone else, or is it a different rationale that goes to other people?

Participant: The rationale heading in that direction is probably much the same.

Interviewer: Is it the same rationale? 

Participant: If you distill it down to its simplest form, yes.

Interviewer: So it’s a simplified rationale.

Participant: It’s simplified going this way, this is more detailed.

Interviewer: So {noun 129} to {position 141} is simplified. 

Participant: I need this {noun 37} because it will tell me something about this event. I need the Data to see what’s going on.

Interviewer: Whereas {noun 129} to {noun 88}...

Participant: It’s detailed. “These things don’t ever add up, I think we need a {noun 37}, and we need this about the {noun 37}” Same basic motivation, it’s got a lot more detail.

Interviewer: Are they both Knowledge?

Participant: Is gut feeling Knowledge?

Interviewer: It’s entirely up to you whether or not you want to categorize it as that.

Participant: I think it’s Knowledge, because it’s something your own makeup that says this is valuable. It’s a form of Knowledge. It’s Knowledge. Knowledge is the sum of experience, activity, that sort of things. “Why are we doing this? Well, in the past, this has worked. I’ve read about this activity in the paper. I’ve put 2 and 2 together and this is my rationale for doing this sort of thing. The Knowledge is the sum of what has been successful in the past, what has been a flop, why was it a flop, where did I go wrong? The formal learning. 

Interviewer: Is learning different from experience?

Participant: That’s a question for an education person. I think it is. ... The activities you’ve engaged in, previous experience, experimental sort of things. Learning, And what you’ve seen other people do. 

Interviewer: What would that be?

Participant: It would be an observation. You’ve seen them doing this, and “jeeze, it worked. I’ll try that.” This is probably not out of the textbook, either. 

...

Interviewer: We have the {noun 97} DB over here. It looks awful lonely. Who plays with the {noun 97} database?

Participant: {noun 88}...

Interviewer: {noun 88} to {noun 97} database... does what?

Participant: They, in my vision of things, they check the integrity of the Data that is actually going into the DB. In the sense that we’ve had a little bit of yesterday’s {noun 97} Information. That point’s not right. Something’s gone wrong there. There’s some sort of quality review of the material that has actually long-term resident in the DB.

Interviewer: In that case, we have a flow... let’s start with the first flow of: stuff goes into the {noun 97} database. From where?

Participant: [list of redacted terms] ... the whole infrastructure. Computers, ... servers. And probably another aside to think about over my working life, there’s been a very distinct change in the perception of those sorts of measurements. In the sense of -- when I did my {noun 91}, measurements were were extremely expensive to do. A {noun 74} cell was an extremely expensive piece of gear. They certainly weren’t refined in the {Noun 4} precision or accuracy that we have today. And the were extremely expensive. And you {noun 95}ed your experiments effectively to extract the maximum amount of Information. Maximum reliable Information from experimentation. That’s probably what I’d go on with. That’s part of the Knowledge DB. This is the experience, this ???. Nowdays, because a {Noun 4}, a DP setting you can get it for a thousand bucks. To wire it into the computer is another thousand. So two grand for a point. Ah, we’ll go for more Data. You see {noun 60}s instead of 20 years ago, you might have a thousand points in a {noun 60}, now you might have twenty thousand. Y’know, we’ve got all this Data, but you think “are we doing anything with it?”  One of my [old] jobs [somewhere else] was condition monitoring and melting. “When are we actually getting to the {noun XX} ? When are they going to {verb XX}?” So they put in a condition monitoring system. My argument was always: “You need to watch this data ??? times ago, it’s your best Data.” “Oh, no, no, we’ve got in a conditioning monitoring system. “ That was the {group 130}. “And we can measure the Data.” But no one looks at it. To me we have this philosophy, oh we’ll just measure everything and hope that the analysis just trickles out some sort of sense. Whereas I come from an era where you {noun 95} the experiment and you have a good idea of how you’re going to extract it. You don’t really much else. It’s planning. You see that, we keep adding more and more {noun 97} {Noun 4}. And you ask the question of [other company], “what are you going to do with this?” “Something.” {noun 88}:”We might have a look at this thing.” but in the context of the whole operating campaign or at least the last year, you don’t know the reliability of the Information, you don’t see the built in trends, and you’re probably shooting in the dark. But you’ve got the Data points. Three quarters of the Data in the DB, you don’t know how it was [collected]. Just a change in the way businesses perceive things now. But we’ve got this and we’ve got that. But you’ve got to be actively worrying the Data. Pulling out the stories, making sense of it. And it’s a little bit upsetting, but that’s the nature of the game. You’d probably hear much of the same story from [name]. Who: “Oh, we’ll go back and have a look at the Data” Uhhh. You’ve got to start sorting it. They’re trying to make the stories out of the Data. Whereas, me, coming here, would probably {noun 95} the concept for the system and how we’re going to extract the Data, what we want to do with it, and you start from times of your own. The focus is on getting the {position 58}ing, and once the {position 58}ing’s there, we’ll be right. No, no, it doesn’t quite work that way. 

...

Interviewer: The {noun 60}, {Noun 4}, and infrastructure send to the {noun 97} DB...

Participant: Data. This is hard numbers.

Interviewer: This Data, these hard numbers, do we have a label?

Participant: {noun 97} variables. This is a measurement of {Noun 12}, or {noun 74}, or flow, or concentrate. 

Interviewer: Some variable of the {noun 97}. These {noun 97} variables, they go into the {noun 97} DB. Does anything else, in a general sense, go into the {noun 97} database?

Participant: There are probably operational items which do go in. Probably from switches, more than anything else. Start a cast. A binary {noun 111}, it’s on or off. Bang, that was the commencement of casting. 

Interviewer: Switches send to the {noun 97} DB what?

Participant: Status Information. Casting, not casting. Power on, power off. 

Interviewer: Is this flow of status Information, Data, Information, or Knowledge?

Participant: probably Data and Information. Some of it are numbers like ladle weight for instance. That’s a hard number. Start a cast, end a cast, that sort of thing is Information. It’s on or it’s off. It’s going to be a combination of hard numerical stuff, quantities, masses. And the Information is we are in this state, we are not in this state. It’s just, tick when we’re not casting, not running {noun 28}. And probably how many bowls of {noun 28}. That will probably be matched up with hard Data. In the sense that there will probably be a chemical analysis.

Interviewer: {noun 97} variables or something different? 

Participant: Something different.

Interviewer: Where is the chemical analysis here?

Participant: It’s probably another one of these {noun 37}.

Interviewer: From {noun 37} to {noun 97} DB?

Participant: Yes. Via the laboratory.

Interviewer: So we’ve got {noun 37} to lab. And we’ve got lab to {noun 97} database?

Participant: Yes. And that doesn’t come through this loop, it’s a completely different loop. It’s another group of {position 90}s out there. 

Interviewer: And they just input their things into the DB. 

Participant: Yeah, they’ll have “we’ve sent the {noun 37} to the lab, this is its number, bang blam bang.” All that is not observational, the {noun 37} is a physical entity, the Data that comes from that {noun 37} is Data. There’s no observation.

Interviewer: {noun 37} to lab, this flow?

Participant: is a physical {noun 37}. 

Interviewer: Would you say the physical {noun 37} embodies anything?

Participant: It embodies Knowledge. It embodies the actual chemical composition. 

Interviewer: And that, you would say, is Knowledge?

Participant: That’s Knowledge. 

Interviewer: So there’s a flow of Knowledge from the {noun 37} to the lab

Participant: Yeah, in the third party. The {noun 37} inherently contains Knowledge. Or is it Data?

Interviewer: You tell me.

Participant: Data. It’s Data. That will come down as a hard analysis and stuff. 

Interviewer: The {noun 37} to the lab sends Data. The label for this Data is?

Participant: Physical {noun 37}. And it embodies Data, hard numbers. 

Interviewer: Versus the physical body to the {position 141} which embodies Information. 

Participant: Information, yeah.

Interviewer: The lab takes this physical {noun 37}, the data flow of physical {noun 37}, 

Participant: Is extracted by the lab, and the lab sends to the {noun 97} DB, the embodied Data.

Interviewer: Is it a data flow?

Participant: I’d say it’s Data flow. They’re not doing anything with it. It spits out of the {noun 133} or the the spark. Bang, how much of these. No one makes an interpretation, it’ just a string of numbers. 

Interviewer: The lab sends Data to the {noun 97} DB of what label?

Participant: {noun 37} chemistry.

Interviewer: The {noun 97} DB takes these three things in, are there any other major components that it takes in. 

Participant: Apart from the time, no. 

Interviewer: Time. Does it take in time from something different?

Participant: It is probably a timing against its own timestamp. When something arrives, {noun 97} Information, it arrived at this time. A clock entity. 24 hour clock starting at midnight, standard {noun 97}. Anything happens, it starts at midnight. Why? that’s 00:00.

Interviewer: The clock sends to the {noun 97} DB a timestamp. 

Participant: Data.

Interviewer: {noun 97} DB. Now there’s interaction between {noun 88} and {noun 97} DB. The {noun 97} DB sends to control, what?

Participant: DB sends to {noun 88} Data. It is interrogated and it downloads number streams. 

Interviewer: So it sends a flow of Data to {noun 88} of what?

Participant: Numerical Information. {noun 97} variable values. 

Interviewer: What do we want to label that flow?

Participant: I would say Data. 

Interviewer: So it’s a Data flow of Data?

Participant: You’re getting me confused. It’s a data flow of hard numerical Information against predefined variables. So this is a ??? it’s its value. 

Interviewer: Do we have a more pithy name than hard defined Information against predefined values?

Participant: Just call them {noun 97} variables.

Interviewer: We have {noun 97} variables here. Are these {noun 97} variables different from the {noun 97} variables that {noun 88} gets?

Participant: No, they will be the same throughout. 

Interviewer: What about all this extra stuff?

Participant: I would call that a {noun 97} variable, in the sense that if you’re doing a chemical analysis, hot metal silicon, it’s like a {noun 97} variable. So they may not be instrumental values, but they would have a defined tag to define them. And it’s a number against the tag. In the same sense as the tag. 

Interviewer: Can we put an adjective before either of these things? Like stored or current?

Participant: The DB goes back goodness knows how many years. So it’s effectively stored and stored for some time. 

Interviewer: Do we want to say historical?

Participant: Historical. 

Interviewer: Historical {noun 97} variables go to {noun 88}.

Participant: From, from. Current time back. Then we can pull them out, and that goes back to {noun 88}. And it can also be interrogated by {noun 129}.

Interviewer: Therefore {noun 88} and {noun 129} need to be able to send stuff to the {noun 97} DB in order to interrogate.

Participant: Yes, that’s the program commands. I want to see this string for this time period of these variables. It commands...

Interviewer: These program commands are?

Participant: Information requests. 

Interviewer: Information requests is the category?

Participant: Yes. You send it a command and the command is to fill an Information requirement. 

Interviewer: So {noun 129} and program control send an Information request to the {noun 97} DB. The {noun 97} DB then sends historical {noun 97} variables to both, back.

Participant: whomever’s requesting it. ... Just on here, you sketch plan there, you’ve probably got from between ten and a hundred people interactions. people, man-machine interactions. You’re looking at ten people. {noun 97}ing, effectively on one shift, 30 people. the side loops, we’re talking generalities, not time based. 

Interviewer: What else do we have here? 

Participant: I think we need a bigger sheet of paper. There’s not much more I think you can talk about as a separate entity. This loop here, there’s probably four strings to it. You can talk about {noun 97} {noun 37}, you can talk about raw material {noun 37}, but I think they all fall into the same sort of concept. And there’s multiple streams for these sorts of things. Probably the next step is how that comes forward and that comes forward to the business given that they’re part of the business , these are one step behind the business . The barriers is, you can probably put it through there. For them, that’s the barrier. For {noun 129} the barrier has come back, because you’ve got this communication loop around here. 

Interviewer: {noun 88} to {noun 129}. 

Participant: This is where it starts to get hazy. And it gets hazy. There’s probably line {group 130}, which is the operational superintendent. His {Position 22}s

Interviewer: Everyone who’s doing work on the line

Participant: On the line, yep. Who supervise the {noun 97}, who probably have a commission to see the {noun 129} of the {noun 97}. This is the maintenance and money. This is getting up to the dollar signs of the business . The next one up here is {group 142}. This is ... line management would be [name]’s superintendent. Raw material {position 92}. The {group 142} would be like general {position}. Very very much removed. You’ve got to summarize it for them. When you go here, it’s a one page summary, and up to here is a one paragraph. 

Interviewer: Because they’ve got so much...

Participant: They’ve got so much, and they’re distilling down...

Interviewer: Any other big big entities here?

Participant: I don’t think we’ll go as far as the shareholders.

Interviewer: They go money...

Participant: Yes, but I think there are some shareholders that care about the businesses they invest in. I’ve invested in this business for ethical reasons. So I’m interested in what they’re doing. Yes, money is probably the root cause of it. People do carry interests about the business, probably some of the interest is “can I get my investment out of this?” and some of them, yes, I’ll invest in this business because of ethical reasons. Some of the {noun 129} work does filter through. This is from experience, some of this work does filter through into annual reports. It was greatly filtered through, I might add, but it does filter through, and it can happen.

Interviewer: Here’s a question. Are the secretaries of any of these bodies a different entity? Or equivalent gatekeepers?

Participant: Yes. They are usually the access point to a lot of these people. Especially when you get up here. And I’ve found that... my access to [person]... the access to him is through his secretary. Because if you go straight to him, whoosh, into the aether. Gone. No response to any e-mail. But you ring up [name], I need to do this, “right [name], Ok, when do you want to see [name] sometime on this day, this day, or this day?” So there are gatekeepers for the very senior levels of {group 130}. 

Interviewer: Are there line secretaries?

Participant: Not in the [company] organization, not in its current form. They can make your life easy or they can make it damned difficult. 

Interviewer: Flows.

Participant: There’s a big one here. This is the day to day reporting, direction... 

Interviewer: So {noun 88} sends reporting to line {group 130}.

Participant: Yes, they report on a day by day, problem by problem basis.

Interviewer: What shall we label that as? Are there two flows?

Participant: It’s a day by day. Day reports. “And we had this problem overnight, and this abnormal report, and I don’t know what the reason is. And the {noun 97} is ticking over. [elements] are down. We’re having problems with the {Noun 21}.” It’s all that sort of Information, Data. Some of it’s numerical, some of it’s observations. 

Interviewer: So it’s Information & Data?

Participant: Information & Data. With some interpretation. 

Interviewer: And the interpretation is the Information? 

Participant: Probably the interpretation is Knowledge.

Interviewer: Is that a separate flow?

Participant: They’re all in the one. Because these are usually [short term hires]. “I’m in this job for a couple of years and I don’t really understand what I’m doing.” These blokes, these are the long term campaigners. They carry all of the sort of history, the Knowledge, what’s been done in the past, sometimes we get it wrong. Most of the time it’s going to be pretty right.

Interviewer: {noun 88} sends the day to day reports to line {group 130}. These reports are Information, Data, and Knowledge.

Participant: Yes. And a whole host of abnormal event reports.

Interviewer: Separate flow?

Participant: They’re all embodied in the day to day Information that’s going across. What do we need to do? What are we going to get roasted on? It’s all this mass of stuff. 

Interviewer: It’s this embolism of reports. Line {group 130} to {noun 88}?

Participant: Are usually directives. What we would like to happen. Where do we need to push the {noun 97}? Why aren’t we... what’s the situation in the {Noun 21} salvage questions. And it’s directives. And in very rare cases there may be Information coming back.

Interviewer: Directives are a flow of?

Participant: Requests, Information?

Interviewer: Information?

Participant: Information. Someone says “I want this to happen in the ... {noun 60}.” 

Interviewer: And you say that that statement is Information?

Participant: In your classification scheme, sure. Mine would be a request. “We want you to do this. I want you to have a look at this problem.”

Interviewer: and this is a request that is neither Data, Information, nor Knowledge.

Participant: No, they’re requests for Information. 

Interviewer: in all of this, are there any that we need to change to match your classification scheme/

Participant: Probably only the one directing the {position 141}. But remember that request comes with a rationale.

Interviewer: So which one needs to be turned into a request?

Participant: Simple rationale. 

Interviewer: So that’s Knowledge and request?

Interviewer: That’s probably a request + Knowledge. “I have a reason for asking that, but I’m asking you to get a {noun 37}.” Yeah, it’s a request. You might almost say that the {noun 97} database is requesting. “I’ve got this hole in my DB, where’s the Data? Send me this number.” 

Interviewer: Who is the DB requesting?

Participant: {noun 97} {Noun 4} or... It’ll send out flags if it....

Interviewer: So {Noun 4} get a request flag?

Participant: Yep. I need a number to fill this box in the DB. 

Interviewer: Fill request? 

Participant: Entry request. You might also get the same from switches. It probably is also interrogating switches. Switches are also effectively a request.

Interviewer: This request, does it have a different label or is it the same flow?

Participant: The same sort of flow. Say the same flow. We’re getting too complicated. 

Interviewer: Over here, day by day reports, Information+D+K. Line {group 130} sends back directives which are requests. It also sends back questions? are they directions?

Participant: they’re questions too.

Interviewer: Is that a separate flow?

Participant: I think it probably is. “I need to know all about this.” as distinct from “You will make sure that this is being done.” “I would like to know about this.” slightly different.

Interviewer: These questions are Data, Information, Knowledge, Request, O?

Participant: I’d say it’s just request. There’s probably some Knowledge involved in making that request, but it’s not transmitted in making that request. Because the link to here: there’s something going on here that triggers some of these requests.

Interviewer: So let’s model that.

Participant: This is starting to get... this is conjecture. Beyond the simplistic bits here and to the development, this, you can only surmise what is going on.

Interviewer: So you believe that {group 142} has a flow to line {group 130}, via secretaries or direct?

Participant: Direct. Because these meet the {noun 97} people, usually once a day if not more. These people will probably have a meeting once a week or they may do the walk around and actually go and see people. That’s getting rarer and rarer too. 

Interviewer: {group 142} to line {group 130}, what’s the flow there? 

Participant: I think there’s flows in both directions, reporting on condition, “this week we made X number of tons, our ... production rate was such and such,” reporting. Usually about weekly is my sense. “You can’t see him, he’s going to the weekly meeting, or gone to a business meeting.”

Interviewer: So weekly reporting is a flow of? 

Participant: Probably more Information than anything else. The Data has dropped out of it. We’re now starting to just give the summary. Yes it is numbers, but we’ve reduced... it’s aggregation. I’ve reduced ten thousand individual data points into one number. All the analyses ... and there’s probably three or four hundred analyses. “The average fill rate was such and such. Our target was such and such. We’re below target, we’re above target. Why? Why?” It is now Information. The Data, the hard numerics, is gone. That number embodies the Information. Slightly probably different sense of... 

Interviewer: but it’s an important difference. {group 142} to line {group 130} sends what?

Participant: Probably the requests again. And the high level business Information. Request: I want {noun 143} to be instituted in the {noun}. {noun 143} is a new managmenent in [company]. Basically it’s “no paper on your desk at the end of the day.” You have a clean office, everything in its place. Have you ever seen a technical person that isn’t mounded up with paper? Me, I want to walk out the end of the day: “That’s where I got to this day.” 

Interviewer: It’s an external memory.

Participant: Bingo. I sit down. Alright, neurons start firing, we’re back into it. This probably comes from the phone center approach, where you don’t have the same desk every day. Wherever the phone is, that’s where your job is. You have a clean desk. I think that’s where it comes from. “You will institute {noun 143} at the mine site” It’s a directive, it comes down, it has to be done.

Interviewer: So basically what you do is you have a desk for management then you have an informal paper center?

Participant: we have another place where we actually go and work.

Interviewer: High level business directives? What would you use?

Participant: Directives that are coming down from further up the pyramid. There is also the Information. That would be the business Information. Kind of the last month, we’ve broken even. We’ve made a profit, we’ve made a loss, this has been the lossmaking area, so it’s that ethereal business Information. And the people down, once you get beyond the line {position 41}. “What’s that all mean? What does it mean on a day to day basis?” Very very hard to see.

Interviewer: This high level business Information is...

Participant: I think it’s Information. Is it Knowledge? No, I don’t think it is. It’s just got an immediacy to it. And people who just basically do what they’re told. The board has said this, we will do this. I don’t care how you implement it. I want want this done. It’s Information and requests, the Knowledge is probably very very subsidiary. ...

Interviewer: other flows?

Participant: {noun 129} to, we talked about this one.

Interviewer: Are there ones besides that?

Participant: {noun 129} which is where I am, yes, I do have a line to there in the normal course of business . And in my current sort of job, I have links to here to. Bing, bing, usually, because to see him, I’ve got to go through [name].

Interviewer: {noun 129} to {position 144} sends what?

Participant: I send, it’s a distilled version of Knowledge, Information, and Data. 

Interviewer: So you send a distillation...

Participant: I send a distillation embodying all of those things.

Interviewer: Distillation of what though?

Participant: The Data. I’m telling a story about...

Interviewer: We’re going to say story.

Participant: A story, a specific problem, event, a series...

Interviewer: This story is Data + Information + Knowledge?

Participant: It’s Data, Information, and Knowledge. It’s built up from my experience. It extracts Data, or it has a Data component, and I’m reducing it to some sort of Information. But they’re not independent. 

Interviewer: Does line {group 130} send anything to you?

Participant: Yes, it also sends directives as to what it... “I like this piece of Information, can you do more?”

Interviewer: Research directives?

Participant: Yeah. We have this problem, can you look at it? That sort of thing. They don’t know what the solution is or what you can do about it. “That report, can you do a photocopy of that too, and can you send it to someone else? Has this {position 41} been included in the circulation of this?” They are, what you say, organizational directives and directions in which way to produce investigations. “Yes, we like that idea, can you do any more.” That’s the one you want to hear and the one you hear the least of. 

Interviewer: These organizational and research directives are... 

Participant: Usually requests. And sometimes there’s Information. 

Interviewer: Is it Requests + Information, or just requests?

Participant: Requests + Information but requests dominates. Requests > Information. I think that’s a fair summary. 

Interviewer: {noun 129} to {position 145}..

Participant: is a request for an audience. “I need to talk to [name] about such and such” or “[name] wants to see you.”

Interviewer: So they are sending back to you a request for audience as well?

Participant: Yes.

Interviewer: So it’s a bidirectional flow.

Participant: Bidirectional, yes. But it’s only to get to see them. There’s no Knowledge, Information, or Data, in that statement.

Interviewer: So it’s a request flow for...

Participant: a request flow for audience.

Interviewer: So it’s audience and type of flow is request. Then {position 145} to {group 142} send what?

Participant: They usually look at their Microsoft diary and see where you’ve got to get.

Interviewer: So the {position 145} send the {position 130} their schedule. 

Participant: “You’ve got a gap there, [name]” “Yeah, I suppose so.” It’s that sort of...

Interviewer: this schedule is...

Participant: Information.

Interviewer: {group 142} to {noun 129} sends?

Participant: I think it’s like the {position 144}. It’s usually requests and Information. Usually the Information flow is large to them. So we’re communicating what we’re finding...

Interviewer: Findings?

Participant: Yep, yep. Where we are in the project. It’s a distillation, it’s the story.

Interviewer: These findings are?

Participant: Primarily Information.

Interviewer: Then {group 142} to you sends?

Participant: Probably requests and Information. Requests: “Look, this looks alright, we don’t want you to do any more on that, we want you to this.” Or, alternatively, “Did you know that someone else is doing this and, you really should talk to them.” So it’s a blurring of the two, but usually, if they’re happy with which way you’re going, whether they support it or keep going. It’s the Do/Don’t Do type...

Interviewer: So this flow is labelled... direction?

Participant: Direction and Information. “That’s very interesting but.”

Interviewer: Direction?

Participant: Yes, Direction. And the Information is: “Have you considered this?”

Interviewer: but that’s a function of direction?

Participant: Yes “We would like you to do this.” Effectively it’s Information, but really it’s probably another form of request to do something.

Interviewer: Shareholders.

Participant: There’re so far into the aether. But really it’s feeding through this line here. I think the Information is that way, as far as the technical is... If you have a look at the report... you know ...

Interviewer: Report?

Participant: Annual reports. Have a look at the [company] annual reports and you’ll see there’s a little spiel about their technology side of things. ... It’s part of our corporate profile and it gets a couple of pictures. In the report it’s usually half a page. It’s saying we’re doing these sorts of things, we’re looking at these problems. We’re trying to be proactive in terms of the environment. All those sorts of things. 

Interviewer: Do the shareholders send anything back?

Participant: Only if the business goes bad.

Interviewer: So it’s just a one way flow.

Participant: Question, has a shareholder said anything of a technical nature that has registered with the board? I don’t think so. Only if a {noun 97} is going really bad. ... But it’s probably only a thousandth of a percent of what the shareholders are communicating about. Institutional shareholders, it’s about the state of the business . But sometimes things will pop up. But it’s only a small dotted line. 

Interviewer: Technical question, concern?

Participant: Someone’s see something, someone’s asked the question. Some of those things might be via investor relations, and you really don’t know. But every organization has an investor relations.

Interviewer: It’s?

Participant: They’re probably trading on Knowledge, 

Interviewer: So they’re sending Knowledge?

Participant: No, they're using their Knowledge to generate a question/request.

Interviewer: So therefore, it’s a request?

Participant: More a request. 

Interviewer: Do we have any flows that we’re missing?

Participant: At this stage, I don’t know. 

Interviewer: The sum of rationale, activities, experiences, learning, and observation create Knowledge?

Participant: Yes.

Interviewer: Let’s have Information, Data, Knowledge, Request. What goes into Information?

Participant: Information is, It’s the technical story. It’s usually is... business Information, what’s the state of the business ? In a technical sense, Information is a distillation of these entities here.

Interviewer: Therefore it distills Knowledge?

Participant: It takes Knowledge, Data, observations...

Interviewer: Is that separate from Knowledge observations?

Participant: Observations build into Knowledge.

Interviewer: So Knowledge + Data... is there a function here that turns it into Information?

Participant: usually my brain. 

Interviewer: Your brain doing?

Participant: Generating the Information is like a jigsaw puzzle.

Interviewer: You take what and combine it with what?

Participant: You take Data and Knowledge and merging the two to generate Information. And it’s an incomplete set. These are incomplete sets.

Interviewer: So you have Knowledge and Data create Information. Is it equality or directionality. Can information be unsausaged?

Participant: Sometimes yes. You want to deconstruct the Information to get back tot he original concepts. So I take this Information and plot it. There is equality.

Interviewer: Whereas with Knowledge you take this and go.

Participant: Yep. Given that these are adding to it all the time.

Interviewer: This sigma is adding to Knowledge all the time?

Participant: Yes. I’ve found out this, and I’m going.

Interviewer: Data is?

Participant: Usually one off.

Interviewer: This, but no sigma?

Participant: Data is the hard numerical material to my way of thinking. It’s the experimental measurements, it’s the {noun 97} measurements. It’s the chemical analyses. It’s the things that I can get which are hard. A number which has a physical association, or a variable association. It represents some property. 

Interviewer: Does Knowledge, Information or Request have any function in Data? Or is it just an experimental measurement?

Participant: It’s just the experimental measurement.

Interviewer: We have experimental measurements. We combine experimental measurements with Knowledge to create Information. We have these perceptive things which are aggregated to create Knowledge.

Participant: Yes. And the request is “We want you to look at these things.” The request brings into play all those factors.

Interviewer: So a request is a one way function of Information, Data, and Knowledge?

Participant: Yes. You are asked, or you ask yourself, “I would like to know about this.” And then the higher factors come into play. “How do I get an answer out of this problem? “and it’s the jigsaw puzzle. Consider a 500 piece jigsaw puzzle. For most of our games, we’re making the story on about 25 pieces of Information. In a really really good situation we might have 100 pieces to actually put the story together. But usually, because it’s in a business nature, we’ve only got 5% of the bits of the story to try and piece it together. 

Interviewer: Would you say that this accurately corresponds to what you consider the ontology of Data, Information, and Knowledge?

Participant: I think it’s pretty close. 
\stopextract


\stopcomponent
