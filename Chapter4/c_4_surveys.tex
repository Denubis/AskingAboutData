\startcomponent c_4_surveys
\product prd_Chapter4
\project project_thesis

\section{Surveys}

To demonstrate that the results of the research were not a fluke of my methodology, I then decided to run a set of surveys to explore another perspective on the reality of data. This section documents the results of my second survey attempt and is a check on the conclusions drawn from my interviews. Many different participants participated in this survey, being drawn from academics, \IST\ professionals, other workers at BlueScope, and intelligence workers from the United States civilian and military intelligence establishments. 

Survey analysis proceeds in a roughly straightforward fashion. I discuss interesting aspects of each incoming survey, and try to derive their conception of data by looking at how they classify data, information, and knowledge. Their classifications and the relationships they articulate between them may give some clue about each participant's conceptions of data. Less time will be spent on each individual survey due to their less comprehensive nature. The terms \quotation{participant} and the singular \quotation{they} will continue to be used to indicate the survey-taker. 

These surveys were the first deployment of my survey methodology that resulted in any kind of success. As such, the instructions were not close to perfect, with a few survey-takers misunderstanding the instructions. I state this with certainty as one participant (name withheld, but permission given) had the following to say in an e-mail:

\startextract
Also, I had a hard time with the questions. About midway through the survey, I realized that some of the questions were looking for a semantic understanding of the meaning of the highlighted words. I began the survey looking at it through an analyst's eyes, with the understanding that data coming in only becomes information or knowledge if it reveals something about the intended target -- usually a person or group of people (network). As I went along, I realized that you were differentiating between data, information, and knowledge as semantic definitions that all described different types of incoming \quotation{media} or \quotation{communications}. I got the impression that you viewed data as something measurable, information as something that feeds understanding, and knowledge as something that is either factual or knowable. Since these are pretty disparate understandings of the terminology, it was difficult to reconcile my answers between the two methods of differentiating the data types.

I offer this as perhaps an anecdotal story about the problems with this kind of question. Like so many issues surrounding understanding within a community (e.g., religious doctrine), the real question here is perhaps not so much a matter of different perceptions within a subculture, but whether or not the subculture has taken the time to clearly define their terminology to enable effective communication. For more \quotation{tangible} elements of conversation, like computer design, engineering, or medicine, the subcultures tend to build a sophisticated lexicon of industry jargon very quickly, because the objects described (whether processes, methods, tools, concepts, or physical things) are easy to understand. With subcultures that focus on more ambiguous topics (freedom, truth, spirituality, goodness, guilt, culpability), the clarity of their jargon becomes diluted by variation in understanding of the meanings of the subculture's terminology. You (purposely?) do not define what information, data, and knowledge mean in your survey. Therefore, these terms will carry a wide variety of denotations and connotations with them. The implication is that effective communication within the intelligence subculture cannot begin until the subculture's definition of these terms is made explicit.
\stopextract

Therefore, the products of this analysis should be taken with a grain of salt, and used to suggest that far more work can be done in this area, as opposed to presenting any kind of definitive or reproducible look at the reality of data. 

As stated in the methodology\footnote{See page \at[MethodSurvey].}, the surveys allowed participants to self-identify by role and then classify a series of scenarios\footnote{See Appendix A.} as to being data, information, knowledge, or other. The self-identification of role will be communicated in summarized form in each survey below. The summary will be generalized to preserve confidentiality. From each survey, I will abstract the most interesting explanations given of categorization. Included in each summary will also be a description of the general proportions of the answers. Each participant not only has their own conception of data, but their own rationale for identifying things as Data, Information, Knowledge, or other. The analysis will also briefly touch on the nature of each participant's categorizations of the O, and how they may influence their construction of data. 

Each introductory paragraph will contain a count, as noted, of {\em X} data, {\em Y} information, and {\em Z} knowledge. Although it is possible that interesting statistical observations may be made from the relative proportions of classification, the main items of interest in my research were the participants' explanations of their decisions, rather than the quantization of the decisions themselves. In addition, the number of surveys is not sufficiently high for any statistically useful generalizations. As ever, this research is just laying the foundation for more intensive research, and I feel that this survey structure could form the basis for an excellent long-term study of many people's conceptions of data and how they change over time. 

\subsection[Survey I]{Survey I}
The participant is working in the database labs of a university. They classified most scenarios as information and knowledge, with the counts of the various classifications being: 3 Data, 11 Information, 10 Knowledge, and 2 other. Participant was careful to classify other scenarios as a \quotation{lack of data} and \quotation{ambiguous} that increases the reliability of prior categorizations as it demonstrates that participant is not afraid of novel categorization. 

Data is a symbol without meaning. Some quotes:
\startitemize
\item  \quotation{This doesn't necessarily contain information and can be simply random.}
\item  \quotation{A design may or may not convey information.}
\item  \quotation{[secret] Code, in this case, is data since it does not have information. However, if it is decrypted, it becomes information.}
\stopitemize

The participant presents a novel interpretation of data. One explanation of the unusual sense of symbol without meaning is that the participant's role is sharply different when compared to the other interviews. Furthermore, when considered in light of databases, a purely technical understanding of data could support the symbol-without-meaning interpretation if the participant does not assign computers the ability to manipulate meaning. 

Information is classified as \quotation{anything with meaning}:
\startitemize
\item  \quotation{Information, since data doesn't necessarily have a meaning and knowledge is factual/practical information. A letter, therefore, can be knowledge, but is normally just information.}
\item  \quotation{'Locations of parts' would [be] basic information. The data has meaning, however it does not detail practical or factual steps of information.}
\item  \quotation{The data is known to be a planning program, so it has meaning.}
\stopitemize

The notation of meaning in information has links with the information-as-communication theme found in some interviews above, but suggests a purely semiotic boundary: symbols-with-meaning versus symbols-without-meaning, without any kind of communicative subtext.

Knowledge is classified as practical information. Some quotes:
\startitemize
\item  \quotation{This is something that contains factual/practical information; i.e., knowledge.}
\item  \quotation{Toughie; the story pertains to steps detailing a story, therefore it has chronological sequence (i.e. - steps, procedure), so it is knowledge.}
\item  \quotation{Step-based, procedural information is knowledge.}
\stopitemize

This position is an interesting one to take, inasmuch as knowledge is a subset of information that relates to action or time. Knowledge is anything with practical meaning, or in other words, anything that is actionable.

The participant belongs in the data-as-communications camp, assigning computers the role of data processors while granting a privileged human-centric view of information and knowledge as symbols with meaning (and utility). Although the relationships between data, information, and knowledge are unusual, they do not present any unexpected new conceptions of data. 

\subsection{Survey II}
The participant works for a defense department as an information processor. They present a strong bias towards classifying scenarios as data, with 11 as Data, 6 as Information, and 9 as Knowledge. They did not take advantage of the other category. One hypothesis is that as an information processor they have a very strictly operationalized definition of information. Still, the lack of any categorization of other suggests that Data or Knowledge were used as catchalls. 

Data is an unanalyzed sign, set of signs, or communication. Some quotes:
\startitemize
\item  \quotation{No analysis or recommendations for action are involved, leaving this as simply data.}
\item  \quotation{The code has not been sorted, so it is still data. The fact that the data is encrypted offers some insight for analysis, so that may be used to create information from this situation.} 
\item  \quotation{Lacking detail about what the letter contains, I would default to the lowest category of analysis, and assume it to be data for the time being. I can be sure that it is data, but cannot determine from the description given if it is information.}
\stopitemize

While \quotation{data as a fundamental building block of information} is unambiguous, the attached semiotic definition results in an uncertain hypothesis. The interesting phrase, \quotation{Lowest category of analysis,} suggests a strict and articulated set of relationships among data, information, and knowledge. The very strictness of the hierarchy, of course, means that accessing it is quite difficult: much of the participant's knowledge about data is tacit and assumed, making it difficult to extract from their explanations of reasoning. 

Information is analyzed data:
\startitemize
\item  \quotation{The data of how the user interface worked is changed to information by the analysis of the users, categorizing things into \quote{good,} \quote{bad,} \quote{useful,} etc. However, this does not include recommendations for action, and so doesn't quite go as far as knowledge.} 
\item  \quotation{Quotes are chosen to reflect a given point, and so are offered with inherent analysis, by the questioner, the respondent, or both. }
\item  \quotation{Poetry, while perhaps not designed as traditional analysis, does offer insight, and information-at least about the poet. The poet has analyzed words, scansion, rhythm, etc., changing raw data (perhaps Dadaist poetry?) into traditional poetry.}
\stopitemize

Analysis of data creates information. However, if we explore the meaning beyond analyzed data, defining the term becomes quite tricky. It is easy to exclude knowledge as actionable information, but the very specificity of the term \quotation{analyzed data} precludes most understanding. In some ways, the meaning of information could be as broad as information as statements about nouns in the world.

Knowledge is information applied to decision making. In many ways, this classification echoes the knowledge of Survey I. Some choice quotes:
\startitemize
\item  \quotation{Bob has used information (prior experience with traffic noise) to make a decision on how to behave, transforming his experience into knowledge-knowledge that he uses to judge what sounds to ignore.}
\item  \quotation{Charlotte has not only analyzed information on behavior (making it information), but has taken that a step farther, into knowledge, based on its applicability.} 
\item  \quotation{Dave is providing not only analyzed data (information), but is doing so to empower action (a change in answers for the next time Eve encounters the situation, on a test or in her outside life), changing it to knowledge.} 
\stopitemize

As knowledge is applied information (similar to Survey I), knowledge must be actionable information. This relationship is a reasonable derivation of Ackoff's hierarchy, put to different ends. While Ackoff's hierarchy generally has knowledge-as-model, another understanding of model could be \quotation{information with practical consequences.} This view necessitates filtered information, as information contains elements both with and without practical consequences. By assigning practicality to knowledge, one can relegate the elements of pure-theory without application  to information until they are filtered and contextualized by use. 

\subsection{Survey III}
The participant is a manager of a consumer electronics repair workshop. They gave a normal sorting of responses, with 7 Data, 11 Information, 6 Knowledge and 2 other. The Os were identified as structure:
\item  \quotation{If it is a program only with no facts or data, perhaps this is just something that transforms data into information.} 
\item  \quotation{I think the empty document is something that assists in organizing data into information, but in itself is just a structure.} 

The wording on these strongly suggests a standard hierarchy. Furthermore, the relationships allow transformation: data becomes information becomes knowledge. 

The participant's definition of data is slightly more problematic. Although data can nominally be defined as a statement without structure, there is little evidence to indicate what data is a statement {\em of}:
\startitemize
\item  \quotation{I'd say this is data, as the facts have no relevance on their own. I'm not entirely clear on whether it is data or information, though.}
\item  \quotation{This one is interesting, also. I'd see each individual mp3 file as information, but the group of them together as data, as it feels like more of a random sample than a structured grouping.} 
\item  \quotation{The randomness and lack of interpretation leads me to think this is data.} 
\item  \quotation{The quotes themselves are data until they are processed, at which time that would become information.}
\stopitemize

The first quote suggests data are facts, but does not indicate the participant's understanding of fact. The second quote belies the traditional hierarchy by placing a set of data above information, though the trend of randomness or \quotation{lack of structure equals data} is still present. Thus, in my analysis, I have to default to the description \quotation{statement} without further elaboration due to the participant's answers. 

Information is analyzed data:
\startitemize
\item  \quotation{Now having the structure to put the information into, I think it is now information.} 
\item  \quotation{Data with structure.}
\item  \quotation{This is a statement with clear intent and purpose, so I'd say the demand is information that the mugger is conveying.} 
\stopitemize

In keeping with the traditional hierarchy, information is data with structure or data post-analysis. The last quote also implies some hints to information-as-communication as well, though that meaning is overwhelmed by the data-with-analysis usage.

Knowledge is understanding:
\startitemize
\item  \quotation{I'd say this is knowledge, as she has taken the information and used it to form an understanding of the situation.} 
\item  \quotation{Someone's impressions of something would be the conclusions they've drawn from the information, so I'd say knowledge.} 
\item  \quotation{That selection is a conclusion based on understanding of what he is doing.} 
\stopitemize

From these quotes, the participant suggests that analysis of information becomes knowledge. Knowledge as understanding or conclusions does not suggest the use to which that knowledge is put, however. 

This survey presented difficulties due to the fuzziness of the participant's explanations. While there is some evidence of a standard hierarchy, the relationships between the terms are poorly articulated. Here the survey did not provide enough guidance to extract from the participant their ideas on how data, information, and knowledge relate to one another. At the very least, this survey presents evidence of how deeply Ackoff's theories have penetrated the \quotation{white knowledge}\footnote{White knowledge is the everyday knowledge that \quotation{everyone knows} but that doesn't need sourcing or attribution due to its sheer ubiquity and acceptance. I believe Sir Terry Pratchett extended the term \quotation{White Noise} for white-knowledge, capturing the same sense of randomness and lack of coherence.
\cite{Pratchett2002} } of society. 

\subsection{Survey IV}
Participant is a social entrepreneur. Their self-description of their role emphasized management and planning skills with a correspondingly high information consumption. Participant overwhelmingly classified things as information, with 5 Data, 13 Information, 5 Knowledge, and 3 other. The use of other suggests a willingness to differentiate, although it seems that from descriptions of what they wrote, some of the categorizations are in error: a few times participant categorized something as information but explained it as data. Participant differentiates emotion from the traditional hierarchy, as they categorize both mugging and poetry as flows of emotion. There is insufficient evidence to articulate how emotion relates to data. Despite these problems there seems to be some evidence for the traditional hierarchy. 

Data is an observation: 
\startitemize
\item  \quotation{Random noise can be data.} 
\item  \quotation{Data only-no context or filters to apply.} 
\item  \quotation{Encrypted information is structured data without a context for Alice to extract relevant knowledge from.} 
\stopitemize

Data is filtered and structured by knowledge, thereby producing information. The participant understands data as subjective observations, as indicated by the keyword \quotation{filter.} These observations are filtered and contextualized into information, in keeping with Tuomi's cyclic hierarchy. This set of responses matches interviews III and IV quite well, suggesting some commonality of background. Most importantly, it suggests that the realities of data explored in those interviews were not flukes, as the \SDFN\ also produced these results. This minimum demonstration of reproducibility is heartening as it validates both methodologies. 

Information is comprised of filtered, structured, and relevant communications:
\startitemize
\item  \quotation{Instructions are information, organized data that conveys....} 
\item  \quotation{A short story is data filtered through the author's perspective.} 
\item  \quotation{An empty word document is information because it has structure and that information may be the presence of structure only.}
\stopitemize

These quotes suggest that structure, while necessary, is a component of the analysis performed upon data to create information. 

Knowledge has explanatory force. It entails conclusions and explanations of the world. The participant is clearly using knowledge in the traditional understanding-the-world context. 
\startitemize
\item  \quotation{The key is knowledge that once learned creates a new set of filters for Alice to perceive the data through.}
\item  \quotation{Dave is passing on the knowledge of why one procedure is right or wrong.}
\item  \quotation{Design is choice, choice involves knowledge.} 
\stopitemize

However, knowledge is sharable and a way of creating context. The context-creation component of knowledge in many ways suggests Tuomi's cyclic hierarchy rather than Ackoff's hierarchy in the way that knowledge plus data create information: \quotation{the key is knowledge that once learned creates a new set of filters for Alice to perceive the data through.} Knowledge is also a way of generating choice.

\subsection{Survey V}
This survey, produced by a research scientist, was incomplete and very terse. The participant categorized six entities as Data, six as Information, one as Knowledge, and one as other. The instance of other is \quotation{probably both data and information,} which tells us little about whether the participant was willing to explore the non-data categories. 

Data are observations of the world:
\startitemize
\item  \quotation{At this stage the numbers and times are data. Study and interpretation of these will lead to information or knowledge about the thing that is unknown.}
\item  \quotation{The noise is a meaningless stream not constructed into anything more significant.}
\item  \quotation{Just listening to a live music stream is receiving data. Gaining pleasure from the story being told by the data stream is a construct inside one's head and is producing information or indeed knowledge of a larger portion of music, e.g. the symphony.}
\stopitemize

There is no evidence for any required activity with respect to the observations. The identification of noise as data suggests that there is no intentionality associated with the data. Unlike some of the interviews, wherein data is the product of sensors and experiments, noise as data suggests that data is inherent in the world, and that the observer's knowledge is what allows for filtering and contextualization into information. 

Information is contextualized data: 
\startitemize
\item  \quotation{A collation of data within a context, not just random.}
\item  \quotation{At one level the planning program might be considered to be knowledge in itself. However, it is information at the level before the unknown thing is revealed.}
\stopitemize

In this instance, the differentiation between data and information is context. The participant applies the term information to the planning program without exploring the term. They give no explanation for how information in the planning program is both unknown and contextualized. 

Knowledge is understanding: 
\startitemize
\item  \quotation{This requires an understanding of the machine as a whole, i.e. a map, so is beyond data and information.}
\stopitemize

The use of the term \quotation{beyond} suggests the traditional hierarchy, but tells us little beyond the simple use of the term understanding. The single categorization of knowledge and the paucity of responses in this survey make intuiting their conceptions of data problematic.

\subsection[Survey VI]{Survey VI}
The participant is a service desk employee for telecoms. They place an unusual emphasis on the term information, with 5 Data, 16 Information, 3 Knowledge, and 2 Os. They note that an experience is not knowledge, nor can art be classified as data, information, or knowledge. 

Data are structured records:
\startitemize
\item  \quotation{The quizzes are on relational algebra, which is more data-driven than say a quiz about philosophy.} 
\item  \quotation{Data because it is a record of data. A record of few parameters. Day + temperature outside her apartment.}
\item  \quotation{Lists of things are perfect data, to me.} 
\stopitemize

Data seems to be a subset of information: structured records containing statements about the world. However, the classification of quizzes as data is strange, although the participant may be responding more to the topic of quizzes than to the entities themselves. Thus, information with structure is data. There are no suggestions as how to utilize either with knowledge.

Information is a statement about the world:
\startitemize
\item  \quotation{What did Alice receive? A letter. The fact that it was a letter is a piece of information.} 
\item  \quotation{The files are information.} 
\item  \quotation{Bob ignores the noise. This is a piece of information.} 
\item  \quotation{What did the mugger do? Demand his wallet and watch. I don't see this as data, it is not a few parameters like a temperature chart.} 
\item  \quotation{I would say this is information. Yes, a lot of knowledge is behind the information, but such is the cases of a lot of information-there is knowledge behind it.} 
\stopitemize

In the last two quotes, the participant defines the boundaries of information using the example of the temperature chart for data and \quotation{background knowledge} for knowledge. In many ways, all the examples are simple facts without relationships and the facts are statements about the world.

Unfortunately, the few answers the participant gave for knowledge do not provide a useful definition:
\startitemize
\item  \quotation{The parts are at different locations. Where they are is a piece of knowledge.}
\item  \quotation{A lecture is knowledge-based. It is not merely a list of items. It is not a list of information.} 
\stopitemize

While there is a suggestion that knowledge consists of factual or true statements about the world, there is no way to reach that generalization from these responses. Furthermore, as the participant uses the term \quotation{fact} when describing information, knowledge has to be something beyond simple fact. 

\subsection{Survey VII}
The participant self-identifies as a corporate strategy manager. They gave balanced categorizations to data, information, and knowledge, with an extraordinary 9 other versus 5 Data, 6 Information, and 6 Knowledge. They used the other category to delineate art, wisdom, and lists. Of all of the surveys so far, I am the happiest with this one because the high incidence of other categories suggests strict definitions for the three categories of interest.

Wisdom structures knowledge, creating behaviors:
\startitemize
\item  \quotation{A learned behavior from a past experience.} 
\stopitemize

Some of the scenarios were not even data, one categorized as a list or sequential phrases:
\startitemize
\item  \quotation{Not even data because there is no metadata to give context to the list.} 
\item  \quotation{I found this a tough one. At first I thought information however because the context in unknown it doesn't fit with my view of what information is. I also don't think this is data because it is not observational and again lack context. The best thing I could think of was sequential phrases. You know each phrase is in a specific order but you do not know the intent.} 
\stopitemize

Data are observations without interpretation: 
\startitemize
\item  \quotation{I think it's either information or data because the code is just a transformation of the content to maintain security. Once transformed the content may be contextual in which case it's information or observations which suggests data.}
\item  \quotation{It's just a list with no interpretation attached.}
\stopitemize

The first quote indicates observations are data; the second indicates a lack of interpretation. Interpretation-as-context suggests a normal hierarchy.

The participant indicates an understanding of information as context:
\startitemize
\item  \quotation{Without any descriptors as to the type of letter the word letter has an ambivalent connotation about it. There is a contextual element to a letter that renders the content more than data but not necessarily knowledge which to me suggests value and worth retaining.}
\item  \quotation{I think it's either information or data because the code is just a transformation of the content to maintain security. Once transformed the content may be contextual in which case it's information or observations which suggests data.}
\stopitemize

The indication of context {\em as} information is slightly unusual because the usual formulation is data {\em with} context. However, information as the contextualizing element is presented well enough here to leave little ambiguity as to how they think.

Actionable responses are knowledge:
\startitemize
\item  \quotation{Because the location is critical for the part to perform its role in the larger machine. The location and the part are linked and relative to another part and location. If you just had the concept of the machine and location, you would only have data.... If you understood how each part worked you would have information.}
\item  \quotation{The secret key provides an actionable response to accessing the coded content therefore it is knowledge.}
\stopitemize

The precision of the answers is fantastic, especially with the examples that the participant gives. They clearly indicate that actionable responses are knowledge and give examples of how the relationship between categories progresses to said action.

\subsection{Survey VIII}
The participant is a counter-terrorism analyst who reads many messages daily, exploring trends for counter terror issues. They identify 6 Data, 10 Information, 6 Knowledge, and 4 other. The participant used the other category when unsure, rather than inventing new categories. They believe that live music and other entertainment do not provide data in themselves, and that entertainment is a minor and irrelevant category: \quotation{one cannot have true knowledge about something when the results are deliberately trivial.} They articulate Ackoff's hierarchy, but seem to employ a cyclic hierarchy in practice.

Data are objective records of activity:
\startitemize
\item  \quotation{Dave has provided himself a reminder, which is a type of data; or he has uploaded a format with appropriate font, pitch, etc., that he needs to use (perhaps he works for a bureaucracy that is very particular about these sorts of things). Because it is empty it cannot be more than data.}
\item  \quotation{Impressions of an interface are simply impressions; the aggregate will be information. It is not }data\quotation{ because impressions are not data, data does not allow of emotive responses. But the impressions are not knowledge or wisdom, they are merely impressions.}
\item  \quotation{Receiving an email is merely a fact; the email could be anything from a joke to a sales proposal to a social invitation to collaboration on a project. the fact of receiving the materiel is unable to be evaluated without content and context.}
\stopitemize

Data are factual records that cannot contain subjective elements such as impressions or thoughts. Entertainment, as stated before, does not produce data, but the record of someone consuming entertainment is data, because it is an objective fact about their person. Furthermore, all facts must be about activity of some sort. The participant is not interested in scientific facts per se, but in records of actions that can be contextualized into information and analyzed into knowledge about the target. 

Information is data with context and assessment:
\startitemize
\item  \quotation{As stated before, the context provides so much about m.o., etc., that this is more than mere data, but because there are critical parts missing or not understood, this is only information.}
\item  \quotation{This is raw data regarding what was watched and when. There is no assessment of the data nor is there a greater purpose that is being described. There is no wisdom, knowledge, or information from this, only data.}
\item  \quotation{There is no historical context as to why this methodology was chosen, so this is not \quote{wisdom,} and because it is a set of instructions, this is more than data.}
\stopitemize

The assessment of the data may be performed by knowledge, but is never explicitly stated as such. If knowledge contextualizes data, then participant is using a cyclic hierarchy, wherein knowledge contextualizes data, which in turn produces information, which can be analyzed to knowledge. 

Knowledge allows someone to make predictions about the world:
\startitemize
\item   \quotation{Prediction moves beyond data and information and -- presuming the analysis is correct -- gets to real knowledge.} 
\stopitemize

From an intelligence standpoint, this makes sense: predictions about the world are one of the ultimate intelligence products. Thus, the outcome of knowledge is a prediction. The nature of a prediction differs from the usual knowledge models of data-as-facts constructions. Although the participant indeed considers knowledge to be models, the nature of a prediction is such that it is an exported and actionable model upon which {\em O} people can base their activity or lack of activity. Of interest here is the traditional view that enemy actions can communicate volumes about their intelligence sources and models of worlds, as unlikely actions may indicate the presence of a leak in one's organization. The ability to make predictions that are given to other people, therefore, is tempered with far more severe complications than physical models of the world. 

\subsection{Survey IX}
The participant is a housekeeper. In this job, they have a low exposure to workplace-enforced understandings of data. The participant has a very unusual categorization: 7 Data, 5 Information, 4 Knowledge, and 10 other. The high rate of other represents a lack of broad categories. In a sense, this represents a lack of formalization from learned patterns.

The participant makes a brief mention of data as technical-communication. In other words, they consider data-as-bits a separate use of the term data:
\startitemize
\item   \quotation{Computer data-a different use of the same word.... Information, in this case, music, stored on computer hardware.} 
\stopitemize

In this, they also conflate the term with information, which is a common usage. However, this separate meaning should not be confused with a conception of data as data-as-communications. While using data as a term for computer records is certainly possible among all of the three realities of data, those who use it as their primary construction elaborate upon it more; the idea of it being a side definition never really occurs in practice, there {\em usually} is a seamless transition in usage depending on context. However, this area is another in which far more research will be needed before it is possible to make any definitive statements, especially the exploration of the use of the two terms, data and information, to mean the same thing. 

Data is a factual scientific observation:
\startitemize
\item  \quotation{My father used to talk to me about the difference between facts and opinions. Short quotes have the potential to contain either, but will more likely be composed entirely of opinions than of facts. Why would anyone collect quotes from students reciting facts? Chances are, the information would already be known, and be of little interest to anyone watching the campus TV station.}
\item  \quotation{This is data of a specific kind collected over a week by observation-albeit by the subject of the observation himself.}
\item  \quotation{The answers given on the quizzes are collected data about the knowledge and correctness of the people who did them.}
\stopitemize

These factual observations are required to have a strong correlation with reality. With the prior term of \quotation{computer data} specifically excepted in their comments, they consider data to be inherently factual and representative of the world.

Information is a statement carrying meaning about the world:
\startitemize
\item  \quotation{A completely empty document contains no information-it is an absence of information except for the fact that the document itself exists. Now, this may in some circumstances be useful information, but the document itself is devoid of meaning, since it's completely empty.}
\item  \quotation{Poetry is an expression of emotion rather than knowledge, confirmed information or collected data most of the time. While one's emotional response to winter wind may be important, and interesting information, poetry about it is in a different class than other forms of information about wind. Its purpose is emotional rather than informative.}
\stopitemize

Information appears to be a fundamental component of data and knowledge, serving as a general communicative vehicle. It also has some elements of meta-data, as the presence or absence of the document is termed \quotation{Information.} More research is required to see how significantly different professions interact with data, information, and knowledge. 

Knowledge is formalized, reliable instructions for activity:
\startitemize
\item  \quotation{Reasons behind an answer could be knowledge or information - it doesn't really seem certain enough for me to call knowledge.}
\item  \quotation{Lectures about database design... or other academic lectures are to confer built-up knowledge from one person to Os. It's also information.}
\stopitemize

Knowledge is transferable, certain, and constructed. The phrase \quotation{built-up knowledge} suggests a subjective, constructivist view of knowledge. Knowledge is viewed as a subset of information because something can be both knowledge and information at the same time.

\subsection{Survey X}
The participant is a senior software architect and acts as a programmer and database modeler. They classified 12 Data, 10 Information, 4 Knowledge, and no other. They make a strong differentiation between data and information, but isolate out knowledge as a privileged case. 

Data are electronically stored observations:
\startitemize
\item  \quotation{A letter is akin to an email which is electronic data. A letter in this case ins handwritten or typewritten data.}
\item  \quotation{This would be digital data. Data that is store on a given medium, in this case, a flash drive.}
\item  \quotation{Live music is data. It is sound waves traveling from the instruments to the user's ear.}
\item  \quotation{This contains specific data points, meaning the temperature outside.}
\stopitemize

While there are strong elements of data as bits, there are also elements of data as observations, which suggests that the participant uses both ideas simultaneously inasmuch as data are stored observations. It seems that collecting data is a database-centric act, i.e., data is something which is stored in a database. There are no semiotic overtones or indications of stored meaning. 

Information is a semiotically significant transfer:
\startitemize
\item  \quotation{A short story is information. Generally, I would think of this as a fun trivial kind of information that is meant to be temporary and enjoyed at that given moment.}
\item  \quotation{This is information, signaling the mugger's intent.}
\item  \quotation{A program is a set of code that will run on a computer. It provides information.}
\stopitemize

There is a component of intentionality to the transfer, but it is any kind of communicatively significant act. 

The participant identifies knowledge as reasons behind interaction with the world:
\startitemize
\item  \quotation{The instructions are knowledge. They are specific pieces of knowledge that Bob is giving to Alice.}
\item  \quotation{This is a knowledge transfer from the professor to the students.}
\item  \quotation{This is an educated decision on what to pick and why.}
\stopitemize

Knowledge as the basis behind the ability to make decisions can then be extended as the ability to make predictions. Knowledge as decision-making is exploring possible future outcomes and making choices that enable a desired visualized outcome, based on the constraints of the search. 


\subsection{Survey XI}
The participant self-described as someone who makes an action plan for a business problem. They identify 10 Data, 9 Information, 5 Knowledge, and 2 other. Os are categorized as \quotation{don't know.} The participant makes a strong differentiation between information and knowledge. 

Data are objective, precise facts:
\startitemize
\item  \quotation{Definitely data. This time chart can be analyzed and information derived from it. A time chart requires no interpretation.}
\item  \quotation{Definitely data. A list. An objective measure such as temperature. You can use it to make information.}
\item  \quotation{Not sure why but I feel this is data. Guess it being files with objective stuff in it (like a digital 1/0 recording) it is data. But I could be convinced in some part it was information depending on the content of the files.}
\item  \quotation{Again, not sure but I feel this is data. Hard to put this question in a work context and }live music\quotation{ is sort of an emotive term. Music requires no interpretation (maybe lyrics do....) and is objective (i.e.. you can follow music notes on a sheet).}
\stopitemize

Data are objective and listable components of information. All data are factual and obvious to all observers without need for interpretation. 

Information is the communication of unambiguous elements such as procedures: 
\startitemize
\item  \quotation{Imagining a situation where I receive instructions on how to take a sample for the task force. Being a procedure, I will call it information. It will be detailed but designed for me to follow, not to interpret. When it doesn't need interpretation, then it is information.} 
\stopitemize

Although the participant differentiates between information and knowledge, there is the strong suggestion that they are similar. The participant articulates an objectivity-subjectivity spectrum: data being the most objective to knowledge being the most internalized and subjective. 

Knowledge is a framework of learning designed to situate problems: 
\startitemize
\item  \quotation{Imagining a scenario where I receive a short technical report to read on the task force. ... This is knowledge. Detailed, contextual and requires interpretation. I can learn something from a report and use it in my own framework to view similar problems.} 
\stopitemize

The act of interpretation indicates that something is knowledge, whereas information is far less subject to interpretation, though there are some suggestions that information is used to build context.

\subsection{Survey XII}
The participant is a research engineer who assists with troubleshooting and does basic research. They provided a standard distribution with 7 Data, 13 Information, 3 Knowledge, and 3 O, with the Os noting categories of music and filtering. Unusually, the participant noted that experimental design was a process. 

Data are specific observations of phenomena as well as stored bits on a computer:
\startitemize
\item  \quotation{Specific observations of events |-| data. No abstraction or generalization.}
\item  \quotation{Data |-| the individual bits that refer to the specific makeup of the songs are data - Its file structure and formatting is information based on the mp3 standard |-| which is information.}
\item  \quotation{The computer sees it as stored data |-| it is also David's preparation so far saved as one Datum.}\stopitemize

There is no need to filter scientific data because they cannot occur without intentional observation and measurements. The duality of data as observations and data is very hard to reconcile. While it seems that data-as-bits is a side definition, there is no evidence to blithely dismiss it. Participant also does not indicate that data are objective, merely that they are specific, intentional observations. This answer illustrates one of the difficulties of the survey: the inability to follow up on interesting leads that are ambiguous. 

Information is a set of analyzed data that can create predictions: 
\startitemize
\item  \quotation{Information made up of pieces of data} and that it is in some way a description of the world: \quotation{The demand is information passed to Bob describing the state of the muggers mind and his potential, this info is based on the muggers generalized knowledge that Bob probably has a wallet |-| the specific demand is too abstract to be data, but too specific to be knowledge. } 
\item  \quotation{Contains some data i.e. measured temperatures etc., but usually has information as well |-| i.e. predictions} 
\item  \quotation{Can't be knowledge because it is not general or transferable, assuming it is primarily fiction-will not contain data. You could possibly stretch the definition a little and say it contains data about a fictional world. But the story does give some information about the fictional content and contained in that Bob's underlying assumptions and experience, etc.}
\stopitemize

Information exists between the abstract generalizations of knowledge and the factual observations of data. This set-based conception resonates with some of the ideas of data sets from the interviews and falls into the same philosophical framework: collections of facts can be analyzed for patterns. 

Knowledge produces highly generalizable concepts about the world:
\startitemize
\item  \quotation{Knowledge because it involves abstract, generalized concepts that presumably have been produced based on previous experience and analysis (data and info).} 
\stopitemize

Knowledge is predicated on both information and data, creating a gradient of specificity. However, as the other descriptions of knowledge were poorly explained, it is difficult to fully generalize knowledge's relationships with data. However, working from the data-as-facts construction, knowledge must be patterns that provide {\em correct} predictions of the world. 

\subsection{Survey XIII}
The participant is a researcher modeling industrial processes from an engineering perspective. They classified 4 Data, 13 Information, 4 Knowledge and 5 Os. Unfortunately, the majority of other responses were not well discussed, but they indicated a special kind of data, a process, and curiously a null set: 
\startitemize
\item  \quotation{Null set. If file contains the correct number of pages and formatted, then \quote{data}}. 
\stopitemize

The participant indicates that meta-data is a form of data, and technical data is a special case.

Data are numbers without context:
\startitemize
\item  \quotation{Only says what he has watched. Does not specify contents of programs. More information however than just a series of time durations (data).}
\item  \quotation{Time series (no attached informations, e.g., weather conditions, etc.).}
\item  Sound is \quotation{special data.}
\stopitemize

All of these data are numerical and contextless. Sound is intrinsically numeric and thus is considered by the participant to be data. They show no evidence, however, for data being anything other than numbers. The use of music/sound rules out measurements as data and there are no suggestions for objectivity. 

Information may be contextualized data: 
\startitemize
\item  \quotation{I would consider instructions to be more than 'blind measurements' but less than a discussion on how the machine works.} 
\stopitemize

Information, as catchall, has a non-standard relationship with knowledge and data. Although there are hints to a hierarchy, there is no discussion of what actually comprises information. From inference, information is contextualized data.

Knowledge is explanation, chief of a hierarchy of primacy: 
\startitemize
\item  \quotation{Depends on quotes: \quote{Day is Hot,} then information. Explanation to the meaning of life: knowledge.} 
\item  \quotation{As it stands, then information, if giving reason for her to take her stuff back then knowledge.}
\stopitemize


Knowledge provides reason and explanation. Only new knowledge is classified as knowledge, and information seems to be used as a catchall.

\subsection{Survey XIV}
The participant is a developer of an in-house optimization application. The most interesting aspect is the very low categorization of data: 3 Data, 14 Information, 9 Knowledge, and 0 other suggest a very interesting conception of data.

Data are facts without context or intrinsic meaning:
\startitemize
\item  \quotation{A letter usually contains a commentary alongside any facts, giving the reader the possibility to interpret those facts.} 
\item  \quotation{If the subject is unknown, the content are data. However, the scenario may be useful to somebody in determining what the data concern.} 
\stopitemize

The contextless aspect is predicted by the information phrasing, but from the second quote, Data is meaningless without said context. Data, without the ability to interpret them or without a use, cannot transmit meaning. The participant may have a constructivist view of knowledge, which suggests that only data contextualized as information have utility. Data as proto-information, therefore, lack utility. 

Information is data with context:
\startitemize
\item  \quotation{There is a sufficient context to determine that it is a 'planning' program (whatever that is)}
\item  \quotation{There is a clear context surrounding the basic demand.}
\item  \quotation{The music itself is just data (unless it contains textual messages, which could be considered information).}
\stopitemize

Context provides meaning to data that support judgments. The process of contextualization is what allows data to become information.

Knowledge is a judgment: 
\startitemize
\item  \quotation{Bob has had to make a judgment on whether this noise is of any interest.} 
\stopitemize

In some ways, the term \quotation{judgment} is disjointed from the other philosophies of knowledge inasmuch as it combines imagining and predicting the future with then assessing the best outcome. Despite the novel term, however, the understanding is fairly consistent. 

\subsection{Survey XV}
The participant identifies as a SIGINT [signals intelligence] analyst and a subject-matter expert in that field. They identify 6 Data, 1 Information, 6 Knowledge and 2 other. The Os identified are: having an experience and art. 

Participant draws a strong distinction between practical matters and art: 
\startitemize
\item  \quotation{To me, art is something different than data, information or knowledge. Art can't be easily classified as data or information, for instance. }
\item  \quotation{The music is live, and not electronically stored as information. As such, the music is 'O' in this case. It isn't data, it isn't information, and it surely isn't knowledge either.} 
\stopitemize

Data is a small, measurable, description of the world:
\startitemize
\item  \quotation{The word \quote{quizzes} does not reveal anything particularly novel about Dave or his situation. Dave grades work from a relational algebra course, which is good knowledge about Dave and his roles and personal knowledge. Working within an educational facility also implies something about Dave. But a quiz is not a revelatory piece of information. It would be more revealing if Dave were grading Theses or Dissertations or reviewing Essays or White Papers.}
\item  \quotation{This is data because it does not reveal anything about Dave or his network or his intentions. A Word document that contains no information does nothing to further analysis into Dave and his network. By the statement above, we already know that he is working on a conference paper.}
\item  \quotation{The weather report is data about current weather conditions, since it is a measurable description of the weather.}
\stopitemize

Data are the traditional building blocks of information, though the quotes above suggest that data stems from non-novel revelations. Data are measurable, small, recorded descriptions of the world or target.

Information is an imprecise statement about the world:
\startitemize
\item  \quotation{This is more than mere data, because we know something about the sender and recipient of the letter. It is less than knowledge, because we do not yet know more about the sender or recipient, beyond their implicit connection to one another.}
\item  \quotation{We know now that Alice is interested in building a device to cook her breakfast, but we do not explicitly know anything about the device itself. This means that we do not know how feasible her breakfast-cooking endeavors are. Because of this, the sentence is merely information and not yet knowledge.}
\item  \quotation{We know that someone is planning something, but we do not know explicitly what they are planning. The details of the planned activity would make this into knowledge.}
\stopitemize

The term \quotation{more than} suggests a traditional hierarchy. However, unlike most hierarchies, information is used to identify statements that do not reveal useful patterns due to their level of detail or other imprecision. 

Knowledge is relationships or patterns: 
\startitemize
\item  \quotation{This is knowledge about the two targets within our tracked network, because it defines loosely the roles that the two targets share between each other. Bob is a keeper of how-to information, and he passes it to Alice. Alice may be a leader who in turn passes this information to the rest of the network. She may be a worker-level associate who executes construction of the machine based on Bob's information. She may be a connector who acts as a conduit to spread information on machine construction throughout the network.}
\item  \quotation{This is knowledge because it implies a relationship between the two targets -- the sender and receiver. A short story may imply a personal relationship between the two targets, or it may reveal that they fulfill roles in publishing or media.}
\item  \quotation{If we are seeking to understand Bob's character, we now know that he enjoys symphony music. Beyond this, we can infer little from the sentence above.}
\item  \quotation{A statistical profile may help reveal patterns of behavior in a target. This would make it knowledge about the target's behavior. There are a lot of unanswered questions here though: how big is her sample? How many different activity types by the target are included in the sampling? How many external environmental variables (associates, external events, etc.) have been considered in the analysis? Finally, the danger of statistical profiling is the Black Swan. Ultimately, this is knowledge that leads to error.}
\stopitemize

Knowledge is a pattern or relationship among entities. It allows for understanding and prediction through application of patterns to future events. 

The most important aspect of this survey is that it is completely analyzable according to the very simple analysis methods that this section employs and that it produces no products that would have it be viewed as an outlier. If I were not the one doing the analysis, it would be difficult to write a rubric that would capture the different understanding of instructions sufficiently for the coders to catch it accurately. Future iterations of this research must have more questions and better instructions that can detect this level of wrong model-of-survey in potential participants.


\section{Survey Analysis}

The results from the surveys successfully triangulate my results. While limited by the same potential bias as the interview analysis, the high variation in conceptions of data from the survey suggests that I may very well be on to something. On the other hand, the phrasing of the survey instructions suggests that I may have indicated to my participants what I was looking for, thereby eliminating scientific rigor from my experiment. For this reason, the survey results are presented as a curiosity, rather than as proof of different conceptions of data in the wider world.

Different roles have different uses for data, and there are differences of opinion as to the methods of knowing, intrinsic nature, and purpose of data. In the surveys, people referred to records, observations, and numbers, roughly mapping to the terms found in the interviews. 

Records, as technologically stored signs, can be data if they are stored properly. Another term for record is row or {\em tuple}. These tuples can be about any topic and can suffer from anomalies: authoritative conflicts in which the database may not represent the thing in the universe of discourse. Despite this, to some people, any records in the database are still data. 

Observations are the act of applying consciousness to a reality. They can be subjective and inconsistent. At the same time, researchers used the term to indicate experimental observations that are objective and provide their own context. The term observation in the surveys suffers from the same traps that it does in the interviews and is worth far more research.

Numbers as data are used by both the database designers and the objective researchers. Numbers are either a more objective refinement of an observation or merely a semiotic rendering that requires context and analysis to render into meaning. Numbers are the {\em sine qua non}\footnote{Without which not.} of measurement, as they are the result and the goal of factual inquiry. Those who believe data are facts can use the idea of numbers to represent this. 

However, numbers are also the {\em without which not} of databases, being a different order of thinking about a record. A number in a database sense can be either the content of a record or the sign by which a record is stored, because a series of bits are simply numbers. Both uses are evident and present evidence of a fascinating linguistic trap for future research. 

From the point of view of question of interest 1, the surveys were a complete success: people demonstrated that they had realities of data that departed in ways far beyond mere semantic differences over subjectivity and objectivity. Furthermore, different roles tended to cluster around different understandings of the term, suggesting that educational background combined with the current uses of the term strongly inform meaning. However, from a methodological perspective, the tainting of the surveys with the item of interest leads to results that are strongly suspect. While these results are encouraging, no real conclusions may be drawn from them.


\stopcomponent
