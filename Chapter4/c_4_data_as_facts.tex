\startcomponent c_4_data_as_facts
\product prd_Chapter4
\project project_thesis

\subsection{Data as Facts}
Conceptions of data in this category tend to be more scientifically oriented, representing data as {\em only} the results of careful consideration and measurement. Here, data as facts orient the observer to a recorder-of-measurements, rather than as someone who must filter observations or interpret semiotic encodings.

This category of interviews is the largest, as can be expected while interviewing a scientific research establishment; because of the scope of the interviews, I probably missed some significant nuances. Exploring the different constructions of data and fact from a philosophy of science standpoint could be extremely productive. 

\subsubsection{Interview V}
\sidebar{
The participant is a scientist. Data are measured observations, objective reflections of the world. Information is a product of human thought to contextualize that data.
}

In strong similarity to Interview I, the participant differentiates between what can be understood by consciousness and what is in the realm of machines. The participant is a highly theoretical researcher, in contrast with some of the more pragmatically trained participants. Data are specific and measured observations.

Data, information, and knowledge are ordered categorically in terms of precision. Information is an unconfirmed interpretation of data and knowledge represents understanding of the real world. The flow from specific to general is interesting, especially when considering the implications of the participant's conception of data.

Data are measured observations, with observations or raw measurements as \quotation{raw data,} and interpretations of data to be data-as-\quotation{interpreted data.} Experiments return data as numbers collected by various instruments, but data is also embodied in the world. Geometry and design of equipment and the operational set points are data. \quotation{Equipment geometry and operational set points are embodied data.} The idea of data embodied in the real is a strong feature of the data-as-facts construction. 

There must be an observer to manipulate things outside of data because, \quotation{[The] experiment itself is not a conscious entity; you cannot send anything but data to it.} While it is interesting to consider information and knowledge as functions of consciousness, the distinction between data as embodied or observed seems to belie my assessment of \quotation{data as measured observations.} If data is allowed to be embodied in the physical universe, then to retrieve the data and thereby to understand the object is to perform some sort of structured and measured observation in the context of an experiment. The participant notes that data are \quotation{raw numbers and no context.} This construction requires that data must be understood as the fundamental relationships of matter that can be observed and quantized.

The category of information trades precision for context. Information is an unconfirmed interpretation of data. There is a clear distinction between the objectivity of data and the interpretation/necessary subjectivity of information. 

\startextract
It's not just raw numbers ... but it's not knowledge because it's the experiment and a very contrived environment.... It is interpreted data about the contrived environment.
\stopextract

This interpreted data-as-information is not knowledge because it does not represent a phenomenon of the object under consideration. Data produced as part of a simple experiment, does not form the basis of action in the real metalworking area.

Experimental information (or interpreted data) as the product of simple experiments may not conform to the real world. However, information may form the basis of understanding of the real world. Information, combined with a statement of \quotation{I think you should do this} (classified as knowledge) may allow design engineers to create new designs. Flows of information wholly contain \quotation{raw data}\footnote{The use of \quotation{raw} here is a specific example of a conceptual trading zone discussed on page \at[Galison].} {\em and} its explanation, whereas an independent flow of data would be subject to external interpretation. 

Knowledge is an understanding of the real world. It may be formed from personal experience, but may not be about \quotation{unreal things.} An unreal thing is \quotation{Information, because it's not a real thing yet. It's not knowledge about a real device. It's still a hypothetical.}

\subsubsection{Interview VI}
\sidebar{
The participant is an Information Architect. Data are facts produced from instrumentation and sensors. Information is multidimensional data. }

The participant is one of the information architects that I interviewed. They explore the implications of computerized data. However, in contrast with other interviewed information workers, they clearly articulate data to be facts exported from measuring instruments. Although there are elements of contextualization here, the view of data as records of measurements indicates a different construction of data when compared to the other information specialists\footnote{Contrast these results with Interview I (p. \at[Interview I]) and Interview IV (p. \at[Interview IV]).}. More research is necessary to explore how interactions with technical devices shape conceptions of data. 

There is a strong relationship between information and knowledge: 

\startextract
An example of information is a time series trend. An example of knowledge is advice (expert correlates information, presents a suggested course of action). Information can be provided to support decision-making. Examples of information: Train timetable, an index of crime levels.
\stopextract

There are suggestions of a relationship between knowledge and data: \quotation{Sensors provide data, something you can work with.} Then, information plots data versus data. Knowledge contextualizes data (and the plot of data). 

Knowledge is a generalized statement about the world. Information is multidimensional data such as data versus time. The act of contextualizing data by forming data sets and comparing it against other parameters like time produces reports as \quotation{information.} These reports are locally true, and may be explored for insights into trends and patterns (knowledge). 

Data produces information, which produces knowledge. Mathematical models are information, which complement experiential knowledge. However, models are useless without data as input. Information-as-model requires contextualization (knowledge) by a person to be useful: \quotation{People's experience, as knowledge, tells them when to transform data into information into knowledge.} Thus, data is treated as if it is atomic, despite being decomposable in other contexts.

Looking at the higher levels of the hierarchy, knowledge must be explicit and encodable (such as into source code), but it has fewer restrictions than information because information is less discrete. Data are components of information. People treat \quotation{derived data} as \quotation{real} despite knowledge that it has been processed. In the local jargon of the research environment, real seems to mean \quotation{authentic and an atomic representation of a measurement.}

Data are measured facts. They can be input to mathematical models, contextualized with information, and generalized with knowledge. They are created from sensors and bits. This construction clearly articulates Ackoff's hierarchy of data, information, and knowledge, viewing data as unimportant building blocks in the path towards generalizable statements. 

\subsubsection{Interview VII}

\sidebar{
The participant is a scientist. Data are numbers formed from data points. Clearly articulates Ackoff's hierarchy.
}

This participant is less computer-focused than those of the prior interviews, instead investigating physical phenomena. Although data are also facts for participants, there exists an entity below data: the \quotation{data point} or specific measurement. Sets of data points become data, restricted purely to the numerical results of measurements. 

Data are numbers, and only numbers. \quotation{Data is a multi-dimensional collection of data points. A data point is a number.} As well as \quotation{Data plus analysis creates information. Analysis is a function of a mental causal model. Understanding of the model is knowledge.} The ultimate goal of data collection is the creation of knowledge as the basis of science. 

Information is the basis of action and a container for data: \quotation{Information are instructions to perform work of a specific nature, guidance into what it is they're going to test.} Information is what allows humans to judge between different outcomes: to order one product over another because of scientific results {\em contained} within the information. Those results, interpreted and packaged, are data. Although the participant uses the container metaphor here, information is not acting as a passive sign-repository. Rather, the contextualization of information is explicitly differentiated from the pure numbers of data, rather than data being transformed into information. 

Data, analyzed into information, may create novel information. Novel information can generate knowledge. Knowledge is a predictive model of the universe, \quotation{based on generated meaning through interpreted results (information).} Knowledge as predictive model is far superior in this hierarchy because of its generalizable quality, something that individual decisions to act do not contain. A meta-analysis of action, the results of acting on information received, can create meaning or knowledge: 

\startextract
Knowledge is a model derived from analysis of information. You analyze data and generate new information. New information can generate new models. Models are knowledge. A shift in [a] model (physical or whatever) is due to new knowledge from discussion. 
\stopextract

This construction demonstrates Ackoff's hierarchy \footnote{See page \at[Ackoff].}, with the scope increasing as each level of abstraction is gained. Models, as the ultimate discussed step, represent knowledge, because they represent that which interprets incoming information. At the base of the hierarchy, however, rests data as facts measured into numbers. 

\subsubsection{Interview VIII}
\sidebar{
The participant is a researcher from a different company. Data are objective observations with no restriction on representation. Data can be generalized in Knowledge and contained within Information.
}

The participant's background is strongly different from that of most of the other interviews, as they are not directly affiliated with the company. The interview was taken spontaneously while they were visiting the site on one of the days I conducted interviews. Data are rendered measurements. Unlike Interview VII, the measurements are not constrained to purely numerical entities, but require the same level of objective rigidity. 

In the interview, the participant did not discuss any hierarchy of data, information, or knowledge. While there seem to be strong verbal differences between data and the Os, there are no transformations from data into information or knowledge. Information is understood quite differently from data or knowledge: \quotation{Information is a container for data and knowledge.}

Knowledge is transmittable experience: \quotation{Knowledge is perception of something, what you think you know.} Experience manifests as intuitions and requests for specific data are an expression of encoded knowledge. In one example, the participant transmits a request for an image to the database: \quotation{I would be sending that knowledge, even if the DB doesn't acknowledge it.} Knowledge is transmitted in the act of using knowledge, despite the receiving components possibly not being able to understand that what they are receiving is indeed knowledge. 

Specific data retrieval is a function of knowledge. Because knowledge is present in the act of using knowledge, data requests to computing devices are formed from knowledge and therefore represent the sender's knowledge, rather than the recipient's received data. This represents an interesting meta-component of this conception of data, representing an explicit trading zone formed between entities sharing what {\em one} of them considers to be data. 

Information is a container. Bits are information and form the substance of the container. Beyond the container metaphor is also a question of quality: 

\startextract
Information is a quality. It mixes with data and knowledge. Something that you relate, it goes from one source to another. Any kind of data or knowledge is some kind of information.
\stopextract

 Thus, information is more than a simple container, and may seem to be an intrinsic building block of data-knowledge interaction with the world. Information seems to equate to semiotic ideas, as it can be a representation or transmission agent. In strong contrast to the constructions of the data-as-communication camp, {\em information} is communication. Information, being formed of signs, requires interpretation for the extraction of the data and knowledge components of use to computers and humans, respectively. 

Data are factual representations of measurements. They usually form numbers, but, \quotation{Information is text and data is numbers, but even text could be data. Text as data are descriptions of measurements.} The fundamental characteristic of data is that it is the factual product of experiment. At the same time, the participant also seems to use data-as-technical, because they describe punching in a few numbers to get a drawing as data as well. 

This interview represents a fascinating departure from the normal affordances expressed as it provides a strong definition of data and knowledge and a very weak, subsidiary, and technical view of information. One possible explanation is that, coming from another country's cultural understandings, participant linked the term information to the computerized bits and had less experience with other uses of information. This interview, alone, suggests a strong need for an international investigation into the different cultural conceptions of data.

\subsubsection{Interview IX}
\sidebar{
The participant is an engineer. Data are numbers that are orderable within a list. Source code is also considered to be data as it is comprised of numbers and instructions for those numbers. 
}

The participant engaged in an interesting interview: they produced two diagrams exploring both an academic project and an engineering project. In both frames of thought, participant describes data as something inherently numeric, with some basis in fact. 

Data, information, and knowledge relate in a cyclic hierarchy. Data interpreted via existing knowledge creates information. Information analyzed creates knowledge. There is a container metaphor in this construction of data as, \quotation{Information can appear alongside knowledge or as a component of knowledge.} As a practical example of this understanding: equipment returns data that is processed into information via an engineer. Analysis is then performed as to \quotation{whether things are working.} Information is produced by an analysis of data. 

Knowledge is fundamentally experiential: \quotation{I would define knowledge as something that an individual or entity... something that an entity possesses. That typically has been learned from somewhere else or gained from experience or whatever that enables them a better understanding then they would have otherwise.} Knowledge as understanding represents knowledge as model, articulating the nature of the universe by making predictions about future events. 

Knowledge acts as the agent that transforms data into information: 

\startextract

My understanding of how things would work is that knowledge is the process by which data is turned into information. So I would say data is things. It is numbers, it is raw information. It is something that, on its own, doesn't mean very much. It's just stuff. Information is an interpretation of that. So it's kind of something that is understandable to someone without knowledge, without even concept of what the raw data is. 

\stopextract

Knowledge is understanding: advice about the causes of unexpected events. Knowledge can be communicated and the creation of understanding of the process under investigation is knowledge. Documentation contains knowledge because it presents a history of actions, and can provide methods for analyzing data. The analyzed data, in some ways the \quotation{understood data,} is collected in things like reports. The report, however, just contains facts about what was done and the results. \quotation{[Less meaningful] stuff is the information and the results are data.} 

Data are numbers: \quotation{You can put them in a list. There is a defined quantity. They can be grouped, they can be ordered or whatever. They probably are data. You can put them in a spreadsheet.} However, the emphasis seems to be on numbers rather than on scientific results: not on the objectivity but on the coding mechanism. The participant believes that code is data: source code is instructions and numbers to work with. 

Information is connected with the communicative act. Facts are transmitted via information, but there does not seem to be any evidence to indicate that facts are information, merely that information is the semiotic encoding\cite{Sperber1995} process by which they are communicated. Knowledge can be communicated through information. The recipient, in possession of this encoded knowledge, must then analyze the information to retrieve its knowledge component. 

In some ways, information described in this construction echoes the information as container construction of Interview VIII. While there is evidence of the participant's cyclic hierarchy, there are interesting ambiguities relating to the relationship of information, \quotation{raw information}, and facts. While data are encoded pre-facts expressed as numbers, this interview presents difficulties in classification and represents an edge case worthy of future research. 

\subsubsection{Interview X}
\sidebar{
The participant is a researcher external to the company. Data is embodied in reality and can be discovered through measurement. Data represents the fundamental relationship of things to one another.
}

This interview also involved someone external to the company. Their conception of data represents something with a strong sense of embodiment; data is physically present in the world, in relations between matter. Measurements and observations can {\em discover} data, but not create it out of subjective whole cloth. 

One of the more interesting concepts of the interview was that of formality. The idea of formality is raising a McLuhan-like \quotation{medium is the message} level of meta-analysis. It notes that the context of a communication and its source within an organization contribute significantly to the actions one takes upon it, despite the same communication being passed from different directions: \quotation{Information passed from knowledge won't be turned into knowledge by the recipient due to a lack of formal power within the organization. The information is just \quote{the background why} that doesn't change actions.} 

Formality, the coding of the \quotation{authority} encoded in the message's medium, is something that modulates knowledge rather than a unique category to itself. Formality represents {\em an actionable instance} of whatever the category is. Written reports, delivered to responsible people in the organization, are an example of formal knowledge. They are incorporated into the organization's stock of knowledge and acted upon. Those same reports delivered to interested parties, informally, do not have the same actionable qualities.

Formal knowledge is an expression and transference of technical understanding. Components of the flow described by the participant include distilled and certified data sets, the analysis of those sets, and other supporting observational material. The certification of the data sets is to insure that the analysis is internally consistent. Certification is checking the fit of \quotation{internal guidelines of quality.} To designate something as certified is to express an opinion of its reliability and repeatability. Because of the additional process of certification, formal {\em x} can be acted upon. The participant believes that normal flows, lacking confirmation, are not processed in the same way.

There is clear evidence for a reverse hierarchy in the interview. The Tuomi hierarchy suggests that we use knowledge to build information that allows us to extract data from the world; this data then allows us to update our models of the world. In this way, as information frames our requests for data (subjective or objective as they may be) it is possible to check the integrity of the data and contextualize it by inspecting the originating information. Experiments can be designed to extract information from sensors and computing devices operating on machinery. When sensors are rare, experiments are designed to maximize \quotation{reliable information from an experiment.} 

Knowledge is expertise. Expertise, however, is expressed as an understanding of the world that {\em shapes} action. The participant observes that \quotation{corridor conversations} discussed in an informal setting can contain information and knowledge. Information, in this case, \quotation{is the things that I can reduce to something technical.} Knowledge \quotation{is their understanding of the process, how and why they turn levers and their perception of the basis of that.} Thus, as expertise accumulates from the past shared precedents of actions, events, and training, it acts as a guide to action. It informs information-gathering methodologies, which then inform data-collection strategies. The use of \quotation{reduce} in the quote above is better understood as \quotation{abstraction} rather than implying that information is \quotation{under} knowledge.

Information has a temporal element or has a different level of reliability. Whereas more sensors, both cheaper and more connected, produce more data, the output of multiple sensors over time produces information. \quotation{In the context of the whole operating campaign, you don't know the reliability of the information, you don't see the built in trends... but you've got the data points. Three quarters of the data in the database, you don't know how it was acquired.} Although information may appear to be used synonymously with data, there is a suggestion that a thing can be classified as data or information based on some kind of underlying or meta-characteristic. 

For example, a physical sample can embody both data and information, depending to whom it is sent. This suggests that observations and numbers are latent in the world. In this way, observation {\em transforms} latent information or data in the atomic or molecular structure of the thing being experimented upon into more easily recognizable numbers with defined error limits. This transformation is not an external imposition of measurement, but a recognition of something that is already there. 

Information is an observation of the world. Observations of physical nature are information assigned from parameters informed by knowledge as previous experience. The participant strongly believes in the objectivity of information. They require that different types of measurements of the same object correspond. In this instance, a sample represents embodied information. There is no inherent knowledge of what it is in the object, but it is information. If the information is analyzed, it then can become data. In this vein, environmental observations surrounding the sample are information, creating a \quotation{picture of the problem.} Binary signals from switches can be information or data, depending on whether they represent state status (information) or measurements of quantities and masses.

The apex of the hierarchy is data. Data are contextualized hard numbers. Data can be generated from \quotation{something you can reduce to a number. It has observations, but it's reducible to a number.} Optimally, samples will be accompanied by recorded site observations, situating the sample in some sort of context. 

Unlike the other interviews, this one holds no indication that contextualization serves to transform the data into information. Instead, contextualization creates data that are more valuable. All things contain within them latent information that, through the correct experiments, may be observable. Only through analysis can the observed and recorded results be promoted to data. One method of analysis is a chemical analysis because the results of chemical analysis are deemed \quotation{real} and \quotation{hard data} numbers according to standards.

At the same time, data has different natures according to the roles that need the data. \quotation{The geologist says I want the sample. The chemist says I want the analysis.} The participant notes that for other roles, data's substance can be process data or the discovery of relations within the set of process data.

Data exist at various levels of accuracy. \quotation{Industrial quality} data has accuracy considerations different from those of academic or theoretical quality. Furthermore, dud data-data that is erroneous-is possible and is a likely result of cost-saving measures. 

\stopcomponent

\section{Methodological assessment}

I presented two questions of interest: do people have different conceptions of data and are my tools capable of capturing someone's personal construction of data? I believe that both questions of interest were satisfied, with my methodologies quite neatly capturing three different schools of thought apparent in both the interview and the survey. As these conceptions roughly correspond to extant definitions in the Oxford English dictionary, and some observations by Zins\footnote{See page \at[Zins].}, there are few suggestions that personal bias interfered with the collection of evidence. 

During the interviews, the \SDFN\ process served as a fantastic tool to evoke the participants' realities of data.  To improve the ability to probe, future researchers should focus on a single topic area with all participants exploring the same universe of discourse. This uniformity may allow the application of more articulated graph theory analysis. With a more standardized topic, not even the other participants would be able to reliably reconstruct identity from simply reading diagrams. More research can be done on better ways to render the \SDFN. Although Graphviz does create adequate representations, it should be possible to create more consistent renderings and a more intuitive labeling scheme than currently exists. 

Examining the interviews, I believe they managed to accurately probe the constructions of data of their participants, especially through the use of the SDFN diagram as a tool for evoking epistemological categorization boundaries. Each interview felt as if it was probing the participant's conception of data, getting them to crystallize their definitions through the SDFN diagram process, which created cognitive dissonance between their use and their theories of use.

With the discussion after the diagrams, participants usually discussed their internal thoughts. The interviews were not a complete success. Although it is probable that I found each participant's personal construction of data, the different topics of each interview severely limit commensurability. In many ways, the different topics made {\em comparing} conceptions of data across interviews very difficult. Recursive analysis, while useful, does not provide any kind of structure to suggest reliability. Future studies must use more reproducible methods of analysis or external coders to minimize or average out bias. Recursive analysis may be profitably employed as a coding technique, as it would merely ask coders to summarize small sections of interview or survey. 

Examining the surveys, I find that they were successful, but suffered from some large problems. The first problem is that the participants did not quite understand the intent of the survey and treated the scenarios in different ways. Furthermore, some participants left the explanations blank, causing great difficulty in trying to understand what they meant by the terms. Future surveys will have more clear instructions, required descriptions of answers, and more choices in the drop down box, including a \quotation{not sure.} 

The surveys' analysis did not employ any formal analysis techniques, having insufficient quantity for any statistical analysis over each survey question and insufficient length of responses for any kind of recursive analysis. Although the responses may be more penetrable to some sort of computerized coding technique, I saw no need for that here, as I was not trying to make statistically rigorous statements of fact. Further studies, however, may have the opportunity for larger research sets with more tested scenarios, enabling interesting and objective conclusions to be drawn.

\section{Summary of Analysis}

The three large realities of data that emerge from all three analyses are: data-as-communications, data-as-observations, and data-as-facts.

Data-as-communications forms a small subset and generally indicates transferred or stored signs with some relationship to computers. Data does not have to be interpreted to be real: it exists as stored signs, and that creates its reality.

Data-as-observations is a broad category suggesting that data is the product of observations that sentient creatures make of their environments. It is subjective and requires filtering to winnow out interesting data from the huge mess of data sleeting around us.

Data-as-facts is a narrower category, indicating that a piece of data is a scientific fact, used for the revelation of knowledge, discovering how the universe works by looking at the relationships and arrangements of things. 

\stopcomponent
