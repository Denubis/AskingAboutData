\startcomponent c_4_data_as_comm
\product prd_Chapter4
\project project_thesis

\subsection{Data as Communications}

One of the three constructions of data is that data are encoded signs. Signs that exist on paper or electronically are passed between humans who may then interpret those signs as data for information or knowledge encoded into them. In this construction, data is a technological artifact rather than something generated and manipulated by humans. Information and knowledge are passed using data, but are not of the same fundamental nature. The relationship is a container metaphor rather than hierarchical.

The few interviews expounding upon this construction suggest that significant research is necessary to explore its tenets. Beyond that, there is some evidence that this construction may be held as a secondary way of understanding data when dealing with technological systems\footnote{This pattern was also apparent in some of the surveys like Survey I (p. \at[Survey I]) and VI (p. \at[Survey VI]).} rather than as a fully-fledged construction of data of its own.

\subsubsection[Interview I]{Interview I} \sidebar{
Technical pilot interview, exploring a Facebook game. Participant holds two realities of data: Data as a menial interpersonal communication and data as a menial technical communication.
}

The first interview I conducted was a technical pilot, assessing the methodology alone without exploring business specific concepts. In this interview, the participant and I explored the leadership dynamics of a Facebook Game. More details about the SDFN diagram and its analysis can be found in chapter 3, in the case study. This first interview will offer the raw quotes, as well as the recursive analysis appropriate to the section in which the quote appears, thanks to explicit permission by participant. 

Data, information, and knowledge relate to each other. Data is a menial category between these three. Data exists only as itself, without reliance on anything else. \quotation{Anything else} here means external signs which frame, or contextualize, data as well as indications that the data requires judgment or discernment on the part of the recipient. I believe that the participant considers the application of context to be something required of ambiguity or importance, and therefore of increased importance. 

Data is not a function of expertise or knowledge. A critical component of data is its lack of necessary arbitration: data is intuitively obvious, objective, and unimportant. 

\startextract
Participant: It's almost like discrete packets.

Interviewer: So, packets of?

Participant: Data?

Interviewer: Commands?

Participant: Like a command. Saying: \quotation{I want to do something. I want rez the fort now.}

Interviewer: So is it OK to label this as commands, or does it encompass something other than commands?

Participant: It does, sometimes...

Interviewer: So commands and..

Participant: No, that's true. It's pretty much just commands. Like I would send it I could do many things but they're all related to telling the game to do something. I don't call it knowledge, because it's like it's just, maybe they're just talking about context thing. It's something fairly simple, discrete. It's not open for arbitration or anything like that. It doesn't require arbitration, just \quotation{do it.}
\stopextract

Two iterations of analysis produce:

\startextract
Data is discrete. Data does not need context nor arbitration. Commands to software are data. Data is simple, it can be executed on its own merits. 
\stopextract

For example, instructions for action are data, both those that are explicit instructions to software agents and implicit orders as \quotation{status updates.} In the game sense, a status update is a message sent out from administrators to the players \quotation{advising them} on the status of the structure players are defending. This \quotation{advice,} however, is intended primarily as a mobilization for action, as positive activity is needed to create the structure. 

Communications are information. Communications are movements of subjective statements about the world encoded in the transmitted signs. Communications must have context. This categorization of data is used to delineate between communication/information and data. Beyond a generalized context, communications must also contain a temporal context: they have to be about somewhen, whatever other elements they contain. 

\startextract
Participant: So I'm telling them, I'm sending them information. That they can now war. That they can start.

Interviewer: And so this is a what? Is this a status, is this a command, is this something else?

Participant: It's information because it does require context. But it's like a status it's saying: \quotation{You can now war.} We can start fighting.
\stopextract

Two iterations produce:

\startextract
Information can be returned by software. Status is a form of information. There is a temporal aspect to changes in status. 

Information requires context. Information provides options.
\stopextract

Expertise is knowledge, being the perception of options for action in information. Knowledge confers  the ability to be persuaded by communications as expertise is used to frame incoming information and thereby \quotation{understand} it. 

\startextract
Participant: I don't know how to class their opinions. Not necessarily knowledge. It's almost like data. It's like, they don't really send me commands, not as in the concept that we talked about, commands in the past that do something. It's almost like data. They just tell me stuff. I'll say, \quotation{We're going to do this.} but they'll say, \quotation{I think we should do this.} I don't think I have to reply to it. It's like they're offering their opinion. But it's where we're distinguishing the different status. And then when we go to active clan masters. Now there's a number of different inputs: \quotation{Clan status, Info, FB.} Same with all of these. They're in a group, they're getting the same as everyone else. But we'll have an extra line which will definitely be MSN as well ... which will be another input to them.

Interviewer: So you're sending them MSN? And what are you sending over MSN?

Participant: I don't know? I can call it anything, can't I? It's like knowledge? I guess? It really is knowledge that I'm sending back to them that way. Similarly, they'll send me back data and opinions via FB messaging. But they'll also have that subsequent line via MSN. It's that knowledge sharing. Even hilariously in the past, I've actually called some of these guys on the phone.
...

Interviewer: Now what are you doing over this MSN & Phone?

Participant: Well, that's the genuine sharing of knowledge. Knowledge of: \quotation{I think this; I want to do this; no, no, no, we can't do that.} It's a collaborative sort of working together. It's a sort of consensus-building. That's where it's knowledge, because by sharing that knowledge you can build a common consensus which we can work around. Because if we don't share each O's opinions or experiences then we never build the consensus to actually operate, otherwise you'd just be individuals because it, generally as a clan, we have to come up with a consensual view of everything..

Interviewer: Do we want to have consensus as another line? Or is the sharing consensus-forming?

Participant: Yeah, the sharing is consensus-forming. We exchange knowledge and out of that comes consensus. Even that knowledge is: \quotation{Do you consent to this?} It's not really a command but it's a knowledge.
\stopextract

Two iterations produce:

\startextract
Opinions, lacking privileged status, are not knowledge. Activity provides context. Responses from active members are classified as knowledge. Responses from semi-active and inactive leaders are data through opinions.

Knowledge can be shared. Collaboratively sharing knowledge builds consensus. 
\stopextract

There is no explicit differentiation between different kinds of data. There are examples in which data is either technical in nature (for example, a command to cause activity) or a communicative activity (wherein the communication was not produced as a function of expertise). Data is interesting only in the sense that it represents the mental activity of the communicator, not as any kind of thing on which to take action. The differentiation only extends to the comment that something is \quotation{sort of data} potentially representing an internal cognitive dissonance between the two ideas.

\startextract
Participant: It's when you're talking to another person and they're passing their knowledge of what happened to them and what happened in the past to you. So you can either learn from your own experience or O's experience. It's really a mixture of that. In relation to knowledge gathering.

Interviewer: Would you say that there's any relationship between knowledge and data in this sense? Or is it just...

Participant: No, I don't... I think. I think the expression of data in this context, it's interesting actually. Now you've got me thinking. No, but yes. Yes, but no. Well, it is sort of, yes. I agree. I don't often express some things like certain relations to [the game] as data. but there definitely is if you stop and think about it, because. But yes.
\stopextract

Two iterations produce:

\startextract
Knowledge can be gained from talking to other people, learning things through the game. Experiential Knowledge is what you get from doing things. You learn from criticism, experimentation, failure. A different kind of knowledge is someone else's experience. Knowledge is privileged over data. Represents \quotation{knowing} over "observation.
\stopextract

There was no discussion on transformations of data. Moreover, no conclusions can be drawn from inference, as it is equally likely that no transformative framework exists such that the standard hierarchy is present. Each term seemed to be discussed on its own merits without reference to the Os. There are two constructions of data found here: data as a menial-interpersonal communication and data as a menial-technical communication. In both senses, data contains no privilege over information or knowledge. As menial-technical, data seems to be considered to be bits or human instructions rendered as bits: self-encapsulating orders for action. As menial-interpersonal, data seems to be expressed observations of the world without a knowledge basis: empty words to be corrected and brought in line with the consensus formed by more experienced people. 

\subsubsection{Interview II}

\sidebar{
The participant is an information worker. Data is a container for human-produced information. Data is the only thing manipulable by machines. Information and Knowledge cannot be transformed into Data, only contained. 
}

The participant works with information systems. Data is a container for human-produced information. Information and knowledge relate to each other as a function of human interpretation, integration, and analysis. The main function of data is as a communicative medium of technology.

There is no hierarchy or relationship between data and information. Data is completely different from information or knowledge.

\startextract
Data does not belong in the hierarchy of Information or Knowledge.

The results of a search are Knowledge, but are formatted and contained within Data.
\stopextract

These quotes indicate a strong differentiation between information/knowledge and a strongly physicalized data. Although information and knowledge are related, the relationship expressed seems to be a function of analysis and encoding.

Communications between people, as discussed in the interview, are clearly information. 

\startextract
Information is non-physical. It can be a communication, a conversation, a story.
\stopextract

Knowledge, or personal experience, is {\em encoded} in information for transmission to other people. This ontology requires this encoding because knowledge is not directly communicable between people. Knowledge guides interpretation of information and, in the case of knowledge encoded in information, the recipient's knowledge must be used to analyze the incoming information-communication to produce new knowledge: 

\startextract
Knowledge is associated with an experience. Information doesn't have to be read or analyzed. Knowledge must be analyzed and used.
\stopextract

Data is a container for human-produced information stored in physicalized entities such as books or computers:

\startextract
Data is a record that can record and contain Information. Records, Data, are merely physical entities. 
\stopextract

As examples of data, the participant offers: \quotation{Physical containers are pieces of paper or e-mail. Digital containers are still physical.} With no evidence for other interpretations. In a simple sense, the participant believes that: \quotation{Data is a bucket for Information and Knowledge.} To restate: data is a semiotic representation of and container for content.

Data as semiotic container is a strong ontological stake. It expands beyond the simple data-as-bits construction that may accompany other realities of \quotation{real} or \quotation{raw} data. The universal application of data-as-container and information as root-of-meaning suggests that this construction extends beyond a simple communicative reality that can be used in conjunction with other constructions. Rather, it strictly defines a role for data in the storage of signs and relegates it to a strictly supportive role with respect to meaning. Nor does this construction leave any room for data as observation or numerical representation. 

\stopcomponent
