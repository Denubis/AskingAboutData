\startcomponent c_4_analysis
\product prd_Chapter4
\project project_thesis

This chapter presents an analysis of the evidence gathered by my interviews and surveys. I will explore whether or not my methodologies have demonstrated different conceptions of data and if they are capable of exploring unique realities of data in the first place. 

The initial discussion is an examination of my research paradigm and approaches, looking at the philosophical basis of my conclusions. I employ an abductive\footnote{ An abductive research strategy features a process of rapid hypothesis formation based on inconclusive evidence. While this shares similarities to the inductive strategy, the approach looks more to suggest boundaries to a region of possible answers and less towards finding (inducing) the governing rule of all the data collected. It is the approach used when there is insufficient evidence to even start guessing\cite{Dubois2002, Levin-Rozalis2008}.} approach to iteratively define universes of interest, rather than applying inductive or deductive approaches. 

I am testing two questions of interest, which guide the course of my analysis: whether people have different realities of data and whether my methodologies can discover someone's personal construction of data. By guiding the abductive research process with these two questions of interest, I may be able to limit my conclusions to the most simple and credible story possible from the evidence.

Using the questions of interest as a guide, I will iterate over the results of a recursive analysis of every interview and survey, looking for evidence of different conceptions of data by looking at their statements about data and the relationships they suggest between data and information and data and knowledge. The evidence suggests that I have successfully answered both questions of interest by demonstrating the existence of three realities of data that emerge from both the surveys and the interviews. The success of this research suggests that my dissertation offers a foundation for future studies and can be deemed successful. 

\section{Summary of Reflections}

My personal reflections found the following constructions of data. While these summaries should not be viewed as \quotation{authoritative} over the \quotation{raw} data, they indicate my personal beliefs as to the interpretation of that data and the final result of the recursive analysis.

\startitemize
\item Pilot interview:
\startitemize
\item Data is a menial-interpersonal communication.
\item Data is a menial-technical communication.
\item No hierarchy specified.
\stopitemize

\item Interview II
\startitemize
\item Data is an electronic container for human-produced information and knowledge.
\item Data does not transform.
\stopitemize

\item Interview III
\startitemize
\item Data is a subjective observation.
\item Experimental data is not data, an exception that allows for quality and less subjectivity.
\item Sometimes data is a communication.
\item Data can create information and knowledge and information is structured data.

\stopitemize
\item Interview IV
\startitemize
\item Data is a subjective, discrete observation with specific provenance and reliability: \quotation{the ability to specify \quote{what, where, when} would define a piece of that, what the measure of it was, and its location at the time.}
\item Representations of data are not data.
\item Data participates in a cyclic hierarchy, being filtered by knowledge and being contextualized into information. 

\stopitemize
\item Interview V
\startitemize
\item Data are the fundamental relationships of matter, which are themselves measurable and observable.
\item Data can become information and knowledge by adding context and abstraction.

\stopitemize
\item Interview VI
\startitemize
\item Data are facts created from objective sensors and experiments.
\item They can be input to mathematical models, contextualized with information, and generalized with knowledge.
\item Standard hierarchy: data produce information, which produces knowledge.

\stopitemize
\item Interview VII
\startitemize
\item Data are numbers: \quotation{Data is a multi-dimensional collection of data points. A data point is a number.} Data points are created from experiments.
\item Standard hierarchy.

\stopitemize
\item Interview VIII
\startitemize
\item Information as container for data and knowledge.
\item Data are factual representations of measurements.
\item No hierarchy.

\stopitemize
\item Interview IX
\startitemize
\item Data are numbers, no requirement of objectivity: \quotation{Look, you can put them in a list. There is a defined quantity. They can be grouped, they can be ordered or whatever. They probably are data. You can put them in a spreadsheet.}
\item Cyclic hierarchy: Data, interpreted with knowledge creates information. Information analyzed creates knowledge.

\stopitemize
\item Interview X
\startitemize
\item Data are contextualized \quotation{hard} numbers as observed representations of reality.
\item Data as apex of ontological hierarchy.
\item Tuomi-reversed hierarchy.

\stopitemize
\section{Summary of Survey Analysis}
\item Survey I
\startitemize
\item Working database labs of university.
\item Data is a symbol without meaning.
\item Standard hierarchy.

\stopitemize
\item Survey II
\startitemize
\item Participant works for a defense department processing information.
\item Data is an unanalyzed sign.
\item Standard hierarchy.

\stopitemize
\item Survey III
\startitemize
\item Participant is a manager of a consumer electronics repair workshop.
\item Data is a statement.
\item Hard to classify.
\item Standard hierarchy.

\stopitemize
\item Survey IV
\startitemize
\item Participant is a social entrepreneur.
\item Data is an observation.
\item No hierarchy.

\stopitemize
\item Survey V
\startitemize
\item Research scientist, incomplete.
\item Data are observations of the world.
\item Standard hierarchy.

\stopitemize
\item Survey VI
\startitemize
\item Participant is a service desk employee for telecoms.
\item Data are structured records.
\item No hierarchy.

\stopitemize
\item Survey VII
\startitemize
\item Participant is corporate strategy manager.
\item Data are observations without interpretation.
\item Standard hierarchy.

\stopitemize
\item Survey VIII
\startitemize
\item Participant is counterterrorism analyst.
\item Data are objective records of activity.
\item Standard hierarchy.

\stopitemize
\item Survey IX
\startitemize
\item Participant is a housekeeper.
\item Data are factual scientific observations.
\item No hierarchy.

\stopitemize
\item Survey X
\startitemize
\item Participant is a senior software architect.
\item Data are electronically stored observations.
\item No hierarchy.

\stopitemize
\item Survey XI
\startitemize
\item Participant ambiguous.
\item Data are objective, precise facts.
\item No hierarchy.

\stopitemize
\item Survey XII
\startitemize
\item Participant is a research engineer.
\item Data are specific observations of phenomena as well as stored bits on a computer.
\item Standard hierarchy.

\stopitemize
\item Survey XIII
\startitemize
\item Participant is researcher and modeler.
\item Data are numbers without context.
\item Possible hierarchy, survey unclear. 

\stopitemize
\item Survey XIV
\startitemize
\item Participant is a developer of an in-house optimization application.
\item Data are facts without context or intrinsic meaning.
\item Standard hierarchy.

\stopitemize
\item Survey XV
\startitemize
\item Participant is a SIGINT analyst.
\item Data is a small, measurable, description of the world.
\item Hierarchy of precision.
\stopitemize
\stopitemize

\stopcomponent
