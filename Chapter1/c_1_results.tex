\startcomponent c_1_results
\product prd_Chapter1_Introduction
\project project_thesis

These questions of interest are posted to the reader to keep in mind in the results section. My personal analysis, presented after the “raw data,” uses these questions of interest as framing devices for my reflections on the individual interviews.

\subsection{Questions of Interest and the Methodology of Analysis}

My \quotation{hypotheses} are described as questions of interest to reflect the rapid iterative nature of abductive explorations. They provide research directions that act as broad guides to the formation of a universe of discourse for future research rather than predictive statements about reality.

The intent of the questions is to frame analysis and guide it towards useful and interesting areas. We need to consider how the evidence relates to these questions of interest.

Each interview, after transcription, was subjected to recursive analysis for my personal reflections on the interviews. I summarized six to ten lines of each interview in a one-line summary. Then between three and six summaries were summarized, filtering for statements about the user’s conception of data. Although self-transcription transmits personal bias, two significant factors prevent a traditional double-blind study. An untested methodology is no place for the mass utilization of volunteer interviewers. The limited scope allowed me to retain control of the interview process and to provide for the best possible interviews for each participant while retaining the basics of the \SDFN. Because I conducted each interview, the bias would have already been introduced; providing for pious-sounding human coding would have lent false reliability to something inherently subjective.

My personal reflections are very simple. I have tried to extract each participant's intuitions about data from the recursive analysis.

\subsubsection{Question of interest 1: do people have different realities of data?}

If this research produces nothing else, it must investigate whether people have different conceptions of data. This idea was the central intuition that prompted this research, and its testing will demonstrate whether or not there is anything to my intuition.

As the organizing factor of my analysis, this question of interest will focus my activities. It will justify further research on the nature and subjective constructions of data from my experimental results, or else its demonstrable failure will justify not doing so.

The question \quotation{Do people have different realities of data?} defines an overly large universe of discourse, one impossible to study at a useful level of granularity in one research project. The very breadth of the question precludes the determination of any useful and specific facts about the world besides simple exploration of the assertion that people have different understandings of data. The intent of this research and of this question is to generate interest in the research of the nature of data, how people understand it, and to demonstrate that there are potential areas of philosophy to research.

I want to see if, beyond my intuitive insight, people actually have different conceptions of data or if my perception of different conceptions is an artifact of the requirements-gathering process of designing a database. It is therefore not sufficient to state that people have different understandings of data depending on whether they are dealing with it in a technical or scientific context. We must look for evidence.

This question of interest, in its reach, is not ambitious. It suggests no predictions about peoples' conceptions of data, how they act with different realities of data, or any other fact about the world. Instead, it simply directs us to see if there is anything of interest for further explorations.

\subsubsection{Question of interest 2: can my methodology probe people's realities of data?}

My methodology has a simple job: to assess what people mean when they use the term \quotation{data.} This question of interest is designed as a sanity check. I am investigating a new idea with an untested methodology. It is vital to consider that the success or failure of Question of interest 1 is directly modulated by the success or failure of Question of interest 2. Therefore, the methodology itself deserves distinct analysis.

The methodology should be of use to more people than just those investigating peoples’ conceptions of data. If the methodology is useful and judged to add value to Question of interest 1, analysis of the methodology should indicate whether other people could use it to investigate matters of interest to them.

Question of interest 2 is asking: do these results make sense? Sense-making is a matter of internal and external consistency. This question should force me to explore whether the \SDFN\ correlates with interview results and whether the types of results make sense relative to the survey.

Beyond consistency, I must also ask: Is it possible to get these results from this methodology? In this case, I need to make sure that I am not reading imaginary meaning in the tea leaves of the results. Because this kind of external self-reflection is difficult, the question must be simplified to: Do the results surprise me? If they do not have elements of surprise, then the probability that I am projecting meaning into them must be strongly considered.

All of these are very self-critical questions, as they must be to explore the impact of an untested methodology. I am trying to consider whether my methodology can present a persuasive story, and if it can, does it?
\section{Interview Analysis}

My interview analysis discovered three different conceptions of data. It would be hard to deny that interviews I and II have data as communication, III and IV have subjective observations (with IX hinting at them) and the rest considering data as objective fact. With these broad differences evident, I feel question of interest 1 has been satisfied.

The observation constructions differ strikingly from the numeric constructions, possibly differing on a fundamental perception of reality. As one interview is trying to render the relationships between matter in the world as numbers (objectivist), another is suggesting that everything emits data and we must filter it. The conflict is records versus measurements versus signs. Does data measure objective reality, record subjective reality, or merely transmit signs? Numbers are seen as a result of precision most of the time, whereas observations are building their way towards knowledge.

\subsection{Result 1: Data as communications}

Data, in the communicative sense, merely requires signs and things to communicate with those signs. The data can be rendered as bits or marks on paper, but it is seen as a factor of semiotic import rather than as something to be discovered or filtered.

This construction is substantively different from the other two inasmuch as it does not uphold data to be an aspect of reality. Instead, data is produced as a function of human intent. Because this understanding does not concern itself with interactions of the real, there is a far greater difference between this and the other two than between the subjective-objective constructions. However, the passivity of this construction allows it to accept facts produced from either source as something to be encoded, stored, and transmitted. Significant research needs to be done to explore how this construction of data relates to the other two.

\subsection{Result 2: Data as subjective observations}

Data, in the subjective observations, requires contextualization and filtering. Everything emits data as sense impressions\footnote{Like the ancient Aristotelian idea of species (particles of sensation). Light was the medium that visual species traveled within.

While this ancient philosophy of image is not hugely useful to us, the same intuitions that led to it could have some parallels with data as subjective observations. This research area could make an interesting bridge between intuitive and experimental philosophies.
} that can be captured by us. Thus, to perform sense-making activities, we must filter and contextualize the interesting data so that it can become information.

Subjective data lends itself more to cyclic hierarchies, where data begets the information and knowledge used to collect more data, reflecting an interestingly constructivist view of knowledge. There is quite a lot here available to future research, and I do not feel sufficiently confident in my sample size to make any assertions as to relationships between data and the various philosophies of knowledge or science, though the subjective nature of observations may tend slightly more towards Latour or Feyerabend.

Of more interest is that this inherently subjective data is constructed from the mind's impressions of the surroundings, rather than revealed through measurement of the surroundings. The understanding of the embodiment of data is a significant difference between the two understandings of data.

\subsection{Result 3: Data as measured facts}
Objective data comes with its own context \quotation{baked in.} It is, in many ways, rare: it requires positive effort to generate, and higher quality data requires a commensurate increase in effort. Data requires analysis to uncover the extant patterns of reality, and with enough data, knowledge about the singular real can be generated.

Objective data requires that data be a fact, usually a numerical, reproducible representation of reality that conveys an understanding of measurement quality and units. Objective data is not filtered, because it is collected with prior intent and all elements of the \quotation{data set} may produce interesting patterns.

Both humans and sensors can reveal objective data, which is embodied in the things being measured. There seems to be no significant link with any of the major philosophies of science. Although my investigations did not explore confirmation, falsifiability, or paradigms, there seems to be a common understanding that data-as-fact accurately represents the universe within the constraints of measurement. This may be because the participants believed data to be a building block upon which their hypotheses or understanding of the universe could be built.

\stopcomponent