\startcomponent c_1_execSummary
\product prd_Chapter1_Introduction % but you can use it in other products anyway
\project project_thesis


What is data? That question is the fundamental investigation of this dissertation. I have developed a methodology from social-scientific processes to explore how different people understand the concept of data, rather than to rely on my own philosophical intuitions or thought experiments about the \quotation{nature} of data. The evidence I have gathered as to different individuals' constructions of data can be used to inform further inquiry of data and the design of information systems. 

My research demonstrates that people have different constructions of data. The methodology of the \infull{SDFN}, created for this dissertation, has proven able to probe those understandings. The \infull{SDFN}, loosely based on a \infull{DFD} and combined with ideas from \infull{SNA}, provides a way of discovering practical definitions of hard-to-operationalize terms like {\em data}. The process of repeatedly categorizing various items as data allows the methodology to explore how participants actually use the term, rather than relying on theoretical dictionary-based definitions.

Analysis of the interviews found three different constructions of data: data as communications, a container for meaning; data as subjective observations, sense-impressions filtered by knowledge; and data as objective facts, measurements revealing the relationships of reality\footnote{ For a longer summary of this research, look at Appendix D. The peer-reviewed paper on page \at[AAD] was presented at the IEEE 5th International Conference on Computer Sciences and Convergence Information Technology in Seoul, Korea during the process of writing the thesis. }. 
 

\stopcomponent
