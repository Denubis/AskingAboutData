\startcomponent c_1_aims
\product prd_Chapter1_Introduction
\project project_thesis

In \infull{IST} studies (\IST), I have noticed that practitioners use and understand the term “data” differently than the people they are helping. The purpose of this research is to explore the different conceptions of data that may exist beyond the domain of \IST\ and demonstrate a methodology that allows practitioners to access the conceptions of data present in their workplace.

Exploring a conception of data is fundamentally a philosophical problem. A person’s conception of data stems from the affordances they attach to it, their belief in its underlying qualities, and their differentiation between data and non-data. However, this philosophical problem cannot be solved through intuition alone: a methodology is necessary to extract a person’s conception of data.

These individual conceptions can then be formalised as “philosophies of data.” By ‘philosophies’ we mean answers to the questions like, ‘What is data?’, ‘What is data for?, ‘How do I know the data is reliable?’, and ‘What are the properties of data?’ While individuals may not “have philosophies,” understanding that individuals engage philosophically with their conceptions of data allows the creation of a tool to probe those philosophical conceptions of data in a workplace. By probing conceptions, the \IST\ practitioner effectively uncovers de facto philosophies of data in individuals.

This research, however, does not propose to uncover fundamental philosophies of data, only some common conceptions of data that may exist in workplaces. These different conceptions of data can produce frustration, error, and miscommunication if people with different conceptions interact unknowingly. Conceptions of data include context, reliability, constraints as to its nature (can it be a description, must it be a number), the means of collection, and the means of manipulation.

I have created a methodology called the \infull{SDFN} (\SDFN). This interview technique has elicited people’s conceptions of data\footnote{their de facto philosophical approaches towards knowing that something is or is not data}, demonstrating three different conceptions within a particular industrial research workplace. A survey developed from the \SDFN\ technique hints that there may be different conceptions of data present in the intelligence analysis community and the \IST\ practitioner community.

It is my hope that \IST\ practitioners can use the \SDFN\ I have developed to make better interfaces and databases: through the understanding of a client’s expectations of data, the system can provide natural interaction methods that conform to the client’s expectations of what data is and is not. The SDFN might also be used within an organization to reduce miscommunication and error: the explicit definition of one particular conception of data for a workplace.

\stopcomponent
