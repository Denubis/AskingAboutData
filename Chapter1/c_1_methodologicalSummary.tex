\startcomponent c_1_methodologicalSummary
\product prd_Chapter1_Introduction % but you can use it in other products anyway
\project project_thesis

The primary result of this thesis is the methodology of the Social Data Flow Network. The SDFN uses repeated categorization to explore how individuals group informational or communicative flows into categories. By eliciting categorizes that focus on data, information, and knowledge, the participants use the categorization to operationalize their epistemological understanding of data: they indicate what is and is not data and how it becomes information and knowledge. This elicitation helps both the interviewer and the participant to discover their own situational conceptualization of data.

The repeated categorization allows participants to generate and resolve cognitive dissonance situated around the differences between their theoretical definitions of data and their practical uses and categorizes of data. In interviews, participants demonstrated a refined understanding of their own conceptions of data at the end of the interview, catalyzed through their participation in the SDFN.

The SDFN involves the articulation of roles as entities, descriptions of content flows between those entities and the categorization of those flows as data, information, knowledge, or other. Participants iterate over a task domain defined at the start of the interview, discussing all the entities and flows between those entities involved in the task. The interview concludes with an opportunity for the participant to self reflect on their \quotation{philosophy} of data, discussing what they categorize data as and how it becomes information and/or knowledge.

A scenario based survey, inspired by the SDFN was also trialled with less satisfactory results. While the survey did demonstrate that intelligence officers, IST professionals, and other industrial research employees did have different conceptions of data, it did not do so with any statistical rigor nor with the depth of discussion that the interviews provided.

The \SDFN\ combines two concepts for a novel purpose. It is a graph\footnote{A graph, strictly speaking, is any diagram that contains edges and nodes. A node is the component of a graph that is a point. The point can be labeled or unlabeled. The node is the element of the graph that is a representation of a thing. Sometimes the thing being represented is a computer or a person, or a place, but in any event the node represents a noun. Edges on the other hand are the relationships or connections between nodes. An edge represents a \quotation{flow} of action or stuff between nodes. Edges traditionally have served as network links, roads, phone lines, and simple representations of adjacency. A graph is a non-topological method of representing the relationships between entities through edges and nodes.

Edges can be directed: they show a flow or relationship from one node to another. The direction on the edge indicates the direction of relationship. For example, consider Alice and Bob. To represent Alice sending a letter to Bob, we would make both of them nodes and draw a directed edge from Alice to Bob indicating the one way flow of the letter. By adding the concept of directionality to edges, a causal element is introduced to the representation-specifically, that the originating node causes a relationship to the recipient node. This addition of causality then precipitates the idea of connectiveness.

A node may or may not be reachable by other nodes. A graph or subgraph where every node can be reached from every other node is called a strongly connected graph. A graph where that's not true is weakly connected. When we apply the idea of strongly connected graphs to social networks, we can identify small groups by identifying strongly connected subgraphs within a larger, weakly connected graph.} that combines the idea of the social network with that of the data flow diagram. In social network analysis, it is possible to represent interactions between people, a social network, through graphs. Each node on a graph represents a person and each edge represents some sort of connection between people, as a function of the interactions of interest to the researcher\cite{Scott1988}.

The Data Flow Diagram\cite{Larsen, Sun2006} contributes its diagrams to the \SDFN. A \DFD\ originally was designed for structured programming. The document produced by the \DFD\ would combine the delineation of a universe of discourse via the context diagram with the highly precise definition of flows into and out of that diagram. A \infull{UoD} (\UoD)\cite{Wiener1950} is the term used for defining the topic under consideration. Everything within the \UoD\ is relevant and must be modeled. Everything outside the \UoD\ is irrelevant. Interestingly, as the \DFD\ was repurposed for business modeling, the \UoD\ remained the same: it is still asking, \quotation{What bit of reality do we care about right now?}

The \DFD\ would then be refined through a process of \quotation{zooming in} on that context diagram to expose the transformations required to produce the outputs from the inputs. Each additional level would seek to conserve inputs and outputs, and thereby produce a diagram that could be mapped to the functions and variables necessary for a structured program.

The \DFD\ contributes great ideas to the \SDFN. It contributes the idea that data {\em is} something that can be modeled. The conception of data embodied by the \DFD\ is that the modeler can translate reality into data-as-bits and that data could be described through text. All actions in the data flow diagram are considered either flows or transformation. Data flows from sources through transformations, and out into sinks. The sources and sinks are entities outside the scope of the diagram. By decomposing these transformations into ever simpler and more detailed sets of sub-transformations, modelers could design an entire software system intended to process and transform data. The modeler acts as translator: taking the described reality by the client and forcing it into a computerized mold. Repurposing the methodology of the \DFD\ by subtracting the modeler's translation suggests that it might be possible to use my method to probe and document a client's subjective reality.

The \DFD\ also contributes an iterative structure for the definition of reality. The iterative techniques explore the \UoD\ in order of increasing specificity from the vague context diagram describing the universe of discourse to highly detailed sub-sub-sub (etc.) transformations required deep in the diagram. By starting with broad generalizations, the \DFD\ insured that the client was thinking about the whole task and did not immediately become fixated on any one aspect. With the \DFD\ iterating across each declared \quotation{transformation} and decomposing it, the details of each transformation were both evoked and then situated in the scaffolding of the broader context. The requirement to conserve inputs and outputs eliminated any question of missing aspects of the diagram or other design-based blind alleys. The idea of iterative exploration and definition is extremely valuable to the \SDFN.

The Social Network Graph provides the concept of a social network\footnote{A social network graph is a mapping of a person's relationships with other people into non-topological graph format. Each relationship is a directed edge; each person, a node. The social network graph is used in many different fields: communications, social media, and sociology are some of them. In many ways, the idea of the social network graph is strongly related to the ideas of actor-network theory\cite{Nejdl2006}.} to the \SDFN. The Social Network Graph also contributes a novel idea about the {\em scope} of edges. Edges in the \DFD\ were simple {\em flows} of data, representing the movement of trivial signs. In the social network graph, edges can be individual communications, orders, relationships, and objects. The huge diversity of edge types suggested by a social network graph, when combined with the \DFD, ruins the \DFD\ for its original purpose: the modeling of software systems. However, they also suggest different possible models that can be applied to the \DFD\ format.

\placefigure[]
[fig:SNA]{Social network graph of \#sla2009 tweet replies to June 19, 2009 \quotation{The thicker the line, the more times you sent an @reply to that person. The more lines you have, the more @replies to different people you sent. If you don't appear on the graph, but know that you sent out @replies, it's because the person you sent your @reply to never sent out an @reply and so that person won't appear on the graph and unfortunately, you can't either! Interestingly, a few people only sent replies to themselves, so they do appear on the graph as a line that goes back to themselves.} -Image used with permission, created by: Daniel P. Lee, MLIS.}
{\externalfigure[Chapter3/SNA.jpg][factor=fit,frame=on]}

In communicative analysis, social network graphs are used for linguistic analysis\footnote{\in{figure}[fig:SNA] provides a trivial example of linguistic analysis as applied to a set of twitter replies during a conference. The different line weights are used to denote quantity of communications along a radially distributes set of nodes. Other approaches can be far more complex, looking at patterns beyond simple frequency\cite{Barnes1983, Reffay2002}.}. It is possible to explore the control structures of a group by noting, with an edge, who is talking to whom. By exploring the frequency and directionality of those notes, analysts gain insights into the power and influence roles of social networks. As such, the \quotation{thought leaders} of the small group can be identified.

Moreover, by graphing flows of communication, it is possible to identify small groups within larger groups, as these small groups will communicate strongly between each other and vaguely to nodes outside. In other circles, this behavior is known as siloing\cite{Jones}. One design intent of the \SDFN\ is to confer the ability to identify siloing. By rendering flows between members of an organization, it should be possible to identify strongly connected sub-graphs, which suggest communicative silos within that organization.

The social network graph contribution alters the diagramming rules of the \DFD. Social network entities can be any actor that participates in a communication. The \SDFN\ is a diagram exploring flows of data between actors, instead of flows between transformations. By creating a web of affiliation\cite{Pondy1967} between these entities, it should be possible to describe the communicative realities that an individual perceives. It should therefore be possible to explore how they understand the nature of data by exploring how they describe its movement from entity to entity in the \SDFN.

Despite the terminology of actors, and the use of a social network, my research does not yet incorporate actor-network theory\cite{Latour2005}. While Latour's work offers many useful ideas for understanding the world, it still imposes a framework from which biases may be imparted. Therefore, while I do not use actor-network theory here, it may be useful in later research exploring the implications of held philosophies on Latour's work.

The \SDFN\ does not try to be explanatory, comprehensive, or objective. The point of the \SDFN\ is to reveal part of how the participant understands a concept, not to build upon that understanding nor transform it into a model for a computer system. Consequently, no design provisions in the methodology allow two or more peoples' categories to be reconciled. More work will be necessary before the \SDFN\ can be used directly as a design methodology.\stopcomponent