\startcomponent c_conclusion
\project project_thesis

\chapter{Conclusion}
This research aimed to understand how users perceive the nature of data. In a more limited sense, I wanted to demonstrate that users have different conceptions of data and a method for discovering hints as to how they use and understand data. In that light, I have been successful. This research should serve as the foundation to more specific and focused studies into the different social constructions of data present in our societies and what epistemological and ontological basis data may have.

With the use of the term “raw data” there is some evidence for trading zones being established as people with different usages of data must discuss it to function in the same way. In a more limited sense, this evidence may just be an artifact of the narrowness of the study. Either way, more indication into the establishment of trading zones with different technological concepts is strongly indicated.

\section{Results}

People have different realities of data. My methods are capable of generally distinguishing between different realities. I have found three distinct realities: data as communications, data as subjective observations, and data as objective facts.

In the data-as-communications category, users generally consider data to be electronically encoded semiotic representations, or bits stored on a computer. Although some people in this category believe that data are any communications (i.e., data are signs), most believe that data are re-encoded and computerized signs. This construction tends to share some space with the other understandings in the sense that people can use the term \quotation{data} in an electronic sense when referring to \quotation{computer data} and in the \quotation{proper} sense of facts or observations. However, even in that case, there is more a blending of the two meanings rather than two distinct meanings.

There are people who classify data as observations. Observations, in this case, are any acts in which a sensor (such as the brain and eye) takes note of the world. These observations are then filtered according to some knowledge and situated with context as information. All observations are data, but not all data is important. Only data situated properly in the correct context can be assessed for utility.

Considering data-as-facts, people equate facts with objective, true, statements about the world, and comprehend data as small facts about specific parts of the world. Facts are a superset of data, information, or knowledge, being objective and truthful statements about the world regardless of scope. People in this category tend to favor strongly objectivist views of science and tend to be scientists and some types of engineers. Data as facts does not require external context, and is packed with its own meta-data: its precision, provenance, reliability, and reproducibility, among other things. Data as facts strongly tend to be numbers.

It is clear from the, admittedly biased, results of the surveys that many other variations exist on these three simple categories. These findings are in no way exhaustive, merely suggestive of what I found in the majority of my interviews and surveys. It is obvious to me that far more research is necessary before any definitive statements can be made about the different realities of data present and, especially, their possible interactions and failure modes with each other.

The Social Data Flow Network has great utility as an reality-discovery methodology. The \SDFN, inspired by the data flow diagram methodology of systems design, is quite able to cause participants to differentiate between practical definitions and theoretical definitions when discussing a participant’s construction of data. And is therefore better than more focused requirements gathering methodologies at this task.

The \SDFN\ produced many useful discussions of data, causing participants to reflect on their own working definitions and on how they contrast with formal or intuitive definitions. The propensity for generating self-reflection is a useful product of the iterative categorization of the \SDFN. This process of iterative categorization provides mechanisms for inspecting peoples' functioning definitions, rather than the theoretical definitions under discussion.

The internal consistency of the results between the analysis of the transcript and the analysis of the diagrams suggests that the generation of the \SDFN\ diagrams provides a useful triangulation method for both reflection during the interview and subsequent analysis. However, the \SDFN\ requires significant refinement: it must be made less intimidating; steps must be made to make analysis easier; it needs to be codified such that other people can perform both the collection and analysis stages of the \SDFN..

\section{Methodological conclusions}

The interviews via \SDFN\ and surveys were successful. The \SDFN\ diagrams were useful in producing directed discussions which uncovered personal realities of data. My chosen research group, the fine engineers, scientists, and researchers at the company I conducted interviews at, were a fantastic initial target. The group presented a number of different trading zones as scientists and engineers interacted in a highly competitive area. The research was both practical and theoretical and had to show progress on both fronts, creating many different goals and many different research languages.

The informal atmosphere also presented significant advantages. My interactions with the team allowed for a sense of reassuring informality in the interviews while the non-interview interactions (mainly sharing meals) allowed for a sense of rapport that helped the difficult sections of the interviews slide past. This informality allowed people to communicate their insights without worries that what they were going to say would backfire.

The survey in practical matters was a partial success. Although the second survey was generally able to probe the thoughts of the participants, a number of people thought the survey had a different intent, and thus their answers presumed different questions than those I asked. As such, the wording and the scenarios could be tightened. All things considered, however, the survey's results are acceptable for use, but could stand quite a lot of work to improve their reproducibility and accuracy. Nonetheless, the survey presented some useful insights as to how wildly different people understand the nature of data.

\section{Further Research}
Further research will be pursuing two primary goals: methodological improvement and philosophical exploration of the philosophy of data. This research is about applied philosophy: using philosophy as an engineer would use science. This research is designed to go some distance towards combating the \quotation{ivory tower}\cite{Frodeman2009} perception of philosophy, especially philosophies of technology\footnote{Our philosophers of technology and science, in the main, are not practitioners. While their outsiderness lends some objectivity to their philosophies, their lack of capability in the fields of which they speak (especially high-tech) means that many actual practitioners in those fields feel no obligation to listen to what these outsiders are saying.}. The ultimate targets for this research are the practitioners who use data every day: engineers, businesspeople, scientists, systems designers. With this research, I hope to encourage philosophical thinking in those areas and to demonstrate that many worthwhile philosophical activities must be rooted in a firm knowledge of their respective areas.

If this research can inspire another \IST\ practitioner to explore aspects of the philosophy of technology in which they have a personal investment, then this research will have been useful. All the further research directions {\em must} be focused in integrating philosophy with practice. Research that involves practitioners and enhances their capabilities for meta-reflection is good research. Research that isolates practitioners and returns nothing is bad research.

This section cannot articulate all the possible research directions, for there are too many unknown unknowns. While most of the known research directions are designed around targeting or identifying flaws in the current philosophy or methodology, those are only short-term goals. In the longer term, as the rate of change increases in the world, our understanding of data will change with it, as data is the reified basis of software, in one way or another. 

\section {Final reflection}

There is science to do and research to be done in understanding different conceptions of data. This dissertation merely paints the broadest outlines of possible questions to ask. The idea of different constructions of data offers many ideas to many disciplines, and borrows just as many. In the discussion of \quotation{what is data?}, there is no consensus on an answer, and investigating that question of the nature of data and the trading zones constructed around it will pose the majority of the work for the next few research projects. 

One major research direction for the philosophy of data is to define a set of open questions of the philosophy, questions that will guide researchers into more specific areas of inquiry, just as Floridi's open questions have guided the philosophy of information. Creating a framework of research as a research project is something that will allow cooperation and mutual work without much unnecessary duplication of effort.

The results must be accessible. It is far too easy to scream, \quotation{This was a triumph,} when the actual conclusions are hidden behind a wall of polysyllabic vocabulary and inscrutable jargon. The worst danger for this field is articulated by Eliezer Yudkowsky: \quotation{Well, sounding wise wasn't difficult. It was a lot easier than being intelligent, actually, since you did not have to say anything surprising or come up with any new insights. You just let your brain's pattern-matching software complete the cliché, using whatever Deep Wisdom you'd stored previously.}\cite{Yudkowsky2011} If research on the nature of data is reduced to pattern-matching Deep Wisdom without any pragmatic export or falsification of results, I will have failed.

People have different realities of data. They can think of data as stored signs of communications, data as subjective observations of the world, or data as objective and numeric facts. They can think about data in ways not articulated here, and I intend to find out what those ways are.

\stopcomponent