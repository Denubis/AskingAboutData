\startcomponent c_3_methodology
\product prd_prd_Chapter3_Methodology
\project project_thesis

The field of requirements generation is heavily overpopulated with methodologies. These methodologies on the whole, generally presume that the participants are attaching similar meaning to the terms they use, especially when the terms are seemingly uncomplicated ones like “data”, “information”, and “knowledge.” The Social Data Flow Network is an elicitation methodology that can be used prior to normal requirements generation. This methodology helps to map the shared and unshared components of a group’s social construction of reality as it relates to data flows.

The \SDFN\ has its methodological roots in the data flow diagram (\DFD). Information technologists currently use a \DFD\ as a tool for probing current data flows within an organization. I have designed the \SDFN\ as a compliment, allowing a practitioner to uncover an individual's subjective constructions of data, one unburdened by the methodological constraints of the \DFD. The \SDFN\ can be used as an artifact for sparking discussion around practical definitions without the investigator having to enter the interview and ask participants about their personal constructions of data directly. Once data has been characterized by the participant, other requirements generation methods can then be employed to extract a formal understanding of what is needed, paying special attention to where different individuals understand the components of the same process differently.

By allowing people to probe their own constructions of data, the \SDFN\ helps them to express their own understanding in their own language without worrying about being judged incorrect. By creating a sense of cognitive dissonance\footnote{While Festinger's original work is important here, I believe that this model represents the satisfaction of constraints imposed by the categorization of terms as per Shultz\cite{Festinger1957, Schultz1996}. } between the participants' application of categories and their theoretical definitions, the methodology discussed in this section will serve as a way to illuminate how people understand the nature of data. It seems quite feasible to extend this methodology to other research endeavors.

\stopcomponent