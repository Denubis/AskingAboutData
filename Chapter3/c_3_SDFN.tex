\startcomponent c_3_SDFN
\product prd_prd_Chapter3_Methodology
\project project_thesis

This chapter documents the methodology I used in conducting the interviews with the company. This section is organized first into definitions, an exploration of the \SDFN\ as a concept, and then a practical discussion of running an interview centered on the \SDFN.

\subsection{Terms}

This section will introduce the major terms of the \SDFN\ and how those terms are used. The introduction of a new methodology, especially one borrowing from many different fields, is fraught with definitional dangers.

An entity is a noun: a role that can manipulate data. A flow is a noun, representing the {\em flow} of communication or symbols between entities. An entity dictionary is a way of brainstorming entities to make the participant feel more at ease.

\subsubsection{Entity}

An entity is something that plays a role receiving, manipulating, or transmitting data. In the \SDFN, this act of input or output is represented by a noun described in a few words, which then have an oval bubble drawn around them. This bubble is a node in graph theory, with all of the corresponding attributes. The nature of the role is not restricted to a person or physical entity. It is anything that can be made-a-thing-of such that it makes an independent manipulation of data.

Roles are anything that can be conceptualized as an independent manipulator of data. What differentiates an interesting role from one worth skipping over is whether the role somehow transforms data passing through it. There is no restriction on the number of roles that can belong to one person or thing. Just as one person can do multiple jobs, one can also have multiple roles. One hypothesis for future testing is that the role determines the perceived affordances of data. Every role has its own unique activities and therefore uses data in its own way. As the framing of the role changes, the definition of data may change along with it.

Participants should never describe themselves as a singular entity due to ambiguity. Their description of an entity as \quotation{self} is ambiguous to other people reading the chart, who do not understand the tacit assumptions of role and interaction from the same perspective as the participant. Instead, participants should articulate the potentially many roles that they play in an organization as separate entities. While participants self-articulating roles adds a certain artificality to the interview, the self-identification of roles also allows participants to adopt some of the framing of those roles. It is thought that the increased precision gained from artificial role distinctions is worth the contrived nature of the process.

Every role should be unique. However, there is no requirement for a one-to-one mapping from person to role. Because people and things are adaptable and can serve many roles in an organization, artificially forcing the participant to select one and only one definition of self would be contrary to the intent of this exploration. This requirement allows and encourages people to represent passing information to themselves in the guise of the different roles they play.

The avoidance of ambiguity is crucial. It makes the \SDFN\ easier to interpret by other people and it forces the individual creating the diagram to define the nature of the entity explicitly. It is far too easy to use the self as a catchall to avoid the cognitive dissonance of thinking about thinking. It is important to document discrete and unambiguous roles, even though they may map to the same person, because it is the role that understands data, not the person. Furthermore, these different roles-as-self can pass data to one another. I used the following example in the interview: An entity as lecture designer (myself) would pass requirements to the database lab developer (myself) who would pass data to the lecturer (myself). Each of those roles has different requirements for the nature of data. Crucial insights would be lost if they were all collapsed into one entity with self-pointing flows of data.

Entities however, should not be ready-to-hand\footnote{Ready-to-hand roughly means tools that form an unconscious extension of the self. However, I will avoid a discussion of Heideggerian Daesin and other terminology. To learn more: Dreyfus's discussion of Heidegger is not too painful (p.230 for ready-at-hand) and Marshall is using the idea in interface research\cite{Dreyfus2004, Marshall_2003}.}. Devices that take on independent roles are fundamentally different from those that function as parts of another entity. The keyboard used for typing these letters into this document should not be considered an entity in the \SDFN\ sense because except when engaging in this self-reflexive behavior, the entity \quotation{author of dissertation} does not explicitly pass data to the keyboard-rather, the \quotation{author of dissertation} passes data to the computer for processing. The keyboard is part of a large entity and does not manipulate data in my own construction of data. Instead, as an input device, the keyboard is an extension of the computer and represents an interface for the electronic recording of symbols.

At the same time, entity creation rules should not be strictly enforced, as each person may have his or her own conceptions of what an entity could be. A role can be a person, machine, place, or group. An entity is any noun that the interviewee regards as accepting or receiving data meaningfully. Participants must define their own entities, as their own conceptualization of roles is one of the strongest sources of insight into their understanding of their construction of data.

\subsubsection{Flow}

A diagram consisting solely of entities, known as an entity dictionary, is not particularly useful. To represent relations among these nouns, however, we need flows. A {\em flow} indicates a transfer of something between one entity and another. We are concerned with the nature of the transfer instead of the act of the transfer, a verb describing how the transfer is accomplished is not particularly useful during categorization.

This expressed relationship, usually, will be a self-categorized flow of data, information, or knowledge. These flows are edges, represented by arcing lines between one or more entities, although most flows link one entity to another, singular entity. There is little objectivity in these indications of relationship. A flow represents a documented relationship, instantiated from the recipient's understanding of reality, not necessarily a true thing in the shared reality of all participants. As the \SDFN\ is intended as a tool for exploring constructions of data there is little need to find a design that corresponds to the real world and the stakeholders' needs. On the contrary, the subjective expressions of reality can be compared against each other to identify where areas of miscommunication arise.

Practically speaking, flows must be represented as arcing lines between one or more entities. The arc allows readers to differentiate the labels of each flow, with a clear distinction between the over and the under component. Recursive flows, which link an entity to itself, are discouraged, as they tend to represent ambiguous and broad entities. Participants selecting recursive flows should be encouraged instead to delineate the starting and ending roles as entities more clearly.

Each flow has a label and a category. The label describes the content of the flow. The category relates the flows to other flows and ideas in the diagram. Each flow, above the arc, should be labeled with the {\em contents} of the flow. The label is a one- to three-word description of the \quotation{stuff} being transmitted. This description must be unambiguous and unique to the contents of the flow. If two entities are transmitting the same content, care should be taken to ensure that the exact same thing is being transmitted. Minor content variations should be indicated by adding adjectives or other modifiers to the name: \quotation{Results} becomes \quotation{Summary of Results} and \quotation{Formal Results.} Each label indicates a result being transmitted, but the different nature of the things changes the understanding of the thing. Reducing ambiguity is the job of the interviewer and is one of the hardest parts of conducting a \SDFN\ session.

For practice in clarifying the nature of results and in the type of thinking needed to conduct an \SDFN\ session, I recommend the game called \quotation{Zendo} by Looney Labs\footnote{The rules of Zendo can be found here: http://www.koryheath.com/games/zendo/

The essence of the game is that players, through the use of transparent colored pyramids, must use inductive logic to find a \quotation{secret rule.}

An example of a secret rule is \quotation{A [set of pieces] [is true] if it has at least one green piece.} And through rating constructions of their own true and false the leader of the game describes a universe of discourse with the secret rule as the governing element.

The critical element of the game, for purposes of this research, is that the leader of the game must, by the rules, refine ambiguity from any guesses the players may make. \quotation{Clarify the Guess: If the Master does not fully understand your guess, or if it is ambiguous in some way, the Master will ask clarifying questions until the uncertainty has been resolved. Your guess is not considered to be official until both you and the Master agree that it is official. At any time before that, you may retract your guess and take back your stone, or you may change your guess. If any koan on the table contradicts your guess, the Master should point this out, and you may take back your stone or change your guess. It is the Master's responsibility to make certain that a guess is unambiguous and is not contradicted by an existing koan; all Students are encouraged to participate in this process.} The process of clarifying guesses to eliminate ambiguity is exactly identical to clarifying entities and the labels of flows. Besides being a fun game, it is crucial practice to get a feel for the level of precision required in the SDFN. }. Specifically, if one can run multiple sessions of the game successfully, the same skills in clarifying statements and assessing the nature of things will be used in this methodology.

The core of the \SDFN\ is the process of categorization. The \SDFN\ encourages participants to discriminate and categorize flows. By relating different flows through the use of category, it is then possible to induce the definition of the category through its flows. The category should be written under the arc. In computerized renderings, the over/under distinction is less important, so long as the label and category of the flow are clear.

When creating the flow, the participant should first be prompted to label and then to categorize the flow according to a pre-formulated short list of categories. This list of categories should contain the most common expected categories of participants. By prompting the participant with a list, the interviewer focuses the categories on the topic of the participant's choice. However, participants should always be able to add their own categories to this list. For example, in the interviews, I always prompted participants with \quotation{Data, Information, Knowledge, or other.}

There should always be the option for other. But the other category should never remain as other; the participant should name it. Some participants used categories such as Emotion, Wisdom, or Request. In no case was a flow allowed to remain other. These new categories were created on the fly and used as part of that participant's diagram from then on.

At the same time, people should not classify their own domains without any initial guidance. All but the most self-reflective will be paralyzed by the many choices available and not entirely clear on the distinctions the interviewer wants them to draw. Thus, my question took the form of \quotation{Data, Information, Knowledge, or O} rather than \quotation{How would you categorize this flow: data or not-data?} Denoting sample categories creates a negotiable universe of discourse for the categories.

\placefigure[]
[fig:flow]
{A trivial SDFN used to illustrate the idea of “flows” and wormholes. Crossed lines become unbelievably messy, and so the \quotation{wormholes} are a far better method for routing lines across other lines.}
{\externalfigure[Chapter3/SDFNWormhole.pdf][factor=fit,frame=on]}

If a diagram becomes too crowded, it is quite acceptable to make \quotation{wormholes} on the paper design during the interview. A wormhole is some symbol (usually an *) and accompanying identifier\footnote{A character, number, or unique symbol all serve well.} placed more than once on the paper. Each symbol sharing an identifier should be considered connected, which may allow for easier routing. In extreme cases, a wormhole may have one arrow leading from it to represent the inbound connections of all of the flows connected to the other wormhole. The only real restriction is that the creation of wormholes should be unambiguous and clear both at the time of creation and afterwards. Sometimes, in the case of major changes, it is better to redraw the design than to use too many wormholes. This action is sometimes quite desirable, due to the edits that the participant may introduce in the entities, flows, or topology on the second draft.

\subsubsection{Entity Dictionary}

Entities and Flows are the core parts of any \SDFN\ diagram. However, not all participants may have the ability to easily understand the nature of entities. For that reason, and as a precursor to group-based \SDFN\ creation, I engaged some participants with creation of an entity dictionary, a simple list of entities that may be involved in the \SDFN.

An entity dictionary is a simple, non-authoritative, brainstorming device in case the participant is unsure about where to start. Instead of starting the \SDFN\ with two entities and a flow connecting them, I will encourage the participant to imagine all the different entities with which they engage on a daily basis, to name them, and to describe their roles. The immediate feedback, both positive and negative, on each described entity teaches participants to think in terms of roles. Once they have filled a page, most will have internalized the meaning of entity.

Through the creation of this entity dictionary, a number of interesting themes will appear, based on participant enthusiasm or repetition. I was especially careful to pay attention to offhand comments about entities or the participant's work during the creation of the dictionary, as these comments will most likely indicate interesting topics for the interview. The dictionary should be started by encouraging each participant to name an entity that represents them in some role, and then the scope should be gradually broadened to things and people they work with.

To those familiar with the \DFD\ methodology, the idea of the entity dictionary is almost completely opposite to that of the \quotation{data dictionary} of the \DFD. While the Data Dictionary is a device for the accurate specification of data in the data flows, compiled during and after the creation of the diagram, the entity dictionary is a piece of scaffolding designed to help participants think the right way about entities.

Unlike a data flow diagram, the entity dictionary is not authoritative\footnote{Authoritative: a canonical listing and extremely precise description of the structure and components of variables.}. In a \DFD, all flows must be decomposed\footnote{Decomposed: simplified by breaking the components of a flow (or transformation) apart into separate components. An example of a decomposition may be an \quotation{Address} flow, that is subsequently decomposed into 4 flows \quotation{street address + city + state + postal code} In the same way, a transformation can be decomposed. \quotation{Mail a letter} can be decomposed into \quotation{Look up address -> Find Zip Code -> Assess Postage -> Attach letter} } to their atomic definitions\footnote{For example, a \quotation{string} is defined as a "series of characters from a to z and A to Z as well as numbers, spaces, and punctuation. This level of excruciating detail is necessary for accurate implementation in a computer.}, which correspond with database or programming structures. This requirement exists because the \DFD\ has its roots as a programming design, and therefore must be able to explicitly define the data structures of a program. Because the \SDFN\ is probing a non-computerized theoretical area, the requirement of precision is unnecessary and counter-productive, as it distracts the participant from their task. The object of the \SDFN\ is to probe functional definitions, not to have all participants arrive at the same constructed definition of the \UoD.

\stopcomponent
