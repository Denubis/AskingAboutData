\startcomponent c_3_bubble_analysis
\product prd_Chapter3_Methodology
\project project_thesis


Participant used the term information to describe a communication of meaning. In this interview, the term was used to describe communications of \quotation{status.} In the sense of the diagram, status is ambiguous. Because people describe their status to the people in charge, it is a communication between people and not a technical communication. Yet it is also the game reporting the status of the data representation of the player to the player. Supporting this, the term \quotation{information} describes communications that are explicitly privileged above data. 

Participant used the term knowledge to refer to expertise. Knowledge can be shared and is asserted to be a communicable view of reality. The players in charge of the group of players explicitly engage in knowledge sharing, and impart that view of reality to their apprentices. 

Participant, in the bubble diagram, seemed to use data to refer to contextless communications. These communications can originate from a computer or a person, but they fall into two significant categories: activity causing and unprivileged. 

In the activity-causing context, participant described communications {\em to} people and computers. While communications from computers are information, the commands to the computer are data to the game, and explicitly contextless. Data can also be transmitted to a person and is a simple alert designed to cause activity. Curiously, in the same category, messages from the players in charge to the rest of the group are also data. They seem to be contextless and simple instructions. There seems to be a different ontological structure between commands to the game and commands to the players, despite both being described as \quotation{data.}

The unprivileged communications context seems to be attached to communications where the senders cannot know what they are talking about. In a sense, such a communication must be viewed with skepticism: it is a minor or inferior form of information without reliability. In this context, it describes communications from apprentices to the people in charge. Participant believed they do not know enough to offer information. As such, their communications are only data, explicitly described as \quotation{opinions} that do not have any basis for action. 

The bubble diagram suggests that the participant has two different understandings of data simultaneously and suggests no way to reconcile the two. 


\stopcomponent
