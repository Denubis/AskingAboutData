\startcomponent c_3_conclude
\product prd_Chapter3_Methodology
\project project_thesis

This section explored the methodologies I used to collect the data that will be analyzed in the next chapter. The methodology should be completely usable outside the context of the philosophy of data as a way of understanding categories and shared meaning between different people\footnote{See Applications page \at[Applications].}. 

As an extension of the \SDFN, the survey and its process of categorization of tiny scenarios may also serve as a useful methodology for probing \XPhi. My discovery of people's utter disinterest in reading more than three sentences as part of any given survey question may help to shape future surveys even outside the context of \XPhi. On a more positive and pragmatic note, the two techniques complement each other, but do not use the same methods of asking questions: they provide elements of triangulation and backup for each other.

The \SDFN\ construction proceeds in three stages: an introduction, the iterative formation of the bubble diagrams, and an optional theoretical interview. The introduction breaks the ice and establishes the universe of discourse. In the negotiation over the universe of discourse, the introduction also serves to educate the participant as to the techniques of the interview, explaining entities and flows, and introduces sufficient scaffolding for the participant to internalize the nature of the components in their own time. 

The iterative formation of the \SDFN\ is also quite simple. The participant identifies a general flow between two entities, the entities are created uniquely and unambiguously if they do not exist, the flow is unambiguously labeled, and then the participant categorizes the flow as to the criteria established during the introduction. This process then loops until the universe of discourse is adequately explored. 

With a completed \SDFN, participants can then discuss the insights they gleaned from the working of the \SDFN. Unfortunately, little guidance is available for this step; participants will engage in philosophy in their own way. Trying to prescribe the interviewer's response beyond \quotation{ask open questions} both dooms it to failure and constrains the discussion.

Both the rapid categorization of the \SDFN\ and the survey lend themselves to allowing participants to articulate their philosophical insights. The next chapter will explore how effectively these methodologies gathered the information of interest. However, these methodologies should serve on their own terms as new methods in \XPhi\ discovery.


\stopcomponent
