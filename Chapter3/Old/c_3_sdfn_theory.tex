\startcomponent c_3_sdfn_theory
\product prd_Chapter3_Methodology
\project project_thesis

Having spent at least a productive half an hour on the \SDFN\ building, participants may then be encouraged to conclude the session with a theoretical discussion on the nature of data. Not all participants will want to have this discussion, and beyond a vague prompt of \quotation{And would you like to talk about your thoughts on the philosophy of data?}, it is not worth the effort to force reevaluations here.

Participants, if not uncomfortable from the unusual thinking demands of the \SDFN, will generally engage in a self-reflective discussion. It is vital for the interviewer to ask open questions that build a scaffold for the participant's self-discovery. 

Of interest in the discussion are how one transitions from one of the categories to another. The \SDFN\ allows for a very solid investigation of interesting philosophical questions of categorization. My research explores how the participant categorizes things as data, information, or knowledge. The main opportunity of the discussion is to study how the participant thinks about the interaction between categories. In many interviews, the participants would construct a hierarchy in the theoretical discussion and discuss how data became information. The self-generation of a theoretical definition from the practical categorization experience is the main purpose of this methodology.

Other items of interest include any departures from the normal flows. The participant, when creating the \SDFN, will prefer certain categories to others, depending on topic and personal preference. When given the opportunity, I tried to ask about the unusual categories: instances in which the participant either made up a category, combined or concatenated categories, or used a rare classification. For example, when certain participants strongly favored data, only a few flows were classified as knowledge, mainly because they fell outside the scope of the discussion. Subsequent conversation then focused on those flows to try to get a balanced understanding of their nature. 

\sidebar{I found that encouraging the participant to create diagrams tended to create a better response. Some participants tried just to jump right into the philosophical discussion by articulating a definition of data, and were tangled up by their own conflicting understandings:

Participant: Coming into this interview, I thought that the hierarchy was Data, Information, Knowledge. Maybe wisdom or something above that. Or something crazy like that. Way out there. During the meeting, I think I... with your ... non-directional coaching, you helped me to see where things were probably Data, where those Data were used by others to form Information. And then, pieces of Information, there's something higher above Information.}

When discussing combined or concatenated flows, it is important to understand the type of metaphor that the participant is using\cite{Lakoff2003}. While this is mostly important in terms of analysis, my participants normally used low key but figurative metaphors. During this process, I tried to ensure that they elaborated on each metaphor and clearly distinguished the type they were using. A container metaphor (\quotation{data holds information}) is different from that of a concatenative metaphor (\quotation{data alongside information}) and both are different from a combinatory metaphor (\quotation{data and information}). 

A brief discussion of metaphor is in order. Participants gave me clues to their philosophy of data through their use of metaphor, especially their use of verbal affordances. Anytime the participants said, \quotation{filter data,} it was a strong clue that they were very interested in subjective masses of data that needed to be winnowed down. Discussions of precision and accuracy, or any kind of implicit meta-data were pointers to objective elements. 

An example of container metaphor in action: 

\startextract
Interviewer: ... To you, what is Information?

Participant: It's something that's not physical, basically. That means it could be a communication, it could be a conversation or a story. Something verbal.

Interviewer: So Information is any non-...

Participant: Data to me is physical, basically. It's an entity that [fuzzy word] an entity of some sort.

Interviewer: You would say that letter is Data?

Participant: I would say that letter is Data. But what is on that Data is Information. Just to be a bit confusing.

Interviewer: Tell me more.

Participant: Coming from the field we're in, in [removed]. You've got different areas. It would be, as I've said, Data or record. That's just the physical entity.
\stopextract

I notice the \quotation{what's on that Data is Information}, because \quotation{on} is a \quotation{container} word.

\startextract
Interviewer: Data is a container...

Participant: Yes, of the Information.

Interviewer: And Information is content of what type? Is there something common to all Information?

Participant: it's not easy. I'm struggling. It's a struggle. Information, what is Information? It's just... No, I'm drawing blank.

Interviewer: We have right now Data

Participant: And then of course you get Knowledge.

Interviewer: We have Data, Data is a container for Information.

Participant: I'm quite happy with that.

Interviewer: You say physical at some point?

Participant: Yes, physical in [unintelligable]. It doesn't actually have to be physical in a piece of paper, but it can be physical as in an e-mail message.
\stopextract

Whereas, in a different interview, we have the combinative \quotation{and}:

\startextract
Interviewer: Analysis, explanation. Analysis is?

Participant: Something like: model outputs or calculations. Explanation is what that actually means in context of your [work environment].

Interviewer: Class of analysis is Data, Information, Knowledge?

Participant: It's probably more on the Information. Well, it's Data and Information. And the explanation is Knowledge, it had better be.

Interviewer: When you say that, what do you mean?

Participant: You hope that when someone gives you their explanation, you know more than before they told you. Not always true. They can tell you stuff, and you can go \quotation{Well, I understand even less than when I started.} Because if it's completely contradictory to your understanding, you are now really confused. And in the cross cultural... you're often doing these meetings with interpreters in the room. Am I not asking it right? You don't necessarily get an interpreter that speaks Technical [other language]. Often they're here for some other meeting, and they just bring someone from one of the marketers who will be bilingual. Now, they all are bilingual, but particularly senior people will choose not to speak English because it's embarrassing when they speak badly. That said, we can't speak [other language], so who are we to criticize?
\stopextract

The keyword \quotation{and} of \quotation{Analysis is data and information} means that they are combinable and can sit in one flow. The concatenative idea is a lot more difficult. Although in the following example I allude to a container, the interaction between data and information is not the strict container of the first example, but rather has more \quotation{alongside} affordances\footnote{Metaphor provides the affordances of the thing being related in the metaphor. Therefore, a container metaphor affords \quotation{putting into.}  It is by analysis of which set of affordances is hinted at most strongly by the participant that important clues are gleaned towards the participant's philosophy of data.}:

\startextract
Participant: As part of the Information flows to those, I may include a little bit of raw Data, but not very much.

Interviewer: Does this raw Data ...

Participant: Usually photos or a graph.

Interviewer: So you would say that photos are raw data.

Participant: Yes.

Interviewer: Would you say that the raw data is contained in the Information? i.e. you send them interpretation of measurements. As part of that interpretation you have to send them some of the measurements that are really interesting. Would you say that the Data is inside the Information flow, and we can just label this as Information? Or would you say that it's Information + Data?

Participant: I'd keep it inside the Information flow, because if it was just raw Data. They could very easily reach what I think is the wrong idea -- misinterpret.

Interviewer: Therefore, you're not going: \quotation{Here's the Information, here's the Data.} You're going \quotation{Here's the Information, here's some Data inside the Information to back it up.}

Participant: Yeah, that's right. But with just the raw numbers and no context, that's Data.
\stopextract

This discussion, if fruitful, will lead into definitions. Having looked at the ontological transitions in the discussion above, the participant may now be prepared to examine the ontological definitions of the various categories. Here, for the interviewer, is one of the more treacherous spots of the presentation, because it is extremely easy to put words into the participant's mouth through suggestions or overly specific leading questions. Instead, the interviewer should try to allow the participants to use their own inductive process on the categories and transitions they have defined. Normally, the basis of the definitions will occur in the discussion of transitions, but it may not happen in every case. If possible, guide the participants to identify and discuss their own thoughts of how they categorized something as data. 

Although the relationship questions are normally deeper, leading into this discussion through transitions means that the various affordances and other philosophical handles of data, information, and knowledge will be discussed first. Data, lacking form, has no \quotation{natural} or non-constructed affordances. The thingification of data through the \SDFN\ will therefore cause the participant to suggest their own affordances, and thereby strongly hint at their philosophy of data. The other component is to ask them to discuss their understanding of how they know something is data or of how they categorize it. The only structure possible here is that provided by the \SDFN\ itself. If possible, the participants should be prompted to explain their categorization methodologies.


\stopcomponent
