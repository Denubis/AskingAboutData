\startcomponent c_3_sdfn_method
\product prd_Chapter3_Methodology
\project project_thesis

\placefigure[]
[fig:flowchart]
{This is a flowchart exploring the complete SDFN interview process. Different portions of this chapter will refer to different elements.}{\externalfigure[Chapter3/SDFNFlowchart.pdf][factor=fit,frame=on]}

When constructing the \SDFN, the interviewer almost always acts as primary scribe. While the participant should have access to a pen so that he or she can scribble corrections, the interviewer will do most of the drawing. Because the activity of the \SDFN\ is to iteratively construct flows of \quotation{data, information, and knowledge} between entities described by the participant as the subject-matter expert, this section will discuss the precise steps necessary to do that.

Describing a flow always begins with entity declaration. The participant declares which two entities the flow is between and then declares the flow itself. The prompt by the interviewer changes throughout the interview. Initially, the questions should be quite explicit. \quotation{Who starts the flow? What do they do?} In this high-scaffolding variant, the interviewer explicitly identifies the source and the destination of the flow. By reducing the focus of the question to the smallest possible parts, the interviewer helps the participant not to feel confused by trying to think about too many unfamiliar things at once. Early in the interview, breaking the questions into tiny sub-questions allows for prompt feedback. As the participant learns through positive and negative reinforcement, their awareness of expectations reduces the need for tiny sub-questions.

The interviewer must mentally examine an entity before committing it to the diagram. When validating about the source entity, the decision tree is simple: Does the entity exist? If it does not exist, is the role that the participant described short and unambiguous? Entities must have short names, as names take up valuable space. If the entity bubble is more than about an inch in diameter, it takes up too much space and is likely to require redrawing the diagram. 

\sidebar{I encouraged the participant to generalize to the point that the entity can reasonably cover all objects in its class. \quotation{Person} is too vague; it does not give the reader any sense of what role the person occupies. \quotation{Brian} is both too vague and too specific: it identifies a specific person, but does not suggest his role (thereby requiring more explanation) and does not allow for similar types of person. Good entities would be \quotation{Dissertation Author} or \quotation{Casual Reader} or \quotation{Examiner.} They can be generalized to one and only one role; they are simple (few words), and they allow for anyone who fits that role to be classified without unnecessary recourse to edge cases. By using examples from my personal experience, I was able to consistently give the same scenario example across multiple interviews, without looking as if I was reading from a script.}

With the source constructed, the participant needs to define the destination using the same methodology. Later, the scaffolding can mostly be withdrawn and the entire process summarized with an \quotation{And then?} as the participant understands what is being asked. The transition should be gradual, rather than abrupt, and is predicated on the error rate of the participant. If the error rate increases, increase the scaffolding to compensate.

Once the entities are identified and written, the participant should describe a flow. Using an arcing path with a clear \quotation{above and below,} connect the two entities. These arcing lines, representing flows, are oriented to the entities, not the page. Rotate the page if it allows for cleaner arcs and more space above and below for description. 

With the role description in mind, the participant should be asked to note what the flow contains with a theme and variations on: \quotation{This is a flow of...} Make sure, when asking these questions, that they are as open-ended as possible. Flows also should be unambiguous. Ambiguity may be introduced via the introduction of other, similarly named flows. When a flow is described, check all the extant flows for identical and similar names. In the case of an identical name, inquire whether the same contents are flowing. If the names are similar, make sure there are enough adjectives around the flow to distinguish the two. Edit the prior flow if it makes more sense to do so. 

\sidebar{By the end of the interview, only the subtlest prompting will be needed for participants who are really engaged with the \SDFN. I used this as an intuitive measure of success. If participants were categorizing without any prompting or with a simple \quotation{and...} from me, they were probably willing to discuss the theoretical underpinnings of their philosophy. If I was still doing the more painful style of prompting, it was a sign to gracefully end the interview early, as I did with two participants who were just not having any fun at all.
}

When the flow has been described, and the noun (with or without adjectives) has been written above the arc, then prompt the participant to categorize the arc as data, information, or knowledge: \quotation{And is this flow of x data, information, knowledge, or other?} Unusually, this element of scaffolding is never completely removed, although it may be shortened as appropriate if the participant has already categorized things of its nature on the diagram. Be careful not to lead the participant in these questions. 


\stopcomponent
