\starttext

\definefloat
[marginfigure]

\setupfloats
[sidespacebefore=none,
sidespaceafter=depth]

\setuptyping
[margin=]

\setupfloat
[marginfigure]
[criterium=.5\textwidth,
maxwidth=\rightmarginwidth,
default={outermargin,none}]


\setupframedtexts[background=screen, 
	corner=round, 
	width=7cm,
	style={\setupbodyfont[10pt]}]

\placemarginfigure{} {\framedtext{In this section, the steps towards creating a SDFN will be described in the main text, while specific examples from my experience will be revealed in sidebars like this one. These will be reflective on the lessons I learned while creating the SDFN in contract with the prescriptive tone of the main text. }}

This section will describe the process of creating a SDFN in full. It is intended as a descriptive manual to provide readers with a way to run the methodology for themselves. In brief, the SDFN begins through the explanation of terms, a summary of the ideas expressed above. If participants do not understand the nature of entities, an entity dictionary should be created. When participants understand entity and flow, a topic is chosen and the diagram is created. 


\stoptext