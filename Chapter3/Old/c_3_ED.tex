\startcomponent c_3_ED
\product prd_prd_Chapter3_Methodology
\project project_thesis

Entities and Flows are the core parts of any \SDFN\ diagram. However, not all participants may have the ability to easily understand the nature of entities. For that reason, and as a precursor to group-based \SDFN\ creation, I recommend the creation of an entity dictionary, a simple list of entities that may be involved in the \SDFN.

An entity dictionary is a simple, non-authoritative, brainstorming device in case the participant is unsure about where to start. Instead of starting the \SDFN\ with two entities and a flow connecting them, I will encourage the participant to imagine all the different entities with which they engage on a daily basis, to name them, and to describe their roles. The immediate feedback, both positive and negative, on each described entity teaches participants to think in terms of roles. Once they have filled a page, most will have internalized the meaning of entity. 

Through the creation of this entity dictionary, a number of interesting themes will appear, based on participant enthusiasm or repetition. Interviewers should be especially careful to pay attention to offhand comments about entities or the participant's work during the creation of the dictionary, as these comments will most likely indicate interesting topics for the interview. The dictionary should be started by encouraging each participant to name an entity that represents them in some role, and then the scope should be gradually broadened to things and people they work with. 

To those familiar with the \DFD\ methodology, the idea of the entity dictionary is almost completely opposite to that of the \quotation{data dictionary} of the \DFD. While the Data Dictionary is a device for the accurate specification of data in the data flows, compiled during and after the creation of the diagram, the entity dictionary is a piece of scaffolding designed to help participants think the right way about entities.

Unlike a data flow diagram, the entity dictionary is not authoritative\footnote{Authoritative: a canonical listing and extremely precise description of the structure and components of variables.}. In a \DFD, all flows must be decomposed\footnote{Decomposed: simplified by breaking the components of a flow (or transformation) apart into separate components. An example of a decomposition may be an \quotation{Address} flow, that is subsequently decomposed into 4 flows \quotation{street address + city + state + postal code} In the same way, a transformation can be decomposed. \quotation{Mail a letter} can be decomposed into \quotation{Look up address -> Find Zip Code -> Assess Postage -> Attach letter} } to their atomic definitions\footnote{For example, a \quotation{string} is defined as a "series of characters from a to z and A to Z as well as numbers, spaces, and punctuation. This level of excruciating detail is necessary for accurate implementation in a computer.}, which correspond with database or programming structures. This requirement exists because the \DFD\ has its roots as a programming design, and therefore must be able to explicitly define the data structures of a program. Because the \SDFN\ is probing a non-computerized theoretical area, the requirement of precision is unnecessary and counter-productive, as it distracts the participant from their task. The object of the \SDFN\ is to probe functional definitions, not to have all participants arrive at the same constructed definition of the \UoD.


\stopcomponent
