\startcomponent c_3_flow
\product prd_Chapter3_Methodology
\project project_thesis

A diagram consisting solely of entities, known as an entity dictionary, is not particularly useful. To represent relations among these nouns, however, we need flows. A {\em flow} indicates a transfer of something between one entity and another. We are concerned with the nature of the transfer instead of the act of the transfer, a verb describing how the transfer is accomplished is not particularly useful during categorization. 

This expressed relationship, usually, will be a self-categorized flow of data, information, or knowledge. These flows are edges, represented by arcing lines between one or more entities, although most flows link one entity to another, singular entity. There is little objectivity in these indications of relationship. A flow represents a documented relationship, instantiated from the recipient's understanding of reality, not necessarily a true thing in the shared reality of all participants. As the \SDFN\ is intended as a tool for exploring understanding, and not as the basis for an information system design as a \DFD\ is, there is little need to find a design that corresponds to the real world and the stakeholders' needs. 

Practically speaking, flows must be represented as arcing lines between one or more entities. The arc allows readers to differentiate the labels of each flow, with a clear distinction between the over and the under component. Recursive flows, which link an entity to itself, are discouraged, as they tend to represent ambiguous and broad entities. Participants selecting recursive flows should be encouraged instead to delineate the starting and ending roles as entities more clearly. 

Each flow has a label and a category. The label describes the content of the flow. The category relates the flows to other flows and ideas in the diagram. Each flow, above the arc, should be labeled with the {\em contents} of the flow. The label is a one- to three-word description of the \quotation{stuff} being transmitted. This description must be unambiguous and unique to the contents of the flow. If two entities are transmitting the same content, care should be taken to ensure that the exact same thing is being transmitted. Minor content variations should be indicated by adding adjectives or other modifiers to the name: \quotation{Results} becomes \quotation{Summary of Results} and \quotation{Formal Results.} Each label indicates a result being transmitted, but the different nature of the things changes the understanding of the thing. Reducing ambiguity is the job of the interviewer and is one of the hardest parts of conducting a \SDFN\ session. 

For practice in clarifying the nature of results and in the type of thinking needed to conduct an \SDFN\ session, I recommend the game called \quotation{Zendo} by Looney Labs\footnote{The rules of Zendo can be found here: http://www.koryheath.com/games/zendo/

The essence of the game is that players, through the use of transparent colored pyramids, must use inductive logic to find a \quotation{secret rule.} 

An example of a secret rule is \quotation{A [set of pieces] [is true] if it has at least one green piece.} And through rating constructions of their own true and false the leader of the game describes a universe of discourse with the secret rule as the governing element. 

The critical element of the game, for purposes of this research, is that the leader of the game must, by the rules, refine ambiguity from any guesses the players may make. \quotation{Clarify the Guess: If the Master does not fully understand your guess, or if it is ambiguous in some way, the Master will ask clarifying questions until the uncertainty has been resolved. Your guess is not considered to be official until both you and the Master agree that it is official. At any time before that, you may retract your guess and take back your stone, or you may change your guess. If any koan on the table contradicts your guess, the Master should point this out, and you may take back your stone or change your guess. It is the Master's responsibility to make certain that a guess is unambiguous and is not contradicted by an existing koan; all Students are encouraged to participate in this process.} The process of clarifying guesses to eliminate ambiguity is exactly identical to clarifying entities and the labels of flows. Besides being a fun game, it is crucial practice to get a feel for the level of precision required in the SDFN. }. Specifically, if one can run multiple sessions of the game successfully, the same skills in clarifying statements and assessing the nature of things will be used in this methodology. 

The core of the \SDFN\ is the process of categorization. The \SDFN\ encourages participants to discriminate and categorize flows. By relating different flows through the use of category, it is then possible to induce the definition of the category through its flows. The category should be written under the arc. In computerized renderings, the over/under distinction is less important, so long as the label and category of the flow are clear.

When creating the flow, the participant should first be prompted to label and then to categorize the flow according to a pre-formulated short list of categories. This list of categories should contain the most common expected categories of participants. By prompting the participant with a list, the interviewer focuses the categories on the topic of the participant's choice. However, participants should always be able to add their own categories to this list. For example, in the interviews, I always prompted participants with \quotation{Data, Information, Knowledge, or Other.} 

There should always be the option for Other. But the Other category should never remain as Other; the participant should name it. Some participants used categories such as Emotion, Wisdom, or Request. In no case was a flow allowed to remain Other. These new categories were created on the fly and used as part of that participant's diagram from then on.

At the same time, people should not classify their own domains without any initial guidance. All but the most self-reflective will be paralyzed by the many choices available and not entirely clear on the distinctions the interviewer wants them to draw. Thus, my question took the form of \quotation{Data, Information, Knowledge, or Other} rather than \quotation{How would you categorize this flow: data or not-data?} Denoting sample categories creates a negotiable universe of discourse for the categories.

\placefigure[]
[fig:flow]
{A trivial SDFN, but an example of how one would look when created organically item by item. While a well-dressed rendering would swap the positions of author and journal authors, it's quite likely that this SDFN put author too close to the top of the page, and added Google scholar as an afterthought. Crossed lines become unbelievably messy, and so the \quotation{wormholes} are a far better method for routing lines across other lines. In this instance, I have routed two flows through the wormhole, as the unlabeled line from the wormhole functionally extends the entity to the other side of the \quotation{compiled text} line. It also produces the shortest line lengths, compared to routing around the \quotation{author} bubble, which in turn, increases readability.}
{\externalfigure[Chapter3/SDFNWormhole.pdf][factor=fit,frame=on]}

If a diagram becomes too crowded, it is quite acceptable to make \quotation{wormholes} on the paper design during the interview. A wormhole is some symbol (usually an *) and accompanying identifier\footnote{A character, number, or unique symbol all serve well.} placed more than once on the paper. Each symbol sharing an identifier should be considered connected, which may allow for easier routing. In extreme cases, a wormhole may have one arrow leading from it to represent the inbound connections of all of the flows connected to the other wormhole. The only real restriction is that the creation of wormholes should be unambiguous and clear both at the time of creation and afterwards. Sometimes, in the case of major changes, it is better to redraw the design than to use too many wormholes. This action is sometimes quite desirable, due to the edits that the participant may introduce in the entities, flows, or topology on the second draft. 


\stopcomponent
