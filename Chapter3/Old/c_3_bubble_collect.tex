\startcomponent c_3_bubble_collect
\product prd_Chapter3_Methodology
\project project_thesis

\placefigure[]
[fig:bubble1]
{A sample entity dictionary. The participant was brainstorming possible entities for us to explore.}
{\externalfigure[Chapter3/bubble1.png][factor=fit,frame=on]}
\placefigure[]
[fig:bubble2]
{A drawn hierarchy by participant, reflective of a misunderstanding of the technique. This will happen if the methodology is explained incorrectly or insufficient guidance is provided.}{\externalfigure[Chapter3/bubble2.png][factor=fit,frame=on]}
\placefigure[]
[fig:bubble3]
{The first \quotation{bubble diagram.} Note how each flow is categorized below the flow and labeled above the flow, showing the necessity for curved flows. }{\externalfigure[Chapter3/bubble3.png][factor=fit,frame=on]}
\placefigure[]
[fig:bubble4]
{Another bubble diagram. Note the presence of wormholes.}
{\externalfigure[Chapter3/bubble4.png][factor=fit,frame=on]}

As this was the technical process, \in{figure}[fig:bubble1] illustrates the creation of an entity dictionary. As we can see, the entities are just sketchy bubbles with entity names in them, which may or may not appear in later cases. 


The following is my explanation of the process to the participant:

\startextract
Interviewer: Well, we'll be getting back to basically this question after we build the data flow diagrams. This is something to let simmer in your subconscious.

Participant: Yeah it would... it's something that we should discuss more. Yeah, there's context of knowledge, and there's context of data. Maybe in a superficial analysis they don't meld. But if you think about it more deeply, you go, 'Oh, hang on.' It's not just making arbitrary distinction that this is knowledge and this is data. Think about, \quotation{Why do I consider that knowledge? Why do I consider that data?} That really is knowledge because its stuff I know not just there You're doing the Ph.D.

Interviewer: Well, yes, but I'm doing the Ph.D. based on what you tell me. So, when we're building this data flow diagram. What I'm going to be doing is two symbols. Well, we're going to be making circle symbol and a circle symbol will have some sort of designator that is important for you in it. It represents a person, or an entity, that communicates, transmits, verbs, data. Whatever you consider data to be.

Participant: Does that include knowledge?

Interviewer: Does it include knowledge?

Participant: Point Taken.

Interviewer: And then are going to be arcing lines sometimes with an arrow on it to another entity. The arcing lines we'll label with whatever you would classify the data/knowledge/information, whatever term you want to do. It's, for example, you said that there was a 'fail res'. As an example I'm not sure you would use this, one transmission of data would be you to the server, res the fort. We would label this line, 'you, server, rez fort.' And if you want to, you could say this is data, or that this isn't data, this is information. Whatever you think is important to define about that transmission. We'll label that line as.

Participant: Oh, so are they different classifications?

Interviewer: There are whatever classifications you want. What I'm going to be doing here, is looking at how you classify these things, and what small groups you identify and how you think that other people classify these things. And then compare it to how other people classify the things, and look at how the perceptions shift from person to person. There's no detail too trivial for this discussion. Because I'll be using it to look at what other people will be doing.
\stopextract


In this, the theory of the \SDFN\ is not deeply explored. The important thing to do before the start of the process is to gently explain the process and diagramming techniques that will be employed in the rest of the interview. As the participants tend to be unsure at this stage, it is also important to avoid giving them definitions of data, information, or knowledge as they may be looking for specific cues to suggest which philosophy of data to use.

In this \SDFN, I did not ask the participant if they wanted to do an entity dictionary, I just started with the process of articulating the entities. It is acceptable if the participant is chatty during this process. Not only will this help to define the universe of discourse, and establish them in their own minds as subject-matter experts, but also the process of being chatty is allowing them to slip more deeply into the role they are discussing.

\startextract
Interviewer: This is literally trying to render how you perceive the game. ... We'll start with a circle. How would you like to label this circle? What do you think is a representation of you? We can write you here, we can write the computer here.

Participant: We can talk about me as a Clan Master. I'd put that one there. That's to start with. I'd like to have another one, for my different roles. Cause I'm also a player of the game. Which is different from their role as a Clan Master for sure. Because they're often in conflict.

Interviewer: What other entities can we identify in general?

Participant: Across the whole game?

Interviewer: Well, that will transmit data, whatever that is, or information or knowledge. When I say data, feel free to assume I'm also talking about information or knowledge if that helps.

Participant: That's good. Cause often, if you get too transfixed on your own view of data, jeez, there's lots. Like the developers. You've got individuals within there. Now, I guess this is when the granularity comes out, because you can talk about other clan masters as a circle. But they're individuals amongst themselves. So I can identify. Within that, you have the concept of a clan master. There's [Active Clan Master 1] but then they have other concepts such as --

Interviewer: [Active Clan Master 1] is a person, yes?

Participant: Person, yeah. You've got active CMs, which [Active Clan Master 1] can be part of this group. And you've got inactives.

Interviewer: Inactive CMs?

Participant: Yeah. So there's individuals, but within there, you could break that. And then say: "Well, look [Active Clan Master 1]'s here, or something like that. And [Clan Master 2]'s here. We'll often talk about semi-active as well. There's me as a thing and there's also others within, and break it down like that.

Interviewer: So what we'll want to do here is create some sort of representation so that you can talk about classes of people. Or individually people if you feel that they're important to be talked about as an individual. So, from this, you can say, \quotation{Well, this is a communication from me, to the active Clan Masters.} \quotation{And I do this sort of communication.} \quotation{Or, this is a communication from me to [Active Clan Master 1].} Whatever represents what you're doing.

Participant: Look, cause sometimes -- Well, the information we can talk about, what, information, data? is transmitted. That's why I sort of did them separately. Cause I want to transmit me and [Active Clan Master 1] will talk about something differently than we'll talk about with others here. That can be a slightly different form of communication with these and with that. You deal with the individual differently even in a group.

Interviewer: And that's what I'm trying to tease out here. Thank you. We've got you as a player, you as a clan master, the Devs. Right now, we're just going to make bubbles and we'll take these bubbles to a third page when we're drawing lines. This is brainstorming.

Participant: We'll keep it sort of then at a higher level.

Interviewer: Whatever you want to do, if you want me to draw it, I'll draw it. If you want to draw it, go for it.

Participant: Nah, you might as well do it. So there's other CMs then, as part of the clan. And, I'll use that term to classify all the people that can rez the fort, per se.

Interviewer: Okay, so you define Clan Master as someone who's able to rez the fort?

Participant: Well, not really. But I'll group them together. There's what we call Martini admirals in [Clan] which aren't Clan Masters per se, but they have almost the same power as a clan Master. They can't be kicked. But I'll group them together. Because they're really, as you know in [Interviewer's Clan]. There's either those at the top and there's the rest of the players. So I'll call that other clan masters. And then there's other players. That's probably an easier, higher level distinction up there. And the next natural bubble is \quotation{other clans.}
\stopextract

The other aspect of the \SDFN\ that I was teaching the participant about here was the appropriate scope of an entity as well as the desired granularity of their universe of discourse, as it is bounded by both scope and detail: only so many actions are of interest, and some actions are too trivial to diagram.

The mistakes of  \in{figure}[fig:bubble2] illustrate the participant demonstrating a misunderstanding of the \SDFN, drawing their own hierarchy of authority within their organization. The creation of these side artifacts as part of the entity diagram is acceptable, especially as a way of pinpointing desired levels of granularity in entities. Participants should not think of the \SDFN\ as a hierarchy.

The start of the \SDFN\ can be quite subtle. In  \in{figure}[fig:bubble3], the rezzing the fort \SDFN\ began as a \quotation{walk me through the process:}

\startextract
Participant: So the next would be -- Maybe if I described the process ...

Interviewer: Walk me through the process.

Participant: Well, this is the case. I say \quotation{It's time clan war.} This is where I bring in the extra bubbles.

Interviewer: Let's trace this and see where we get off these bubbles from the process.

Participant: We're at clan war. \quotation{I want to rez the fort.} A command to rez the fort, it sends me back. Now, I guess to introduce the other bubble here as other clan members.

Interviewer: So, we're going to say internal clan members? Can we say other than members because that conflicts with clan master?

Participant: Clan Players?

Interviewer: Players. ICP.

Participant: I'll leave you with the acronyms. I tell them something now as well.

Interviewer: Now, when you tell them something.

Participant: There's a lot. That's a really detailed line between us and

Interviewer: Do we want to multiple lines?

Participant: Yeah. That's cool What we're dealing with at the moment. I'm going through the process of rezzing the fort, which is a common thing we want to do.

Interviewer: So shall we label this rezzing the fort?

Participant: So I'm telling them, I'm sending them information. That they can now war. That they can start.

Interviewer: And so this is a what? Is this a status, is this a command, is this something else?

Participant: It's information because it does require context. But it's like a status it's saying: \quotation{you can now war. We can start fighting.}

Interviewer: So it's a status. What other flows do you have to the internal clan players?

Participant: Apart from that? We obviously maintain that they're sending stuff back to me.

Interviewer: What are they doing there?

Participant: The line going back. They're sending me, also status updates. Whether they're ready to play, whether they're there. How much AP they have and things like that.

Interviewer: And they're sending you these as?

Participant: Textual information.

Interviewer: So it's text information?

Participant: Received via MSN.

Interviewer: Do we want to have MSN here or is MSN not at this level?

Participant: No, MSN is at that level. I'd say it is. I see MSN as -- it's true, these would in essence, I don't see them as MSN. MSN is a like a tool or a spanner. As an intermediary because I send it here (MSN bubble) and then to there (Player bubble.) And that is true because I don't talk directly to them, per se.
\stopextract

I start by exploring the entities that we described in the entity diagram along with a process that has come about out of small talk. The process of diagramming a single process is about the right complexity for an \SDFN. As we can see, the advent of a second \SDFN\ in Case-4 meant that the process of \quotation{rezzing the fort} was a little too simple. It costs nothing to make a second diagram if there is sufficient time remaining. 

The other important element is the requirement of asking questions. The point of the interview is to tease out the understanding of the participant, and to do that, they have to keep talking. Open questions, confirmations, and other prompts keep them talking without guiding down them any specific direction. 

\stopcomponent
