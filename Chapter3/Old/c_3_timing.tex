\startcomponent c_3_timing
\product prd_Chapter3_Methodology
\project project_thesis

The interviewer should allocate around 15 minutes on both sides of the interview for equipment preparation. During the interview, another 10 to 15 minutes should be spent on breaking the ice and making the interviewee comfortable. Creating the \SDFN\ will take half an hour to an hour, depending on how complex a diagram the participant desires. Although it is theoretically possible to compile the answers for a bubble diagram very quickly, try to encourage the participant until either a page is filled or he or she is clearly horribly lost.

A subsequent discussion, once the \SDFN\ has been completed, is completely optional. Some of the people I interviewed wanted to reflect, whereas others, uncomfortable with the procedure, did not. Due to this huge variation, there is no standard duration for this discussion, because it can go as long as the participant would like it to go. It is unusual, however, for it to go more than half an hour. If the participant is still interested after half an hour, attempt to schedule a second, follow-up interview. The \SDFN\ creation tends to be quite draining, and new insights may appear after a few days off for internal self-reflection.

Preparation is fairly trivial with enough pre-interview logistics work. It is important to have liquids and treats for both parties available, as there will be a significant amount of oral discussion. In the meeting area, try to position the discussion around a corner of a table. Having large separation between the interviewer and participant is contraindicated on both theoretical and practical levels. Theoretically, it is a bad idea to introduce any sense of distance or remoteness, as it will just increase the difficulties of icebreaking. Practically, the sheet will change hands many times, and a short distance will allow both parties to read edits and additions as they happen. Normally, the interviewer will serve as scribe to render the participant's descriptions in a common and consistent format. The participant should nevertheless see what is being scribed in real time, to offer feedback and corrections of his or her own thoughts.

The final element of preparation is to ensure the operation of both recorders. The recorders should be positioned out of the direct line between interviewer and participant. If possible, they should be positioned to pick up the participant clearly and isolated from the table to reduce the thumps and scratches transmitted by the table. Recording devices will pick up hand movements and emphatic gestures that hit the table depressingly well. 

Have some sort of subtle timing device to ensure the interview is proceeding according to schedule. Make sure that it is possible to look at the timer without disrupting the concentration of the participant. A watch in this regard is a poor choice, as looking at a watch is a social cue for many people. Cell phones pose a similar problem (and in any case, they and other radio devices should be off during the interview to prevent transmission interference.) Try to record at least 30 seconds of silence before the interview begins, and turn off the option on the recorders to not record white noise, because those measures will help with post-processing operations.


\stopcomponent
