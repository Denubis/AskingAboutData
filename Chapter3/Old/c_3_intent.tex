\startcomponent c_3_intent
\product prd_prd_Chapter3_Methodology
\project project_thesis

The \SDFN\ takes elements from both \infull{SNA} (\SNA) and a \DFD. It is a methodology that allows people to describe their own communicative reality in terms of flows of data, information, and knowledge. It diverges strongly from the goals and the methods of the \DFD. Although the \DFD\ implements its iterative methodology with a strict context diagram and recursively \quotation{drills down} into it, the \SDFN\ does not need the rigor of that method and can allow entities to be added at the highest level. At the same time, it is far more intrusive than typical social network analysis. 

SNA practitioners typically compile the network from observations of participants. Even in the social media sense, analysis comes from data created by participants for other purposes: people do not designate people their \quotation{friend} on Facebook out of some desire to participate in a social network study, though that is a consequence of their action. 

Just as the \DFD\ can accurately describe the data reality of a client's workplace by synthesizing many viewpoints, I intend the \SDFN\ to probe an individual's subjective philosophies of data. The \DFD\ integrates many different viewpoints by making one \quotation{true} graph. By refusing to \quotation{correct} the different graphs against each other, the \SDFN\ can highlight areas of philosophical or definitional incompatibility, and explore how individuals within an organization can share, create, and disagree about meaning.

\stopcomponent