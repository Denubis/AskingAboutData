\startcomponent c_3_conducting_intro
\product prd_Chapter3_Methodology
\project project_thesis

The introduction serves multiple purposes. Primarily, it diffuses anxiety, explains the background of the subject, and creates a scaffold for the intuitive prompting of the \SDFN. In these interviews, people display many different sources of anxiety. 

The most common is a sort of performance anxiety, wherein they do not believe their opinions are sufficiently privileged to contribute to a \quotation{philosophy of data.} Another common difficulty is job anxiety. Participants may feel that they are revealing secrets of their job to an outsider who, either as a spy for management or for some other reason, would steal the secrets to the participant's detriment. It is vital, in this stage, to reassure the participants of the intent of the interview and to make them feel in control-as they in fact are. 

The participant is in control of the interview. One way to demonstrate that control is with the ethics declaration. Besides expressing the uses to which the interview will be put, the ethics declaration can serve to break a coercive loop. The loop begins with management at the participant's location being involved in the recruitment process. Employees feel as if they need to participate or face negative repercussions at their job. The counter to this fear is to explicitly give participants the option to opt out at any time {\em while preserving the fiction of their excellent engagement in the interview}. My phrasing was, \quotation{By signing this ethics form, we have already had a fantastic interview, by definition. At any time, you may choose to revoke consent by signing that other form, but this won't change the fact that we've had an excellent interview.} It is in this explicit transfer of control that provides one of the primary functions of reassurance to the participant.

The other goal of the introduction is to provide the interviewer with an understanding of the background of the participant. This background understanding will provide for demographics and will hint at the topic of the bubble diagram. By investigating their work and educational experience, I was gathering data regarding any possible links between work, education, and their philosophy of data. Understanding educational background is also important because it shapes the nature of the jargon used, and is an explicit way of changing vocabulary.

As the participant discusses their work experience, especially in relation to their understanding of the topic of \quotation{philosophy of data,} incidents that are important to them will arise. By drawing out these incidents for any significance of data flows, one can choose a topic for the \SDFN\ that is both engaging to the participant and a fruitful for examination during the \SDFN. If repurposing this methodology for other tasks, at this point the task-specific goals should be emphasized, because by choosing a topic for discussion, the participant is implicitly assuming a role and engaging in a particular mindset. 

After the participant engages in the discussion, it is important to explain the nature of the \SDFN. Lightly explain flows and entities, the purpose of the diagram, and the nature of categorization. This explanation should be far less philosophical than even the descriptions presented above. A flow, to participants, is \quotation{any flow of data, information, or knowledge between one entity or another}; an entity is \quotation{a person, place, or thing that can interact with the flows.} This is a significant point of divergence for participants. Some people will understand the nature of entities quite clearly, as shown by their body language, and others will not. If it looks as though the participant does not understand, correct that problem by building an entity dictionary. The discussion of categorization should explain that, \quotation{The content of the flow will go above the flow. Content is roughly what is flowing between the two entities. Then I'll ask you to categorize the nature of the flow, whether it's data, information, knowledge, or other.} 

\sidebar{As an authentic \quotation{\IT\ person,} I used that knowledge base in some interviews. That knowledge did not shape interviews so much as enabled me to use jargon during the icebreaking to authentically communicate, \quotation{Hey, I'm one of you. I'm not an ivory-tower egghead.} By engaging the participant from an area of mutual understanding, I created a mutuality of concern that defused a significant amount of the tension from the interview. More than defusing tension, however, it also established a common jargon that we both knew that we both knew\footnote{The act of creating shared knowledge is important in making both parties comfortable.}, making the communication of supporting concepts far easier. 

With some of the other interviews, I had to default to a more neutral language because I did not share the pure scientific or other educational background of the speaker. While icebreaking was accomplished in these interviews as well, I was not able to establish a mutuality of concern as quickly, and the interview speed suffered as a result. 
}

If the participant is confused about entities, help them to create an entity dictionary. Ask them to describe typical entities from their workday and to describe themselves in various roles. Then ask them to describe other roles and things with which they work. The building of the entity dictionary provides the maximum scaffolding for teaching them about the nature of entities. 


\stopcomponent
