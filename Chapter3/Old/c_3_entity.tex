\startcomponent c_3_entity
\product prd_prd_Chapter3_Methodology
\project project_thesis

An entity is something that plays a role receiving, manipulating, or transmitting data. In the \SDFN, this act of input or output is represented by a noun described in a few words, which then have an oval bubble drawn around them. This bubble is a node in graph theory, with all of the corresponding attributes. The nature of the role is not restricted to a person or physical entity. It is anything that can be made-a-thing-of such that it makes an independent manipulation of data. 

Roles are anything that can be conceptualized as an independent manipulator of data. What differentiates an interesting role from one worth skipping over is whether the role somehow transforms data passing through it. There is no restriction on the number of roles that can belong to one person or thing. Just as one person can do multiple jobs, one can also have multiple roles. One hypothesis for future testing is that the role determines the perceived affordances of data. Every role has its own unique activities and therefore uses data in its own way. As the framing of the role changes, the definition of data may change along with it. 

Participants should never describe themselves as a singular entity due to ambiguity. Their description of an entity as \quotation{self} is ambiguous to other people reading the chart, who do not understand the tacit assumptions of role and interaction from the same perspective as the participant. Instead, participants should articulate the potentially many roles that they play in an organization as separate entities. 

Every role should be unique. However, there is no requirement for a one-to-one mapping from person to role. Because people and things are adaptable and can serve many roles in an organization, artificially forcing the participant to select one and only one definition of self would be contrary to the intent of this exploration. This requirement allows and encourages people to represent passing information to themselves in the guise of the different roles they play.

The avoidance of ambiguity is crucial. It makes the \SDFN\ easier to interpret by other people and it forces the individual creating the diagram to define the nature of the entity explicitly. It is far too easy to use the self as a catchall to avoid the cognitive dissonance of thinking about thinking. It is important to document discrete and unambiguous roles, even though they may map to the same person, because it is the role that understands data, not the person. Furthermore, these different roles-as-self can pass data to one another. I used the following example in the interview: An entity as lecture designer (myself) would pass requirements to the database lab developer (myself) who would pass data to the lecturer (myself). Each of those roles has different requirements for the nature of data. Crucial insights would be lost if they were all collapsed into one entity with self-pointing flows of data. 

Entities however, should not be ready-to-hand\footnote{Ready-to-hand roughly means tools that form an unconscious extension of the self. However, I will avoid a discussion of Heideggerian Daesin and other terminology. To learn more: Dreyfus's discussion of Heidegger is not too painful (p.230 for ready-at-hand) and Marshall is using the idea in interface research\cite{Dreyfus2004, Marshall_2003}.}. Devices that take on independent roles are fundamentally different from those that function as parts of another entity. The keyboard used for typing these letters into this document should not be considered an entity in the \SDFN\ sense because except when engaging in this self-reflexive behavior, the entity \quotation{author of dissertation} does not explicitly pass data to the keyboard-rather, the \quotation{author of dissertation} passes data to the computer for processing. The keyboard is part of a large entity and does not manipulate data in my own philosophy of data. Instead, as an input device, the keyboard is an extension of the computer and represents an interface for the electronic recording of symbols. 

At the same time, entity creation rules should not be strictly enforced, as each person may have his or her own conceptions of what an entity could be. A role can be a person, machine, place, or group. An entity is any noun that the interviewee regards as accepting or receiving data meaningfully. Participants must define their own entities, as their own conceptualization of roles is one of the strongest sources of insight into their understanding of the philosophy of data. 

\stopcomponent
