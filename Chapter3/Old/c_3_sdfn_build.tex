\startcomponent c_3_sdfn_build
\product prd_Chapter3_Methodology
\project project_thesis

After the participant is comfortable with the topic, and an entity dictionary is built (if appropriate), the \SDFN\ should begin. To reiterate, this section should commence only when a topic has been established and the participant is comfortable with the idea of the \SDFN. Avoid asking participants to define the philosophy of data at this point, as that will, if anything, reinforce their awareness of their theoretical definitions. 

\sidebar{When participants gave me a category for \quotation{other,} I asked them a few questions immediately to define the category. This initial definition is not canonical, but allowed me to iterate over all the prior categorizations to see if any should be reclassified. In addition, when they gave me this category, I made sure to incorporate it into subsequent prompts. By changing my prompting based on their input, I demonstrated the control they had over the interview, which both reassured them and saved time by eliminating the need to define \quotation{other} every time they used one of their own self-established categories.}

The \SDFN\ is designed to encourage participants to intuitively define their philosophies of data. Classification is a way of probing operational (rather than theoretical) understanding. Repeatedly confronting people with their \quotation{gut reactions} creates a cognitive dissonance\footnote{See Schultz for a theory of cognitive dissonance\cite{Schultz1996}.} between the theory and practice that, one hopes, the participant will articulate during the process. One of the subtlest pitfalls of this process, however, is \quotation{interview-aware} participants who try to guess which role you want them to play and respond accordingly. 

One of the most risky components of the interview is the categorical prompting. Participants may not have a hierarchical understanding of data, information, and knowledge, but may rather understand them as synonyms. When asked to categorize something as \quotation{data, information, knowledge, or other,} interview-aware participants may try to create artificial distinctions between the categories despite their internalization of similarity. This danger, however, is viewed to be a lesser problem than providing insufficient scaffolding\footnote{This area requires further investigation, and is one of the easiest future experiments to perform.}. 

Participants, especially at the beginning of the interview, are so unsure of what is being asked of them that asking them to classify something as \quotation{data or other} may overwhelm them with possible choices. Judging the amount of assistance required and adjusting the amount of assistance given during the interview based on increasing expertise is one of the more difficult aspects of the \SDFN. To combat this, a potential change to the methodology in the future is to have participants create an initial set of categories {\em before} the \SDFN\ creation begins, an idea I wish I had thought of before I was halfway done with the interview process.

It is important to engage the participant as a subject-matter expert. The \SDFN\ should explore a safe topic within the subject's expertise. A project, a process, or everyday interactions are excellent topical areas, as long as the participants have a strong familiarity with the domain. The choice of topic is important because it empowers the participants. Their experience in the domain reduces their uncertainty and fears of being wrong. Explaining what you do every day and are good at to someone who is interested and willing to listen also tends to be pleasant for most people, because of the validation\footnote{Validation is a confirmatory statement that increases a person's self-worth\cite{Leary2005}.}inherent in the discussion. 


\stopcomponent
