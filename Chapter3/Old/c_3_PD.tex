\startcomponent c_3_PD
\product prd_prd_Chapter3_Methodology
\project project_thesis

The process of intuitive interviewing to try to understand the user's reality is primarily seen in the sociological practices of human-computer interaction (HCI)\cite{Schuler1993}. The practices of participatory design and, in particular, the meta-communications of a Joint Application Design (\JAD) were particularly striking. This methodology applies the insights I gleaned from teaching and running a \JAD\ session to the core of the \SDFN. 

In Participatory Design (\PD), the central requirement is to design {\em for} the users, instead of forcing the users to learn a new system designed purely through developer fiat. This process of design requires that the designers actually involve the users in the design practice by letting them propose interfaces and thereby evoking their intuitive understanding of the topic. 

\PD\ rejects the idea of the privileged intuition. It also directs research towards the ultimate consumers of the interface: the users. The successes of \PD\ provide a counterpoint to the idea of a privileged intuition. The success of Volvo's participatory design projects\cite{Ehn1993} in re-engineering their assembly lines suggests that designers do not occupy some lofty and superior ideological intuitive space simply by virtue of being designers. 

The rejection of the privileged intuition applied to the discipline of philosophy validates the discipline of \XPhi. \XPhi\ rejects the traditional logical/intuitive scholarship of philosophy and substitutes the act of actively testing and applying scientific and sociological practices to philosophical inquiry\footnote{However, just as intuitive philosophy produces significant utility, so too does Norman argue that too much Human-Centered (participatory) design can be harmful\cite{Norman2005}.}. Both \PD\ and \XPhi\ reverse the normal approaches of their field, and engage the ultimate consumers in creating the \quotation{product.}

The participants should value all of the products of the effort. Be it a design for a system or the products of the process, the outputs of the process at every stage should be useful to everyone involved in the process. In this way, the \SDFN\ creates an artifact that is a useful representation of data flows within the company and a worthwhile product in its own right. In this, just as in participatory design, the participants are co-creators, not test subjects.

Another theoretical contributor is the process of Joint Application Design (\JAD). The process of a \JAD\ is to bring all the stakeholders\footnote{In an interesting study about the risks of stakeholder-focused design, Greene argues: \quotation{Following the original architect of stakeholder-based evaluation, the  stakeholder concept was defined as people whose lives are affected by the program and people whose decisions can affect the future of the program}\cite{GREENE1987}. } into one room to allow them to \quotation{jointly design the application.}\cite{Carmel1993c} Yet this process still frequently results in conflict between the stakeholders because different stakeholders have different understandings of the organizational reality being aritculated and designed. Consequently, they have to spontaneously create evaluative accents\footnote{See page \at[Accent] for more details on evaluative accents. An evaluative accent is the mechanism by which different meanings are attached to the same words based on the context understood by both communicator and recipient.} and create a trading zone\footnote{A trading zone (page \at[Galison]) occurs in any communication where people from different cultures or languages must meet and communicate useful concepts to one another. The essence of the trading zone is that locally true definitions of terms are evolved to allow for effective communication.} where they can all speak a locally true language.

The reality conflict produced by a \JAD\ stems from forcing all stakeholders to articulate and confront their own understandings of what the business/application does. When they articulate those understandings and contrast them from the articulated understandings of the other participants, with some of whom they share an antagonistic relationship, their understanding of the joint reality resolves as they are forced to create trading zones. The observation of the creation of these zones is what allows this method to expose the reality of the business to the outside designers conducting the \JAD.

This evidence of local trading zone formation was one of the core inspirations behind my realization that the \SDFN\ can serve as a process of intuitive prompting for the participants. If a group of people can evolve a locally true definition of reality in a room over a strongly contentious subject, a suitably modified methodology should be able to probe a less painful topic in far less time.

\stopcomponent
