\startcomponent c_3_conducting
\product prd_Chapter3_Methodology
\project project_thesis

The interview necessary to explore a participant’s individual construction of data has three phases. Initially, the interviewer should collect demographics through the introduction. The introduction is primarily a means to diffuse anxiety and to gain the critical basis for comparison between parties. The second phase is that of constructing the \SDFN. The \SDFN\ exposes the practical understandings of data of the participant through repeated categorization. The final component of the interview, the denouement, tends to be a discussion of the participant’s understanding of data uncovered by the \SDFN. As the process can cause a construction to change as it is articulated, this self-reflection period is an excellent opportunity for the participant to air their thoughts and describe their new or revealed understanding.

\subsection{Introduction}

The introduction serves multiple purposes. Primarily, it diffuses anxiety, explains the background of the subject, and creates a scaffold for the intuitive prompting of the \SDFN. In these interviews, people display many different sources of anxiety.

The most common is a sort of performance anxiety, wherein they do not believe their opinions are sufficiently privileged to describe their \quotation{understanding of data.} Another common difficulty is job anxiety. Participants may feel that they are revealing secrets of their job to an outsider who, either as a spy for management or for some other reason, would steal the secrets to the participant's detriment. It is vital, in this stage, to reassure the participants of the intent of the interview and to make them feel in control-as they in fact are.

The other goal of the introduction is to provide the interviewer with an understanding of the background of the participant. This background understanding will provide for demographics and will hint at the topic of the \SDFN\ diagram. By investigating their work and educational experience, it is possible to gather data regarding any possible links between work, education, and their understanding of data. Understanding educational background is also important because it shapes the nature of the jargon used, and is an explicit way of changing vocabulary.

As the participant discusses their work experience, especially in relation to their understanding of data, incidents that are important to them will arise. By drawing out these incidents for any significance of data flows, one can choose a topic for the \SDFN\ that is both engaging to the participant and a fruitful for examination during the \SDFN. If repurposing this methodology for other tasks, at this point the task-specific goals should be emphasized, because by choosing a topic for discussion, the participant is implicitly assuming a role and engaging in a particular mindset.

After the participant engages in the discussion, it is important to explain the nature of the \SDFN. Lightly explain flows and entities, the purpose of the diagram, and the nature of categorization. This explanation should be far less philosophical than even the descriptions presented above. A flow, to participants, is \quotation{any flow of data, information, or knowledge between one entity or another}; an entity is \quotation{a person, place, or thing that can interact with the flows.} This is a significant point of divergence for participants. Some people will understand the nature of entities quite clearly, as shown by their body language, and Os will not. If it looks as though the participant does not understand, correct that problem by building an entity dictionary. The discussion of categorization should explain that, \quotation{The content of the flow will go above the flow. Content is roughly what is flowing between the two entities. Then I'll ask you to categorize the nature of the flow, whether it's data, information, knowledge, or other.}

If the participant is confused about entities, help them to create an entity dictionary. Ask them to describe typical entities from their workday and to describe themselves in various roles. Then ask them to describe other roles and things with which they work. The building of the entity dictionary provides the maximum scaffolding for teaching them about the nature of entities.

\subsection{SDFN Building}

After the participant was comfortable with the topic, and an entity dictionary was built (if appropriate), the \SDFN\ began. I avoided asking for definitions of data to avoid contaminating their categories with half-remembered definitions from their educational days.

The \SDFN\ is designed to encourage participants to intuitively define their understanding of data. Classification is a way of probing operational (rather than theoretical) understanding. Repeatedly confronting people with their \quotation{gut reactions} creates a cognitive dissonance\footnote{See Schultz for a theory of cognitive dissonance\cite{Schultz1996}.} between the theory and practice that the participant will articulate during the process.

It was important to engage the participant as a subject-matter expert. The \SDFN\ should explore a safe topic within the subject's expertise. A project, a process, or everyday interactions are excellent topical areas, as long as the participants have a strong familiarity with the domain. The choice of topic is important because it empowers the participants. Their experience in the domain reduces their uncertainty and fears of being wrong. Explaining what you do every day and are good at to someone who is interested and willing to listen also tends to be pleasant for most people, because of the validation\footnote{Validation is a confirmatory statement that increases a person's self-worth\cite{Leary2005}.} inherent in the discussion.

\subsubsection{Methodology of the SDFN}

\placefigure[]
[fig:flowchart]
{This is a flowchart exploring the complete SDFN interview process. Different portions of this chapter will refer to different elements.}{\externalfigure[Chapter3/SDFNFlowchart.pdf][factor=fit,frame=on]}

When constructing the \SDFN, I acted as primary scribe. While the participant should have access to a pen so that he or she can scribble corrections, the interviewer will do most of the drawing. Because the activity of the \SDFN\ is to iteratively construct flows of \quotation{data, information, and knowledge} between entities described by the participant as the subject-matter expert, this section will discuss the structure I provided to participants.

Describing a flow always began with entity declaration. The participant declared which two entities the flow is between and then declares the flow itself. My prompting for categorization changed throughout the interview. Initially, the questions were quite explicit. \quotation{Who starts the flow? What do they do?} In this high-scaffolding variant, I explicitly identified the source of the flow and then guided participants to identify the destination and then the nature of the flow. By reducing the focus of the question to the smallest possible parts, I helped the participant not to feel confused by trying to think about too many unfamiliar things at once. Early in the interview, breaking the questions into tiny sub-questions allows for prompt feedback. As the participant learned through positive and negative reinforcement, their awareness of expectations reduced the need for tiny sub-questions.

I mentally examined an entity before committing it to the diagram. When validating the source entity, my decision tree was simple: Does the entity exist? If it does not exist, is the role that the participant described short and unambiguous? Entities must have short names, as names take up valuable space. If the entity bubble is more than about an inch in diameter, it takes up too much space and is likely to require redrawing the diagram.

I encouraged the participant to generalize to the point that the entity can reasonably cover all objects in its class. \quotation{Person} is too vague; it does not give the reader any sense of what role the person occupies. \quotation{Brian} is both too vague and too specific: it identifies a specific person, but does not suggest his role (thereby requiring more explanation) and does not allow for similar types of person. Good entities would be \quotation{Dissertation Author} or \quotation{Casual Reader} or \quotation{Examiner.} They can be generalized to one and only one role; they are simple (few words), and they allow for anyone who fits that role to be classified without unnecessary recourse to edge cases. By using examples from my personal experience, I was able to consistently give the same scenario example across multiple interviews, without looking as if I was reading from a script.

With the source constructed, the participant needs to define the destination using the same methodology. Later, the scaffolding can mostly be withdrawn and the entire process summarized with an \quotation{And then?} as the participant understands what is being asked. The transition should be gradual, rather than abrupt, and is predicated on the error rate of the participant. If the error rate increases, I increased the scaffolding to compensate.

Once the entities are identified and written, the participant should describe a flow. Using an arcing path with a clear \quotation{above and below,} connect the two entities. These arcing lines, representing flows, are oriented to the entities, not the page. I rotated the page if it allowed for cleaner arcs and more space above and below for description.

With the role description in mind, the participant was asked to note what the flow contains with a theme and variations on: \quotation{This is a flow of...} I made sure, when asking these questions, that they were as open-ended as possible. Flows also must be unambiguous. Ambiguity may be introduced via the introduction of O, similarly named flows. When a flow is described, check all the extant flows for identical and similar names. In the case of an identical name, inquire whether the same contents are flowing. If the names are similar, make sure there are enough adjectives around the flow to distinguish the two. I edited the prior flow if it makes more sense to do so.

When the flow has been described, and the noun (with or without adjectives) has been written above the arc, then prompt the participant to categorize the arc as data, information, or knowledge: \quotation{And is this flow of x data, information, knowledge, or O?} Unusually, this element of scaffolding is never completely removed, although it may be shortened as appropriate if the participant has already categorized things of its nature on the diagram. While other is a category, and the participants should be allowed the luxury of defining new categories, the greatest utility is derived if most participants classify most flows as data, information, or knowledge. As such, insure that only few new categories are made.

\subsection{Theoretical Discussion}

Having spent at least a productive half an hour on the \SDFN\ building, participants were then encouraged to conclude the session with a theoretical discussion on their own revealed constructions of data. Not all participants will want to have this discussion, and beyond a vague prompt of \quotation{And would you like to talk about your thoughts on the philosophy of data?}, it is not worth the effort to force reevaluations here. This discussion generally explores the ontological and epistemological questions and novel categorizations that arose during the \SDFN creation.

Participants, if not uncomfortable from the unusual thinking demands of the \SDFN, generally engaged in a self-reflective discussion. It was vital to ask open questions that build a scaffold for the participant's self-discovery.

Of interest in the discussion are how the participant transitioned from one of the categories to another. The \SDFN\ allows for a very solid investigation of interesting questions of categorization. In this case, I was exploring how the participant categorized things as data, information, or knowledge, their boundaries, and their transitions.

The main opportunity of the discussion is to study how the participant thought about the interaction between categories. In many interviews, the participants would construct a hierarchy in the theoretical discussion and discuss how data became information. The self-generation of a theoretical ontology of data from the practical categorization experience is the main purpose of this methodology.

other items to query include any departures from the normal flows. The participant, when creating the \SDFN, will prefer certain categories to Os, depending on topic and personal preference. When given the opportunity, I tried to ask about the unusual categories: instances in which the participant either made up a category, combined or concatenated categories, or used a rare classification. For example, when certain participants strongly favored data, only a few flows were classified as knowledge, mainly because they fell outside the scope of the discussion. Subsequent conversation then focused on those flows to try to get a balanced understanding of their nature.

When discussing combined or concatenated flows, it is important to understand the type of metaphor that the participant is using\cite{Lakoff2003}. While this is mostly important in terms of analysis, my participants normally used low key but figurative metaphors. During this process, I tried to ensure that they elaborated on each metaphor and clearly distinguished the type they were using. A container metaphor (\quotation{data holds information}) is different from that of a concatenative metaphor (\quotation{data alongside information}) and both are different from a combinatory metaphor (\quotation{data and information}).

A brief discussion of metaphor is in order. Participants gave me clues to their personal construction of data through their use of metaphor, especially their use of verbal affordances. Anytime the participants said, \quotation{filter data,} it was a strong clue that they were very interested in subjective masses of data that needed to be winnowed down. Discussions of precision and accuracy, or any kind of implicit meta-data were pointers to objective elements.

An example of container metaphor in action:

\startextract
Interviewer: ... To you, what is Information?

Participant: It's something that's not physical, basically. That means it could be a communication, it could be a conversation or a story. Something verbal.

Interviewer: So Information is any non-...

Participant: Data to me is physical, basically. It's an entity that [fuzzy word] an entity of some sort.

Interviewer: You would say that letter is Data?

Participant: I would say that letter is Data. But what is on that Data is Information. Just to be a bit confusing.

Interviewer: Tell me more.

Participant: Coming from the field we're in, in [removed]. You've got different areas. It would be, as I've said, Data or record. That's just the physical entity.
\stopextract

I notice the \quotation{what's on that Data is Information}, because \quotation{on} is a \quotation{container} word.

\startextract
Interviewer: Data is a container...

Participant: Yes, of the Information.

Interviewer: And Information is content of what type? Is there something common to all Information?

Participant: it's not easy. I'm struggling. It's a struggle. Information, what is Information? It's just... No, I'm drawing blank.

Interviewer: We have right now Data

Participant: And then of course you get Knowledge.

Interviewer: We have Data, Data is a container for Information.

Participant: I'm quite happy with that.

Interviewer: You say physical at some point?

Participant: Yes, physical in [unintelligible]. It doesn't actually have to be physical in a piece of paper, but it can be physical as in an e-mail message.
\stopextract

Whereas, in a different interview, we have the combinative \quotation{and}:

\startextract
Interviewer: Analysis, explanation. Analysis is?

Participant: Something like: model outputs or calculations. Explanation is what that actually means in context of your [work environment].

Interviewer: Class of analysis is Data, Information, Knowledge?

Participant: It's probably more on the Information. Well, it's Data and Information. And the explanation is Knowledge, it had better be.

Interviewer: When you say that, what do you mean?

Participant: You hope that when someone gives you their explanation, you know more than before they told you. Not always true. They can tell you stuff, and you can go \quotation{Well, I understand even less than when I started.} Because if it's completely contradictory to your understanding, you are now really confused. And in the cross cultural... you're often doing these meetings with interpreters in the room. Am I not asking it right? You don't necessarily get an interpreter that speaks Technical [other language]. Often they're here for some other meeting, and they just bring someone from one of the marketers who will be bilingual. Now, they all are bilingual, but particularly senior people will choose not to speak English because it's embarrassing when they speak badly. That said, we can't speak [other language], so who are we to criticize?
\stopextract

The keyword \quotation{and} of \quotation{Analysis is data and information} meant that they are combinable and can sit in one flow. The concatenative idea is a lot more difficult. Although in the following example I allude to a container, the interaction between data and information is not the strict container of the first example, but rather has more \quotation{alongside} affordances\footnote{Metaphor provides the affordances of the thing being related in the metaphor. Therefore, a container metaphor affords \quotation{putting into.}  It is by analysis of which set of affordances is hinted at most strongly by the participant that important clues are gleaned towards the participant's conception of data.}:

\startextract
Participant: As part of the Information flows to those, I may include a little bit of raw Data, but not very much.

Interviewer: Does this raw Data ...

Participant: Usually photos or a graph.

Interviewer: So you would say that photos are raw data.

Participant: Yes.

Interviewer: Would you say that the raw data is contained in the Information? i.e. you send them interpretation of measurements. As part of that interpretation you have to send them some of the measurements that are really interesting. Would you say that the Data is inside the Information flow, and we can just label this as Information? Or would you say that it's Information + Data?

Participant: I'd keep it inside the Information flow, because if it was just raw Data. They could very easily reach what I think is the wrong idea -- misinterpret.

Interviewer: Therefore, you're not going: \quotation{Here's the Information, here's the Data.} You're going \quotation{Here's the Information, here's some Data inside the Information to back it up.}

Participant: Yeah, that's right. But with just the raw numbers and no context, that's Data.
\stopextract

This discussion, if fruitful, will lead into definitions. Having looked at the ontological transitions in the discussion above, the participant may now be prepared to examine the ontological definitions of the various categories. Here was one of the more treacherous spots of the presentation, because it would have been extremely easy to put words into the participant's mouth through suggestions or overly specific leading questions.

Instead, I tried to allow the participants to use their own inductive process on the categories and transitions they have defined. Normally, the basis of the definitions will occur in the discussion of transitions, but it may not happen in every case. If possible, guide the participants to identify and discuss their own thoughts of how they categorized something as data.

Although the relationship questions are normally deeper, leading into this discussion through transitions means that the various affordances and other philosophical handles of data, information, and knowledge will be discussed first. Data, lacking form, has no \quotation{natural} or non-constructed affordances. The reification of data through the \SDFN\ caused the participant to suggest their own affordances, and thereby strongly hint at their conception of data. The other component is to ask them to discuss their understanding of how they know something is data or of how they categorize it. The only structure possible here is that provided by the \SDFN\ itself. The participants were also prompted to explain their categorization methodologies.

\stopcomponent
