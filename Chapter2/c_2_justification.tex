\startcomponent c_2_justification
\product prd_Chapter2_LiteratureReview
\project project_thesis

\IST\ both suffers and benefits from its multidisciplinarity. One of the key tools taught in \IST-training programs is merely the ability to understand the jargon of the other sub-disciplines of \IST. As a database professional, I can still speak the terminologies of networking, web design, and enterprise systems. The process of communicating in these various jargons, however, necessitates different mental models of the world\footnote{Exploring computer problems and their solutions is an exercise in quickly changing levels of abstraction. Small, vital, and technical details fight tooth and claw against the broad vision of the designer\cite{Medin1989}.}. In addition, each of those individual levels will have its own uses for the term “data”. The field is not amenable to a single probe, and even if it were, each sub-discipline understands reality in its own way, as it must to solve problems according to the constructed protocols of that profession.

One thing underlying all of \IST, however, is the use of the term \quotation{data.} Every aspect of \IST\ uses data, but their understanding of what constitutes that data is significantly different. Moreover, in this computerized age, everyone interacts daily with data to some degree. The difficulty is in the question: what is data? 

There is a need for to understand how people understand data because conflicting definitions of \quotation{data} inform communications. Peoples' inherent conceptions of data inform how they interact with the constructed data of the world\footnote{From a pragmatic point of view, many linguistic elements are socially constructed and our understanding of them is shaped by our linguistic interactions with other people. Data, being something categorized by humans, is a great example of a linguistic construction\cite{Berger1967}.}. Some people consider data to be objective facts, Os consider it to be subjective observations, and still Os consider it to be electronically stored signs\footnote{For more details, see the results section, page \at[Summary].}.

When people discuss data, information, and knowledge, their understanding of data informs their understanding of information and knowledge - be it synonymous, Ackoff's hierarchy from 1989, or any of the other hierarchies suggested by the literature\footnote{For more details, see page \at[Ackoff].}. When these different understandings collide, the best case is that the people involved recognize that they have different understandings and create a local trading zone with words that have functionally identical meanings to both people. In the worst case, both people use the term in the way to which they are accustomed, and errors go uncaught until large mistakes are made.

Data is defined by its use. It is a socially constructed term\footnote{While the term Data, as language is socially constructed, there are a large number of people who feel that the content of Data, as measurements of reality, cannot be so constructed\cite{Bruffee1986}.} rather than a reflection of some property of the universe. Therefore, data is subjective relative to the person using the term. I have identified a need to probe other peoples' understandings of data. It is easy to mistake professional training as a single, true, definition of data. The problem with intuitive definitions is that their elegance may not be used or tested in reality. For research to be useful to practitioners, it must deal with the philosophical problems that they face, not add another definition onto the large heap. This research aims only to provide a tool and a reason for practitioners to use that tool. 

\section{Aims}

I want to help improve communications, and I believe that a means for understanding different constructions of data could be one way to do so that has not been thoroughly explored. It may offer a theory towards explaining some errors in intergroup and intragroup communication. Furthermore, it may offer some direction towards exploring the philosophical basis of error by offering another take on system maps transmitted through communications\footnote{A system map is simply a person's internal mental model of how a thing operates and of how to get it to transition between different states. These maps may be communicated through instruction or alluded to\cite{Roy2008}.}.

I will, while exploring our ability to define and communicate data to people around us, lay a foundation for the exploration of the reality created by our use of data in computer systems. Our systems use data in multiple levels, from the hardware and simulated hardware through software and into fantastic constructions and games that embody and then produce their own data from any philosophical meaning. This study will not explore the various sub-constructions of data present on the Internet, in games, or in virtual worlds. Nevertheless, I hope that the methodology I create and validate can be applied to all sorts of computerized data: from the traditional bits down a wire into a simulation of a physics experiment inside Second Life\cite{Brown2008}. As humans use and create all of these tools, our constructions of reality inform them. To serve that end, this research creates and tests a methodology that can probe peoples' understanding of data. 

In database design, the hardest task is trying to understand the client's reality. Modeling a current organization's memory structures, its files and paperwork, and the relations among them in the minds of practitioners is an extremely difficult task. To facilitate understanding, this methodology is a tool for designers. The tool may allow them to understand what their clients think data actually {\em is}. 

By understanding the type of data being modeled, database designers make two significant gains. First, their data models can correspond with how their clients think about reality, and thereby create intuitive relationships and map the computerized model to their client's mental model more capably. Second, and in some ways more critically, they can then explain the database design {\em to} their clients in their clients' language, potentially shortening design times by reducing miscommunications.

In the same way, the proposed methodology should help extend normal modeling practice: simply making designers more aware of the different types of data constructions may make more responsive designs possible. The demonstration of different conceptions of data is important to designers because it offers another meta-aspect of reality to be captured and incorporated.

I also want to create a method that can help extend \HCI\ design practice. This methodology should be applicable to all sorts of design, as it is a tool for rendering clients' realities and not a specific kind of technical reality. The discovery of practical meaning of terms, ideas, and affordances\cite{Norman1999} of data is another tool with which HCI designers can understand how to render data presented in an interface. A tool that can make elements of private jargon explicit, and that is focused on that task (rather than treating it as a happy byproduct) can significantly contribute to the HCI design cycle.

This research investigates individual constructions of data, because there is no clear consensus on the exact nature of data, much less on the exact nature of data in technical design. However, as there is no recognized domain of the philosophy of data, this research, as a more practical matter, must lay the simplest foundations for that multidisciplinary field. 

My basic discoveries, both methodological and philosophical, should have pragmatic results. I hope to create a methodology that improves communication and database design. I explore how we socially construct and use the term \quotation{data}. From this investigation, I can offer potential insights into how we create trading zones between different cultures of data use. While true understanding of the nature of data may be outside the scope of this present research, the construction of a foundation is not. Any methodology created must be robust enough to provide useful observations and a compelling story. 

\stopcomponent

