\startcomponent c_2_hierarchiesOfData
\product prd_Chapter2_LiteratureReview
\project project_thesis

This work is not the first to ponder the nature of data. There exist two significant and pre-established relationships of data to information and knowledge: Ackoff's and Tuomi's. My findings mostly tend to echo the realities of data described by Ackoff or Tuomi. While not every interview or survey articulates a hierarchical relationship between data, information, and knowledge, it is clear that Ackoff's work has entered the \quotation{common knowledge}. A number of interviewees discussed a hierarchy of data first promulgated by Ackoff. While they never cite his influence, their descriptions of relationships between data, information, and knowledge match his quite precisely.

Interestingly, this hierarchy has a management basis, rather than one grounded in philosophy or practical understanding of how we appreciate and use data. There are many signs of cognitive dissonance between what is perceived as the traditional hierarchy and how data is used in practice. Ackoff has updated his hierarchy many times, including a book \quotation{Management F-Laws}\cite{Ackoff2007}  wherein he not only states that \quotation{to managers, a pound of wisdom is worth an ounce of understanding} but also belabors the useless metaphor with the extension that an ounce of wisdom is worth \quotation{65,536 ounces of data.}

Tuomi presents a more philosophically rigorous conception of data in his cyclic hierarchy, with data feeding into information, which in turn feeds knowledge, which provides the ability to collect more data. In the analyses, I will refer to this as a cyclic hierarchy.

\subsection[Ackoff]{Ackoff's Traditional Hierarchy}

The traditional hierarchy linking data, information, knowledge, and wisdom in a strict hierarchy of dominance and importance was created by Ackoff\cite{Ackoff} in 1989. In a summary of his work, Bernstein notes\cite{Bernstein2009}:

\startextract
Ackoff was a management consultant and former professor of management science at the Wharton School specializing in operations research and organizational theory. His article formulating what is now commonly called the Data-Information- Knowledge-Wisdom hierarchy (or DIKW for short) was first given in 1988 as a presidential address to the International Society for General Systems Research. This background may help explain his approach. Data in his terms are the product of observations, and are of no value until they are processed into a usable form to become information. Information is contained in answers to questions. Knowledge, the next layer, further refines information by making \quotation{possible the transformation of information into instructions. It makes control of a system possible} (Ackoff, 1989, 4), and that enables one to make it work efficiently. A managerial rather than scholarly perspective runs through Ackoff's entire hierarchy, so that \quotation{understanding} for him connotes an ability to assess and correct for errors, while \quotation{wisdom} means an ability to see the long-term consequences of any act and evaluate them relative to the ideal of total control (omnicompetence). While a scholarly perspective on this hierarchy might prioritize the processes of inquiry and discovery, Ackoff does not account for them. But his concept of omnicompetence, which refers to \quotation{the ability to satisfy any and every desire} (Ackoff, 1989, 8), does encompass the satisfaction of user-defined needs.

\stopextract

In this ontology, data are subjective observations. Curiously, despite data being subjective observations, Ackoff does not suggest any need for filtering (a common theme in subjective/observation conceptions of data).

\subsection[Tuomi]{Tuomi's Cyclic Hierarchy}

Tuomi's ontology is simple and counter-intuitive: knowledge is a framework of the world from which we build information. Information provides a local framework from which to extract data from the world. Thus, the apex of the hierarchy is data, which then filters downwards to modify knowledge and information. This approach represents an abductive approach towards the philosophy of knowledge and the research about data as the hypotheses being tested are information, generated from knowledge-of-world, rather than induced from data points\cite{Tuomi}:

\startextract
The generally accepted view sees data as simple facts that become information as data is combined into meaningful structures, which subsequently become knowledge as meaningful information is put into a context and when it can be used to make predictions. This view sees data as a prerequisite for information, and information as a prerequisite for knowledge. ... [Exploring] the conceptual hierarchy of data, information and knowledge, showing that data emerges only after we have information, and that information emerges only after we already have knowledge.
\stopextract

Tuomi's conception of a reverse hierarchy is useful to my research in two significant ways. Obviously, it allows me a prior idea to which to compare analyzed hierarchies of data. Although Tuomi's research is not explicitly about user’s conceptions of data, his analysis of hierarchies is an acceptable complement to it. By presenting a novel relationship hierarchy, Tuomi challenges the \quotation{everyone knows} mentality of much of knowledge management.

Tuomi's research into the fundamental questions of knowledge is one of the fundamentals of my research, for he demonstrates that it is possible to have a different understanding of data from an intuitive-philosophical standpoint. This demonstration of difference allows a questioning of the nature of data and acts as an external source of validation for my analysis.

This ontology of data is supported by a study of \quotation{intelligence} published in {\em Nature} by McNab and Klingberg\cite{McNab2008}:

\startextract

Thus, high-capacity individuals (who can remember more information at once and who tend to do better on aptitude tests) might simply be better at keeping irrelevant information \quotation{out of mind,} whereas low-capacity individuals may allow more irrelevant information to clutter up the mental in-box. The difference may just be a matter of having better spam filters.

Some of our own recent work on differences in controlling access to working memory has provided evidence favoring this mental spam-filtering idea. In one experiment, measuring electrical signals emitted by the brain enabled us to show that high-capacity people were excellent at controlling what information was represented in working memory: they let in information about relevant objects but completely filtered out that about irrelevant objects. Low-capacity individuals, in contrast, had much weaker control over what information entered the mental in-box; they let in information about both relevant and irrelevant objects roughly equally. Surprisingly, these results mean that we found that low-capacity people were actually holding more total information in mind than high-capacity individuals were-but much of the information they held was irrelevant to the task.

\stopextract

The idea of consciousness as filter discussed by all of these researchers is not particularly novel, although the localization of filtering activities by fMRI to those physical regions of the brain suggests that this ontology has a closer connection to our biological minds than does Ackoff's.

\stopcomponent