\startcomponent c_2_litReviewIntro
\product prd_Chapter2_LiteratureReview
\project project_thesis

The aim of this thesis is to ultimately facilitate better workplace communication, user interfaces, and database design and management. In order to do that, I borrow heavily from concept elicitation methodologies in order to produce personal constructs of data. These personal constructs of data, rendered in a concept map, allow for explicit exposition of the concept of data in a workplace and thereby reduce miscommunication through self-aware modification to available mental maps of the purpose and role of data.

Concept elicitation methodologies are a subset of knowledge elicitation methods, a tool used in many disciplines to \quotation{obtain the information needed to solve problems}\cite{burge1998}. Knowledge elicitation, in the main, is focused on direct problem solving: exploring requirements and understanding the meanings of those requirements. However, by turning the techniques of knowledge elicitation onto epistemological questions of category, we can discover not the direct meaning behind requirements, but some of a person’s semiotic models of the constructions behind those requirements.

My research looks to investigate a person’s personal construction of data. I borrow from data flow diagrams with a similar intent to the RepGrid methodology, though the end product differs significantly. The idea of personal constructs, discussed by Kelly and Tan\cite{kelly1955, tan2002repertory} and reformulated under many names: Terms that have been used to describe these things include \quotation{schemas}\cite{cossette1992mapping, jelinek1994toward} \quotation{cognitive maps} \cite{eden1992nature, weick2000making}, \quotation{technological frames} \cite{orlikowski1994technological}, and \quotation{mental models} \cite{daniels1995validating}. I, like Tan, will use personal constructions as the operative term.

Kelly\cite{kelly1955} describes a personal construction as a combination of philosophy and psychology. A construct, being subjective, is a personal epistemological tool of categorization and differentiation: \quotation{A construct is a way in which some things are construed as being alike and yet different from Os.} His thesis denotes constructs as framing devices where we can situate objects-as-signs in our way of knowing. He continues, \quotation{We have departed from conventional logic by assuming that the construct is just as pertinent to some of the things which are seen as different as it is to the things which are seen as alike.} Here, the fact that an object is not categorized as something can be an important factor in a person’s personal construction of reality. Constructs are bipolar, admitting knowledge of the sign/concept and its opposite rather than simple negation. The \SDFN\ extends this bipolar methodology of construction construction by asking people to categorize elements as data, information, or knowledge. By articulating a tripolar construction, we not only can articulate the positive categorizations of data, but can more closely examine data as it transforms into specifically delineated categories.

Much would be lost if participants were asked to categorize \quotation{data or not data} as the \quotation{not data} construction comprises everything that is not data, and is therefore not particularly interesting as a means of indicating the ontological and epistemological affordances of data. By requiring positive categorization, relationships between data and other concepts can be elicited more easily than simple negation would warrant. However, I also recognize that a a given categorization may simply be irrelevant in respect to data (relevancy is a far more useful and pragmatic benchmark than negation). Kelly notes that personal constructions are bounded\cite{kelly1955}, and are not necessarily \quotation{convenient} methods of categorization. In that light, the interview methodology will allow participants to articulate other categories that do not belong to the trinary construct of data-information-knowledge.

The repGrid\cite{tan2002repertory}, is a similar concept elicitation method. Tan describes the \IST\ uses of the technique as: \quotation{a set of procedures for uncovering the personal constructs individuals use to structure and interpret events relating to the development, implementation, use, and management of \IST\ in organizations.} While it is more overtly focused on organizational modelling, and the interpretation of events, it is a study of cognitive processes in an organizational setting to more effectively articulate information system requirements. The repgrid relies on participants sorting a pre-established schema of entities or objects, defined as a common set of \quotation{nouns or verbs} to constructs, the framing understanding around those concepts. Tan describes repGrid concepts as: \quotation{Constructs represent the research participant’s interpretations of the elements. Further understanding of these interpretations may be gained by eliciting contrasts resulting in bi-polar labels. Using the same example, research participants may come up with bi-polar constructs such as “high user involvement – low user involvement to differentiate the elements (i.e., IS projects).} The creation of framing dichotomies echos the construct framework of Kelly and then allows users to sort elements within those constructs with a variety of different methods.

However, the repgrid is not the best tool for understanding constructions of data: while it does articulate a dichotomy, it fails to expose the manipulations attached to data. Elicitation of affordances and transformations of data is crucial to understanding a person’s construction of data in sufficient detail to provide useful tools designed for them. Furthermore, while the statistical reliability of the repGrid is appreciated, especially as it can be subject to content analysis through simple frequency counting, the lack of an explicit period of participants to articulate their self-schemata robs interviewers of the potential insights of an articulated schema.

A representation grid draws on the personal construct framework for its own purposes of organizational knowledge modelling. In many ways, a \quotation{RepGrid} is a means of evaluating a social construction of reality, as discussed by Berger and Luckman\cite{Berger1967}. The social construction of reality echos the idea of personal constructions (though never explicitly calls out the term) by evoking the different realities of objects, \quotation{Different objects present themselves to consciousness as constituents of different spheres of reality. I recognize the fellowmen I must deal with in the course of everyday life as pertaining to a reality quite different from the disembodied figures that appear in my dreams. The two sets of objects introduce quite different tensions into my consciousness and I am attentive to them in quite different ways.} This evocation of personal constructions framing the affordances of interaction was one of the other inspirations behind this project. While Berger & Luckman articulate the primacy of our shared reality, this investigation explores one area where that shared understanding may break down.

Shared understandings of reality as encoded as self-schemata and expressed as understandings of terms. While this practice should just as easily be expressed as a linguistic pursuit, the aim of this investigation is to uncover elements of that primal construction of reality, not in differences in linguistic expression of that construction. I have found that the best way to explore an individual’s construction of reality is to ask them to express that reality in database design. The act of rendering the real-in-mind into diagrams expressing that causes an awareness of the self-schemata to coalesce simply by bring it into the forefront of consciousness. Through introspection into cognitive activity, self-schemata are formed: \quotation{attempts to organize, summarize, or explain one's own behavior in a particular domain will result in the formation of cognitive structures about the self or what might be called self-schemata. Self-schemata are cognitive generalizations about the self, derived from past experience, that organize and guide the processing of self-related information contained in the individual's social experiences.}\cite{markus1977self} It is this very process which the creation of the data flow diagram occasions in regards to an individual’s data manipulation activities. Furthermore, it is this act of schemata creation and subsequent discussion that I aim to elicit with the \SDFN.

The idea of schemata {\em qua} personal constructions of reality influencing human computer interfaces and system design is not novel. Though in the \HCI\ field, the term \quotation{mental model} is used. Wilson and Rutherford were exploring this very topic in 1989. Specifically, while they identify a significant variation in the definitions of the term \quotation{mental model,} they generalize the term to: \quotation{a representation formed by a user of a system and/or task, based on previous experience as well as current observation, which provides most (if not all) of their subsequent system understanding and consequently dictates the level of task performance.}\cite{wilson1989mental} The definitions they synthesize this from extend back into the seventies, and there is no fundamental disagreement that the practice of human-computer interaction is, in some way, the practice of presenting an interface to these mental models.

It is important to note that there are philosophical distinctions between the terms mental model, personal construction, and self-schemata. A personal construction is, in many ways, the philosophical reality of a term. The construction provides for understanding of when and how to use the term for all use cases as well as its personal and cultural semiotic identifications. A self-schemata is the articulated and explicit epistemological conceptions of the term: it is the developed understanding of an individual understanding how they categorize and use a term. A mental model, on the other hand, is the situated understanding in procedural memory. These mental model are, themselves, socially constructed through routines in organizations\cite{cohen1994organizational}. The mental model is the procedural manifestation of he personal construction in the recognized semiotic affordances of the concept of data.

Extending the mental model to expected manipulations of data, rather than expected interactions with a system is the providence of the \DFD, though the \DFD\ holds to an objective reality which synthesizes many mental models. The \SDFN, therefore, is a way to inspect the subjective mental models of humans as they relate to the expected interactions and transformations that their world applies to the thing they call data. As the term is never formally taught, we must evolve our models by experience with the world. Rasmussen asserts that mental models evolve with world-experience: \quotation{A mental model of a physical environment is a causal model structured in terms of objects with familiar functional properties. The objects interact in events, i.e., by state changes that propagate through the system “Kelly argues that individuals use their own personal constructs to understand and interpret events that occur around them and that these constructs are tempered by the individual’s experiences.}\cite{rasmussen1987cognitive}

As our experience with the world differs, so to must our models diverge to make individual predictions about the systems we encounter in our subjective, constructed, reality. Through articulated schema creation, we can expose a person’s mental map in a sufficiently valid framework for database designers and philosophers to puzzle over.
\stopcomponent
