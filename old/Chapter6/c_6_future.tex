\startcomponent c_6_Future
\product prd_Chapter6
\project project_thesis

\chapter[Further]{Further Research Directions}

Unlike potential applications, the research directions offered in this chapter are less useful to industry. Instead, this chapter focuses on detailing the limits of my research and some {\em research directions} in which it can be expanded. As this is the first foray of the philosophy of data into \XPhi, there are a {\em great many} research directions, far beyond the scope of this document. I will explore questions of breadth, depth, other philosophies, and methodology, as all of these areas can profitably expand the philosophy of data. Most of this chapter will be in the nature of open questions, rather than detailed research protocols into specific areas.

Questions of breadth explore the design space. My current findings have not demonstrated many of the theoretically possible philosophies of data nor their interactions. My open questions of breadth rather aim to suggest potential explorations into different philosophies of data, defining some of the boundaries of the field or pointing toward the unknown unknowns. 

Questions of depth seek to elaborate on my current research, generating more evidence and increasing the persuasive reliability of what I have discovered so far. New applications of the present research methodology and further triangulation of my results are necessary before any particularly persuasive case may be made.

There are many areas where useful intuitive philosophical exploration may be found. Although this research is \XPhi, certain areas can be profitably explored by someone using analytic or continental philosophy. This section suggests some potential links in those areas.

The \SDFN\ is a useful, if barely tested, innovation. Quite a lot of research may be performed on the \SDFN\ itself, making it a better and more accurate tool for the purposes of \XPhi. This section therefore also details some potential changes or extensions to the methodology.

Finally, I take some of the open questions and explore potential specific research projects that may be useful. This discussion provides a shortcut for future research proposals and a way of dividing interesting work.

\section{Questions of Breadth}

The philosophy of data warrants further investigation. My research indicates that different people have different philosophies of data, but many major questions remain to be explored. So far, my research has merely illuminated part of the \quotation{design space} of the field. The idea of a design space can be applied to any discipline. Put simply, a design space articulates a range of possible options, orthogonal to each other. It would be silly to think that this research has even uncovered many of the axes.

How does data relate to context? The data-as-observation philosophies believe that context allows the construction of information from data, allowing the brain to fit it into its own internal web of knowledge. The data-as-bits people believe that context is included in the meta-data between database tables. From this, how do data-as-facts philosophies view the context of data? Are there hidden contexts that indicate data's factual nature? Are relationships context, or do they form their own category?

How does measurement relate to data? Does measurement create data, or is it merely a transformation of intrinsic properties? Is data creatable via observation and is that observation then a measurement? How does error in measurement, and even further, imprecision, change the nature of the data represented?

Do numbers matter? Some philosophies within data|-|as|-|communications and data|-|as|-|facts require their data to be numerical. Is this requirement a reaction against the imprecision of language or a function of some other element of peoples' philosophy? How are numbers viewed to produced objective data, and can subjective interpretations be rendered as data to turn them into objective assertions?

How do sender and receiver matter? Has Shannon's\cite{Shannon1949} specification of information communication created philosophies of data? Are there realistic ways to use the model outside of ill thought-out metaphors? How does the nature of semiotics inform the philosophy of data? When people think of data as signs, which semiotic model fits better, and does the semiotic model change for different people?\cite{Floridi2003, Floridi2004, Floridi2005a}

Is data embodied in the universe? This question is another way of exploring the subjectivity/objectivity split, with some interesting implications and links to other philosophical fields. Exploring the belief of embodiment could be a good way to integrate this research into other multidisciplinary studies.

One of the most pressing aspects of the philosophy of data is the question of data's fundamental nature and how it relates to information and knowledge. Although the philosophy of data is not strongly interested in the fundamental philosophies of information and knowledge, an exploration of the hierarchies from the perspective of the philosophy of data may produce some novel results. The philosophies of both information and knowledge explore their fields' questions of being, but they explore them mainly from a \quotation{top-down} perspective, noting that information or knowledge is the most important element in the relationship.

The philosophy of data can contribute to a bottom-up hierarchy. It explores not the final and polished product, but the fundamental components of that product as perceived by people. Many of the intuitions of both the philosophy of knowledge and that of information can be explored through various \XPhi\ protocols. By looking at these other philosophies through the lens of the philosophy of data, it may be possible to confirm, substantiate, and question their findings.

Exploring relationships between data and other elements of knowing from the philosophy of data's point of view is also useful. We can explore the relationships between the different elements and examine how different people conceptualize and use those relationships, articulating the differences between theoretical definition and practical use. Methodologies can be created to see if people (and computer systems) understand data to be the first element of the hierarchy, the ultimate element, or merely a component of a circuit. From there, investigations can be conducted to see how people know and categorize these different \quotation{priorities} or products of data.

Another investigation concerns the semiotic representations of data. There are many avenues of investigation to see whether data is more appropriately understood in the traditional or the Peircian views of semiotics\cite[sep-peirce-semiotics]. Beyond that, the nature of the representation, being sign or electronic, can be investigated. Jacob Voss\cite[Voss2011] is studying the linguistic patterns of data, extending Gray's work in differentiating data from information\cite[Gray2003], exploring the potentially unique relationships between bits, bytes, strings, and the other artifacts of computer representation. Research to falsify data's linguistics or to explore the differences between it and the linguistics of written data can be quite profitable.

Questions on objectivity and subjectivity persist. A person's beliefs about data may suggest whether or not data, as an abstract, can be objective, but their assessment of each incoming data, piece of data, data set, or datum (depending on their academic background) indicates the subjectivity of any given element.

Nevertheless, questions of subjectivity and objectivity require significant research. The first question is the essential one: how does someone know whether something is objective or subjective? Beyond the direct questions-of-knowledge, \quotation{how do people know that something is data?} are questions such as: Are the elements that contribute to subjectivity or objectivity themselves data? Are they situated in a knowledge or information framework? How does an assessment of objectivity influence future interactions with the data?

There are also interesting interactions with the philosophy and sociology of science. Data, to many people, informs the nature of fact. Whether data is a fact or merely informs the construction of facts may therefore inform peoples' philosophies of data. The question of reproducibility, while related to subjectivity and objectivity, is one of the fundamental questions of the {\em theory} of science, with many different discussions as to objects and people transporting scientific knowledge. The relationship of data and facts to data and reproducibility is a vital one to establish through \XPhi.

Exploring the scientific nature of data also raises questions of analysis. Specifically, analysis is a transformation applied by one thing to another thing. Some people, in the interviews, stated the belief that analysis transformed data into information. Yet there are interesting questions of how people analyze data so that they can contextualize, filter, or refine it. Questions of analysis are those of affordances, virtual though they may be. How do people conceptualize data such that they know they can manipulate it? Where do these conceptualizations come from? How do they know what they know, especially from the perspective of data?

The other aspect to be investigated of data-as-fact is the nature of its structure and potential meta-data. How do people conceptualize relationships, meta-data, and other structural elements of data? Is the structure of data itself data? How does the structure of data influence our understanding of it, and are we aware of the structure when we understand a piece of data? How does software implement our mental maps of structure, and how can we improve that structural mapping?

Although all of these questions are a start, the main goal of short-term research in this area should be the generation of new questions. Too many unknown unknowns exist:  the boundaries of the question \quotation{what is data?} are not well defined.

\section{Questions of Depth}

The preceding section raises many links with other philosophies. Moreover, for many identified areas, I simply do not know the right questions to ask. However, other avenues of explorations are available. This research has explored an extremely small sample size, merely proving that there is something here. Further research must explore the two basic questions of interest.

The research needs to be applied to the same and new organizations. In the same organization, performing a similarly scoped interview process on a different group would provide an interesting comparison with my original research. Extending from there, the low-key interviews (taking very little organization time) must also be conducted across different types of organizations, including research, business, and others. Many useful intuitions can be gleaned by looking at different groups, organizations, and industries, even with the \quotation{choose your own topic} technique used in the interviews and survey above.

At the same time, a way to further increase the depth of the studies would be to run a series of more rigorous (and therefore longer) interviews exploring the same topical area with every participant. Because every participant would be exploring the same area, their diagrams would be made commensurate and differences could be found and articulated. The area diagrammed would have to be something in which many people in the organization participated, because the answers from workers in the same area are likely to be similar even if they have different philosophies of data. The similarity can be predicted from the people's need to share a common vocabulary and therefore a roughly common worldview of the scenario.

Exploration of this scenario could also be exported into a survey, once the interviews were complete. The survey would postulate elements within the area of concern about which people would express the most agreement and disagreement. If the survey is held to be commensurate with the interviews and if participation is restricted to those inside an organization, the survey could become a much more valuable tool for understanding the different philosophies of data present.

\section{Intuitive-philosophical justification}

Although increases in breadth and increases in depth would be needed to pursue it, the other significant area of research required is in formal philosophy. The initial, experimental investigation has suggested many different potential philosophies of data, but there is significant room for intuitive-philosophical thinking.

It is not an objective of this research to burn bridges. Nothing in this document should be construed to indicate any kind of negative view of intuitive-philosophical work. While \XPhi\ is excellent at probing the philosophies of people and even generating surprising conclusions, the acts of synthesis and imagination fall mainly in the domain of the intuitive philosophers.

Beyond synthesis, the inductive exploration of discovered philosophies from \XPhi\ methodologies, 
there exists a requirement for intuitive-philosophy to cover the same basic ground. Is it possible for an intuitive-philosopher to find and discuss the three domains of data that I have discovered in this research? Plying an imaginative gaze over the unmapped design space, the intuitive philosopher may discover different nodes, or different ways of thinking about the three major types of philosophies.

The research I have performed here should be reproducible through normal philosophy. Because the gathered evidence roughly reflects extant definitions, the armchair philosopher would find no novel, unexpected cases. A long and deep consideration of the relationships of the three philosophies may provide additional elements to test or project unexpected consequences of one of the three philosophies.

Although my intent is to create a practical philosophy of data, one that is functional and useful to practitioners, a good theory is itself quite practical. Involving formal philosophers, philosophers of science, and philosophers of technology in the exploration of the philosophy of data can generate a great deal of theory. If the ultimate goal is kept the same, to generate theory that can drive real practice, then the participation of philosophers from other domains could be quite fruitful.

Philosophies of science and technology tend to involve data in some degree. By involving the practical aspect of this philosophy in their research, philosophers may be able to export functional practices to people in the working world. Their participation as translators is encouraged, but beyond translation, the philosophers of technology and science can serve to frame questions of data: they can situate, via footnotes, different philosophies of data within their work, and in that way both frame their work and indicate why certain people believe certain things about science and technology.

Finally, the intuitive-philosophical component of philosophy may be a useful and practical matter of exploration for practical engineering and database communities. Although the experimental component may be better at describing what {\em is}, it is likely that the intuitive component is better at describing what should be. Arranging a collaborative exploration between the three peoples, intuitive, experimental, and practical, may encourage the conduct of research into new ways of thinking about all these conceptions of data. Once those new ways are explored, they can be tested by the practical users of the philosophy and then assessed by the experimental side of things.

\section{Methodological}

The \SDFN, as a tool, requires refinement and research. It may be an excellent way to model a business, but it requires significant research before it may produce useful results. It creates an extremely subjective look at how people inside the business see their own business. Quite a lot of study will be necessary before the \SDFN\ can present a compelling story that both communicates effectively and provides value to participants and sponsors.

As a way of understanding what people think about what is going on inside an organization, the \SDFN\ could be a very useful tool. More research is needed to create methodologies for performing interviews, especially in choosing a topic that will cover multiple peoples' interviews, exploring how people create categories, presenting category lists without introducing too much bias, and exploring more ephemeral definitions than those of a relatively well known term such as \quotation{data.}

At present, the \SDFN\ is an excellent way of creating an artifact for discussion. Assessment purely on the methodology is needed to see if it can produce consistent results within groups of people. Furthermore, the methodological description in an earlier chapter, while sufficient, probably fails to articulate many unconscious adaptations performed to increase the quality of interviews. In many ways, meta-research needs to be conducted with an external party modeling the \SDFN\ process before other people can make full use of it.

The bubble diagram is quite useful, though limited due to its subjectivity. New ways of modeling and representing the \SDFN\ are required, especially for analyzing the diagram outside the context of the interview. This research should also be related to the \quotation{objectivity} research mentioned above, to try to produce useful stories or reproducible results from the methodology. It is essential that the \SDFN\ be a useful tool for people who require objectivity and people who embrace subjectivity in their research. If the bubble diagram is attached to one doctrinal camp or the other, large segments of potential researchers will dismiss it out of hand, when it could potentially be a very useful tool for their work.

The \SDFN\ requires significant further research. Fundamentally, it is barely tried and tested as a methodology. Creating a way to assess its efficacy and reduce uncertainties regarding potential bias is vital. Researching more methods of analysis, especially ones that can be performed by groups of people is also worthwhile.

The first large area of methodological validation should be in the failure modes of the \SDFN. What behaviors during an interview cause an \SDFN\ to fail? How is failure defined? How does failure affect other interviews? Where are the sources of subjectivity and bias? What is an appropriate amount of scaffolding for the interviewer to provide? Where does the documentation fail, such that someone reading the methodology is not able to run their own \SDFN? Where does the bubble analysis fail? What conclusions should not be reached when considering the \SDFN?

All of the above questions can lead to larger or smaller amounts of failure before, after, or during the running of the \SDFN. Beyond those questions, however, are questions of reliability. As a subjective tool, the idea of reproducibility is difficult to credit: interviewing a person multiple times will, by necessity, produce different results as they update their mental models with the contents of prior interviews. However, that result increases the difficulties in discussing the reliability of the \SDFN. Subjective social research relies on telling a \quotation{compelling story}\cite{Stake1995} that may resonate with aspects of what the audience understands about the world, by providing explanatory power for things that the researchers may not have conceptualized. The pure-subjective story here is also limited, as the \SDFN\ and interview are designed primarily to elicit definitions.

As the \SDFN\ straddles the subjective-objective barrier, quite a lot of research will need to be done on its reliability. Measurements or assessments of reliability need to be created. Different \SDFN\ runs must be commensurate in terms of reliability such that it is possible to say \quotation{this was a bad run} and to refer to some external reason for the interview failure. The danger of subjectivity is that it is easy to have interviews \quotation{fail} when the participant does not agree with the interviewer; a rubric of reliability can in many ways reduce that problem.

Another area of methodological research is the categorical survey. Both the \SDFN\ and the survey present ways of probing meaning, although the classification-through-scenario survey is clearly in need of significant refinement. The survey suffers from many failure modes and reliability problems, just as the \SDFN\ does.

Both reliability and failure modes hinge on a single problem: everyone needs to understand that they are taking the same survey. Investigation into the reliability of results within an organization is also warranted, with time being dedicated to crosschecking the survey against the \SDFN\ and both against other methods. Proving this survey methodology is important because the survey deviates significantly from most established practice, namely in that it invites self-reflection at every question, which inherently turns the survey into a far more subjective research vehicle with the attendant risks of reproducibility, falsification, and unpersuasive case studies.

On a more practical side, the refinement necessary is in two parallel courses. Primarily, the survey instructions need to be refined so that everyone understands the survey {\em in the same way}. The identical understanding of questions is vital when investigating a large organization; otherwise, interpretation bias of the questions themselves will make the results incommensurate. Along with the instructions for the questions, the method of probing an interviewee's initial role must be refined so that it can explore demographic and educational backgrounds while not breaking anonymity.

Even after the instructions are well tested and presented, the scenarios and the methodology for creating scenarios for a particular institution need to be refined so that data which fits all discovered philosophies of data is described in a number of scenarios. People must be allowed to resonate with many different interpretations of data, and the scenarios need to be short but sufficiently clear to resonate with the different kinds of philosophies.

Finally, an analysis methodology needs to be developed to compare explicitly the results from the \SDFN\ interview in a given department with a broader survey conducted amongst the department. Although the simple method of reviewing the various philosophies of data and searching for correspondences was sufficient to fulfill my simple goals in this research, more sophisticated techniques will be necessary to draw more refined and useful conclusions.

\section{Long-term projects}

There are a number of long-term projects that could usefully and critically extend this research. While the prior section explored the rough questions to inform the design space of the philosophy of data and the failings of my methodologies, this section will propose specific research paths based on the above section.

\subsection{Organizational modeling}

The first major research path should involve a large organizational modeling operation. The intent is to expand the scope of the research covered by the philosophy of data and the \SDFN\ to more interviews, more surveys, and, if possible, use an iterated approach to extend the scope of both of them.

The initial plan for my current research was to explore the interactions in a small group as an explicit pilot test of the methodology. Once conducted, I was to brief the company on the results and then run another series of tests with the company as a whole, exploring the interactions between different arms of the company. The first 10 pilot interviews provided more than sufficient research material.

With another opportunity to do deep research in a company, I would first conduct an interview in every interested department. The \SDFN\ would be topically decided by the participant, and would serve both as advance advertising of the methodology to the company and as a way of familiarizing myself with the activities of the various departments. This research project, however, would not involve other interviewers, because the methodology requires another, larger, more strenuous test before proper multi-interviewer participation is possible. A formal set of instructions, guidelines, and analysis methods, beyond that given in chapter 3, must be developed before other interviewers can reliably run the \SDFN. Without a strict written framework, each interviewer will take the \SDFN\ research in their own direction, reducing possible commensurability.

With the initial interviews out of the way, I would execute my survey of the company. By running the survey first, I can use it to select participants and to get an idea of what the company as a whole thinks about data. By shaping the scenarios of the survey to those found in the various interviews, I can make the survey topically relevant for the company.

The survey will have an \quotation{enter contact information here} component that will allow me to extend the survey into a proper \SDFN\ interview. Although the contact information eliminates anonymity to me\footnote{While there are ways to unlink contact information, the dissociation isn't useful because surveys should provide an initial sampling for subsequent interviews. The ethical implications are not as important here due to the theoretical domain of research, but still pose some interesting problems.}, subsequent interview/survey processing will insure that the entire thing remains anonymous in any publications, depending on what serves the research and company interests the most.

In this first iteration, I will not try to have unified \SDFN\ topics, as that requires a far more polished \SDFN\ methodology than allowing people to be their own subject matter experts. Trying to find a topic to which multiple, self-selected people in different parts of an organization can relate is an unnecessary difficulty in this proposed project.

\subsection{Group modeling}

Another research path is exploring the group dynamics of an \SDFN. This requires a smaller group of 5-to-10 people, all of whose members are willing to take part in the experiment. As in the prior methodology, the first task will be to present the group with a survey containing scenarios tailored to their particular mission. These surveys will identify the question-taker in each one, to see if their philosophies of data change when interacting within the group

Once the surveys are conducted, six \SDFN's will be performed. The first \SDFN\ will be with one of the \quotation{group leaders} and will ask to model something that has happened to the entire group. This initial interview will be long running or multi-part, such that the diagram can be sufficiently complete to cover any eventualities in subsequent groups. Another important reason to involve the group leader in the first interview is that then \quotation{legitimate research concerns} will prevent their subsequent participation in the group \SDFN\ session. A group leader usually has coercive disciplinary and fiscal power over the other members of the group. Although in egalitarian organizations this coercion is less of a problem, any organization with a culture of fear will have its group results undermined by the group leader's presence due to the perception of inevitable punishment for disagreement. 

With the topic area defined by the group leader, the next phase is to run a control set of approximately four interviews, enough to capture methodologies of about half the group. These interviews will be conducted alone, and will test the effectiveness of the externally chosen topic. The last interview will use the other half of the group and traditional focus group methodologies to encourage them to jointly create an \SDFN. By comparing results from the control tests, it may be possible to understand the impact of using focus groups when it comes to \XPhi.

The \SDFN\ itself will try to understand the activities of the group, soliciting opinions, flows, and discussion from all of its members. It will be vital to test this methodology in a friendly group before trying it on the research group, for I suspect there are many different failure modes. The main area of interest, here, is to see how peoples' answers change when they are no longer simply discussing things with me, but rather exploring options through the group. 

\subsection{Breadth-first research}

Another major research track is a wide search, looking less at the \SDFN\ and its connection with the philosophy of data and more at how other methodologies from \XPhi, the social sciences, and intuitive philosophy can contribute to the philosophy of data. A great deal of work needs to be performed to build the \quotation{foundations} of the research.

The breadth-first search would look at the three categories of philosophy of data that I have identified and try to engage them using other methodologies. Functionally, this research is designed to test whether the different categories of data are real, what implications they have on peoples' use of data, and how that interpretation of data is communicated. It aims to be an experimental look at the nature of trading zones and how those communications influence actual work functions.

One of the primary areas of complementary methodological research for this is the use of role-playing games in \XPhi\ research\footnote{Role-playing games have effectively performed and researched the art of engaging participants in scenarios for the last 30 years. Sadly, little of this expertise has been integrated into academia. Studying the philosophy of data of Intelligence Agents may be an excellent way to bridge that gap.}. The idea of articulated roles and scenario interactions could be an extremely valuable way of exploring philosophical problems. In the ludic\footnote{Play, or philosophy of play.} sense, role-playing is a way to map someone's ideas to a not-self persona. Although it is a form of interactive storytelling in game form, as a game it is deeply revealing of someone's personal philosophies and view of the world {\em because} they are engaging a fictional world with a fictional character who is emphatically not them.

Role-playing as an experimental discipline suffers from a stigma of being \quotation{just a game.} It is my hope to use my prepared scenarios to explore how different people understand the idea of military intelligence as data. This area of research is also valuable in exploring how role-playing can influence research, exploring the nature of asking people to role-play situations where their roles have different philosophies of data. As part of this process, different game mechanics can be explored, to see which mechanics produce more useful experimental results and whether randomness and uncertainty in the mechanics increase engagement sufficiently to offset their decrease in experiment reliability and reproducibility.

The main groups for this research would be various role-playing organizations, including school clubs, table-top, role-playing gamers,  and \quotation{live action} groups. Although it is possible to engage companies with this level of research, the extreme experimental aspect of the research suggests that a more cooperative user-base is preferred.

Another aspect of the breadth-first search is greatly expanding the scope of my research, while minimizing its depth. Nominally, the organization research of the depth section accomplishes that scope, but it still is primarily focused on digging deeply into how an organization conceptualizes data.

The other alternative is to explore how diverse organizations conceptualize data. By creating a generalized survey, longer and tested for instructional accuracy, it becomes possible to assess many people in many organizations in one field at the same time. This large-scale survey would not provide the depth of the first \SDFN-based interviews, but it would provide something just as meaningful: the ability to apply statistical analysis to the survey results, especially with multiple choice-based demographics to complement the long-form role coding and to test whether there are any demographic links between different ways of thinking about the philosophy of data.

This methodology requires far more sophisticated survey design, applying multiple disciplines' survey-making techniques to properly \quotation{triangulate} on what an individual survey-taker believes. From assessments of 100-200 individuals across many organizations in the same field, patterns may emerge based on various aspects of demographic background. These patterns would be available to any kinds of correlation analysis and may even objectively demonstrate the various philosophies of data.

Novel research without clear prior results represents a very risky venture for most companies. One significant obstacle to this research is to convince participating companies of its potential benefit without having research beyond this document. While the in-depth research can investigate the requirements and communication habits of a particular company, it is remarkably difficult to get companies to cooperate with their potential competitors. One of the alternatives would be to ask to target a company that is a holding company: this arrangement would allow for the sub-companies to be interviewed across a country or state and provide a tool for the large company to potentially understand some of its internal data-driven disconnects.

Alternatively, this style of broad survey would be very well suited for an academic setting, exploring how different people in many different universities perceive the nature of data. It is also a low-impact way to interest many academics in the discipline and the meta-reflective questions of the philosophy of data.

\subsection{Longitudinal research}

Another research direction is temporal. Both of the above research directions stipulate one-off studies, looking at an organization only once. This limitation makes sense as these studies are designed both to assess an organization and to assess the methodologies assessing that organization at the same time. A third research dimension, time, can go deep or wide as the study dictates, but the process of longitudinal research requires minimal methodological modifications.

The essence of longitudinal research is to look at changes over time. Looking at an organization, there is little reason to suppose that someone's philosophy of data will change much over time {\em except} when their surroundings change significantly with them. Practical philosophies of data offer ways to deal with constructed input from a person's environment. The old saw \quotation{If it ain't broke, don't fix it} applies here, as a working philosophy does not generate any cognitive dissonance, and therefore fails to create any impetus for change. 

The best time to do longitudinal research on an organization is in the process surrounding a merger. Specifically, four studies would be optimal. The first would be during the initial planning of the merger, when I would look at two departments that would in theory be merged. The other three times, I would look at the departments after the merger had been finalized (to assist with integration), in the middle of integration, and sometime after the dust had settled.

This investigation would investigate two primary questions: do an organization's philosophies of data change over time, and how does the discussion of the philosophy of data inform an organization's thoughts? At present, I do not have any ideas of what could cause an organization's philosophies of data to change over time, though I suspect that changes in software use and tools will gradually change how an organization thinks about data.

Beyond changes in tools, I suspect that active discussions of the philosophy of data will indeed change how individuals in the organization think about the philosophy of data. Although they may not change their own, individual opinions, the explicit discussion of the practice will strongly inform how data is thought about at the organizational level. I predict that the meta-reflection, taken from after-the-fact comments from my contacts at BlueScope, will inform how the organization designs data processing systems and communicates the nature of data.

\section{Meta-reflection}

Further research will be pursuing two primary goals: methodological improvement and philosophical exploration of the philosophy of data. This research, however, is more than a contribution to the \quotation{field} of the philosophy of data, and more than a methodology. It is a way of looking at \XPhi\ and extending {\em that} discipline to more pragmatic ends.

In many ways, this research is about applied philosophy: using philosophy as an engineer would use science. This research is designed to go some distance towards combating the \quotation{ivory tower}\cite{Frodeman2009} perception of philosophy, especially philosophies of technology\footnote{Our philosophers of technology and science, in the main, are not practitioners. While their outsiderness lends some objectivity to their philosophies, their lack of capability in the fields of which they speak (especially high-tech) means that many actual practitioners in those fields feel no obligation to listen to what these outsiders are saying.}. The ultimate targets for this research are the practitioners who use data every day: engineers, businesspeople, scientists, systems designers. With this research, I hope to expand philosophical thinking in those areas and to demonstrate that many worthwhile philosophical activities must be rooted in a firm knowledge of their respective areas.

If this research can inspire another \IT\ practitioner to explore aspects of the philosophy of technology in which they have a personal investment, then this research will have been useful. All the further research directions {\em must} be focused in integrating philosophy with practice. Research that involves practitioners and enhances their capabilities for meta-reflection is good research. Research that isolates practitioners and returns nothing is bad research.

This section cannot articulate all the possible research directions, for there are too many unknown unknowns. While most of the known research directions are designed around targeting or identifying flaws in the current philosophy or methodology, those are only short-term goals. In the longer term, as the rate of change increases in the world, our understanding of data will change with it, as data is the reified basis of software, in one way or another. Research into the philosophy of data is, in some ways, a look at the philosophy of technology.


\stopcomponent
