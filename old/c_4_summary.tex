\startcomponent c_4_summary
\product prd_Chapter4
\project project_thesis



\section{Interview Analysis}
Interview analysis discovered three different philosophies of data. Although the bubble diagrams hinted at the third philosophy, there was insufficient evidence to brand observations of the world with their own philosophy.

It would be hard to deny that interviews I and II have data as communication, III and IV have subjective observations (with IX hinting at them) and the rest considering data as objective fact. With these broad differences evident, Question of interest 1 has been satisfied. 

The observation philosophies differ strikingly from the numeric philosophies, possibly differing on a fundamental perception of reality. As one interview is trying to render the relationships between matter in the world as numbers (objectivist), another is suggesting that everything emits data and we must filter it. The conflict is records versus measurements versus signs. Does data measure objective reality, record subjective reality, or merely transmit signs? Numbers are seen as a result of precision most of the time, whereas observations are building their way towards knowledge.


\subsection{Data as communications}

Data, in the communicative sense, merely requires signs and things to communicate with those signs. The data can be rendered as bits or marks on paper, but it is seen as a factor of semiotic import rather than as something to be discovered or filtered. 

This philosophy is substantively different from the other two inasmuch as it does not uphold data to be an aspect of reality. Instead, data is produced as a function of human intent. Because this philosophy does not concern itself with interactions of the real, there is a far greater difference between this and the other two than between the subjective-objective philosophies. However, the passivity of this philosophy allows it to accept facts produced from either source as something to be encoded, stored, and transmitted. Significant research needs to be done to explore how this philosophy of data relates to the other two.

The interviews successfully explored peoples' philosophies of data and I consider the interviews, in conjunction with the bubble diagrams, to completely answer Question of Interest 1: yes, people have different philosophies of data. My interviews generated results that I was not expecting (mainly the container metaphors), which suggests that the interpretation and differences are not merely an artifact of my analysis.

However, as I conducted the interviews and analyzed them, the possibility exists that I have infected the analysis with my own philosophy of data, seeing in all three philosophies aspects of my own. One area of methodological research would be to design a methodology and subsequent analysis protocol that can be conducted by other people, and then to perform a much larger-scale experiment to try to reproduce these results. My sole hope here was to tell a persuasive story, rather than to create a model for predictions of the world.

\subsection{Data as subjective observations}

Data, in the subjective observations, requires contextualization and filtering. Everything emits data as sense impressions\footnote{Like the ancient Aristotelian idea of species (particles of sensation). Light was the medium that visual species traveled within. 

While this ancient philosophy of image is not hugely useful to us, the same intuitions that led to it could have some parallels with data as subjective observations. This research area could make an interesting bridge between intuitive and experimental philosophies. 
} that can be captured by us. Thus, to perform sense-making activities, we must filter and contextualize the interesting data so that it can become information.

Subjective data lends itself more to cyclic hierarchies, where data begets the information and knowledge used to collect more data, reflecting an interestingly constructivist view of knowledge. There is quite a lot here available to future research, and I do not feel sufficiently confident in my sample size to make any assertions as to relationships between data and the various philosophies of knowledge or science, though the subjective nature of observations may tend slightly more towards Latour or Feyerabend. 

Of more interest is that this inherently subjective data is constructed from the mind's impressions of the surroundings, rather than revealed through measurement of the surroundings. The understanding of the embodiment of data is a significant difference between the two philosophies of data. 

\subsection{Data as measured facts}
Objective data comes with its own context \quotation{baked in.} It is, in many ways, rare: it requires positive effort to generate, and higher quality data requires a commensurate increase in effort. Data requires analysis to uncover the extant patterns of reality, and with enough data, knowledge about the singular real can be generated.

Objective data requires that data be a fact, usually a numerical, reproducible representation of reality that conveys an understanding of measurement quality and units. Objective data is not filtered, because it is collected with prior intent and all elements of the \quotation{data set} may produce interesting patterns. 

Both humans and sensors can reveal objective data, which is embodied in the things being measured. There seems to be no significant link with any of the major philosophies of science. Although my investigations did not explore confirmation, falsifiability, or paradigms, there seems to be a common understanding that data-as-fact accurately represents the universe within the constraints of measurement. This may be because the participants believed data to be a building block upon which their hypotheses or understanding of the universe could be built. Understanding this particular subset of the philosophy of data and linking it to ramifications in the philosophy of science could produce many interesting future research projects. 

\section{Survey Analysis}

The results from the surveys successfully triangulate my results. While limited by the same potential bias as the interview analysis, the high variation in philosophies from the survey suggests that I may very well be on to something. 

Different roles have different uses for data, and there are differences of opinion as to the methods of knowing, intrinsic nature, and purpose of data. In the surveys, people referred to records, observations, and numbers, roughly mapping to the terms found in the interviews. 

Records, as technologically stored signs, can be data if they are stored properly. Another term for record is row or {\em tuple}. These tuples can be about any topic and can suffer from anomalies: authoritative conflicts in which the database may not represent the thing in the universe of discourse. Despite this, to some people, any records in the database are still data. 

Observations are the act of applying consciousness to a reality. They can be subjective and inconsistent. At the same time, researchers used the term to indicate experimental observations that are objective and provide their own context. The term observation in the surveys suffers from the same traps that it does in the interviews and is worth far more research.

Numbers as data are used by both the database designers and the objective researchers. Numbers are either a more objective refinement of an observation or merely a semiotic rendering that requires context and analysis to render into meaning. Numbers are the {\em sine qua non}\footnote{Without which not.} of measurement, as they are the result and the goal of factual inquiry. Those who believe data are facts can use the idea of numbers to represent this. 

However, numbers are also the {\em without which not} of databases, being a different order of thinking about a record. A number in a database sense can be either the content of a record or the sign by which a record is stored, because a series of bits are simply numbers. Both uses are evident and present evidence of a fascinating linguistic trap for future research. 

From the point of view of question of interest 1, the surveys were a complete success: people demonstrated that they had philosophies of data that departed in ways far beyond mere semantic differences over subjectivity and objectivity. Furthermore, different roles tended to cluster around different understandings of the term, suggesting that educational background combined with the current uses of the term strongly inform meaning. 

\section{Methodological assessment}

I presented two questions of interest: do people have different philosophies of data and are my tools capable of capturing someone's philosophy of data? I believe that both questions of interest were satisfied, with my methodologies quite neatly capturing three different schools of thought apparent in both the interview and the survey. As these philosophies roughly correspond to extant definitions in the Oxford English dictionary, and some observations by Zins\footnote{See page \at[Zins].}, there are few suggestions that personal bias interfered with the collection of evidence. 

During the interviews, the \SDFN\ process served as a fantastic tool to evoke the participants' philosophies.  To improve the ability to probe, future researchers should focus on a single topic area with all participants exploring the same universe of discourse. This uniformity may allow the application of more articulated graph theory analysis. With a more standardized topic, not even the other participants would be able to reliably reconstruct identity from simply reading diagrams. More research can be done on better ways to render the \SDFN. Although Graphviz does create adequate representations, it should be possible to create more consistent renderings and a more intuitive labeling scheme than currently exists. 

Examining the interviews, I believe they managed to accurately probe the philosophies of data of their participants, especially through the use of the bubble diagram as a tool for evoking philosophies. Each interview felt as if it was probing the participant's philosophy of data, getting them to crystallize their definitions through the bubble diagram process, which created cognitive dissonance between their use and their theories of use.

With the discussion after the diagrams, participants usually discussed their internal thoughts. The interviews were not a complete success. Although it is probable that I found each participant's philosophy of data, the different topics of each interview severely limit commensurability. In many ways, the different topics made {\em comparing} philosophies of data across interviews very difficult. Recursive analysis, while useful, does not provide any kind of structure to suggest reliability. Future studies must use more reproducible methods of analysis or external coders to minimize or average out bias. Recursive analysis may be profitably employed as a coding technique, as it would merely ask coders to summarize small sections of interview or survey. 

Examining the surveys, I find that they were successful, but suffered from some large problems. The first problem is that the participants did not quite understand the intent of the survey and treated the scenarios in different ways. Furthermore, some participants left the explanations blank, causing great difficulty in trying to understand what they meant by the terms. Future surveys will have more clear instructions, required descriptions of answers, and more choices in the drop down box, including a \quotation{not sure.} 

The surveys' analysis did not employ any formal analysis techniques, having insufficient quantity for any statistical analysis over each survey question and insufficient length of responses for any kind of recursive analysis. Although the responses may be more penetrable to some sort of computerized coding technique, I saw no need for that here, as I was not trying to make statistically rigorous statements of fact. Further studies, however, may have the opportunity for larger research sets with more tested scenarios, enabling interesting and objective conclusions to be drawn.

\section{Summary of Analysis}

The three large philosophies of data that emerge from all three analyses are: data-as-communications, data-as-observations, and data-as-facts.

Data-as-communications forms a small subset and generally indicates transferred or stored signs with some relationship to computers. Data does not have to be interpreted to be real: it exists as stored signs, and that creates its reality.

Data-as-observations is a broad category suggesting that data is the product of observations that sentient creatures make of their environments. It is subjective and requires filtering to winnow out interesting data from the huge mess of data sleeting around us.

Data-as-facts is a narrower category, indicating that a piece of data is a scientific fact, used for the revelation of knowledge, discovering how the universe works by looking at the relationships and arrangements of things. 

\subsection{Links with the literature}

There are two authors whose research into the various relationships of data, information, and knowledge presented useful assistance in my analysis. The first is Dr. Ilkka Tuomi, who articulates a reverse hierarchy of data, information, and knowledge, a hierarchy to which at least one of my participants subscribes. The second researcher is Dr. Chaim Zins, whose surveys in many ways validated Ackoff's \quotation{standard} hierarchy with his experiments, despite showing that there are many definitions of data, information, and knowledge.

\subsubsection[Zins]{Zins' concepts}

Zins performed a \quotation{collective knowledge mapping} of a number of researchers using a critical Delphi methodology over three rounds of research\cite[Zins2007,Zins2007a,Zins2007b,Zins2007c]. Although his approach looked at definitions directly, he found two different concepts of data and a similar ontological split over subjectivity, objectivity, and communication. 

While I have utilized his works for justification of my literature, I did not closely examine his conclusions to avoid biasing my analyses. Zins, in 2003, identified areas of difference similar to those I identified\cite{Zins2007b}:

\startextract

Six distinctive concepts. Having established the distinction between the subjective and the universal domains, we are in a position to define the three key concepts data, information, and knowledge. In fact, we have six concepts to define, divided into two distinctive sets of three. One set relates to the subjective domain, and the other-to the universal domain. 
[Data-Information-Knowledge] in the subjective domain. In the subjective domain, data are the sensory stimuli, which we perceive through our senses. Information is the meaning of these sensory stimuli (i.e., the empirical perception). For example, the noises that I hear are data. The meaning of these noises (e.g., a running car engine) is information. Still, there is another alternative as to how to define these two concepts-which seems even better. Data are sense stimuli, or their meaning (i.e., the empirical perception). Accordingly, in the example above, the loud noises, as well as the perception of a running car engine, are data. Information is empirical knowledge. Accordingly, in the example above, the knowledge that the engine is now on and the car is leaving is information, since it is empirically based. Information is a type of knowledge, rather than an intermediate stage between data and knowledge. Knowledge is a thought in the individual's mind, which is characterized by the individual's justifiable belief that it is true. It can be empirical and non-empirical, as in the case of logical and mathematical knowledge (e.g., \quotation{every triangle has three sides}), religious knowledge (e.g., \quotation{God exists}), philosophical knowledge (e.g., \quotation{Cogito ergo sum}), and the like. Note that knowledge is the content of a thought in the individual's mind, which is characterized by the individual's justifiable belief that it is true, while \quotation{knowing} is a state of mind which is characterized by the three conditions: (1) the individual believe[s] that it is true, (2) S/he can justify it, and (3) It is true, or it is appear to be true.

[Data-Information-Knowledge] in the universal domain. In the universal domain, data, information, and knowledge are human artifacts. They are represented by empirical signs (i.e., signs that one can sense through his/her senses). They can take on diversified forms such as engraved signs, painted forms, printed words, digital signals, light beams, sound waves, and the like. Universal data, universal information, and universal knowledge mirror their cognitive counterparts. Meaning, in the objective domain data are sets of signs that represent empirical stimuli or perceptions, information is a set of signs, which represent empirical knowledge, and knowledge is a set of signs that represent the meaning (or the content) of thoughts that the individual justifiably believes that they are true.

Signs Versus Meaning. Defining the Data-I-Knowledge phenomena as sets of signs needs to be refined. There is a fundamental distinction between documented (i.e., written, spoken, or physically expressed) propositions and meanings. \quotation{E = MC2},\quotation{E = MC2}, and \quotation{E = MC2} are not three different types of knowledge. These are three different sets of signs that represent the same meaning. In other words, they are three different utterances of the same knowledge. Knowledge, in the collective domain, is the meaning that is represented by written and spoken statements (i.e., sets of symbols). However, because we cannot perceive with our senses the meaning itself, which is an abstract entity, we can relate only to the sets of signs (i.e., written, spoken, or physically expressed propositions), which represent it. Apparently, it is more useful to relate to the data, information, and knowledge as sets of signs rather than as meaning and its building blocks.
\stopextract

My work profoundly agrees with the discoveries he made, though my research focuses far more on data and differentiates three different orders of data to his two. 

\subsubsection[Galison]{Trading Zone}

Also of interest is the way that the interviewees demonstrated the idea of a trading zone. As Galison defines it\cite{Galison1997b}, 


\startextract

These considerations so exacerbated the problem [of physicists communicating] that it seemed as if any two cultures (groups with very different systems of symbols and procedures for their manipulation) would be condemned to pass one another without any possibility of significant interactions. Here we can learn from the anthropologists who regularly study unlike cultures that do interact, most notably by trade. Two groups can agree on rules of exchange even if they ascribe utterly different significance to the objects being exchanged; they may even disagree on the meaning of the exchange process itself. Nonetheless, the trading partners can hammer out a local coordination despite vast global differences. In an even more sophisticated way, cultures in interaction frequently establish contact languages, systems of discourse that can vary from the most function-specific jargons, through semi-specific pidgins, to full-fledged creoles rich enough to support activities as complex as poetry and metalinguistic reflection. The anthropological picture is relevant here. For in focusing on local coordination, rather than on global meaning, one can understand the way engineers, experimenters, and theorist interact. At last, I come to the connection between place, exchange, and knowledge production. Instead of looking at laboratories simply as the places at which experimental information and strategies are generated, my concern is with the site -- partly symbolic and partly spatial -- at which the local coordination between beliefs and action takes place. It is a domain I call the trading zone.

\stopextract


The requirement of locally true definitions applies across the original trading zones between cultures and, more interestingly, to the various cultures of physics. In the interviews, I noticed some evidence for trading zones in the interview material. Specifically, when the various participants referred to the terms \quotation{raw data} and \quotation{derived data,} they seemed to be using a local definition of data that did not correspond with their own philosophy, strictly speaking. Instead, they were referring to various sensor products that were, indeed, \quotation{raw data} to every member of the team. 

The extension of the Galisonian trading zone concept is not new to this research. In fact, business researchers have used the idea of trading zones and some sophisticated ideas of boundary demarcation for quite some time, going so far as to use graphs and knowledge maps (as opposed to my \SDFN) to identify different groups. Wilson and Herndl use this methodology when they describe their understanding of knowledge maps and trading zones\cite{Wilson2007}:


\startextract

The knowledge maps we created and shared with project participants encouraged cooperation and mutual understanding rather than the slash-and-burn rhetoric of demarcation events. When technical experts discuss the parts and subfunctions they have made, they get to describe their local practice, explain their knowledge, and open up their community-specific lexicon within the ecological relations of the boundary object. As they trace the lines connecting the boxes on the knowledge map, participants articulate communities of practice: each distinct but also connected through the boundary object. Because it is plastic and robust, the knowledge map balances the demands of identification and division in Burke's terms. As boundary objects, the knowledge maps help to create a rhetorical space that is best understood through Galison's notion of the trading zone.

\stopextract


This methodological description of their work is focused in the anthropological study of finding sub-cultures, rather than language differences. Despite this, my analysis produced many of the same results as theirs (methodologically speaking, if not with respect to content) because both my research and theirs tried to understand different philosophies/cultures with the metaphor of trading zones. 

Just as they extend the concept of trading zone to consultants on the Washington beltway creating local definitions for a program already approved by the Pentagon, I extend the idea to how different philosophies of data interact. Their use of Graphviz-generated graphs as talking points to determine cultural ramifications matches my experience of using said graphs to generate philosophical insight: 


\startextract

In the case we have been exploring, the knowledge map is crucial to the emergence of something like Galison's (1997) trading zone. Participants develop Galison's \quotation{possibility of communication and joint action} (p. 803) through the map as it emerges. The map continually structures how the team understands and explains the project.


\stopextract

The differences between their study and mine are quite pronounced, although just as they observed different groups creating temporary trading zones through the use of knowledge maps, I observed something similar with the pidgin concepts of \quotation{derived data} and \quotation{raw data.} While waiting for interviews, I saw researchers passing around sheets of paper with pictures of phenomena on them and referring to them explicitly as raw data. This practice almost certainly serves to inform the local definitions and create the trading zone necessary for successful research practice. 

\subsubsection[Accent]{Evaluative accents}

An evaluative accent is the set of interpretive filters a recipient applies to incoming communication, thereby changing its meaning based on the biases applied by the recipient, shared understanding, and cultural mores. It was originally used to explore the effects of Marxist propaganda, but it can also be an interesting way to explore how trading zones operate effectively. 

V. N. Voloshinov suggests an idea of an evaluative accent\cite{Volosinov1994}: 


\startextract

Any word used in actual speech possesses not only theme and meaning in the referential, or content, sense of these words, but also value judgment: i.e., all referential contents produced in living speech are said or written in conjunction with a specific evaluative accent. There is no such thing as word without evaluative accent. 


What is the nature of this accent, and how does it relate to the referential side of meaning?
The most obvious, but at the same time, the most superficial aspect of social value judgment incorporated in the word is that which is conveyed with the help of expressive intonation. In most cases, intonation is determined by the immediate situation and often by its most ephemeral circumstances. To be sure, intonation of a more substantial kind is also possible. ...

All six [uses of a single word in a removed quote] by the artisans are different, despite the fact that they all consisted of one and the same word. That word, in this instance was essentially only a vehicle for intonation. The conversation was conducted in intonations expressing the value judgments of the speakers.

\stopextract

Beyond the verbal intonation is context and use. In a more modern sense, people of different generations use what amounts to steganographic\footnote{Steganography is the act of hiding messages within other messages, where, only if you know the pattern or encoding scheme, can you identify the hidden message.

Social steganography is the use of shared context to provide polymorphic (meaning-changing) meanings to one's social communications, depending on context and other available social cues.
} encryption in their status messages, relying on context and the source of quoted material to produce different meaning for different people. 

An instance of this social steganography appears in the following example to pass different meaning through a Facebook post to the subject's friends and mother:

\startextract

When Carmen broke up with her boyfriend, she \quotation{wasn't in the happiest state.} The breakup happened while she was on a school trip and her mother was already nervous. Initially, Carmen was going to mark the breakup with lyrics from a song that she had been listening to, but then she realized that the lyrics were quite depressing and worried that if her mom read them, she'd \quotation{have a heart attack and think that something is wrong.} She decided not to post the lyrics. Instead, she posted lyrics from Monty Python's \quotation{Always Look on the Bright Side of Life.} This strategy was effective. Her mother wrote her a note saying that she seemed happy which made her laugh. But her closest friends knew that this song appears in the movie when the characters are about to be killed. They reached out to her immediately to see how she was really feeling.


\stopextract

The use of \quotation{Always Look on the Bright Side of Life,} as Boyd discusses, is an example of a successful steganographic encoding of a message. Her friends could decrypt the hidden message because they shared a private context of culture with Carmen, a shared evaluative accent\cite{Boyd2010}.

Although that process is called social steganography, its unintentional practice causes gulfs in evaluative accent. Failed obscure jokes are an example of an incorrectly parsed communication. The obscure joke, in this case, relies on a shared commonality to be correctly anticipated by the recipient, and this mode of receptive listening is informed by the evaluative accent. In more common use, the language in a business memo may be so full of \quotation{business-speak} that someone who is not used to the company may misunderstand the provided references. This misunderstanding is especially deadly if it makes sense within the reader's incorrectly applied accent. As the statement can be parsed by the listener, only the mismatch of reality models in later conversations can hint at the source of the problem: the misinterpreted statement.

Voloshinov believes that the evaluative accent partially belongs with the speaker, but also that there exist \quotation{side bands} of communication, such as intonation and body language, that are specifically interpreted by the recipient of any communication, in the recipient's context. 

Literal intonation has very little to do with the philosophy of data. Yet the term is used in everyday, technical, engineering, and scientific speech. The full weight of the evaluative accent, as seen in the interviews above, falls into the use of context and role. While originally it was seen as way to frame ideologies; almost a post-hermeneutic way of explaining some failures of Marxism\cite{Voloshinov1929}. The idea of an evaluative accent can be combined profitably with the philosophy of the trading zone.

The idea of an evaluative accent corresponds well with the idea of a trading zone. The construction of misunderstanding the same language could hardly be the result of a simple linguistic misunderstanding. When a data-as-subjective-observation person says \quotation{data} to a data-as-objective-hard-numbers person, both of them are using a \quotation{functionally correct} definition, a definition shared by many people. They are encountering a trading zone. As Galison states, \quotation{In the trading zone, where two webs meet, there are knots, local and dense sets of quasi-rigid connections that can be identified with partially autonomous clusters of actions and beliefs.}\cite{Galison1997b} My identification of different philosophies of data certainly corresponds with these diverse beliefs. And those beliefs inform the evaluative accents that people use when they use and receive the term \quotation{data.} 

While different uses of data may be boundary objects for more profound cultures,\cite{Chrisman1999, Collins2007, Gorman2002, Star1989} this minimal investigation can scarcely provide an anthropological look into the various research cultures in existence. I can present research into the intentional creation of a local language. The use of \quotation{raw} and \quotation{derived} as a semiotic prefix presents a linguistic indicator to switch evaluative accents to recipients who are {\em aware} of that indicator. The practice of forming local trading zones, by repeatedly presenting symbols to the other party in an environment where people are aware that bridging must occur\cite{Galison1997b}, is the non-ideological practice of causing sufficient cognitive dissonance in the recipients for them to \quotation{bud off} a new evaluative accent for interpreting the incoming sign-set. 

The evaluative accent of the local definition of a word can be understood in the context of Peircian semiotics. As Aktin notes\cite{sep-peirce-semiotics}: 

\startextract
In Peirce's theory the sign relation is a triadic relation that is a special species of the genus: the representing relation. Whenever the representing relation has an instance, we find one thing (the \quote{object}) being represented by (or: in) another thing (the \quote{representamen}) and being represented to (or: in) a third thing (the \quote{interpretant}).
\stopextract

The interpretant serves as the developed representation of meaning. Thus, we can understand local trading zones to be the product of evaluative accents present in the interpretant\cite{Chandler2001}.

\section{Conclusion}

My findings fundamentally agree with and extend both Chaim Zins' work and the definitions of the {\em Oxford English Dictionary}. While more research is necessary, there is little doubt in my mind that I have answered the two questions of interest: 

\startitemize
\item I have discovered philosophies of data that profoundly differ from each other and that agree with other, related research. 
\item Some evidence points to the existence of trading zones between different philosophies of data. The use of the terms \quotation{raw data} and \quotation{derived data} in the interviews strongly suggests this. However, more research is necessary before there is any useful evidence for how trading zones form with philosophies of data and what impact evaluative accents have on data and misunderstandings.
\item There is evidence for both Ackoff's and Tuomi's hierarchies of data, information, and knowledge. Ackoff's work has penetrated modern thinking, especially management thinking, quite deeply. At the same time, the more \quotation{modern} relativistic philosophies of science seem to correspond quite well with Tuomi's philosophy. Far more evidence is needed to create suggested \quotation{trading zones} that can bridge these and other discovered philosophies.
\item Some indications hint that the philosophy of data is a factor in database design, though the current evidence cannot create useful rules for database changes based on the recipient's philosophy of data. More implementation research is needed here, as well. 
\item Both the interviews and the surveys discovered different philosophies of data, but not without failures. More research is necessary to explore the utility of my methodology and to refine it for different purposes. 
\stopitemize

\stopcomponent
