\startcomponent c_2_3_basis
\product prd_Chapter2_Justification
\project project_thesis

I use the terms pragmatism, social construction, and constructed reality in this document. In no way should these terms be taken as assertions about any kind of {\em fundamental} reality. I simply do not know enough about the \quotation{real} to make any kind of philosophical claims about it. Instead, this research deals with the reality of computing as understood by humans and their protocols. While computing has some mathematical-algorithmic basis, the application of specific mathematics and the choice of particular logics for use within a computer system mean that the computing subset of the real is socially constructed. There is no fundamental definition of a \quotation{bit}\footnote{Even within a computer, a bit can be represented in many different ways, depending on the specific architecture. It is this social construction of computing that creates the gulf between programs designed for the set of protocols called OS X and the set of protocols called Windows.}, much less any higher construction in a computer. 

In the same way, this document is pragmatic about pragmatism. The goal of this exercise is to discover a methodology that works and a reason to use that methodology, not to make sweeping claims. 

This research uses \XPhi\ to document pragmatic and realistic philosophies that reflect how people understand data. In my experience, many of the major philosophies of technology that explore questions of computing do it from an outsider's perspective: they are philosophers first, looking into the computing areas\footnote{On searches of the term \quotation{philosophy of computing} on Google scholar in 2011, the best cited and most articulate members of that field don't actually practice computing. When discussing the practical elements of computing, this is certainly a problem.
Even this discussion of the uses of computing in philosophy seems to regard the computer as an adjunct to the philosophy of computing or of science, rather than as an important component\cite{Thagard2004}.}.

\XPhi\ looks mainly at the philosophical implications of psychology. The philosophy of technology explores the impact of technology in our lives. There is a middle ground between them, a space for \quotation{American pragmatism,}\footnote{On why philosophy matters on a practical level:
\quotation{But there are some people, nevertheless-and I am one of them-who think that the most practical and important thing about a man is still his view of the universe. We think that for a landlady considering a lodger, it is important to know his income, but still more important to know his philosophy. We think that for a general about to fight an enemy, it is important to know the enemy's numbers, but still more important to know the enemy's philosophy. We think the question is not whether the theory of the cosmos affects matters, but whether in the long run, anything else affects them.}\cite{Chesterton} As referenced in James' pragmatism\cite{James2007}.} that can create a philosophy from the opinions of practitioners and create the philosophy for that same audience. As this document and philosophy are intended {\em for} multiple audiences, the use of jargon is minimized. 

The aim of the philosophy of data is to understand this thing everyone calls data. Despite professionally designing databases, I still do not know what data is in any fundamental sense. Data is a tacit thing, with a fluid meaning that depends on the participant in the conversation. My professional definition of data, evolved for a class of high-level database, is \quotation{data is an electronically recorded observation of reality.} This definition may be unique to database designers and is certainly meant to satisfy how a database {\em uses} data. Due to its specificity, this definition will not be explored further in this document. The intent is not to imprint my own philosophy onto the people I interview, but find out how they view and make-a-thing-of\footnote{Reify. For an excellent discussion of reification and our propensity to engage in it, see Pratchett's Science of Discworld\cite{Pratchett2002}.} data.

I want to understand how we classify this thing called data and how we manipulate it. Each conception of data has its own mental affordances. Different conceptions {\em must} have different affordances, as the set of possible actions on a thing is one of the practical definitions of a thing. By looking at the transfer of data between people of different philosophies, I want to understand how different people interact with data, and how it changes due to those interactions.


\stopcomponent