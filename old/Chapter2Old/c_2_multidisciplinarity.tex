\startcomponent c_2_multidisciplinarity
\product prd_Chapter2_Justification
\project project_thesis


The basic principles surrounding the philosophy of data, if they exist, probably exist in other fields\footnote{See the Venn diagram in my paper for AMCIS 2009\cite{Ballsun-Stanton2009}.}. The diverging understandings of data are partially a product of its adoption by different fields, because the word has different affordances and utility for an engineer's practice than it does for a librarian's activities. Jargon is the special shorthand that the practice of a field creates for a very specific concept\footnote{For a fascinating study of the linguistics of \IT\, I recommend Shortis' book as it develops many different forms of linguistic evolution in what he calls \quotation{ICT}\cite{Shortis2001}.}. Problems arise when different fields use the same word as jargon, but with different meanings. Such is the problem with data. Every field contributes its own understanding, its own jargon. By demonstrating the deep philosophical differences in the usage and understanding of data, one can hope to set up future research that may be able to bridge those differences. 

To form that theoretical basis, I need to demonstrate that people construct data differently through different conceptualized affordances\footnote{Affordance: a perceived means of interaction with the target. \quotation{A flat metal plate on a door affords pushing, and a handle on a door affords pulling, and the thing to do with a testable hypothesis is to go test it.}\cite{Yudkowsky2011} For a look at affordances in interfaces, Norman's book {\em The Design of Everyday Things} is highly recommended\cite{Norman2002}. } based on how they think they can interact with data. My intent in this investigation is to investigate the dynamics of how people interact with data. Those interactions include categorizing something as data, basing actions upon data, and considering data's relationship with information and knowledge. 

One of the dangers in forming a theoretical basis is the borrowing of intents from parallel investigations into the philosophy of information or the linguistics of data. Floridi's philosophy of information is exploring computer-theoretical uses of information\cite{Floridi2002}. Voss is exploring the linguistics of data\cite{Voss2011}. Both of these investigations, at their core, are computer-centric: they define data by the computing architecture and protocols surrounding it, and theorize and make observations on that basis. When exploring the possibility of different philosophies of data, I cannot begin my exploration by assuming a computational component or basis. 

The term \quotation{data} predates the idea of digital computing. It even predates human computing\footnote{Computers were originally human workers applying mathematical functions in assembly-line fashion.}. To assume that the technological adoption of the word grants any definitional privilege is fundamentally erroneous. This investigation into the philosophy of data will not assume a technological basis despite the significant and technical uses of the term in engineering, science, and mathematics. The use of data in these three disciplines is not inherently computational\footnote{1640s, plural of datum, from L. datum \quotation{(thing) given,} neuter pp. of dare \quotation{to give} (see date (1)). Meaning \quotation{transmittable and storable computer information} first recorded 1946. Data processing is from 1954\cite{Harper}.}. The impact of computing on these fields' understandings of data, therefore, becomes a question to explore rather than an assumed answer. 

The essence of \XPhi\ is to abandon privileged intuition for gathered evidence of philosophy. To use a philosophy of data practically, I need to understand what people mean when they use \quotation{data.} To do this, I cannot prejudice any possible definitions, even to the extent of defining in my methodology the idea of data under investigation. By allowing participants to construct their own definitions of data by example, rather than by theory, this research will explore the native philosophies of data, rather than those constructed to meet an imposed definition. The practical requirement exists because of the contradictory dictionary definitions\footnote{A curious function of the OED definitions is that they comprise most of the uses of data that I found, though it is not without significant reflection that it is possible to match a particular use with a particular definition. My methodology of the Social Data Flow Network will help developers do just this. 
OED defines datum as:
\quotation{A thing given or granted; something known or assumed as fact, and made the basis of reasoning or calculation; an assumption or premise from which inferences are drawn}, \quotation{datum of sense}, \quotation{The quantities, characters, or symbols on which operations are performed by computers and other automatic equipment, and which may be stored or transmitted in the form of electrical signals, records on magnetic tape or punched cards, etc.}, \quotation{ In pl. Facts, esp. numerical facts, collected together for reference or information.}, \quotation{Used in pl. form with sing. construction.}\cite{Oxford2011}.}. By creating a methodology to route around theoretical definitions, this research will uncover the practical uses, affordances, and relationships of data. 

The requirement to solicit philosophies of data precludes the use of prior philosophies or even definitions as developed in fields like Information Sciences. My disuse of prior definitions of data may invalidate what some people claim to be philosophy in the first place. Some would claim that, without a unifying definition of data it is impossible to look at what the philosophy of data actually is. This statement, while valid, misses the essence of my exploration: we do not know enough about data, philosophically speaking, to construct any kind of definitive framework without making assumptions. I want to find out how different people understand the term. By intentionally not defining the term, and by constructing my methodology to let the participants' practice, rather than imposed theory, define the term, I can gather evidence upon which to frame {\em future} debates.

\stopcomponent
