\startcomponent c_2_intent
\product prd_Chapter2_Justification
\project project_thesis


Data is defined by its use. It is a socially constructed term\footnote{While the term Data, as language is socially constructed, there are a large number of people who feel that the content of Data, as measurements of reality, cannot be so constructed\cite{Bruffee1986}.} rather than a reflection of some property of the universe. Therefore, data is subjective relative to the person using the term. When exploring the philosophy of data, though, one needs a methodology to probe other peoples' philosophies of data. It is easy to come up with intuitive guesses as to the universal nature of data. The problem with intuitive definitions is that their elegance may not be used or tested in reality. For research to be useful to practitioners, it must deal with the philosophical problems that they face, not add another definition onto the large heap. This research aims only to provide a tool and a reason for practitioners to use that tool; the practice of finding new philosophies of data and persuading people of their use is left for further research\footnote{For more details on further research directions, see page \at[Further].}. Creating a philosophy of data is only useful in so far as people in the field can use that philosophy for practical ends.

To create such a philosophy of data, one must base it on results found in the world. \XPhi\ is a tool for philosophers that allows them to assess philosophies present in the world rather than creating their own, novel philosophies from whole cloth\footnote{In many ways, this is a straw man argument. However, the practice of intuitive philosophy is to explore and formalize the philosophers' own experiences and observations of the world. The practice of \XPhi\ is to explore and formalize others' experiences of the world.}. In a description of the assessment of intuition by \XPhi, Alexander and Weinberg state\cite{Alexander2007}:

\startextract
It has been standard philosophical practice in analytic philosophy to employ intuitions generated in response to thought-experiments as evidence in the evaluation of philosophical claims. In part as a response to this practice, an exciting new movement -\XPhi - has recently emerged. This movement is unified behind both a common methodology and a common aim: the application of methods of experimental psychology to the study of the nature of intuitions. 
\stopextract

\XPhi, however, is more than the simple application of experimental psychology methods to traditional philosophical problems. It is the application of scientific or social scientific techniques to understand and document the philosophies present in the world. The application of \XPhi\ to the philosophy of data is just one possible approach, setting aside anthropological, sociological, linguistic, and psychological methodologies for understanding both language use and the concepts behind specific components of language. 

To provide for the practical ends, the \XPhi\ used in this research must tell an engaging story\footnote{The case study, being non-reproducible, must tell an engaging story. While this research is not a case study per se, it shares the same central requirement of a case study: \quotation{the persuasive story.}\cite{Stake1995}.}. For the story to be credible, the methodologies used must be assessed and proven. A consequence of this process is to add those methodologies to the collective body of knowledge of both experimental philosophers and \IT\ practitioners. This research therefore will extend \XPhi\ beyond investigating moral philosophy and folk intuitions\footnote{One of the more notable studies discusses the role of investigating the folk intuition\cite{Knobe2007}. While there are other domains, as listed on Phillips' exemplary hub of \XPhi\ research, they all involve investigating moral or analytic philosophies and how normal intuitions interact with them\cite{Phillips2011}.} and will start exploring and testing elements of the philosophies of technology and science. 

This research may also contribute to the theoretical basis of \IT. As a pragmatic discipline, most of the \IT\ specific research has focused on practical matters of integration and translation between humans and computers and discovering useful concepts within other disciplines. As the differences in the philosophy of data between developer and client have real consequences in design, error, and usage patterns, however, the theoretical output of this research provides useful theories to \IT\ practitioners. These theories cover the critical and unique elements of \IT\ and should therefore be counted as part of the theoretical basis of \IT\footnote{While no footnote is sufficient to cover even a survey of the theories of IT, the following paper is a good example of a theory of \IT\cite{Orlikowski1991a}.}.

However, this research is not investigating basic principles. In the interest of producing a rough and pragmatic philosophy and a methodology for understanding data in local and practical terms, one needs to review differences of opinion, and to understand these differences so that \IT\ models and research models can be adjusted to fit all participants' philosophies of data.

\stopcomponent
