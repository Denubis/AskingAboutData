\startcomponent c_2_otherDomains
\product prd_Chapter2_Justification
\project project_thesis


 Whether or not this study of the philosophy of data is itself philosophy is beside the point. I intend it to ground the understanding of future philosophy and the pragmatic application of its tools and theories to various practicing disciplines, and in some ways to start thinking about the many, almost incommensurate, actual philosophies of data that exist. 

 I make no claims as to this research being part of the philosophy of knowledge's exploration into questions-of-knowing\footnote{This document will try to avoid philosophy specific jargon like epistemological and ontological. Consequently, there will be lots of hyphenated words and footnotes, but it should be more acceptable to a larger audience.}. Data may be an essential component of knowing, but trying to understand, model, and debate the interactions of various forms of knowing with the results of this research is far beyond the scope of this experiment. The philosophy of knowledge integrates well with more methods of \XPhi, and my exploration is designed to produce enough evidence to ask these questions. 

 The more technical side of the philosophy of knowledge, affiliated with artificial intelligence, also has a limited relationship with this research. The philosophy of artifical intelligence, in many ways, is trying to describe the human brain such that we can simulate it, or describe thinking such that we can simulate {\em that} without a brain\footnote{Some debates about the basis of different models of artifical intelligence are discussed in Catherine Howard's Thesis\cite{Howard2011}. There exists significant discussion on what aspect of the brain to model in artifical intelligence, if any.}. The idea of data is quite valuable to both of these philosophies. 

Exploring how humans understand data may lead to useful insights into how to make computers fundamentally understand, rather than simply process, data. However, the very aspect of engineering data to standards is what destroys this initial connection: instead of proposing a construction, the intent of this research is to explore the ideas of data currently extant. By looking at what different people think about data without constraining its use to a particular domain, my purpose is opposite to that of the philosophy of artificial intelligence. 

My research is not knowledge management. I do not seek to create a \quotation{data management} sub-field of knowledge management; there are no business-worthy buzzwords in my research that people could use to \quotation{leverage knowledge tools to create synergies, etc...} Although I do think that differences in the philosophy of data may contribute to siloing\footnote{The practices of small groups in organizations to communicate only within themselves and not to each other\cite{Jones}.}, little in this research suggests that it fits well with current knowledge management methodologies.

My hope is that the methodologies I propose will be adopted by fields like knowledge management, despite differences in their ultimate aims. Knowledge management aims to be a prescriptive field: offering \quotation{best practices} to achieve knowledge goals. This research is entirely descriptive, rather than prescriptive: only by seeing what is currently extant can we make useful hypotheses about better models.

Without understanding the current philosophies of data, data modelers, philosophers, and other interested people have very little save their own perceptions of reality on which to base their models of data. Therefore, this work, while useful to knowledge management, is not knowledge management.

 Nor is this philosophy of data formally part of \IT. Although I aim to supply theory and tools to \IT, the investigation of the philosophy of data is not strictly a concern of \IT. \IT\ is interested in adapting computer tools for specific real-world situations such that those situations become more effective and manageable through the application of {\em technology} that manipulates {\em information}. In this application of technology, just as in the case of knowledge management, the field turns prescriptive.

 This research cannot yield a grand unified theory of data. The traditional definitions of data stretch back over many centuries, and the different usages have become well entrenched. I do not propose creating a philosophy with predictive power. Rather, I propose a methodology with descriptive power and a reason to employ that methodology. In this way, regardless of their philosophy of data, practitioners may be able to use my methodology to discover other people's philosophies of data, and thereby enhance their understanding and their designs. 

\stopcomponent
