\startcomponent c_2_2_reality
\product prd_Chapter2_Justification
\project project_thesis

The realms of computing and, to an extent, the business of business are effectively Baudrillardian\cite{Baudrillard1994a} simulacra, unreal images that exist in their own locally true domains. They are beholden in their structure and content only to themselves, rather than to any underlying objectively true reality. Because protocols and contracts are human constructions, not all people understand their computing or business environment in the same way, and may attribute causes and effects to different things based on their own knowledge. 

Even our practices of science and engineering have their own human-created protocols, identifying matters of interest and precision simply because they use computing technology to investigate the world. When designing computing systems for people, the designer must understand the segment of reality constructed by and understood by the client, rather than imposing their own understanding of computing onto the problem. 

With the clients' mental affordances mapped into a computer-system, the computer will both \quotation{behave} as the client expects and store observations in keeping with the way that the client understands the world. Without this mapping, the computer system will be routed around until it either does its job according to the client's point of view or is marginalized such that it cannot do any further harm. Information technologists must understand the realities of the client to create computer systems usable by the client. 

\stopcomponent
