\startcomponent c_2_1_goals
\product prd_Chapter2_Justification
\project project_thesis

I want to help improve communications, and I believe that a philosophy of data could be one way to do so that has not been thoroughly explored. It may offer a theory towards explaining some errors in intergroup and intragroup communication. Furthermore, it may offer some direction towards exploring the philosophical basis of error by offering another take on system maps transmitted through communications\footnote{A system map is simply a person's internal mental model of how a thing operates and of how to get it to transition between different states. These maps may be communicated through instruction or alluded to\cite{Roy2008}.}.

I will, while exploring our ability to define and communicate data to people around us, lay a foundation for the exploration of the reality created by our use of data in computer systems. Our systems use data in multiple levels, from the hardware and simulated hardware through software and into fantastic constructions and games that embody and then produce their own data from any philosophical meaning. This study will not explore the various sub-constructions of data present on the Internet, in games, or in virtual worlds. Nevertheless, I hope that the methodology I create and validate can be applied to all sorts of computerized data: from the traditional bits down a wire into a simulation of a physics experiment inside Second Life\cite{Brown2008}. As humans use and create all of these tools, our philosophies inform them. 

To serve that end, this research creates and tests a methodology that can probe peoples' understanding of data. This methodology must be accessible to others in many different fields. By offering a tool that is independent of philosophy of data, I hope to allow researchers to probe other peoples' understandings of aspects of reality, or at least offer a method by which the practical usage of terms can be discovered.

I want to improve design methodologies. By contributing a theory of the philosophy of data, as well as a methodology for understanding specific philosophies of data to \IT, I hope to provide a tool to database designers and \infull{HCI} (\HCI) practitioners.

In database design, the hardest task is trying to understand the client's reality. Modeling a current organization's memory structures, its files and paperwork, and the relations among them in the minds of practitioners is an extremely difficult task. To facilitate understanding, this methodology is a tool for designers. The tool may allow them to understand what their clients think data actually {\em is}. 

By understanding the type of data being modeled, database designers make two significant gains. First, their data models can correspond with how their clients think about reality, and thereby create intuitive relationships and map the computerized model to their client's mental model more capably. Second, and in some ways more critically, they can then explain the database design {\em to} their clients in their clients' language, potentially shortening design times by reducing miscommunications.

In the same way, the proposed methodology should help extend normal modeling practice: simply making designers more aware of the different types of data philosophies may make more responsive designs possible. The demonstration of different philosophies of data is important to designers because it offers another meta-aspect of reality to be captured and incorporated.

I also want to create a method that can help extend \HCI\ design practice. This methodology should be applicable to all sorts of design, as it is a tool for rendering clients' realities and not a specific kind of technical reality. The discovery of practical meaning of terms, ideas, and affordances\cite{Norman1999} of data is another tool with which HCI designers can understand how to render data presented in an interface. A tool that can make elements of private jargon explicit, and that is focused on that task (rather than treating it as a happy byproduct) can significantly contribute to the HCI design cycle.

\stopcomponent