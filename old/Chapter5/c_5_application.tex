\startcomponent c_5_application
\product prd_Chapter5
\project project_thesis

\chapter[Applications]{Applications}

My research must demonstrate practical applications for other people. In order for my methodology to be useful, and for the philosophy of data to be applicable, there must be applications in the real world of both metholodolgy and analysis. This chapter explores those applications. I demonstrate that the \SDFN\ can be used beyond exploring the philosophy of data and some practical consequences of people considering the philosophy of data.  Each section of this chapter stands alone and is a potential application for the \SDFN\ or the philosophy of data. 

The SDFN is a methodology for exploring flows between entities and peoples' categorization of those flows. Application of this methodology may allow businesses to become more efficient and present academic researchers with new techniques for understanding how research participants differentiate concepts. Application of the philosophy of data may enable better designs by database designers and human-computer interface designers. It is, in general, useful to everyone needing to communicate \quotation{data} across disparate groups, as it raises awareness of the different interpretations of data, and can thereby reduce mistakes, errors, and misunderstandings. 

\section{Usage of the SDFN}

Throughout my analysis are hints as to how I could use the \SDFN\ Methodology for purposes other than discovering participants' philosophies of data. This section takes some of the more interesting possible uses in different fields and documents them. 

Both academia and business can profit from the adoption of the \SDFN. The methodology is one of many discovery methodologies, but its simplicity and definition by categorization may be a useful mind-mapping solution or a way to construct a trading zone between disciplines with vastly different jargons.

\subsection{Academic Applications of the SDFN}

One of the most important uses of the \SDFN\ is as a meaning-making methodology: one that researchers can use to extract the operational meaning from terms in an interview. The success of Question of interest 2 indicates that the categorization focused by the \SDFN\ is now a tested and valid way of probing operational definitions. 

As a meaning-making activity, the repeated categorization creates and then resolves cognitive dissonance in the mind of the participant. The participant categorizes flows according to how they use each term, as opposed to their theoretical definition of the term. Subsequent thought about the categorization then introduces a conflict between how they categorized and how they think about it. Researchers can use this research anyplace where discovering operational definitions of categorizable terms is important.

When the cognitive dissonance develops, participants try to relieve the mental pressure through discussion. This discussion, more than the \SDFN, is the real value of the methodology. By helping the participant articulate their differences of definition, and by recording that articulation, a researcher can capture actual operational definitions as the participant understands them. The process exposes how the participant thinks and their internal conflicts to the view of the interviewer.

The \SDFN\ provides a framework for the interview discussion. While many interview frameworks exist, especially knowledge-mapping frameworks\cite{Choi2006}, the \SDFN\ may offer a more practical access to how people use concepts, rather than to how they think about them. Therefore, by offering the methods of philosophy without philosophical jargon, we can understand how the participant understands certain concepts without spending significant time teaching philosophy's vocabulary. 

Asking philosophical questions of non-philosophers is mostly a pointless exercise. Specifically, asking philosophical questions {\em worded in philosophical jargon} is pointless. The intent of the \SDFN\ is to remove the intimidating and almost-taboo jargon from deep questions of philosophy.

By allowing people to approach the problem laterally, through categorization rather than definition, interviewers can inspect the evidence of the understanding without requiring participants to directly articulate their practical definitions. By focusing the categorization in a domain where the participant is a subject-matter expert, the participant feels confident and in control while meaning is generated.

The \SDFN\ interview methodology is not only useful for experimental philosophers performing interviews. It is extendable into focus groups and \infull{JAD} (\JAD) sessions. The DFD methodology has been used  successfully to diagram consensus in a focus group. 

Consensus building through \DFD\ construction requires that all members of the focus group be considered \quotation{subject-matter experts} in the topic of the universe of discourse. As the transforms are created, people in the group discover places where their understanding of reality is not reciprocated across the group, and argument erupts. This argument serves to generate a local trading zone if necessary or, more to the point, forces people to clarify their arguments and generally refine their statements. 

The \SDFN\ offers a more streamlined approach to knowledge discovery than the \DFD\ does. The modern use of the \DFD's focus is to translate business practices into technological systems. Although this usage is an excellent basis for collaborative and participatory software design, it is difficult to adopt fully when looking at more theoretical matters. The \DFD's focus on data, transformations, and atemporailty\footnote{A DFD is atemporal and acausal, a significant sticking point in trying to design one. No transformation can depend on any other transformation at the same level of detail, and therefore there can be no causality. } means that explicit definitions of data fields and their transformations are required. Such explicit definitions can be useful, but they can also serve as a major obstacle, taking valuable interviewing time and increasing frustrations.

\placefigure[]
[fig:dfd]
{This is an example of functional decomposition, with every transformation (the squares) being resolved into smaller and smaller squares. The transformation is \quotation{decomposed} into sub-transformations working together to achieve the primary transformation. 
}{\externalfigure[Chapter5/FunctionalDecomposition.pdf][factor=fit,frame=on]}


The \SDFN\ would potentially allow for more useful arguments in the focus group and the \JAD. By indicating how different stakeholders categorize flows, it can identify specific points of departure without getting bogged down in excruciatingly exact definitions of the data structures needed to successfully program an application. Most people do not naturally think in the required fractal decompositions of the \DFD, and arguing about the level on which a particular transformation belongs is another significant distraction. Arguments in focus groups tend to be definitional, articulating the nature of a transform or a flow. Looking at entities and their categorizations, rather than focusing on the necessary and tedious-to-define elements needed to create a working software specification, can direct the focus group's focus towards philosophical challenges rather than implementation. 

When presenting an \SDFN\ in a group environment, it is better to use it as a brainstorming document rather than as a definitional document\footnote{Maier explored how to get groups to problem solve optimally and found that focusing on defining the problem reduced fighting due to no-one getting too attached to early sub-optimal solutions\cite{Maier1970}.}. Thus, as groups disagree, the disagreements can be noted and saved for later. Batching the areas of disagreement, both in a small group and with the many stakeholders of a \JAD, means that patterns can emerge as part of the exercise, and the focus can be kept on abstract meaning-making. The \SDFN, when presented to a focus group, is a method for modeling the presence and absence of flows and their nature via categorization, rather than their specific qualities and manipulations. 

Although the \SDFN\ is useful in focus groups for academic research, it is also very useful in the \IT\ discipline. One of the largest areas of difficulty that \IT\ faces is in trying to map an organization's specific reality to a computer program. To make correct recommendations about which programs to use, which customizations are necessary, and which new interfaces and databases to correct, a new \IT\ project must model the \quotation{reality} or understanding that the final users have of how their business operates. This entirely subjective experience, facilitating and instantiating something found only in the minds of the members of an organization, requires careful exploration during the investigation phase. 

The \SDFN, like other \IT\ tools, offers a way of modeling the customer's understanding of the business reality. The process of categorization can be applied to more unusual aspects of meaning-making than operational definitions of data. Categories can reflect urgency, importance, relevancy, difficulty, or even utility. These categories are a way of both operationalizing the terms and evoking a deeper understanding of the reality that they model within their consciousness. For example, asking participants to categorize flows with respect to difficulty will generate diagrams reflecting their subjective perceptions of difficulty, and therefore areas upon which to focus \IT\ expertise. 

As a technique, the \SDFN\ is quite suited to finding holes in communication. In a large study, multiple interviews would be necessary to help every participant in an organization create an \SDFN\ covering all of their important business activities. Then, when the researchers compare entity diagrams and the relationships among them, topic areas diagrammed by various people will not only identify communicative silos (by revealing where few links between entities exist), but may also identify completely different understandings of those areas. In this sense, the requirement that every variable be completely and objectively defined is a poor fit for finding these areas of operational difference.

The \SDFN\ presents far less temptation to represent the \quotation{reality as it should be.}\footnote{When creating a database design with a DFD, the designer's first task is to model the \quotation{current reality.} When that model is completed, the \quotation{current DFD} is considered and \quotation{streamlined} to represent a future reality with the database within it. It is far too tempting to skip to the \quotation{reality as it should be} step without modeling the client's current reality.} The data-definition aspects of the \DFD\ present an urge to operationalize to a consistent definition. All flows of type X are operationalized as X and therefore the differences in understanding and opinion between the different participants are mostly eliminated through the conscious or unconscious use of the same or similar data dictionaries (which are themselves biasing towards a technical view of things.)

The \SDFN\ is an excellent tool for \IT\ because, although it is suggestive of the \DFD, it does not have the software legacy that the \DFD\ has and is ideally suited towards discovering the meaning that individuals attach to entities and to the relationships between them.

The other advantage of the \SDFN\ as a meaning-making technique in \IT\ is that it is far more approachable, both as a participant and as an interviewer. The \DFD\ has strict requirements for atemporality and acausality, which present considerable difficulties in intuitive rendering. As it is forbidden to note that one transformation happens \quotation{before} another transformation (because function design in structured language is itself atemporal with respect to incoming and outgoing data flows), the act of transforming the usually causal account into a \DFD\ distances the diagram from the participant's reality. This distancing, as noted before, is less useful when trying to understand the participant's reality and especially less useful when trying to model interactions of people in a company. 

\subsection{The \SDFN\ and \XPhi }

The \SDFN\ is an excellent way of probing definitions of philosophical understanding. It is a useful tool of \XPhi, which, as Sosa notes\cite{Sosa2007}:

\startextract
... bears on traditional philosophy in at least two ways. It puts in question what is or is not believed intuitively by people generally. Moreover, it challenges the truth of beliefs that are generally held, ones traditionally important in philosophy. Each challenge is based on certain experimental results.
\stopextract

The \SDFN\ is a way of generating experimental results. Specifically, by having all participants in the interview explore the same topic of the \SDFN, it is possible to render all diagrams commensurate. 

The \SDFN\ is a tool for probing practical understanding. Most current methodologies depend on scenarios and multiple choice questions to probe \quotation{what would you do?} Although the scenario-based survey questions are excellent at testing approaches to moral questions, by their very nature, they do not allow for inventive answers. In many ways, the \SDFN\ and its categorizations would be ideal complements to the survey, by allowing people to differentiate various types of moral action and to engage in what amounts to a structured role-play. 

At the same time, the split between practical and theoretical definitions in moral philosophy may produce unusually large cognitive dissonance in participants, decreasing the accuracy of the technique. Although assessing the philosophy of data is tricky, there are no overriding imperatives to present a \quotation{socially accepted} behavior towards data. Due to the lack of a moral dimension, there are few socially approved \quotation{right answers} that would impede a participant's true views of the topic. Hewing to the party line is important to the participant, but the articulation of the organization's policy is sufficient to demonstrate their own philosophies of data: they interpret the organization's desires through their own mental framework, producing answers that sound right to themselves.

To take advantage of the nature of flows in this instance, a potential experimental dynamic would be to categorize flows as \quotation{the right thing to do, the wrong thing to do, or neutral.} Specifically, the interviewer would have a set of two- to three-sentence scenarios befalling a group of related entities. Participants would then build the \SDFN\ from the scenario, gradually categorizing flows between people. 

\subsection{The \SDFN's role in business modeling}

The \SDFN\ is an excellent tool for business modeling. Just as Information Technologists need to understand a business to create devices and software for it, consultants and managers need to be able to know not only the logistical nature of their business, but also the subjective or practical structure. The \SDFN\ is one way to model the structure of an organization.

The \SDFN, if all participants engage, is a way to derive the actual structure of an organization organically. Depending on management philosophy, be it coordinative or authoritarian, people will identify communications from those who are their managers in practice, without regard to the organization chart. By creating an \SDFN\ with categories that are types of business communication, it should be possible to identify small groups, the communication hubs, and the actual functions each group performs.

Most formal charts apply a template to reality. They assert that because someone has the title of \quotation{director,} that word indeed describes what he or she is and should be doing\cite{Spolsky2008}. The nominal method of business organizations is to apply pressure to make reality conform to organizational charts. With the current rate at which demands on business are increasing, businesses cannot be that inflexible. This problem is compounded with more sophisticated management schemes, which make it even more difficult to model the reality of a corporation versus its actual practices. Data mining tools only go so far and do not actually track the most vital elements of management: communications.

The \SDFN\ can become a tool for understanding what people do in an organization. It is vital that it cannot be used in any kind of coercive setup, because it depends on people honestly self-reporting and actively engaging with the interviewer. However, by having interviewers spend time with people and asking them to model the entities of an organization, the SFDN will not only demonstrate their philosophies of organization, but will show where cognitive splits are occurring and where one group thinks it is communicating one thing but a different group is hearing something else. 

This technique might be extremely valuable to the academic side of business, because it can allow academicians to explore the utility of various management practices and to identify their consequent effects in an organization. Looking at the patterns of communications from all entities, the \SDFN\ might be a tool to supplement standard models of technological adoption. 

When modeling different perceptions of the organization, the \SDFN\ makes no attempt to be objective. In fact, the lack of objectivity is the very essence of the methodology's utility. By explicitly modeling, one can identify how the logical constructs of the organization diverge. Noting how management understands business reality and how programmers understand the same reality makes it possible to create a more reflective and accurate model of the organization.

The model reflects that different people have different areas of expertise. Combining the synthetic, subjective, and individual viewpoints where they match and creating what amounts to an organizational \SDFN, with every \SDFN\ mapped into one large document, will allow patterns with identified sub-groupings to emerge, as people detail the areas of their expertise. 

At the same time, the \SDFN\ is a way of identifying organizational silos: areas where there should logically be communications of vital information but, for one reason or another, the communication does not occur. By identifying strongly connected groups talking about the same self-categorized entities without links between them, organizations can improve communications.

It is also a means by which individual departments can advocate for changes within an organization. By identifying how they are viewed internally and externally, departments and individuals can identify problems caused by a lack of trading zones. The lack of border regions can, in part, then be rectified by intentional planning {\em once the problem has been identified}. 

This methodology may be able to expose the flows of data within a government organization, producing a map of practical reality and the functioning of government departments. Doing so would require an enormous scope and appropriate security clearances, but the identification of data flows could even help to identify redundant activities between remote departments. Identifying this redundancy may not convince either department to reduce its activities, but it may at least improve accuracy by giving each department someone with whom to compare results. At the same time, it may also identify communicative silos in departments and suggest useful and practical lines of communication out of those silos.

The process of organization modeling through the \SDFN\ focuses less on the exploration of definitions. Instead, the \SDFN\ creates a data-focused graph of communications within an organization, which can identify unlinked but mutual schools of thought, communications silos, and redundant activities. This is a way to expose the undocumented links that form the pragmatic reality of any organization. 

\subsection{Workflow Modeling and the \SDFN}

Beyond organizational modeling, computer-requirements modeling, or the academic evocation of philosophical definitions, it is possible to use the \SDFN\ for the sociological imaging of an organization. Although the organizational modeling focused on looking at communicative hierarchies, workflow modeling seeks to understand the communicative flows among people, objects, and other sources of data as a way of increasing the efficiency of an organization. The simplicity of the \SDFN\ allows people to describe their workflows in roughly their own terminology. The advantages of increased comprehension outweigh the disadvantage of decreased precision: it is better to get more takes on an organization's workflow then to get detailed but partially correct models from a few experts.

The \SDFN\ used in workflow modeling is an excellent tool for any consulting group, because it provides an interview technique that allows participants to walk the interviewers through their work without revealing the \quotation{deep secrets} of their work. One of the most difficult obstacles in most consulting jobs is the sense that participants have that the interviewer is trying to take away their jobs\cite{Nikolova2007, Young2004, Czarniawska2003, Kipping2002}. Stressing that this methodology is looking at interpersonal relations to improve communication and treating the participant as a subject matter expert empowers the participant and could subsequently make them more cooperative with the technique.

For purposes of workflow modeling, the categorization methods will change. It is important to note that this methodology must explore the flows between both devices and people, not just the organizational model of people. And, that the flows are \quotation{work units} rather than any communication. Work units are functionally a description of incoming and outgoing artifacts of work processes, reports, orders, and discussions to support the creation of those orders, among other things. Therefore, the \SDFN\ will take on more aspects of the \DFD\ and look at how entities pass and transform data between them. The various jobs or transformations should form the basis of categorization, rather than philosophical definitions. 

Even if two people have many different flows connecting them, the workflow model should be able to distinguish, based on the self-assigned categories, the nature of different jobs. Therefore, with the identification of different jobs between various entities, the \SDFN\ is a tool to answer the difficult question of \quotation{What is it, exactly, that you do here?} without relying on the usually unhelpful \quotation{job title.}
\subsection{Design consulting uses for the \SDFN}

The \SDFN\ is also a tool for software designers outside the bounds of the normal \IT\ academic uses. Although it can serve as an excellent check on the validity of \DFD\ and state transition diagrams, the process of defining terms through categorization serves as a way to understand what the clients are expressing as part of their data-flow diagrams and identifications. The normal data-flow diagram requires participants to define elements through iterative decomposition, but this process focuses less on specific definitions of flows (which may well just be discarded) and more on grouping those flows into categories representing the intuitions and realities of the client. 

This technique is a way to reconcile differences in the \DFD\ and other modeling techniques when the created models come from different people. The differences are expressed as categories, and then the people who showed the differences can be interviewed through this technique. 

The \SDFN\ is also a way to generate a rough model of an organization. This model is a handy side benefit beyond the categorization, such that detailed diagramming techniques can be employed at interesting junctures. The model of data flows must model the organization's practical realities, representing how data and information pass between entities. From an \IT\ standpoint, all organizations are simple information processors: they transform certain inputs into certain outputs. This process may cause a delivery or matter to exchange hands, but it is either outside the scope of the business or is represented as an information exchange. Thus, as a way of quickly modeling what different people consider to be the data exchange within an organization, the \SDFN\ is an excellent way to map the data flows of an organization, and thereby get an idea for how it functions. 

The rough mapping of an organization for purposes of design means that the system design can focus on the practical and real problems of the organization as discovered, rather than on the problems as articulated at the start. This mapping can be accomplished through what amounts to a random sampling of an organization without any of the categorization aspects. Once a rough communications map is generated, more focus can be applied to \quotation{interesting} areas: areas that diverge on many maps or areas that are flagged during interviews as a problem by many people. 

More vitally, the \SDFN\ is a useful tool for user experience designers. It is a way of exploring semantic expectations. Users would be able to create an \SDFN\ of applications they currently use and categorize different interaction methodologies within the purview of the user's current understanding of software systems. Beyond that, by describing entities within a software system, it is a tool for reconciling the needs of a group of users and helping them to jointly articulate what they want from an application using the same language. 

Because the \SDFN\ evokes the definitions of items through the process of categorization, I believe it is able to build trading zones by making participants aware of their own and others' evaluative accents. By applying the \SDFN\ across a group of users' expectations of the same planned software environment, it may be possible to learn each user's distinct categories and terms for entities and data within that environment. Using those terms where appropriate and explicitly highlighting possible misunderstandings may create a language among all potential users of the software.

Analyzing user experience, this methodology can diagram the nature of the normal inputs and articulate the differences beyond simple data-driven definitions such as \quotation{field, 14 characters long.} The level of detail that a \DFD\ requires is far beyond that needed for most enterprise models. However, the \DFD\ fails to function if the most trivial of flows is not precisely defined, because its philosophical basis of functional decomposition breaks down. Changing the perspective of the diagram from objective reality to subjective user's viewpoint removes the requirement for precisely specified data flows. 

By allowing designers to explore the expected inputs and outputs of the system {\em from the perspective of the users}, it becomes possible to build more sophisticated information architectures and user experiences tailored for the needs and expectations of those users. 

\section{Uses of the Philosophy of Data}

The philosophy of data as presented in this research is an intensely pragmatic philosophy. The exploration of real peoples' philosophies of data has little utility on its own, and derives most of its power from providing a means of exploring someone's internal reality as it relates to data.

Therefore, for the philosophy of data to be of value, it must be used. There are uses for the philosophy of data both in the business and academic worlds, outside the study of the philosophy as philosophy. The philosophy of data can be used as a unique consulting lens, for looking at the client's requirements in such a way that their data realities are weighted most heavily. Besides the sheer novelty of this approach, it may offer unique insights into clients' problems. 

The reason for the creation of the philosophy of data, however, was to improve database and user-interface design. There is a clear need to understand clients' philosophies of data when creating systems for them, and as such, the philosophy of data will be very useful to designers.

\subsection{Using the Philosophy of Data in buisness efficiency consulting}

The philosophy of data is primarily a means for understanding how people think and use data. In a consulting scenario, the primary product is transmitted understanding. Consultants can employ this understanding in design, in reorganization, or as a way to change lines of communication within an organization. By looking at how different people within a team relate to their sources of data, they can reduce sources of inefficiency and error in non-intuitive ways.

As an \XPhi, the philosophy of data offers a novel take on a whole suite of methodologies for understanding aspects of an organization that are not normally considered, but that profoundly affect day-to-day operations. The application of \XPhi\ protocols beyond those proposed in this dissertation may provide insights into the group philosophy of data within an organization, which may then be exported as suggestions for change to the organization so that people who do not share that philosophy are made aware of it. These protocols may expose philosophical disconnects in an organization. Disconnects may predict communicative siloing, detrimental political maneuvering, or sources of error.

The philosophy of data and \XPhi\ in general offer options similar to those provided by sociological and anthropological explorations in companies. They offer novel tools for describing, documenting, and then presenting the companies back to themselves.

The process of exploring an organization's philosophy of data, as encoded into its practices and machines, provides a novel framework for understanding an organization. Using the philosophy of data is far more than a venue for the \SDFN, as the methodology used does not determine the focus of the research. In this vein, although the \SDFN\ can be used as a methodology for many different inquiries, many different methodologies can be used to probe an organization's philosophy of data.

The survey methodology, with appropriate changes such as adapting the scenarios to typical business concerns, provides an excellent way of grasping how different aspects of an organization understand, and therefore use, data. By applying the survey first, practitioners may be able to identify \quotation{interesting} areas within an organization in which to do more thorough interviews, or they may just leave the survey as the only tool for a smaller-scoped investigation.

Understanding an organization's philosophy of data is important because data is vital to organizations. The databases of an organization serve as its collective memory. Individual entities, especially in a large organization, are functionally unable to operate the many concerns the organization has without a database. By exploring how different people understand the database, it may be possible to understand some of the actual problems that stem from data anomolies and corruption within the organization. 

In {\em Image and Logic}\footnote{See Image and Logic\cite{Galison1997b}, frontispiece, page xxi.}, Galison reproduces a comic describing how different departments at Stanford saw the \quotation{magnetic detector} in one of their highly expensive scientific devices: 


\startextract

In the early 1970s, as work came to completion on an extraordinary new particle detector at the Stanford Linear Collider, George Lee, an engineer and amateur cartoonist, sketched the frontispiece illustration. To the Accounting Department, the SPEAR Magnetic Detector was a pile of gold -- quite a bit of gold by the standards of the time; to Plant Engineering, the hybrid device was a substantial problem in water cooling and electrical feeding; to Electronics there were the delicate strands of instrument wire, and the powerful conduits to the solenoid. This was not a world of vanishing technical assistants, like those artisanal helpers who wrapped wire for Maxwell, or assisted Rutherford in counting scintillator flashes; this was a laboratory where the physicists played a leading role, but where everyone knew that the engineers were by no means mere extras. Amid such a coordinated cacophony of construction, physicists could be lampooned by likening them to bewildered scavengers, walking parks and beaches in the hopes of culling a gold ring from the discarded flip tops and bent nails. The humor of the cartoon -- and the reality of physics, I would argue -- resides in its startling diversity in how the machine looked from the vantage points of the various subcultures of physics. 

\stopextract


In many ways, the study of the philosophy of data is attempting to do the same thing, though to the intangible database rather than to the extremely obvious scientific instrument. Questioning the philosophies of data within an organization is a way of investigating how an organization understands, produces, and manipulates its critical signs.

Providing an organization with a discussion of its philosophy of data essentially creates a framework within the company that allows for self-reflection. Moreover, this self-reflection will seem to come from a privileged and external source, bypassing many of the feuds and filters in intra-company communications, thereby reducing any instinctive and historical antipathy to change by providing a new way of thinking about daily tasks. The external source repackaging the participant's own contributions, especially when stripped of identifying features, increases the sense of the new or alien, and allows people in a corporation to come up with new reasons to accept the presentation rather than to stick with their normal habits.

\subsection{Design and the Philosophy of Data}
The philosophy of data may also be an interesting way of thinking about usability engineering. It presents novel ways of both understanding usability elements and, more to the point, assessing the understanding someone has of an interface's underlying functionality. In many ways, combining the idea of the philosophy of data with usability engineering shifts the focus of the engineering toward understanding the person's relationship with data and what affordances they ascribe to the interface's presentation of data. 

When a group's philosophy of data is explored, it allows the designer to understand the interface-based trading zones of the group. The nature of data is strongly tied to its technological manifestation when dealing with interfaces, and the nature of the group's philosophy of data may strongly inform its presentation of data. For example, a group with a more subjective philosophy of data would require more ability to filter, especially the ability to export mental filters into the program. An objective group would want more ability to dig into the provenance of data and look at what they would consider to be the meta-data in an attempt to assess the quality of the data. 

The philosophy of data is also a way for designers to understand the subjective reality of the group. The philosophy of data touches on many questions of knowing related to business practices. While not directly related, assessing how a group decides that things are \quotation{known} and how they assess data informs their use of language and their expectations of \quotation{information} technology. Better understanding of how they think can lead to the presentation of both more novel and (one hopes) more intuitively useful prototypes to those who would use them.

This philosophy can also form part of the basis of design for larger systems. One of the many problems confronting interfaces that require mass adoption is that people's differing philosophies of data cause them to form different expectations of the interfaces, simply by changing the users' mental models of the system's affordances. A designer who thinks about philosophies of data can set up appropriate scaffolding in a program in anticipation of users' differing philosophies, and thereby head off user-experience complaints before they arrive.

\section{Conclusion}

The \SDFN\ is a powerful tool that can be used in many different disciplines. Although it was  conceived as a tool for \XPhi, no inherent restrictions prevent its use for meaning discovery. It is just as valid for exploring moral philosophy as it is for exploring organizational dynamics and behavior.

In the business world, it is an excellent tool for identifying potential silos, and for enhancing user experience and database design: it can uncover peoples' subjective realities without any need to reconcile multiple realities into any kind of \quotation{objective world.} As a practical and pragmatic tool refreshingly free of jargon, it may prove quite useful in the business world. 

The philosophy of data also has its use, independently of the \SDFN, as an item of great consideration in consulting efforts and design efforts. The nature of an organization's philosophy of data should inform the tenor and nature of suggestions to the organization and the basis of a consultant's understanding. It is also an important mechanism for designers of all stripes to understand as another factor affecting how their clients' perceptions of constructed reality differ from their own. 

\stopcomponent
